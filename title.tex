% !Mode:: "TeX:UTF-8"
%\renewcommand\refname{参考文献}
%\renewcommand\abstractname{摘要}
%\ctexset{section = {
%	name = {\S},
%	number = \arabic{section},
%	}
%}
\ctexset{
	contentsname = {\zihao{3}\mdseries\heiti 目录},
	part = {format = {\zihao{3}\mdseries\heiti\centering}},
	section = {
		format = {\zihao{-4}\mdseries\heiti\flushleft},
		number = \bfseries{\arabic{section}}
	},
	subsection = {
		format = {\zihao{5}\mdseries\heiti\flushleft},
		number = \bfseries{\arabic{section}.\arabic{subsection}}
	},
	subsubsection = {format = {\zihao{5}\mdseries\songti\flushleft}},
}

\titlecontents{part}
			[0em]
			{\zihao{3}\mdseries\heiti}
			{\contentslabel{0em}}
			{\hspace*{0em}}
			{\hfill \bfseries\contentspage}

\titlecontents{section}
			[2.3em]
			{\zihao{-4}\mdseries\heiti}
			{\contentslabel{2.3em}}
			{\hspace*{-2.3em}}
			{\titlerule*[1pc]{.} \bfseries\contentspage}

\titlecontents{subsection}
			[5.5em]
			{\zihao{5}\mdseries\heiti}
			{\contentslabel{3.2em}}
			{\hspace*{-3.2em}}
			{\titlerule*[1pc]{.} \bfseries\contentspage}

\titlecontents{subsubsection}
			[8.5em]
			{\zihao{5}\mdseries\songti}
			{\contentslabel{3.9em}}
			{\hspace*{-3.9em}}
			{\titlerule*[1pc]{.} \contentspage}

\floatname{algorithm}{\mdseries\heiti 算法}
\renewcommand{\algorithmicrequire}{\heiti 输入:}
\renewcommand{\algorithmicensure}{\heiti 输出:}
\renewcommand\appendixname{附录}
\renewcommand\appendixtocname{附录}
\renewcommand\appendixpagename{\zihao{-4}\mdseries\heiti 附录}

\numberwithin{equation}{section}
\numberwithin{figure}{section}
\numberwithin{table}{section}

\DeclareCaptionFont{song}{\songti}
\DeclareCaptionFont{hei}{\heiti}
\DeclareCaptionFont{minusfive}{\zihao{-5}}
\DeclareCaptionFont{five}{\zihao{5}}
\captionsetup*[figure]{	% 图标题设置
	font={song, minusfive}	% 宋体小五
}
\captionsetup*[table]{	% 表标题设置
	font={hei, minusfive}	% 黑体小五
}
\captionsetup*[algorithm]{	% 算法标题设置
	font={song, minusfive}	% 宋体小五
}

\title{\zihao{0}\mdseries\heiti 泛函分析笔记}
\author{\zihao{-0}\mdseries\kaishu 阙嘉豪}
\date{\zihao{1}\mdseries 最后编译时间: \number\year ~年 \number\month ~月 \number\day ~日 \currenttime}
