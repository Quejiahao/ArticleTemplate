% !Mode:: "TeX:UTF-8"
%\renewcommand\refname{参考文献}
%\renewcommand\abstractname{摘要}
%\ctexset{section = {
%	name = {\S},
%	number = \arabic{section},
%	}
%}
\ctexset{section = {format = {\zihao{-4}\heiti\flushleft}}}
\ctexset{subsection = {format = {\zihao{5}\heiti\flushleft}}}
\ctexset{subsubsection = {format = {\zihao{5}\songti\flushleft}}}
\floatname{algorithm}{算法}
\renewcommand{\algorithmicrequire}{\heiti 输入:}
\renewcommand{\algorithmicensure}{\heiti 输出:}
\renewcommand\appendixname{附录}
\renewcommand\appendixtocname{附录}
\renewcommand\appendixpagename{\zihao{-4} 附录}

\numberwithin{equation}{section}
\numberwithin{figure}{section}
\numberwithin{table}{section}

\DeclareCaptionFont{song}{\songti}
\DeclareCaptionFont{hei}{\heiti}
\DeclareCaptionFont{minusfive}{\zihao{-5}}
\DeclareCaptionFont{five}{\zihao{5}}
\captionsetup*[figure]{
	font={song, minusfive}	% 图的字体, 宋体小五
}
\captionsetup*[table]{
	font={hei, minusfive}	% 表的字体, 黑体小五
}
\captionsetup*[algorithm]{
	font={five}	% 算法的字体, 宋体小五
}

\title{\zihao{0}\heiti 泛函分析笔记}
\author{\zihao{-0}\kaishu 阙嘉豪}
\date{\zihao{1}最后编译时间: \number\year ~年 \number\month ~月 \number\day ~日 \currenttime}
