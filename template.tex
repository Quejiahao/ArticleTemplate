% !Mode:: "TeX:UTF-8"
%-------------------- 文类 --------------------
\documentclass[UTF8, a4paper, 12pt, oneside, onecolumn]{article}

%-------------------- 宏包 --------------------
% !Mode:: "TeX:UTF-8"
\usepackage[table]{xcolor}	% 表格着色
\usepackage[toc, page]{appendix}
\usepackage{amscd}	% 交换图
\usepackage[tbtags]{amsmath}	% 数学, 底部序号
\usepackage{amsopn}
\usepackage{amssymb}
\usepackage{array}	% 数组环境
\usepackage{anyfontsize}	% 消除 Font shape `OT1/cmss/m/n' in size <4> not available
\usepackage{animate}	% 插入 gif
\usepackage{algorithm}	% 算法环境
\usepackage{algpseudocode}	% 算法环境
\usepackage{bm}	% 数学粗体斜体
\usepackage{calc}
\usepackage{cases}	% 括号宏包
\usepackage{changes}	% 标注批改
\usepackage[space,	% 保留汉字与英文或数字之间的空格
			heading,	% 开启章节标题设置功能
			UTF8,	%编码为 UTF-8
			fontset = windowsnew	% 使用 windowsnew 中文字体
			]{ctex}	% 文档类为 article 或 book 时需要开启, ctexart 或 ctexbook 则不需要
\usepackage{dsfont}	% \mathds{} 字体
\usepackage{enumerate}	% 编号
\usepackage{enumitem}
\usepackage{fancyhdr}	% 页眉页脚等
\usepackage[T1]{fontenc}
\usepackage{fontspec}	% 字体设置, 需要 XeLaTeX
\usepackage{geometry}	% 调节纸张等
\usepackage{latexsym}
\usepackage{mathrsfs}
\usepackage[amsmath, thmmarks]{ntheorem}	% 定理宏包
\usepackage{setspace}	% 用于设置行距
\usepackage{verbatim}	% 提供 comment 环境
\usepackage{commath}	% 微分算子, 偏微算子
\usepackage{lastpage}	% 进阶功能用 pageslts 替代
\usepackage{layout}
\usepackage{graphicx}	% 插图
\usepackage{booktabs}
\usepackage{longtable}	% 长表格
\usepackage{ifthen}	% 这个宏包提供逻辑判断命令
\usepackage[nodayofweek]{datetime}
\usepackage{lipsum}
%\usepackage{titlesec}	% 标题形式
\usepackage{titletoc}	% 标题形式
\usepackage{multicol}	% 分栏显示
\usepackage{listings}	% 显示代码
\usepackage{blkarray}	% 标记矩阵???
\usepackage{cite}	% 参考文献标注
\usepackage{comment}	% 注释
\usepackage[stable]{footmisc}	% 脚注
%\usepackage{pageslts}
\usepackage{pdfpages}	% 嵌入 PDF
\usepackage{tikz}	% 画图
\usepackage{tikz-cd}	% 交换图
\usetikzlibrary{calc}
\usetikzlibrary{decorations.markings}
\usepackage{textcomp}
%\IfFileExists{trackchanges.sty}{\usepackage{trackchanges}}{\usepackage{../template/packages/trackchanges}}
\usepackage{gensymb}
\usepackage{float}	% 浮动体
\usepackage{bbm}	% \mathbbm
\usepackage{subcaption}	% 图表标题
\usepackage{multirow}	% 表格跨行
\usepackage{diagbox}	% 表格斜线
\usepackage{extarrows}	% 箭头
\usepackage{eso-pic}	% 水印
\usepackage{mathtools}
\usepackage{emptypage}	% 空白页不显示页眉
\usepackage{qrcode}	% 二维码
\usepackage{printlen}	% 显示长度变量
\usepackage[all, cmtip]{xy}	% 交换图
\usepackage[unicode,
			colorlinks	= true,
			linkcolor	= black,
			urlcolor	= black,
			citecolor	= black,
			anchorcolor	= blue]{hyperref}	% 参考文献超链接
\IfFileExists{\jobname.aux}{}{\renewcommand{\filemoddate}[1]{D:\pdfdate+08'00'}}	% 在没有 \jobname.aux 文件的时候防止 hyperxmp 报错
\usepackage{hyperxmp}	% pdfinfo


%-------------------- 杂 --------------------
\newcommand\blfootnote[1]{\begingroup\renewcommand\thefootnote{}\footnote{#1}\addtocounter{footnote}{-1}\endgroup}
\newcommand{\upcite}[1]{\textsuperscript{\textsuperscript{\cite{#1}}}}

\geometry{left = 2.5 cm, right = 2.5 cm, top = 2.5 cm, bottom = 2.5 cm,	% 页边距
			headheight = 40 pt, headsep = 15 pt,	% 页眉距离,
			marginparwidth = 4 cm, marginparsep = 10 pt}	% 边注设置
%\renewcommand{\baselinestretch}{1.5}	% 行距, 系统默认约 1.2, ctex 默认约 1.56
%\linespread{1}	% 行距

\setcounter{MaxMatrixCols}{20}	% 矩阵最大列数

% 矩阵行距
\makeatletter
\renewcommand*\env@matrix[1][\arraystretch]{%
	\edef\arraystretch{#1}%
	\hskip -\arraycolsep
	\let\@ifnextchar\new@ifnextchar
	\array{*\c@MaxMatrixCols c}}
\makeatother

% 水印
\newcommand{\watermark}[3]{\AddToShipoutPictureBG{
\parbox[b][\paperheight]{\paperwidth}{
\vfill%
\centering%
\tikz[remember picture, overlay]%
  \node [rotate = #1, scale = #2] at (current page.center)%
    {\textcolor{gray!80!cyan!30}{#3}};
\vfill}}}
%\newcommand{\watermarkoff}{\ClearShipoutPictureBG}

%\xeCJKsetup{CJKecglue={}}

\raggedbottom	% 防止报 Underfull \vbox (badness 10000) has occurred while \output is active []

% 代码样式设置
\lstset{
	backgroundcolor = \color{lightgray!30},
	keywordstyle    = \color{blue},
	stringstyle     = \color{purple!40},
	basicstyle      = {\small\ttfamily},
	breaklines      = true,
	tabsize         = 4,
	gobble          = 2,
	numbers         = left,
%	numberstyle     = \tiny,
	frame           = single,
	xleftmargin     = \ccwd,
	numbersep       = \ccwd,
	columns         = fullflexible,
	emphstyle       = {\color{blue}\small\ttfamily},
	emph            = {mkdir,rmdir,sudo,mount,umount,rm},
}

\allowdisplaybreaks[2]	% 公式内允许换页

%-------------------- 文档自定义 --------------------



%-------------------- 字体设置 --------------------
% !Mode:: "TeX:UTF-8"
\newcommand{\chuhao}{\fontsize{42.2pt}{\baselineskip}\selectfont}
\newcommand{\xiaochu}{\fontsize{36.1pt}{\baselineskip}\selectfont}
\newcommand{\yihao}{\fontsize{26.1pt}{\baselineskip}\selectfont}
\newcommand{\xiaoyi}{\fontsize{24.1pt}{\baselineskip}\selectfont}
\newcommand{\erhao}{\fontsize{22.1pt}{\baselineskip}\selectfont}
\newcommand{\xiaoer}{\fontsize{18.1pt}{\baselineskip}\selectfont}
\newcommand{\sanhao}{\fontsize{16.1pt}{\baselineskip}\selectfont}
\newcommand{\xiaosan}{\fontsize{15.1pt}{\baselineskip}\selectfont}
\newcommand{\sihao}{\fontsize{14.1pt}{\baselineskip}\selectfont}
\newcommand{\xiaosi}{\fontsize{12.1pt}{\baselineskip}\selectfont}
\newcommand{\wuhao}{\fontsize{10.5pt}{\baselineskip}\selectfont}
\newcommand{\xiaowu}{\fontsize{9.0pt}{\baselineskip}\selectfont}
\newcommand{\liuhao}{\fontsize{7.5pt}{\baselineskip}\selectfont}
\newcommand{\xiaoliu}{\fontsize{6.5pt}{\baselineskip}\selectfont}
\newcommand{\qihao}{\fontsize{5.5pt}{\baselineskip}\selectfont}
\newcommand{\bahao}{\fontsize{5.0pt}{\baselineskip}\selectfont}
%\fontsize{10pt}{\baselineskip}

%\setCJKfamilyfont{FZQTJW}{方正启体简体}
%\newcommand{\qiti}{\CJKfamily{FZQTJW}}

% 设置字体
%\setCJKmainfont{方正启体简体}
%\setmainfont{Times New Roman}
%\setmainfont{CMU Serif}	% 实现英文, 希腊文, 拉丁文, 西班牙文, 俄文, 中文的混排, macOS 需安装字体
%\setsansfont{Cambria Math}
%\setmonofont{Cambria Math}
%\setmathfont{Cambria Math}


%-------------------- 标题样式 --------------------
% !Mode:: "TeX:UTF-8"
%\renewcommand\refname{参考文献}
%\renewcommand\abstractname{摘要}
%\ctexset{section = {
%	name = {\S},
%	number = \arabic{section},
%	}
%}
\ctexset{section = {format = {\zihao{-4}\heiti\flushleft}}}
\ctexset{subsection = {format = {\zihao{5}\heiti\flushleft}}}
\ctexset{subsubsection = {format = {\zihao{5}\songti\flushleft}}}
\floatname{algorithm}{算法}
\renewcommand{\algorithmicrequire}{\heiti 输入:}
\renewcommand{\algorithmicensure}{\heiti 输出:}
\renewcommand\appendixname{附录}
\renewcommand\appendixtocname{附录}
\renewcommand\appendixpagename{\zihao{-4} 附录}

\numberwithin{equation}{section}
\numberwithin{figure}{section}
\numberwithin{table}{section}

\DeclareCaptionFont{song}{\songti}
\DeclareCaptionFont{hei}{\heiti}
\DeclareCaptionFont{minusfive}{\zihao{-5}}
\DeclareCaptionFont{five}{\zihao{5}}
\captionsetup*[figure]{
	font={song, minusfive}	% 图的字体, 宋体小五
}
\captionsetup*[table]{
	font={hei, minusfive}	% 表的字体, 黑体小五
}
\captionsetup*[algorithm]{
	font={five}	% 算法的字体, 宋体小五
}

\title{\zihao{0}\heiti 泛函分析笔记}
\author{\zihao{-0}\kaishu 阙嘉豪}
\date{\zihao{1}最后编译时间: \number\year ~年 \number\month ~月 \number\day ~日 \currenttime}


%-------------------- 自定义符号 --------------------
% !TeX program	= XeLaTeX
% !TeX encoding	= UTF-8

\newcommand\<{\left\langle}
\renewcommand\>{\right\rangle}
\renewcommand\({\left(}
\renewcommand\){\right)}
\renewcommand\-{\textrm{-}}	% 数学环境内使用 -, 而不是减号
\def\1{\mathbbm{1}}
\renewcommand\a{\alpha}
\newcommand\A{~\mathrm{A}}
\newcommand\AC{\mathrm{AC}}
\newcommand\al{\bm\alpha}
\DeclareMathOperator{\argmin}{argmin}
\newcommand\ba{\beta}
\newcommand\bb{\bm b}
\newcommand\bA{\bm A}
\providecommand{\bR}{\bm R}
\newcommand\bbH{\mathbb{H}}
\newcommand\bbS{\mathbb{S}}
\newcommand\be{\bm\beta}
\newcommand\bh{\bm h}
\newcommand\bk{\bm k}
\newcommand\br{\bm r}
\newcommand{\bs}[2]{{\raisebox{.2em}{$#1$}\left/\raisebox{-.2em}{$#2$}\right.}}	% 斜线除号
\newcommand\bT{\mathbb{T}}
\newcommand\bu{\bm u}
\newcommand\bU{\bm U}
\newcommand\BV{\mathrm{BV}}
\newcommand\bx{\bm x}
\newcommand\C{\mathbb{C}}	% 复数 C
\newcommand\mca{\mathcal}
\newcommand\cis{\displaystyle\bigcap_{k = 1}^\infty}
\newcommand\cov{\mathbf{Cov}}
\newcommand\csum{\displaystyle\sum_{k = 1}^\infty}
\newcommand\cT{\mathcal{T}}
\newcommand\cu{\displaystyle\bigcup_{k = 1}^\infty}
\newcommand\curl{\mathbf{curl}}
\DeclareMathOperator{\ch}{ch}	% 双曲余弦
\DeclareMathOperator{\diam}{diam}
\newcommand\de{\delta}
\newcommand\dba{\displaystyle\bigcap}	% 集合交
\newcommand\dbigcap{\displaystyle\bigcap}	% 集合交
\newcommand\dbigcup{\displaystyle\bigcup}	% 集合并
\newcommand\dbu{\displaystyle\bigcup}	% 集合并
\newcommand\di{\mathrm{d}}	% 微分算符 d
\newcommand\diff{\mathrm{d}}	% 微分算符 d
\newcommand\dinf{\displaystyle\inf}
\newcommand\divr{\mathbf{div}}
\DeclareMathOperator{\diag}{diag}	% 对角矩阵 diag
\newcommand\dint{\displaystyle\int}
\newcommand\dlim{\displaystyle\lim}
\newcommand\dmax{\displaystyle\max}
\newcommand\dmin{\displaystyle\min}
\newcommand\dsum{\displaystyle\sum}	% 求和号
\newcommand\dsup{\displaystyle\sup}
\newcommand\dmmm{~\mathrm{dm^3}}
\newcommand\D{\Delta}
\newcommand\Di{\mathrm{D}}	% 微分算符 D
\newcommand\e{\mathrm{e}}	% 自然对数的底数
\newcommand\E{\mathbb{E}}	% \R 上赋予了欧氏拓扑
\newcommand\et{\bm\eta}
\newcommand\ep{\varepsilon}
\newcommand\fb{\bm f}
\newcommand\g{~\mathrm{g}}
\newcommand\ga{\bm\gamma}
\renewcommand\geq{\geqslant}	% 大于或等于号, 下面一划是斜的
\newcommand\grad{\mathbf{grad}}
\newcommand\h{~\mathrm{h}}
\newcommand\heq{\mathbin{\widehat{=}}}
\newcommand\hin{\mathbin{\widehat{\in}}}
\renewcommand\H{\mathrm{H}}	% 共轭转置 H
\renewcommand\i{\mathrm{i}}	% 虚数单位 i
\DeclareMathOperator{\id}{id}
\DeclareMathOperator{\im}{Im}
\newcommand\I{\bm{I}}		% 单位矩阵 I
\newcommand\Int{\mathrm{Int}}	% 内部
\newcommand\J{~\mathrm{J}}
\newcommand\JK{~\mathrm{J}~\cdot ~\mathrm{K}^{-1}}
\newcommand\kJ{~\mathrm{kJ}}
\newcommand\K{~\mathrm{K}}
\DeclareMathOperator{\Ker}{Ker}
\def\l[{\left[}
\newcommand\lb{\left\{}
\newcommand\ld{\left.}
\newcommand\lllcdots{$%
\cdots\cdots\cdots\cdots\cdots%
\cdots\cdots\cdots\cdots\cdots%
\cdots\cdots\cdots\cdots\cdots%
\cdots\cdots\cdots\cdots\cdots$}
\newcommand{\lrb}[1]{\left\{ #1 \right\}}
\newcommand{\lrv}[1]{\left| #1 \right|}
\newcommand{\lrvv}[1]{\left\| #1 \right\|}
\newcommand\lv{\left|}
\renewcommand\leq{\leqslant}	% 小于或等于号, 下面一划是斜的
\newcommand{\mf}[1]{\marginpar{\footnotesize #1}}
\newcommand\m{\mathrm{m}}
\newcommand\mol{~\mathrm{mol}}
\newcommand\mr{\mathring}
\newcommand{\ms}[1]{\marginpar{\scriptsize #1}}
\newcommand\N{\mathbb{N}}	% 自然数集 N
\newcommand\om{\omega}
\newcommand\oR{\overline{\mathbb{R}}}
\newcommand\ol{\overline}
\newcommand\p{\varphi}
\DeclareMathOperator{\proj}{proj}	% 向量的投影 proj
\newcommand\pa{\partial}
\newcommand\Pa{~\mathrm{Pa}}
\newcommand\pl{\mathbin{/\mskip-2.5mu/}}
\newcommand\Q{\mathbb{Q}}	% 有理数 Q
\newcommand{\qrh}[1]{\href{#1}{\XeTeXLinkBox{\qrcode{#1}}}}	% XeLaTeX 下使得二维码是超链接
\def\r]{\right]}
\newcommand\rb{\right\}}
\newcommand\rd{\right.}
\newcommand\rv{\right|}
\DeclareMathOperator{\rank}{rank}	% 矩阵的秩 rank
\newcommand\R{\mathbb{R}}	% 实数域 R
\newcommand\rel{\mathrm{rel}}
\newcommand\Rie{\mathcal{R}}	% 黎曼可积 R
\newcommand\RP{\mathbb{RP}}	% 实数域 R
\newcommand\RR{\mathrm{R}}
\newcommand\s{~\mathrm{s}}
\newcommand\scr{\mathscr}
\DeclareMathOperator{\sign}{sign}	% 映射度
\newcommand\sg{\sigma}
\DeclareMathOperator{\sgn}{sgn}	% 符号函数
\DeclareMathOperator{\sh}{sh}	% 双曲正弦
\newcommand\sn{\mathrm{span}}
\def\st{~\textrm{s.t.}~}
\newcommand\sx{\mathscr{X}}
\DeclareMathOperator{\supp}{supp}	% 支撑集
\newcommand\T{\mathrm{T}}	% 转置 T
\newcommand\te{\theta}
\newcommand\tr{\mathrm{tr}}	% 矩阵的迹 tr
\DeclareMathOperator{\tah}{th}	% 双曲正切
\newcommand\U{\mathring{U}}	% 去心邻域 U 上面有一圈
\newcommand\V{~\mathrm{V}}
\newcommand\wh{\widehat}
\newcommand\wt{\widetilde}
\newcommand\xra{\xrightarrow}
\newcommand\xle{\xlongequal}
\newcommand\xlra{\xlongrightarrow}
\newcommand\xLra{\xLongrightarrow}
\newcommand\Z{\mathbb{Z}}	% 整数 Z


%-------------------- 自定义环境 --------------------
% !TeX program	= xelatex
% !TeX encoding	= UTF-8

\theoremstyle{nonumberplain}	% 预定格式
\theoremheaderfont{\normalfont\heiti}	% 标题字体
\theorembodyfont{\songti}	% 陈述字体, 默认 \itshape
\theoremseparator{}	% 标题与陈述分割符号
\theoremindent 0em	% 左缩进
\theoremnumbering{arabic}	% 计数形式
\theoremsymbol{$\square$}	% 结束符, 应用于所有定理类, 默认空
\newtheorem{Proof}{\hspace*{4.5ex}证明}
\newtheorem{Solve}{\hspace*{4.5ex}解}

\theoremstyle{plain}	% 预定格式
\theoremheaderfont{\normalfont\heiti}	% 标题字体
\theorembodyfont{\songti}	% 陈述字体, 默认 \itshape
\theoremseparator{}	% 标题与陈述分割符号
\theoremindent 0em	% 左缩进
\theoremnumbering{arabic}	% 计数形式
\theoremsymbol{}	% 结束符, 应用于所有定理类, 默认空
\newtheorem{Example}{\hspace*{4.5ex}例}
\newtheorem*{Rem}{\hspace*{4.5ex}注}
\newtheorem{Remark}{\hspace*{4.5ex}注}

\theoremstyle{plain}	% 预定格式
\theoremheaderfont{\normalfont\heiti}	% 标题字体
\theorembodyfont{\kaishu}	% 陈述字体, 默认 \itshape
\theoremseparator{}	% 标题与陈述分割符号
\theoremindent 0em	% 左缩进
\theoremnumbering{arabic}	% 计数形式
\theoremsymbol{}	% 结束符, 应用于所有定理类, 默认空
\newtheorem{Corollary}{\hspace*{4.5ex}推论}
\newtheorem{Definition}{\hspace*{4.5ex}定义}
\newtheorem{Theorem}{\hspace*{4.5ex}定理}
\newtheorem{Lemma}{\hspace*{4.5ex}引理}
\newtheorem{Proposition}{\hspace*{4.5ex}命题}
\newtheorem{Exercise}{\hspace*{4.5ex}习题}
\newtheorem{Axiom}{\hspace*{4.5ex}公理}
\newtheorem{Conclusion}{\hspace*{4.5ex}结论}

\theoremstyle{plain}	% 预定格式
\theoremheaderfont{\normalfont\bf}	% 标题字体
\theorembodyfont{\normalfont}	% 陈述字体, 默认 \itshape
\theoremseparator{}	% 标题与陈述分割符号
\theoremindent 0em	% 左缩进
\theoremnumbering{arabic}	% 计数形式
\theoremsymbol{}	% 结束符, 应用于所有定理类, 默认空
\newtheorem{corollary}{Corollary}
\newtheorem{definition}{Definition}
\newtheorem{proposition}{Proposition}
\newtheorem{theorem}{Theorem}
\newtheorem{lemma}{Lemma}
\newtheorem{conclusion}{Conclusion}

\theoremstyle{plain}	% 预定格式
\theoremheaderfont{\normalfont\itshape}	% 标题字体
\theorembodyfont{\normalfont}	% 陈述字体, 默认 \itshape
\theoremseparator{}	% 标题与陈述分割符号
\theoremindent 0em	% 左缩进
\theoremnumbering{arabic}	% 计数形式
\theoremsymbol{}	% 结束符, 应用于所有定理类, 默认空
\newtheorem{example}{Example}
\newtheorem{remark}{Remark}

\theoremstyle{nonumberplain}
\theoremheaderfont{\normalfont\itshape}	% 标题字体
\theorembodyfont{\normalfont}	% 陈述字体, 默认 \itshape
\theoremseparator{}	% 标题与陈述分割符号
\theoremindent 0em	% 左缩进
\theoremnumbering{arabic}	% 计数形式
\theoremsymbol{$\square$}	% 结束符, 应用于所有定理类, 默认空
\newtheorem{proof}{Proof}
\newtheorem{solution}{Solution}


%-------------------- \item 编号 --------------------
% !TeX program	= XeLaTeX
% !TeX encoding	= UTF-8

\AddEnumerateCounter{\chinese}{\chinese}{}	% 中文编号
\renewcommand{\theenumi}{\roman{enumi}}
\renewcommand{\labelenumi}{(\theenumi)}	% 设置第一级编号为一、
\renewcommand{\theenumii}{\arabic{enumii}}
\renewcommand{\labelenumii}{\theenumi .\theenumii}	% 设置第二级编号为 1.1
\renewcommand{\theenumiii}{\arabic{enumiii}}
\renewcommand{\labelenumiii}{\theenumi .\theenumii .\theenumiii}	% 设置第三级编号为 1.1.1

\setlist[enumerate]{itemindent = 2em, leftmargin = 0ex, listparindent = 2em}
\setlist[itemize]{itemindent = 2em, leftmargin = 0ex, listparindent = 2em}


%-------------------- 页眉页脚 --------------------
% !TeX program	= xelatex
% !TeX encoding	= UTF-8

\newcommand{\makefirstpageheadrule}{	% 定义首页页眉线绘制命令, 这里为等宽双线
\makebox[0pt][l]{\rule[0.55\baselineskip]{\headwidth}{0.4pt}}%
\rule[0.7\baselineskip]{\headwidth}{0.4pt}}
\newcommand{\makeheadrule}{	% 定义正文页页眉线绘制命令, 单线
\rule[0.7\baselineskip]{\headwidth}{0.4pt}}

\newboolean{first}	% 定义一个布尔变量用于判断是否为首页
\setboolean{first}{true}	% 设定 first 变量初值为 true, 根据布尔变量 first 为 true 或 false 分别执行不同的页眉线绘制命令
\renewcommand{\headrule}{\ifthenelse{\boolean{first}}{\makeheadrule}{\makefirstpageheadrule}}

% \ctexset {today = old}
\newdateformat{monthyeardate}{\monthname[\THEMONTH], \THEYEAR}
\renewcommand{\dateseparator}{\shortdate}
\fancypagestyle{plain}{\setboolean{first}{false}	% 在 plain 样式的定义中将 first 重置为 false
\lhead{\songti\zihao{-5} \number\year ~年 \number\month ~月} \chead{\zihao{-5} 实\quad 变\quad 函\quad 数\quad 笔\quad 记} \rhead{\zihao{-5}\currenttime, \today}%\monthyeardate\today}
\lfoot{} \cfoot{} \rfoot{}}

\pagestyle{fancy}
\fancyhf{}
\lhead{} \chead{\zihao{-5} 阙嘉豪: 矩阵分解} \rhead{\zihao{-5}\thepage}
\lfoot{} \cfoot{} \rfoot{}

% 奇偶页页眉
%\pagestyle{fancy}
%\fancyhf{}
%\fancyhead[LE]{\kaishu\zihao{-5}\thepage\quad \leftmark}
%\fancyhead[RO]{\kaishu\zihao{-5}\rightmark\quad \thepage}
%\renewcommand\sectionmark[1]{%
%\markright{\CTEXifname{\CTEXthesection\quad}{}#1}}


%-------------------- 正文 --------------------
\begin{document}

\thispagestyle{plain}

%\columnseprule = 1pt	% 栏线
\begin{center}
	{\zihao{-2}\heiti 常微分方程自测题~2} \\
	\vspace{1.5ex}
	{\zihao{-4}\fangsong 阙嘉豪\textsuperscript{1}%
	\blfootnote{\zihao{6}\heiti 作者简介: \songti 阙嘉豪~(1999—), 男, 广东深圳人, 北京师范大学数学科学学院本科生}} \\
	{\zihao{6}\songti (1. 北京师范大学 数学科学学院, 北京~~100875)}
\end{center}

{\zihao{-5}\heiti 摘要: \songti 本文主要介绍了矩阵分解的三种办法, 分别讨论了分解的存在性及唯一性的问题. 在三角分解中, 通过~Gauss 消元法引出了分解, 并给出了消元过程能进行到底的条件, 最后得到了~LDU 基本定理. 在~QR 分解中, 分别使用了~Gram–Schmidt 正交化方法、Givens 变换和~Householder 变换来得到分解式. 在最大秩分解中, 通过使用初等行变换将矩阵化为阶梯形矩阵获得了最大秩分解.}

{\zihao{-5}\heiti 关键字: \songti 矩阵;~LU 分解;~QR 分解; 最大秩分解}

{\zihao{-5}\heiti 中图分类号: O 151.21 \qquad\quad 文献标识码: A \qquad\quad}

\zihao{5}

\tableofcontents

%\watermark{60}{10}{\currenttime}

Hello \LaTeX

你好\LaTeX

% 行内公式
方程 1 $ a + b = c $,

方程 3 \begin{math}  g \div h = i \end{math}. % 行间公式

Newton-Leibniz 公式
\begin{equation}
	\int_a^b f(x) = F(b) - F(a).
\end{equation}

质能方程
$$E = mc^2.$$

线性组合
\[\bm\gamma = \lambda_1\bm\alpha + \lambda_2\bm\beta.\]	% 需要 \usepackage{bm}

正弦定理
\begin{displaymath}
\dfrac{a}{\sin A} = \dfrac{b}{\sin B} = \dfrac{c}{\sin C}.
\end{displaymath}

$\forall\exists$
\{ \}

$\varepsilon \partial \cdots \aleph \rtimes$	% \rtimes 需要 \usepackage{amssymb}

$\lim$

矩阵的秩 $a\rank\bm{M} rank$

\begin{center}
最大值 $\max_{i}$	\[\max_{i}\]

$\displaystyle\max_{i}$

积分 $\int_a^b$	\[\int_a^b\]

$\displaystyle\int_a^b$

\[\int\limits_a^b\]

\[\intop_a^b\]
\end{center}

\[\sum_{\substack{0 \leq i\\ 0 < j < n}}\]

{\bf 练习 1}
\[(\lim_{\mathcal{B}}f(x) = A) := (\forall V(a) \subset Y \exists B \in \mathcal{B}(f(B) \subset f(A))).\]

$$(\lim_{n \to \infty} x_n = A) := \forall \varepsilon > 0, \exists N \in \mathbb{N}, \forall n \geqslant N(|x_n - A| < \varepsilon).$$

$(\lim_{n \to \infty} x_n = A) := \forall \varepsilon > 0, \exists N \in \mathbb{N}, \forall n \geqslant N(|x_n - A| < \varepsilon).$

$(\displaystyle\lim_{n \to \infty} x_n = A) := \forall \varepsilon > 0, \exists N \in \mathbb{N}, \forall n \geqslant N(|x_n - A| < \varepsilon).$

勾股定理\begin{equation}
c^2 = a^2 + b^2.
\end{equation}

勾股定理\begin{equation*}
c^2 = a^2 + b^2.
\end{equation*}

% 需要 \usepackage{array}	% 数组宏包
\begin{equation*}
	\left.%
	\begin{array}{>{} r c@{}l@{}l}
		\text{\kaishu 常数})	& y	& =	& c \\
		\text{\kaishu 抛物线})	& y	& =	& cx + d \\
		\text{\kaishu 直线})	& y	& =	& bx^2 + cx + d \\
	\end{array}\right\}\text{多项式}
\end{equation*}

\begin{gather}
y = \mathrm{e}^x \\
x = \ln y
\end{gather}

\begin{equation}
\left\{\begin{gathered}
y = \mathrm{e}^x \\
x = \ln y
\end{gathered}\right.\text{与}
\left\{\begin{gathered}
y = x + 1 \\
x = y - 1
\end{gathered}\right.
\end{equation}

\begin{align}	% 列对之间的空白宽度与列对两侧的空白宽度相等
c_{11}	&= a_{11}\times b_{11}	& c_{12}	&= a_{12}\times b_{12} \\
c_{21}	&= a_{21}\times b_{21}	& c_{22}	&= a_{22}\times b_{22}
\end{align}

\begin{equation}
\begin{aligned}	% 列对之间的空白宽度与列对两侧的空白宽度相等
c_{11}	&= a_{11}\times b_{11}	& c_{12}	&= a_{12}\times b_{12} \\
c_{21}	&= a_{21}\times b_{21}	& c_{22}	&= a_{22}\times b_{22}
\end{aligned}
\end{equation}

\begin{flalign}	% 两端对齐
c_{11}	&= a_{11}\times b_{11}	& c_{12}	&= a_{12}\times b_{12} \\
c_{21}	&= a_{21}\times b_{21}	& c_{22}	&= a_{22}\times b_{22}
\end{flalign}
\newpage

\begin{subequations}	% 子公式编号
向量相加\begin{equation}
\bm{a} = \bm{b} + \bm{c}
\end{equation}
\begin{align}
a_1	& = b_1 + c_1 \\
a_2	& = b_2 + c_2 \\
a_2	& = b_2 + c_2
\end{align}
\end{subequations}

完全平方展开计算
\begin{equation}
\begin{split}	% 编号位置通过 \usepackage[tbtags]{amsmath} 移至最下方
f(x)	& = 2(x + 1)^2 - 1 \\
		& = 2(x^2 + 2x + 1) - 1 \\
		& = 2x^2 + 4x + 1
\end{split}
\end{equation}

% 括号环境
\begin{equation}
|x|=
\begin{cases}	% 只给一个序号
x	& x \geq 0 \\
-x	& x \leq 0
\end{cases}\label{abs}
\end{equation}

% 需要 cases 括号宏包
\begin{numcases}{|x|=}	% 分别给出序号
x	& $ x \geq 0 $ \\
-x	& $ x \leq 0 $
\end{numcases}

\begin{subnumcases}{|x|=}	% 两行共用一个序号
x	& $ x \geq 0 $ \\
-x	& $ x \leq 0 $
\end{subnumcases}

{\bf 练习 2}
真空中电磁场的麦克斯韦方程组:
\begin{equation}
\begin{aligned}
\mathrm{div} \bm{E} &= \dfrac{\rho}{\varepsilon_0},& \mathrm{div} \bm{B} &= 0,\\
\mathrm{rot} \bm{E} &= -\dfrac{\partial \bm{B}}{\partial t},& \mathrm{rot} \bm{B} &= \dfrac{\bm{j}}{\varepsilon_0 c^2} + \dfrac{1}{c^2} \dfrac{\partial \bm{E}}{\partial t}.
\end{aligned}
\end{equation}

% 公式序号
$$U = Q + W\eqno{[T.1]}$$
$$U = Q + W\leqno{[T.1]}$$
\begin{equation}U = Q + W \notag\end{equation}
\begin{equation}U = Q + W \tag{$*$}\end{equation}
\begin{equation}U = Q + W \tag*{$*$}\end{equation}
\begin{equation*}U = Q + W \tag*{$*$}\end{equation*}

% 交叉引用
绝对值函数 \ref{abs}

绝对值函数 \eqref{abs}

分块矩阵
\begin{equation}
\left(
\begin{array}{c@{}c@{}}
\begin{array}{|cc|}\hline
a_{11}	& a_{12} \\
a_{21}	& a_{22} \\ \hline
\end{array}	& \bm{O} \\
\bm{O}		& \begin{array}{|ccc|}\hline
b_{11}	& b_{12}	& b_{13} \\
b_{21}	& b_{22}	& b_{23} \\
b_{31}	& b_{32}	& b_{33} \\ \hline
\end{array} \\
\end{array}\right)
\end{equation}

三阶循环矩阵
\begin{gather*}
	\begin{matrix}
		0 & 1 & 0 \\
		0 & 0 & 1 \\
		1 & 0 & 0
	\end{matrix}\quad
	\begin{pmatrix}
		0 & 1 & 0 \\
		0 & 0 & 1 \\
		1 & 0 & 0
	\end{pmatrix}\quad
	\begin{bmatrix}
		0 & 1 & 0 \\
		0 & 0 & 1 \\
		1 & 0 & 0
	\end{bmatrix}\\
	\begin{vmatrix}
		0 & 1 & 0 \\
		0 & 0 & 1 \\
		1 & 0 & 0
	\end{vmatrix}\quad
	\begin{Vmatrix}
		0 & 1 & 0 \\
		0 & 0 & 1 \\
		1 & 0 & 0
	\end{Vmatrix}
\end{gather*}

% Givens 矩阵
\[\bm T_{ij} = \begin{pmatrix}
	1 \\
		& \ddots \\
		&		& 1 \\
		&		&	& c		&	 &		&	 & s \\
		&		&	&		& 1 \\
		&		&	&		&	& \ddots \\
		&		&	&		&	&		& 1 \\
		&		&	& -s	&	&		&	& c \\
		&		&	&		&	&		&	&	& 1 \\
		&		&	&		&	&		&	&	&	& \ddots \\
		&		&	&		&	&		&	&	&	&		& 1 \\
\end{pmatrix} \begin{matrix}
\\
\\
\\
i \\
\\
\\
\\
j \\
\\
\\
\\
\end{matrix},\quad (i < j)\]

\[a\mspace{1mu}b\quad c\]
\vspace{1cm}
\[d\]

分数
\[\frac{a}{b}\dfrac{a}{b}\displaystyle\frac{a}{b}\tfrac{a}{b}\textstyle\frac{a}{b}\]

\centerline{$\frac{a}{b}\dfrac{a}{b}\displaystyle\frac{a}{b}\tfrac{a}{b}\textstyle\frac{a}{b}$}

%二项式系数
%\[{n + 1 \choose k} = {n \choose k} + {n \choose k + 1}.\]

二项式系数
\[\binom{n + 1}{k} = \tbinom{n}{k} + \dbinom{n}{k - 1}.\]


\centerline{二项式系数
$\binom{n + 1}{k} = \tbinom{n}{k} + \dbinom{n}{k - 1}$.}

柯西不等式
\[(\sum_{i = 1}^n u_i v_i)^2 \leq (\sum_{i = 1}^n u_i^2)(\sum_{i = 1}^n v_i^2).\]

柯西不等式
\[\left(\sum_{i = 1}^n u_i v_i\right)^2 \leq \left(\sum_{i = 1}^n u_i^2\right)\left(\sum_{i = 1}^n v_i^2\right).\]

{\bf 练习 3} (Cauchy-Binet 公式)\quad {\kaishu 设 $\bm{A} = \left(a_{ij}\right)$ 是 $m \times n$ 矩阵, $\bm{B} = \left(b_{ij}\right)$ 是 $n \times m$ 矩阵. $\bm{A}\begin{pmatrix}
	i_1 & \cdots & i_s \\
	j_1 & \cdots & j_s
\end{pmatrix}$ 表示 $\bm{A}$ 的一个 $s$ 阶子式, 它由 $\bm{A}$ 的第 $i_1, \cdots, i_s$ 行与第 $j_1, \cdots, j_s$ 列交点上的元素按原次序排列组成的行列式. 同理可定义 $\bm{B}$ 的 $s$ 阶子式.

(1) 若 $m > n$, 则必有 $|\bm{AB}| = 0$;

(2) 若 $m \leq n$, 则必有
\[|\bm{AB}| = \sum_{1 \leq j_1 < j_2 < \cdots < j_m \leq n} \bm{A}\begin{pmatrix}
	1	& 2		& \cdots & m \\
	j_1	& j_2	& \cdots & j_m
\end{pmatrix} \bm{B}\begin{pmatrix}
	j_1	& j_2	& \cdots & j_m \\
	1	& 2		& \cdots & m
\end{pmatrix}.\]
}

{\bf 练习 4} \begin{equation*}\begin{split}
\varphi^*\omega(t)(\bm{\tau}_1, \bm{\tau}_2) :&= \omega(\varphi(\bm{\tau}_1, \bm{\tau}_2)) = \mathrm{d} x^{i_1} \mathrm{d} x^{i_2}(\bm{\xi}_1, \bm{\xi}_2) \\
&= \begin{vmatrix}
	\xi_1^{i_1}	& \xi_1^{i_2} \\
	\xi_2^{i_1}	& \xi_2^{i_2}
\end{vmatrix} = \begin{vmatrix}
	\dfrac{\partial x^{i_1}}{\partial t^{j_1}} \tau_1^{j_1}	& \dfrac{\partial x^{i_2}}{\partial t^{j_2}} \tau_1^{j_2} \\
	\dfrac{\partial x^{i_1}}{\partial t^{j_1}} \tau_2^{j_1}	& \dfrac{\partial x^{i_2}}{\partial t^{j_2}} \tau_2^{j_2}
\end{vmatrix} \\
&= \sum_{j_1, j_2 = 1}^m \dfrac{\partial x^{i_1}}{\partial t^{j_1}} \dfrac{\partial x^{i_2}}{\partial t^{j_2}} \begin{vmatrix}
	\tau_1^{j_1} & \tau_1^{j_2} \\
	\tau_2^{j_1} & \tau_2^{j_2}
\end{vmatrix} \\
&= \sum_{j_1, j_2 = 1}^m \dfrac{\partial x^{i_1}}{\partial t^{j_1}} \dfrac{\partial x^{i_2}}{\partial t^{j_2}} \mathrm{d} t^{j_1} \wedge \mathrm{d} t^{j_2} (\bm{\tau}_1, \bm{\tau}_2) \\
&= \sum_{1 \leqslant j_1 < j_2 \leqslant m} \left(\dfrac{\partial x^{i_1}}{\partial t^{j_1}} \dfrac{\partial x^{i_2}}{\partial t^{j_2}} - \dfrac{\partial x^{i_1}}{\partial t^{j_2}} \dfrac{\partial x^{i_2}}{\partial t^{j_1}}\right) \mathrm{d} t^{j_1} \wedge \mathrm{d} t^{j_2} (\bm{\tau}_1, \bm{\tau}_2) \\
&= \sum_{1 \leqslant j_1 < j_2 \leqslant m} \begin{vmatrix}
	\dfrac{\partial x^{i_1}}{\partial t^{j_1}} & \dfrac{\partial x^{i_2}}{\partial t^{j_1}} \\
	\dfrac{\partial x^{i_1}}{\partial t^{j_2}} & \dfrac{\partial x^{i_2}}{\partial t^{j_2}}
\end{vmatrix}(t) \mathrm{d} t^{j_1} \wedge \mathrm{d} t^{j_2} (\bm{\tau}_1, \bm{\tau}_2).
\end{split}\end{equation*}

% 交换图
\begin{equation}\begin{CD}
	a		@>j>>	b \\
	@VVV			@VV{\lim P}V \\
	c		@=		d
\end{CD}\end{equation}

\begin{equation}\label{cdd}
\xymatrix{
A \ar[d] \ar[r] & B \\
B \ar[r] & C \ar[u]}
\end{equation}

引用参考文献超链接\upcite{RongYuan}

\hyperref[cdd]{text}

$\_\_\_\_\_\_\_\_\_\_\_\_\_\_\_\_$

\begin{equation*}
\begin{split}
    a \\
     bbbbbbbbbbb \\
      c
  \end{split}
  \begin{array}{c}
    a \\
    b \\
    c
  \end{array}
\end{equation*}

\begin{figure}[ht]
	\centering
	\begin{tikzpicture}
		\draw[fill = green!20] (0, 2.866) arc[start angle = 60, end angle = 300, x radius = 2, y radius = 1] arc[start angle = 240, end angle = 120, x radius = 2, y radius = 1] (-2, 2) node{$U_\alpha$};
		\draw[fill = blue!20] (0, 2.866) arc[start angle = 120, end angle = -120, x radius = 2, y radius = 1] arc[start angle = -60, end angle = 60, x radius = 2, y radius = 1] (2, 2) node{$U_\beta$};
		\draw[fill = red!20] (0, 2.866) arc[start angle = 120, end angle = 240, x radius = 2, y radius = 1] arc[start angle = -60, end angle = 60, x radius = 2, y radius = 1] (0, 2) node{$W$};
		\draw (-1, 0) -- (-4, 0) -- (-5, -2) -- (-2, -2) -- (-1, 0) (-0.5, -0.25) node{$\R^n$};
		\draw (4, 0) -- (1, 0) -- (0, -2) -- (3, -2) -- (4, 0) (4.5, -0.25) node{$\R^n$};
		\draw[rotate around = {4 : (-3, -1)}, color = black!50, fill = green!20] (-3, -1) ellipse [x radius = 1.4, y radius = 0.7];
		\draw[rotate around = {4 : (2, -1)}, color = black!50, fill = blue!20] (2, -1) ellipse [x radius = 1.4, y radius = 0.7];
		\draw[rotate around = {4 : (-3, -1)}, color = black!50, fill = red!20] (-3, -0.3) arc[start angle = 150, end angle = 210, radius = 1.4] arc[start angle = -90, end angle = 90, x radius = 1.4, y radius = 0.7];
		\draw (-3, -1) node{$f_\alpha\(U_\alpha\)$};
		\draw[rotate around = {4 : (2, -1)}, color = black!50, fill = red!20] (2, -0.3) arc[start angle = 30, end angle = -30, radius = 1.4] arc[start angle = 270, end angle = 90, x radius = 1.4, y radius = 0.7];
		\draw (2, -1) node{$f_\beta\(U_\beta\)$};
		\draw[->] (-2, 1.5) arc[start angle = 130, end angle = 160, radius = 3.4] node[midway, sloped, above]{$f_\alpha$};
		\draw[->] (2, 1.5) arc[start angle = 50, end angle = 20, radius = 3.4] node[midway, sloped, above]{$f_\beta$};
		\draw[<-] (-0.5, 2) arc[start angle = 120, end angle = 185, radius = 2.9] node[midway, sloped, below]{$f_\alpha^{-1}$};
		\draw[->] (0.5, 2) arc[start angle = 60, end angle = -5, radius = 2.9] node[midway, sloped, below]{$f_\beta$};
	\end{tikzpicture}
	\caption{Condition}
\end{figure}

\begin{figure}[H]
\centering
\begin{minipage}[t]{.25\linewidth}\centering
\qrh{https://jq.qq.com/?_wv=1027&k=57DLax6}
\end{minipage}
\begin{minipage}[t]{.25\linewidth}\centering
\qrh{http://weixin.qq.com/r/Zjn-54fE5tqZrcOE92x0}
\end{minipage}
\caption*{\heiti\zihao{3}欢迎扫码加群及关注公众号}
\end{figure}

\begin{lstlisting}[language = TeX]
	\documentclass{article}
	
	\begin{document}
		Hello \LaTeX !
	\end{document}
\end{lstlisting}

``"

\verb”%”

\begin{enumerate}
	\item
	\item aaaa

aaaa
\end{enumerate}

\begin{thebibliography}{1}
\addcontentsline{toc}{section}{参考文献}	% 将 "参考文献" 加入目录中

\bibitem{RongYuan} 袁荣. 常微分方程[M]. 北京: 高等教育出版社, 2012:59,62.
\end{thebibliography}
%\end{multicols}

\begin{center}
{\zihao{-3} \bf Matrix decomposition}

{\zihao{-4} QUE~Jia-hao}

{\zihao{-5} (School of Mathematical Sciences, Beijing Normal University, Beijing 100875, China)}
\end{center}

{\zihao{5} \bf Abstract}~.

{\zihao{5} \bf Key words}~Matrix;~LU decomposition;

\begin{appendices}

\section{Some Appendix2}\label{aaa}
The contents...

\end{appendices}
\end{document}
