% !TeX program	= xelatex
% !TeX encoding	= UTF-8

%-------------------- 文类 --------------------
\documentclass[UTF8, a4paper, 12pt, oneside, onecolumn]{article}

%-------------------- 宏包 --------------------
% !TeX program	= XeLaTeX
% !TeX encoding	= UTF-8

\ExplSyntaxOn
\msg_redirect_name:nnn{fontspec}{no-script}{info}	% 抑制 fontspec 警告
\ExplSyntaxOff
\usepackage[table]{xcolor}	% 表格着色
\usepackage[toc, page]{appendix}
\usepackage{amscd}	% 交换图
\usepackage[tbtags]{amsmath}	% 数学, 底部序号
\usepackage{amsopn}
\usepackage{amssymb}
\usepackage{array}	% 数组环境
\usepackage{anyfontsize}	% 消除 Font shape `OT1/cmss/m/n' in size <4> not available
\usepackage{animate}	% 插入 gif
\usepackage{algorithm}	% 算法环境
\usepackage{algpseudocode}	% 算法环境
\usepackage{bm}	% 数学粗体斜体
\usepackage{calc}
\usepackage{cases}	% 括号宏包
\usepackage{changes}	% 标注批改
\usepackage[space,	% 保留汉字与英文或数字之间的空格
			heading,	% 开启章节标题设置功能
			UTF8,	%编码为 UTF-8
			fontset = fandol	% 使用 fandol 中文字体
			]{ctex}	% 文档类为 article 或 book 时需要开启, ctexart 或 ctexbook 则不需要
\usepackage{dsfont}	% \mathds{} 字体
\usepackage{enumerate}	% 编号
\usepackage{enumitem}
%\usepackage{etex}
\usepackage{fancyhdr}	% 页眉页脚等
\usepackage[T1]{fontenc}
\usepackage{fontspec}	% 字体设置, 需要 XeLaTeX
\usepackage{geometry}	% 调节纸张等
\usepackage{latexsym}
\usepackage{mathrsfs}
\usepackage[amsmath, thmmarks]{ntheorem}	% 定理宏包
\usepackage{setspace}	% 用于设置行距
\usepackage{verbatim}	% 提供 comment 环境
\usepackage{commath}	% 微分算子, 偏微算子
\usepackage{lastpage}	% 进阶功能用 pageslts 替代
\usepackage{layout}
\usepackage{graphicx}	% 插图
\usepackage{booktabs}
\usepackage{longtable}	% 长表格
\usepackage{ifthen}	% 这个宏包提供逻辑判断命令
\usepackage[nodayofweek]{datetime}
\usepackage{lipsum}
%\usepackage{titlesec}	% 标题形式
\usepackage{titletoc}	% 标题形式
\usepackage{multicol}	% 分栏显示
\usepackage{listings}	% 显示代码
\usepackage{blkarray}	% 标记矩阵???
\usepackage{cite}	% 参考文献标注
\usepackage{comment}	% 注释
\usepackage[stable]{footmisc}	% 脚注
%\usepackage{pageslts}
\usepackage{pdfpages}	% 嵌入 PDF
\usepackage{tikz}	% 画图
\usepackage{tikz-cd}	% 交换图
\usetikzlibrary{calc}
\usetikzlibrary{decorations.markings}
\usepackage{textcomp}
%\IfFileExists{trackchanges.sty}{\usepackage{trackchanges}}{\usepackage{../template/packages/trackchanges}}
\usepackage{gensymb}
\usepackage{float}	% 浮动体
\usepackage{bbm}	% \mathbbm
\usepackage{subcaption}	% 图表标题
\usepackage{multirow}	% 表格跨行
\usepackage{diagbox}	% 表格斜线
\usepackage{extarrows}	% 箭头
\usepackage{eso-pic}	% 水印
\usepackage{mathtools}
\usepackage{metalogo}	% \XeLaTeX 等
\usepackage{emptypage}	% 空白页不显示页眉
\usepackage{qrcode}	% 二维码
\usepackage{printlen}	% 显示长度变量
\usepackage[all, cmtip]{xy}	% 交换图
\usepackage{pgf}
\usepackage[unicode,
			colorlinks	= true,
			linkcolor	= black,
			urlcolor	= black,
			citecolor	= black,
			anchorcolor	= blue]{hyperref}	% 参考文献超链接
\IfFileExists{\jobname.aux}{}{\renewcommand{\filemoddate}[1]{D:\pdfdate+08'00'}}	% 在没有 \jobname.aux 文件的时候防止 hyperxmp 报错
\usepackage{hyperxmp}	% pdfinfo


%-------------------- 杂项 --------------------
% !TeX program	= XeLaTeX
% !TeX encoding	= UTF-8

\geometry{left = 3.18 cm, right = 3.18 cm, top = 2.54 cm, bottom = 2.54 cm,	% 页边距
	headheight = 40 pt, headsep = 15 pt,	% 页眉距离
	marginparwidth = 4 cm, marginparsep = 10 pt}	% 边注设置
%\renewcommand{\baselinestretch}{1.5}	% 行距, 系统默认约 1.2, ctex 默认约 1.56
%\linespread{1}	% 行距

\newcommand\blfootnote[1]{%
	\begingroup%
	\renewcommand\thefootnote{}\footnote{#1}%
	\addtocounter{footnote}{-1}%
	\endgroup
}

\newcommand{\upcite}[1]{\textsuperscript{\textsuperscript{\cite{#1}}}}
\newcommand{\upref}[1]{\textsuperscript{\textsuperscript{\ref{#1}}}}

\setcounter{MaxMatrixCols}{20}	% 矩阵最大列数

% 矩阵行距
\makeatletter
\renewcommand*\env@matrix[1][\arraystretch]{%
	\edef\arraystretch{#1}%
	\hskip -\arraycolsep
	\let\@ifnextchar\new@ifnextchar
	\array{*\c@MaxMatrixCols c}}
\makeatother

% 水印
\newcommand{\watermark}[3]{\AddToShipoutPictureBG{
\parbox[b][\paperheight]{\paperwidth}{
\vfill%
\centering%
\tikz[remember picture, overlay]%
	\node [rotate = #1, scale = #2] at (current page.center)%
		{\textcolor{gray!80!cyan!30}{#3}};
\vfill}}}
%\newcommand{\watermarkoff}{\ClearShipoutPictureBG}

%\xeCJKsetup{CJKecglue={}}

\raggedbottom	% 防止报 Underfull \vbox (badness 10000) has occurred while \output is active []

\allowdisplaybreaks[2]	% 公式内允许换页

%\pagenumbering{arabic}

\hypersetup
{
	% 颜色
	colorlinks	= true,
	linkcolor	= black,
	urlcolor	= black,
	citecolor	= black,
	anchorcolor	= blue,
}


%-------------------- 字体设置 --------------------
% !Mode:: "TeX:UTF-8"
\newcommand{\chuhao}{\fontsize{42.2pt}{\baselineskip}\selectfont}
\newcommand{\xiaochu}{\fontsize{36.1pt}{\baselineskip}\selectfont}
\newcommand{\yihao}{\fontsize{26.1pt}{\baselineskip}\selectfont}
\newcommand{\xiaoyi}{\fontsize{24.1pt}{\baselineskip}\selectfont}
\newcommand{\erhao}{\fontsize{22.1pt}{\baselineskip}\selectfont}
\newcommand{\xiaoer}{\fontsize{18.1pt}{\baselineskip}\selectfont}
\newcommand{\sanhao}{\fontsize{16.1pt}{\baselineskip}\selectfont}
\newcommand{\xiaosan}{\fontsize{15.1pt}{\baselineskip}\selectfont}
\newcommand{\sihao}{\fontsize{14.1pt}{\baselineskip}\selectfont}
\newcommand{\xiaosi}{\fontsize{12.1pt}{\baselineskip}\selectfont}
\newcommand{\wuhao}{\fontsize{10.5pt}{\baselineskip}\selectfont}
\newcommand{\xiaowu}{\fontsize{9.0pt}{\baselineskip}\selectfont}
\newcommand{\liuhao}{\fontsize{7.5pt}{\baselineskip}\selectfont}
\newcommand{\xiaoliu}{\fontsize{6.5pt}{\baselineskip}\selectfont}
\newcommand{\qihao}{\fontsize{5.5pt}{\baselineskip}\selectfont}
\newcommand{\bahao}{\fontsize{5.0pt}{\baselineskip}\selectfont}
%\fontsize{10pt}{\baselineskip}

%\setCJKfamilyfont{FZQTJW}{方正启体简体}
%\newcommand{\qiti}{\CJKfamily{FZQTJW}}

% 设置字体
%\setCJKmainfont{方正启体简体}
%\setmainfont{Times New Roman}
%\setmainfont{CMU Serif}	% 实现英文, 希腊文, 拉丁文, 西班牙文, 俄文, 中文的混排, macOS 需安装字体
%\setsansfont{Cambria Math}
%\setmonofont{Cambria Math}
%\setmathfont{Cambria Math}


%-------------------- 标题样式 --------------------
% !TeX program	= XeLaTeX
% !TeX encoding	= UTF-8

%\renewcommand\refname{参考文献}
%\renewcommand\abstractname{摘要}
%\ctexset{section = {
%	name = {\S},
%	number = \arabic{section},
%	}
%}
\ctexset{
	contentsname = {\zihao{3}\mdseries\heiti 目录},
	part = {format = {\zihao{3}\mdseries\heiti\centering}},
	section = {
		format = {\zihao{-4}\mdseries\heiti\flushleft},
		number = \bfseries{\arabic{section}}
	},
	subsection = {
		format = {\zihao{5}\mdseries\heiti\flushleft},
		number = \bfseries{\arabic{section}.\arabic{subsection}}
	},
	subsubsection = {format = {\zihao{5}\mdseries\songti\flushleft}},
}

\titlecontents{part}
			[0em]
			{\zihao{3}\mdseries\heiti}
			{\contentslabel{0em}}
			{\hspace*{0em}}
			{\hfill \bfseries\contentspage}

\titlecontents{section}
			[2.3em]
			{\zihao{-4}\mdseries\heiti}
			{\contentslabel{2.3em}}
			{\hspace*{-2.3em}}
			{\titlerule*[1pc]{.} \bfseries\contentspage}

\titlecontents{subsection}
			[5.5em]
			{\zihao{5}\mdseries\heiti}
			{\contentslabel{3.2em}}
			{\hspace*{-3.2em}}
			{\titlerule*[1pc]{.} \bfseries\contentspage}

\titlecontents{subsubsection}
			[8.5em]
			{\zihao{5}\mdseries\songti}
			{\contentslabel{3.9em}}
			{\hspace*{-3.9em}}
			{\titlerule*[1pc]{.} \contentspage}

\floatname{algorithm}{\mdseries\heiti 算法}
\renewcommand{\algorithmicrequire}{\heiti 输入:}
\renewcommand{\algorithmicensure}{\heiti 输出:}
\renewcommand\appendixname{附录}
\renewcommand\appendixtocname{附录}
\renewcommand\appendixpagename{\zihao{-4}\mdseries\heiti 附录}

\numberwithin{equation}{section}
\numberwithin{figure}{section}
\numberwithin{table}{section}

\DeclareCaptionFont{song}{\songti}
\DeclareCaptionFont{hei}{\heiti}
\DeclareCaptionFont{minusfive}{\zihao{-5}}
\DeclareCaptionFont{five}{\zihao{5}}
\captionsetup*[figure]{	% 图标题设置
	font={song, minusfive}	% 宋体小五
}
\captionsetup*[table]{	% 表标题设置
	font={hei, minusfive}	% 黑体小五
}
\captionsetup*[algorithm]{	% 算法标题设置
	font={song, minusfive}	% 宋体小五
}

\title{\zihao{0}\mdseries\heiti 泛函分析笔记}
\author{\zihao{-0}\mdseries\kaishu 阙嘉豪}
\date{\zihao{1}\mdseries 最后编译时间: \number\year ~年 \number\month ~月 \number\day ~日 \currenttime}


%-------------------- 自定义符号 --------------------
% !Mode:: "TeX:UTF-8"
\def\<{\left\langle}
\def\>{\right\rangle}
\def\({\left(}
\def\){\right)}
\def\-{\textrm{-}}	% 数学环境内使用 -, 而不是减号
\def\1{\mathbbm{1}}
\def\a{\alpha}
\def\A{~\mathrm{A}}
\def\AC{\mathrm{AC}}
\def\al{\bm\alpha}
\DeclareMathOperator{\argmin}{argmin}
\def\ba{\beta}
\def\bA{\bm A}
\def\bbH{\mathbb{H}}
\def\bbS{\mathbb{S}}
\def\be{\bm\beta}
\def\bh{\bm h}
\newcommand{\bs}[2]{{\raisebox{.2em}{$#1$}\left/\raisebox{-.2em}{$#2$}\right.}}	% 斜线除号
\def\bT{\mathbb{T}}
\def\bU{\bm U}
\def\BV{\mathrm{BV}}
\def\bx{\bm x}
\def\C{\mathbb{C}}	% 复数 C
\def\mca{\mathcal}
\def\cis{\displaystyle\bigcap_{k = 1}^\infty}
\def\cov{\mathbf{Cov}}
\def\csum{\displaystyle\sum_{k = 1}^\infty}
\def\cT{\mathcal{T}}
\def\cu{\displaystyle\bigcup_{k = 1}^\infty}
\def\curl{\mathbf{curl}}
\DeclareMathOperator{\ch}{ch}	% 双曲余弦
\DeclareMathOperator{\diam}{diam}
\def\de{\delta}
\def\dba{\displaystyle\bigcap}	% 集合交
\def\dbigcap{\displaystyle\bigcap}	% 集合交
\def\dbigcup{\displaystyle\bigcup}	% 集合并
\def\dbu{\displaystyle\bigcup}	% 集合并
\def\di{\mathrm{d}}	% 微分算符 d
\def\diff{\mathrm{d}}	% 微分算符 d
\def\dinf{\displaystyle\inf}
\def\divr{\mathbf{div}}
\DeclareMathOperator{\diag}{diag}	% 对角矩阵 diag
\def\dint{\displaystyle\int}
\def\dlim{\displaystyle\lim}
\def\dmax{\displaystyle\max}
\def\dmin{\displaystyle\min}
\def\dsum{\displaystyle\sum}	% 求和号
\def\dsup{\displaystyle\sup}
\def\dmmm{~\mathrm{dm^3}}
\def\D{\Delta}
\def\Di{\mathrm{D}}	% 微分算符 D
\def\e{\mathrm{e}}	% 自然对数的底数
\def\E{\mathbb{E}}	% \R 上赋予了欧氏拓扑
\def\et{\bm\eta}
\def\ep{\varepsilon}
\def\fb{\bm f}
\def\g{~\mathrm{g}}
\def\ga{\bm\gamma}
\def\geq{\geqslant}	% 大于或等于号, 下面一划是斜的
\def\grad{\mathbf{grad}}
\def\h{~\mathrm{h}}
\def\heq{\mathbin{\widehat{=}}}
\def\hin{\mathbin{\widehat{\in}}}
\def\H{\mathrm{H}}	% 共轭转置 H
\def\i{\mathrm{i}}	% 虚数单位 i
\DeclareMathOperator{\id}{id}
\DeclareMathOperator{\im}{Im}
\def\I{\bm{I}}		% 单位矩阵 I
\def\Int{\mathrm{Int}}	% 内部
\def\J{~\mathrm{J}}
\def\JK{~\mathrm{J}~\cdot ~\mathrm{K}^{-1}}
\def\kJ{~\mathrm{kJ}}
\def\K{~\mathrm{K}}
\DeclareMathOperator{\Ker}{Ker}
\def\l[{\left[}
\def\lb{\left\{}
\def\ld{\left.}
\def\lllcdots{$%
\cdots\cdots\cdots\cdots\cdots%
\cdots\cdots\cdots\cdots\cdots%
\cdots\cdots\cdots\cdots\cdots%
\cdots\cdots\cdots\cdots\cdots$}
\def\lrb#1{\left\{ #1 \right\}}
\def\lrv#1{\left| #1 \right|}
\def\lrvv#1{\left\| #1 \right\|}
\def\lv{\left|}
\def\leq{\leqslant}	% 小于或等于号, 下面一划是斜的
\def\mf#1{\marginpar{\footnotesize #1}}
\def\m{\mathrm{m}}
\def\mol{~\mathrm{mol}}
\def\mr{\mathring}
\def\ms#1{\marginpar{\scriptsize #1}}
\def\N{\mathbb{N}}	% 自然数集 N
\def\om{\omega}
\def\oR{\overline{\mathbb{R}}}
\def\ol{\overline}
\def\p{\varphi}
\DeclareMathOperator{\proj}{proj}	% 向量的投影 proj
\def\pa{\partial}
\def\Pa{~\mathrm{Pa}}
\def\pl{\mathbin{/\mskip-2.5mu/}}
\def\Q{\mathbb{Q}}	% 有理数 Q
\def\qrh#1{\href{#1}{\XeTeXLinkBox{\qrcode{#1}}}}	% XeLaTeX 下使得二维码是超链接
\def\r]{\right]}
\def\rb{\right\}}
\def\rd{\right.}
\def\rv{\right|}
\DeclareMathOperator{\rank}{rank}	% 矩阵的秩 rank
\def\R{\mathbb{R}}	% 实数域 R
\def\rel{\mathrm{rel}}
\def\Rie{\mathcal{R}}	% 黎曼可积 R
\def\RP{\mathbb{RP}}	% 实数域 R
\def\RR{\mathrm{R}}
\def\s{~\mathrm{s}}
\def\scr{\mathscr}
\DeclareMathOperator{\sign}{sign}	% 映射度
\def\sg{\sigma}
\DeclareMathOperator{\sgn}{sgn}	% 符号函数
\DeclareMathOperator{\sh}{sh}	% 双曲正弦
\def\sn{\mathrm{span}}
\def\st{~\textrm{s.t.}~}
\def\sx{\mathscr{X}}
\def\supp{\mathrm{supp}}
\def\T{\mathrm{T}}	% 转置 T
\def\te{\theta}
\def\tr{\mathrm{tr}}	% 矩阵的迹 tr
\DeclareMathOperator{\tah}{th}	% 双曲正切
\def\U{\mathring{U}}	% 去心邻域 U 上面有一圈
\def\V{~\mathrm{V}}
\def\wh{\widehat}
\def\wt{\widetilde}
\def\xra{\xrightarrow}
\def\xle{\xlongequal}
\def\xlra{\xlongrightarrow}
\def\xLra{\xLongrightarrow}
\def\Z{\mathbb{Z}}	% 整数 Z


%-------------------- 自定义环境 --------------------
% !TeX program	= XeLaTeX
% !TeX encoding	= UTF-8

\theoremstyle{nonumberplain}	% 预定格式
\theoremheaderfont{\normalfont\heiti}	% 标题字体
\theorembodyfont{\songti}	% 陈述字体, 默认 \itshape
\theoremseparator{}	% 标题与陈述分割符号
\theoremindent 0em	% 左缩进
\theoremnumbering{arabic}	% 计数形式
\theoremsymbol{$\square$}	% 结束符, 应用于所有定理类, 默认空
\newtheorem{Proof}{\hspace*{4.5ex}证明}
\newtheorem{Solve}{\hspace*{4.5ex}解}

\theoremstyle{plain}	% 预定格式
\theoremheaderfont{\normalfont\heiti}	% 标题字体
\theorembodyfont{\songti}	% 陈述字体, 默认 \itshape
\theoremseparator{}	% 标题与陈述分割符号
\theoremindent 0em	% 左缩进
\theoremnumbering{arabic}	% 计数形式
\theoremsymbol{}	% 结束符, 应用于所有定理类, 默认空
\newtheorem{Example}{\hspace*{4.5ex}例}
\newtheorem*{Rem}{\hspace*{4.5ex}注}
\newtheorem{Remark}{\hspace*{4.5ex}注}

\theoremstyle{plain}	% 预定格式
\theoremheaderfont{\normalfont\heiti}	% 标题字体
\theorembodyfont{\kaishu}	% 陈述字体, 默认 \itshape
\theoremseparator{}	% 标题与陈述分割符号
\theoremindent 0em	% 左缩进
\theoremnumbering{arabic}	% 计数形式
\theoremsymbol{}	% 结束符, 应用于所有定理类, 默认空
\newtheorem{Corollary}{\hspace*{4.5ex}推论}
\newtheorem{Definition}{\hspace*{4.5ex}定义}
\newtheorem{Theorem}{\hspace*{4.5ex}定理}
\newtheorem{Lemma}{\hspace*{4.5ex}引理}
\newtheorem{Proposition}{\hspace*{4.5ex}命题}
\newtheorem{Exercise}{\hspace*{4.5ex}习题}
\newtheorem{Axiom}{\hspace*{4.5ex}公理}

\theoremstyle{plain}	% 预定格式
\theoremheaderfont{\normalfont\bf}	% 标题字体
\theorembodyfont{\normalfont}	% 陈述字体, 默认 \itshape
\theoremseparator{}	% 标题与陈述分割符号
\theoremindent 0em	% 左缩进
\theoremnumbering{arabic}	% 计数形式
\theoremsymbol{}	% 结束符, 应用于所有定理类, 默认空
\newtheorem{corollary}{Corollary}
\newtheorem{definition}{Definition}
\newtheorem{proposition}{Proposition}
\newtheorem{theorem}{Theorem}
\newtheorem{lemma}{Lemma}

\theoremstyle{plain}	% 预定格式
\theoremheaderfont{\normalfont\itshape}	% 标题字体
\theorembodyfont{\normalfont}	% 陈述字体, 默认 \itshape
\theoremseparator{}	% 标题与陈述分割符号
\theoremindent 0em	% 左缩进
\theoremnumbering{arabic}	% 计数形式
\theoremsymbol{}	% 结束符, 应用于所有定理类, 默认空
\newtheorem{example}{Example}
\newtheorem{remark}{Remark}

\theoremstyle{nonumberplain}
\theoremheaderfont{\normalfont\itshape}	% 标题字体
\theorembodyfont{\normalfont}	% 陈述字体, 默认 \itshape
\theoremseparator{}	% 标题与陈述分割符号
\theoremindent 0em	% 左缩进
\theoremnumbering{arabic}	% 计数形式
\theoremsymbol{$\square$}	% 结束符, 应用于所有定理类, 默认空
\newtheorem{proof}{Proof}
\newtheorem{solution}{Solution}


%-------------------- \item 编号 --------------------
% !TeX program	= xelatex
% !TeX encoding	= UTF-8

\AddEnumerateCounter{\chinese}{\chinese}{}	% 中文编号
\renewcommand{\theenumi}{\roman{enumi}}
\renewcommand{\labelenumi}{(\theenumi)}	% 设置第一级编号为一、
\renewcommand{\theenumii}{\arabic{enumii}}
\renewcommand{\labelenumii}{\theenumi .\theenumii}	% 设置第二级编号为 1.1
\renewcommand{\theenumiii}{\arabic{enumiii}}
\renewcommand{\labelenumiii}{\theenumi .\theenumii .\theenumiii}	% 设置第三级编号为 1.1.1

\setlist[enumerate]{itemindent = 2em, leftmargin = 0ex, listparindent = 2em}
\setlist[itemize]{itemindent = 2em, leftmargin = 0ex, listparindent = 2em}


%-------------------- 代码样式设置 --------------------
% !TeX program	= XeLaTeX
% !TeX encoding	= UTF-8

\lstset{
	backgroundcolor	= \color{lightgray!30},
	stringstyle		= \color{purple!40},
	basicstyle		= {\small\ttfamily},
	breaklines		= true,
	keywordstyle	= \color{blue},
	tabsize			= 4,
	gobble			= 2,
	numbers			= left,
%	numberstyle		= \tiny,
	frame			= single,
	xleftmargin		= \ccwd,
	numbersep		= \ccwd,
	columns			= fullflexible,
	emphstyle		= {\color{blue}\small\ttfamily},
	emph			= {mkdir, rmdir, sudo, mount, umount, rm},
}


%-------------------- PDF 元信息, 建模时注释掉 --------------------
% !TeX program	= xelatex
% !TeX encoding	= UTF-8

\hypersetup
{
	pdftitle			= {题目},
	pdfauthor			= {阙嘉豪},
	pdfauthortitle		= {头衔},
	pdfcreator			= {阙嘉豪的模板},
	pdfcreationdate		= {1999-12-07T23:00:00+08:00},
	%pdfmoddate			= {1999-12-07T23:00:00+08:00},
	%pdfmetadate			= {2014-09-23T14:15:09-06:00},
	%pdfdate				= {2014-09-23T14:15:09-06:00},
	pdfcopyright		= {Copyright (C) 2020, Coumarin},
	pdfsubject			= {主题},
	pdfkeywords			= {关键字},
	pdflicenseurl		= {https://creativecommons.org/licenses/by-nc-sa/4.0/deed.zh},
	pdfcaptionwriter	= {阙嘉豪},
	pdfcontactaddress	= {新街口外大街 19 号},
	pdfcontactcity		= {北京市},
	pdfcontactpostcode	= {100875},
	pdfcontactcountry	= {中华人民共和国},
	pdfcontactphone		= {+8688888888888},
	pdfcontactemail		= {i@kumarino.com},
	pdfcontacturl		= {%
		https://kumarino.com,
	},
	pdflang				= {zh-cn},
	pdfmetalang			= {zh-cn},
	baseurl				= {https://kumarino.com},
}


%-------------------- 页眉页脚 --------------------
% !TeX program	= xelatex
% !TeX encoding	= UTF-8

\newcommand{\makefirstpageheadrule}{	% 定义首页页眉线绘制命令, 这里为等宽双线
\makebox[0pt][l]{\rule[0.55\baselineskip]{\headwidth}{0.4pt}}%
\rule[0.7\baselineskip]{\headwidth}{0.4pt}}
\newcommand{\makeheadrule}{	% 定义正文页页眉线绘制命令, 单线
\rule[0.7\baselineskip]{\headwidth}{0.4pt}}

\newboolean{first}	% 定义一个布尔变量用于判断是否为首页
\setboolean{first}{true}	% 设定 first 变量初值为 true, 根据布尔变量 first 为 true 或 false 分别执行不同的页眉线绘制命令
\renewcommand{\headrule}{\ifthenelse{\boolean{first}}{\makeheadrule}{\makefirstpageheadrule}}

% \ctexset {today = old}
\newdateformat{monthyeardate}{\monthname[\THEMONTH], \THEYEAR}
\renewcommand{\dateseparator}{\shortdate}
\fancypagestyle{plain}{\setboolean{first}{false}	% 在 plain 样式的定义中将 first 重置为 false
\lhead{\songti\zihao{-5} \number\year ~年 \number\month ~月} \chead{\zihao{-5} 实\quad 变\quad 函\quad 数\quad 笔\quad 记} \rhead{\zihao{-5}\currenttime, \today}%\monthyeardate\today}
\lfoot{} \cfoot{} \rfoot{}}

\pagestyle{fancy}
\fancyhf{}
\lhead{} \chead{\zihao{-5} 阙嘉豪: 矩阵分解} \rhead{\zihao{-5}\thepage}
\lfoot{} \cfoot{} \rfoot{}

% 奇偶页页眉
%\pagestyle{fancy}
%\fancyhf{}
%\fancyhead[LE]{\kaishu\zihao{-5}\thepage\quad \leftmark}
%\fancyhead[RO]{\kaishu\zihao{-5}\rightmark\quad \thepage}
%\renewcommand\sectionmark[1]{%
%\markright{\CTEXifname{\CTEXthesection\quad}{}#1}}


%-------------------- 正文 --------------------
\begin{document}

\thispagestyle{plain}

%\columnseprule = 1pt	% 栏线
\begin{center}
	{\zihao{-2}\heiti 常微分方程自测题~2} \\
	\vspace{1.5ex}
	{\zihao{-4}\fangsong 阙嘉豪\textsuperscript{\hyperref[auth:1]{1}}%
	\blfootnote{\zihao{6}\heiti 作者简介: \songti 阙嘉豪~(1999—), 男, 广东深圳人, 北京师范大学数学科学学院本科生.}} \\
	{\zihao{6}\songti \label{auth:1}(1. 北京师范大学 数学科学学院, 北京~~100875)}
\end{center}

{\zihao{-5}\heiti 摘要: \songti 本文主要介绍了矩阵分解的三种办法, 分别讨论了分解的存在性及唯一性的问题. 在三角分解中, 通过~Gauss 消元法引出了分解, 并给出了消元过程能进行到底的条件, 最后得到了~LDU 基本定理. 在~QR 分解中, 分别使用了~Gram–Schmidt 正交化方法、Givens 变换和~Householder 变换来得到分解式. 在最大秩分解中, 通过使用初等行变换将矩阵化为阶梯形矩阵获得了最大秩分解.}

{\zihao{-5}\heiti 关键字: \songti 矩阵;~LU 分解;~QR 分解; 最大秩分解}

{\zihao{-5}\heiti 中图分类号: O 151.21 \qquad\quad 文献标识码: A \qquad\quad}

\tableofcontents

\zihao{5}

%\watermark{60}{10}{\currenttime}

\part{部分}
\section{一级标题}
\subsection{二级标题}
\subsubsection{三级标题}

Hello \LaTeX

你好\LaTeX

% 行内公式
方程 1 $ a + b = c $,

方程 3 \begin{math}  g \div h = i \end{math}. % 行间公式

Newton-Leibniz 公式
\begin{equation}
	\int_a^b f(x) = F(b) - F(a).
\end{equation}

质能方程
$$E = mc^2.$$

线性组合
\[\bm\gamma = \lambda_1\bm\alpha + \lambda_2\bm\beta.\]	% 需要 \usepackage{bm}

正弦定理
\begin{displaymath}
\dfrac{a}{\sin A} = \dfrac{b}{\sin B} = \dfrac{c}{\sin C}.
\end{displaymath}

$\forall\exists$
\{ \}

$\varepsilon \partial \cdots \aleph \rtimes$	% \rtimes 需要 \usepackage{amssymb}

$\lim$

矩阵的秩 $a\rank\bm{M} rank$

\begin{center}
最大值 $\max_{i}$	\[\max_{i}\]

$\displaystyle\max_{i}$

积分 $\int_a^b$	\[\int_a^b\]

$\displaystyle\int_a^b$

\[\int\limits_a^b\]

\[\intop_a^b\]
\end{center}

\[\sum_{\substack{0 \leq i\\ 0 < j < n}}\]

{\bf 练习 1}
\[(\lim_{\mathcal{B}}f(x) = A) := (\forall V(a) \subset Y \exists B \in \mathcal{B}(f(B) \subset f(A))).\]

$$(\lim_{n \to \infty} x_n = A) := \forall \varepsilon > 0, \exists N \in \mathbb{N}, \forall n \geqslant N(|x_n - A| < \varepsilon).$$

$(\lim_{n \to \infty} x_n = A) := \forall \varepsilon > 0, \exists N \in \mathbb{N}, \forall n \geqslant N(|x_n - A| < \varepsilon).$

$(\displaystyle\lim_{n \to \infty} x_n = A) := \forall \varepsilon > 0, \exists N \in \mathbb{N}, \forall n \geqslant N(|x_n - A| < \varepsilon).$

勾股定理\begin{equation}
c^2 = a^2 + b^2.
\end{equation}

勾股定理\begin{equation*}
c^2 = a^2 + b^2.
\end{equation*}

% 需要 \usepackage{array}	% 数组宏包
\begin{equation*}
	\left.%
	\begin{array}{>{} r c@{}l@{}l}
		\text{\kaishu 常数})	& y	& =	& c \\
		\text{\kaishu 抛物线})	& y	& =	& cx + d \\
		\text{\kaishu 直线})	& y	& =	& bx^2 + cx + d \\
	\end{array}\right\}\text{多项式}
\end{equation*}

$$\begin{array}{r@{~}ll}
x_k &= \left\{
\begin{array}{l}
\eta, \\
a+(k-1)h,
\end{array}\right. &
\begin{array}{l}
k=0,\\
k=1,2,\cdots,n,
\end{array}
\\
y_k &= \left\{
\begin{array}{l}
\eta,\\
y_{k-1}+\dfrac{h}{8}(t_{1,k}+3t_{2,k}+3t_{3,k}+t_{4,k}),
\end{array}\right. &
\begin{array}{l}
k=0,\\
k=1,2,\cdots,n,
\end{array}
\end{array}$$

\begin{gather}
y = \mathrm{e}^x \\
x = \ln y
\end{gather}

\begin{equation}
\left\{\begin{gathered}
y = \mathrm{e}^x \\
x = \ln y
\end{gathered}\right.\text{与}
\left\{\begin{gathered}
y = x + 1 \\
x = y - 1
\end{gathered}\right.
\end{equation}

\begin{align}	% 列对之间的空白宽度与列对两侧的空白宽度相等
c_{11}	&= a_{11}\times b_{11}	& c_{12}	&= a_{12}\times b_{12} \\
c_{21}	&= a_{21}\times b_{21}	& c_{22}	&= a_{22}\times b_{22}
\end{align}

\begin{equation}
\begin{aligned}	% 列对之间的空白宽度与列对两侧的空白宽度相等
c_{11}	&= a_{11}\times b_{11}	& c_{12}	&= a_{12}\times b_{12} \\
c_{21}	&= a_{21}\times b_{21}	& c_{22}	&= a_{22}\times b_{22}
\end{aligned}
\end{equation}

\begin{flalign}	% 两端对齐
c_{11}	&= a_{11}\times b_{11}	& c_{12}	&= a_{12}\times b_{12} \\
c_{21}	&= a_{21}\times b_{21}	& c_{22}	&= a_{22}\times b_{22}
\end{flalign}
\newpage

\begin{subequations}	% 子公式编号
向量相加\begin{equation}
\bm{a} = \bm{b} + \bm{c}
\end{equation}
\begin{align}
a_1	& = b_1 + c_1 \\
a_2	& = b_2 + c_2 \\
a_2	& = b_2 + c_2
\end{align}
\end{subequations}

完全平方展开计算
\begin{equation}
\begin{split}	% 编号位置通过 \usepackage[tbtags]{amsmath} 移至最下方
f(x)	& = 2(x + 1)^2 - 1 \\
		& = 2(x^2 + 2x + 1) - 1 \\
		& = 2x^2 + 4x + 1
\end{split}
\end{equation}

% 括号环境
\begin{equation}
|x|=
\begin{cases}	% 只给一个序号
x	& x \geq 0 \\
-x	& x \leq 0
\end{cases}\label{abs}
\end{equation}

% 需要 cases 括号宏包, 报 Bad Boxes 以后解决
%\begin{numcases}{|x|=}	% 分别给出序号
%x & $ x \geq 0 $\\
%-x & $ x \leq 0 $
%\end{numcases}

%\begin{subnumcases}{|x|=}	% 两行共用一个序号
%x	& $ x \geq 0 $ \\
%-x	& $ x \leq 0 $
%\end{subnumcases}

{\bf 练习 2}
真空中电磁场的麦克斯韦方程组:
\begin{equation}
\begin{aligned}
\mathrm{div} \bm{E} &= \dfrac{\rho}{\varepsilon_0},& \mathrm{div} \bm{B} &= 0,\\
\mathrm{rot} \bm{E} &= -\dfrac{\partial \bm{B}}{\partial t},& \mathrm{rot} \bm{B} &= \dfrac{\bm{j}}{\varepsilon_0 c^2} + \dfrac{1}{c^2} \dfrac{\partial \bm{E}}{\partial t}.
\end{aligned}
\end{equation}

% 公式序号
$$U = Q + W\eqno{[T.1]}$$
$$U = Q + W\leqno{[T.1]}$$
\begin{equation}U = Q + W \notag\end{equation}
\begin{equation}U = Q + W \tag{$*$}\end{equation}
\begin{equation}U = Q + W \tag*{$*$}\end{equation}
\begin{equation*}U = Q + W \tag*{$*$}\end{equation*}

% 交叉引用
绝对值函数 \ref{abs}

绝对值函数 \eqref{abs}

分块矩阵
\begin{equation}
\left(
\begin{array}{c@{}c@{}}
\begin{array}{|cc|}\hline
a_{11}	& a_{12} \\
a_{21}	& a_{22} \\ \hline
\end{array}	& \bm{O} \\
\bm{O}		& \begin{array}{|ccc|}\hline
b_{11}	& b_{12}	& b_{13} \\
b_{21}	& b_{22}	& b_{23} \\
b_{31}	& b_{32}	& b_{33} \\ \hline
\end{array} \\
\end{array}\right)
\end{equation}

三阶循环矩阵
\begin{gather*}
	\begin{matrix}
		0 & 1 & 0 \\
		0 & 0 & 1 \\
		1 & 0 & 0
	\end{matrix}\quad
	\begin{pmatrix}
		0 & 1 & 0 \\
		0 & 0 & 1 \\
		1 & 0 & 0
	\end{pmatrix}\quad
	\begin{bmatrix}
		0 & 1 & 0 \\
		0 & 0 & 1 \\
		1 & 0 & 0
	\end{bmatrix}\\
	\begin{vmatrix}
		0 & 1 & 0 \\
		0 & 0 & 1 \\
		1 & 0 & 0
	\end{vmatrix}\quad
	\begin{Vmatrix}
		0 & 1 & 0 \\
		0 & 0 & 1 \\
		1 & 0 & 0
	\end{Vmatrix}
	\begin{smallmatrix}
		0 & 1 & 0 \\
		0 & 0 & 1 \\
		1 & 0 & 0
	\end{smallmatrix}
\end{gather*}

% Givens 矩阵
\[\bm T_{ij} = \begin{pmatrix}
	1 \\
		& \ddots \\
		&		& 1 \\
		&		&	& c		&	 &		&	 & s \\
		&		&	&		& 1 \\
		&		&	&		&	& \ddots \\
		&		&	&		&	&		& 1 \\
		&		&	& -s	&	&		&	& c \\
		&		&	&		&	&		&	&	& 1 \\
		&		&	&		&	&		&	&	&	& \ddots \\
		&		&	&		&	&		&	&	&	&		& 1 \\
\end{pmatrix} \begin{matrix}
\\
\\
\\
i \\
\\
\\
\\
j \\
\\
\\
\\
\end{matrix},\quad (i < j)\]

\[a\mspace{1mu}b\quad c\]
\vspace{1cm}
\[d\]

分数
\[\frac{a}{b}\dfrac{a}{b}\displaystyle\frac{a}{b}\tfrac{a}{b}\textstyle\frac{a}{b}\]

\centerline{$\frac{a}{b}\dfrac{a}{b}\displaystyle\frac{a}{b}\tfrac{a}{b}\textstyle\frac{a}{b}$}

%二项式系数
%\[{n + 1 \choose k} = {n \choose k} + {n \choose k + 1}.\]

二项式系数
\[\binom{n + 1}{k} = \tbinom{n}{k} + \dbinom{n}{k - 1}.\]


\centerline{二项式系数
$\binom{n + 1}{k} = \tbinom{n}{k} + \dbinom{n}{k - 1}$.}

柯西不等式
\[(\sum_{i = 1}^n u_i v_i)^2 \leq (\sum_{i = 1}^n u_i^2)(\sum_{i = 1}^n v_i^2).\]

柯西不等式
\[\left(\sum_{i = 1}^n u_i v_i\right)^2 \leq \left(\sum_{i = 1}^n u_i^2\right)\left(\sum_{i = 1}^n v_i^2\right).\]

{\bf 练习 3} (Cauchy-Binet 公式)\quad {\kaishu 设 $\bm{A} = \left(a_{ij}\right)$ 是 $m \times n$ 矩阵, $\bm{B} = \left(b_{ij}\right)$ 是 $n \times m$ 矩阵. $\bm{A}\begin{pmatrix}
	i_1 & \cdots & i_s \\
	j_1 & \cdots & j_s
\end{pmatrix}$ 表示 $\bm{A}$ 的一个 $s$ 阶子式, 它由 $\bm{A}$ 的第 $i_1, \cdots, i_s$ 行与第 $j_1, \cdots, j_s$ 列交点上的元素按原次序排列组成的行列式. 同理可定义 $\bm{B}$ 的 $s$ 阶子式.

(1) 若 $m > n$, 则必有 $|\bm{AB}| = 0$;

(2) 若 $m \leq n$, 则必有
\[|\bm{AB}| = \sum_{1 \leq j_1 < j_2 < \cdots < j_m \leq n} \bm{A}\begin{pmatrix}
	1	& 2		& \cdots & m \\
	j_1	& j_2	& \cdots & j_m
\end{pmatrix} \bm{B}\begin{pmatrix}
	j_1	& j_2	& \cdots & j_m \\
	1	& 2		& \cdots & m
\end{pmatrix}.\]
}

{\bf 练习 4} \begin{equation*}\begin{split}
\varphi^*\omega(t)(\bm{\tau}_1, \bm{\tau}_2) :&= \omega(\varphi(\bm{\tau}_1, \bm{\tau}_2)) = \mathrm{d} x^{i_1} \mathrm{d} x^{i_2}(\bm{\xi}_1, \bm{\xi}_2) \\
&= \begin{vmatrix}
	\xi_1^{i_1}	& \xi_1^{i_2} \\
	\xi_2^{i_1}	& \xi_2^{i_2}
\end{vmatrix} = \begin{vmatrix}
	\dfrac{\partial x^{i_1}}{\partial t^{j_1}} \tau_1^{j_1}	& \dfrac{\partial x^{i_2}}{\partial t^{j_2}} \tau_1^{j_2} \\
	\dfrac{\partial x^{i_1}}{\partial t^{j_1}} \tau_2^{j_1}	& \dfrac{\partial x^{i_2}}{\partial t^{j_2}} \tau_2^{j_2}
\end{vmatrix} \\
&= \sum_{j_1, j_2 = 1}^m \dfrac{\partial x^{i_1}}{\partial t^{j_1}} \dfrac{\partial x^{i_2}}{\partial t^{j_2}} \begin{vmatrix}
	\tau_1^{j_1} & \tau_1^{j_2} \\
	\tau_2^{j_1} & \tau_2^{j_2}
\end{vmatrix} \\
&= \sum_{j_1, j_2 = 1}^m \dfrac{\partial x^{i_1}}{\partial t^{j_1}} \dfrac{\partial x^{i_2}}{\partial t^{j_2}} \mathrm{d} t^{j_1} \wedge \mathrm{d} t^{j_2} (\bm{\tau}_1, \bm{\tau}_2) \\
&= \sum_{1 \leqslant j_1 < j_2 \leqslant m} \left(\dfrac{\partial x^{i_1}}{\partial t^{j_1}} \dfrac{\partial x^{i_2}}{\partial t^{j_2}} - \dfrac{\partial x^{i_1}}{\partial t^{j_2}} \dfrac{\partial x^{i_2}}{\partial t^{j_1}}\right) \mathrm{d} t^{j_1} \wedge \mathrm{d} t^{j_2} (\bm{\tau}_1, \bm{\tau}_2) \\
&= \sum_{1 \leqslant j_1 < j_2 \leqslant m} \begin{vmatrix}
	\dfrac{\partial x^{i_1}}{\partial t^{j_1}} & \dfrac{\partial x^{i_2}}{\partial t^{j_1}} \\
	\dfrac{\partial x^{i_1}}{\partial t^{j_2}} & \dfrac{\partial x^{i_2}}{\partial t^{j_2}}
\end{vmatrix}(t) \mathrm{d} t^{j_1} \wedge \mathrm{d} t^{j_2} (\bm{\tau}_1, \bm{\tau}_2).
\end{split}\end{equation*}

% 交换图
\begin{equation}\begin{CD}
	a		@>j>>	b \\
	@VVV			@VV{\lim P}V \\
	c		@=		d
\end{CD}\end{equation}

\begin{equation}\label{cdd}
\xymatrix{
A \ar[d] \ar[r] & B \\
B \ar[r] & C \ar[u]}
\end{equation}

\pageref{TotPages}

引用参考文献超链接\upcite{RongYuan}

\hyperref[cdd]{text}

$\_\_\_\_\_\_\_\_\_\_\_\_\_\_\_\_$

\begin{equation*}
\begin{split}
    a \\
     bbbbbbbbbbb \\
      c
  \end{split}
  \begin{array}{c}
    a \\
    b \\
    c
  \end{array}
\end{equation*}

\begin{figure}[ht]
	\centering
	\begin{tikzpicture}[decoration={
		markings,% switch on markings
		mark=at position .5 with {\arrow[line width=1pt]{>}}},]
		\draw[fill = green!20] (0, 2.866) arc[start angle = 60, end angle = 300, x radius = 2, y radius = 1] arc[start angle = 240, end angle = 120, x radius = 2, y radius = 1] (-2, 2) node{$U_\alpha$};
		\draw[fill = blue!20] (0, 2.866) arc[start angle = 120, end angle = -120, x radius = 2, y radius = 1] arc[start angle = -60, end angle = 60, x radius = 2, y radius = 1] (2, 2) node{$U_\beta$};
		\draw[fill = red!20] (0, 2.866) arc[start angle = 120, end angle = 240, x radius = 2, y radius = 1] arc[start angle = -60, end angle = 60, x radius = 2, y radius = 1] (0, 2) node{$W$};
		\draw (-1, 0) -- (-4, 0) -- (-5, -2) -- (-2, -2) -- (-1, 0) (-0.5, -0.25) node{$\R^n$};
		\draw (4, 0) -- (1, 0) -- (0, -2) -- (3, -2) -- (4, 0) (4.5, -0.25) node{$\R^n$};
		\draw[rotate around = {4 : (-3, -1)}, color = black!50, fill = green!20] (-3, -1) ellipse [x radius = 1.4, y radius = 0.7];
		\draw[rotate around = {4 : (2, -1)}, color = black!50, fill = blue!20] (2, -1) ellipse [x radius = 1.4, y radius = 0.7];
		\draw[rotate around = {4 : (-3, -1)}, color = black!50, fill = red!20] (-3, -0.3) arc[start angle = 150, end angle = 210, radius = 1.4] arc[start angle = -90, end angle = 90, x radius = 1.4, y radius = 0.7];
		\draw (-3, -1) node{$f_\alpha\(U_\alpha\)$};
		\draw[rotate around = {4 : (2, -1)}, color = black!50, fill = red!20] (2, -0.3) arc[start angle = 30, end angle = -30, radius = 1.4] arc[start angle = 270, end angle = 90, x radius = 1.4, y radius = 0.7];
		\draw (2, -1) node{$f_\beta\(U_\beta\)$};
		\draw[->] (-2, 1.5) arc[start angle = 130, end angle = 160, radius = 3.4] node[midway, sloped, above]{$f_\alpha$};
		\draw[->] (2, 1.5) arc[start angle = 50, end angle = 20, radius = 3.4] node[midway, sloped, above]{$f_\beta$};
		\draw[<-] (-0.5, 2) arc[start angle = 120, end angle = 185, radius = 2.9] node[midway, sloped, below]{$f_\alpha^{-1}$};
		\draw[->] (0.5, 2) arc[start angle = 60, end angle = -5, radius = 2.9] node[midway, sloped, below]{$f_\beta$};
	\end{tikzpicture}
	\caption{Condition}
\end{figure}

\begin{figure}[H]
	\resizebox{0.9\linewidth}{!}{%% Creator: Matplotlib, PGF backend
%%
%% To include the figure in your LaTeX document, write
%%   \input{<filename>.pgf}
%%
%% Make sure the required packages are loaded in your preamble
%%   \usepackage{pgf}
%%
%% and, on pdftex
%%   \usepackage[utf8]{inputenc}\DeclareUnicodeCharacter{2212}{-}
%%
%% or, on luatex and xetex
%%   \usepackage{unicode-math}
%%
%% Figures using additional raster images can only be included by \input if
%% they are in the same directory as the main LaTeX file. For loading figures
%% from other directories you can use the `import` package
%%   \usepackage{import}
%%
%% and then include the figures with
%%   \import{<path to file>}{<filename>.pgf}
%%
%% Matplotlib used the following preamble
%%   \usepackage{fontspec}
%%   \setmainfont{DejaVuSerif.ttf}[Path=C:/Users/Quejiahao/.julia/conda/3/Lib/site-packages/matplotlib/mpl-data/fonts/ttf/]
%%   \setsansfont{DejaVuSans.ttf}[Path=C:/Users/Quejiahao/.julia/conda/3/Lib/site-packages/matplotlib/mpl-data/fonts/ttf/]
%%   \setmonofont{DejaVuSansMono.ttf}[Path=C:/Users/Quejiahao/.julia/conda/3/Lib/site-packages/matplotlib/mpl-data/fonts/ttf/]
%%
\begingroup%
\makeatletter%
\begin{pgfpicture}%
\pgfpathrectangle{\pgfpointorigin}{\pgfqpoint{19.200000in}{9.830000in}}%
\pgfusepath{use as bounding box, clip}%
\begin{pgfscope}%
\pgfsetbuttcap%
\pgfsetmiterjoin%
\definecolor{currentfill}{rgb}{1.000000,1.000000,1.000000}%
\pgfsetfillcolor{currentfill}%
\pgfsetlinewidth{0.000000pt}%
\definecolor{currentstroke}{rgb}{1.000000,1.000000,1.000000}%
\pgfsetstrokecolor{currentstroke}%
\pgfsetdash{}{0pt}%
\pgfpathmoveto{\pgfqpoint{0.000000in}{0.000000in}}%
\pgfpathlineto{\pgfqpoint{19.200000in}{0.000000in}}%
\pgfpathlineto{\pgfqpoint{19.200000in}{9.830000in}}%
\pgfpathlineto{\pgfqpoint{0.000000in}{9.830000in}}%
\pgfpathclose%
\pgfusepath{fill}%
\end{pgfscope}%
\begin{pgfscope}%
\pgfsetbuttcap%
\pgfsetmiterjoin%
\definecolor{currentfill}{rgb}{1.000000,1.000000,1.000000}%
\pgfsetfillcolor{currentfill}%
\pgfsetlinewidth{0.000000pt}%
\definecolor{currentstroke}{rgb}{0.000000,0.000000,0.000000}%
\pgfsetstrokecolor{currentstroke}%
\pgfsetstrokeopacity{0.000000}%
\pgfsetdash{}{0pt}%
\pgfpathmoveto{\pgfqpoint{2.400000in}{1.081300in}}%
\pgfpathlineto{\pgfqpoint{17.280000in}{1.081300in}}%
\pgfpathlineto{\pgfqpoint{17.280000in}{8.650400in}}%
\pgfpathlineto{\pgfqpoint{2.400000in}{8.650400in}}%
\pgfpathclose%
\pgfusepath{fill}%
\end{pgfscope}%
\begin{pgfscope}%
\pgfsetbuttcap%
\pgfsetroundjoin%
\definecolor{currentfill}{rgb}{0.000000,0.000000,0.000000}%
\pgfsetfillcolor{currentfill}%
\pgfsetlinewidth{0.803000pt}%
\definecolor{currentstroke}{rgb}{0.000000,0.000000,0.000000}%
\pgfsetstrokecolor{currentstroke}%
\pgfsetdash{}{0pt}%
\pgfsys@defobject{currentmarker}{\pgfqpoint{0.000000in}{-0.048611in}}{\pgfqpoint{0.000000in}{0.000000in}}{%
\pgfpathmoveto{\pgfqpoint{0.000000in}{0.000000in}}%
\pgfpathlineto{\pgfqpoint{0.000000in}{-0.048611in}}%
\pgfusepath{stroke,fill}%
}%
\begin{pgfscope}%
\pgfsys@transformshift{3.073052in}{1.081300in}%
\pgfsys@useobject{currentmarker}{}%
\end{pgfscope}%
\end{pgfscope}%
\begin{pgfscope}%
\definecolor{textcolor}{rgb}{0.000000,0.000000,0.000000}%
\pgfsetstrokecolor{textcolor}%
\pgfsetfillcolor{textcolor}%
\pgftext[x=3.073052in,y=0.784078in,,top]{\color{textcolor}\sffamily\fontsize{36.000000}{12.000000}\selectfont 0}%
\end{pgfscope}%
\begin{pgfscope}%
\pgfsetbuttcap%
\pgfsetroundjoin%
\definecolor{currentfill}{rgb}{0.000000,0.000000,0.000000}%
\pgfsetfillcolor{currentfill}%
\pgfsetlinewidth{0.803000pt}%
\definecolor{currentstroke}{rgb}{0.000000,0.000000,0.000000}%
\pgfsetstrokecolor{currentstroke}%
\pgfsetdash{}{0pt}%
\pgfsys@defobject{currentmarker}{\pgfqpoint{0.000000in}{-0.048611in}}{\pgfqpoint{0.000000in}{0.000000in}}{%
\pgfpathmoveto{\pgfqpoint{0.000000in}{0.000000in}}%
\pgfpathlineto{\pgfqpoint{0.000000in}{-0.048611in}}%
\pgfusepath{stroke,fill}%
}%
\begin{pgfscope}%
\pgfsys@transformshift{4.728777in}{1.081300in}%
\pgfsys@useobject{currentmarker}{}%
\end{pgfscope}%
\end{pgfscope}%
\begin{pgfscope}%
\definecolor{textcolor}{rgb}{0.000000,0.000000,0.000000}%
\pgfsetstrokecolor{textcolor}%
\pgfsetfillcolor{textcolor}%
\pgftext[x=4.728777in,y=0.784078in,,top]{\color{textcolor}\sffamily\fontsize{36.000000}{12.000000}\selectfont 50}%
\end{pgfscope}%
\begin{pgfscope}%
\pgfsetbuttcap%
\pgfsetroundjoin%
\definecolor{currentfill}{rgb}{0.000000,0.000000,0.000000}%
\pgfsetfillcolor{currentfill}%
\pgfsetlinewidth{0.803000pt}%
\definecolor{currentstroke}{rgb}{0.000000,0.000000,0.000000}%
\pgfsetstrokecolor{currentstroke}%
\pgfsetdash{}{0pt}%
\pgfsys@defobject{currentmarker}{\pgfqpoint{0.000000in}{-0.048611in}}{\pgfqpoint{0.000000in}{0.000000in}}{%
\pgfpathmoveto{\pgfqpoint{0.000000in}{0.000000in}}%
\pgfpathlineto{\pgfqpoint{0.000000in}{-0.048611in}}%
\pgfusepath{stroke,fill}%
}%
\begin{pgfscope}%
\pgfsys@transformshift{6.384502in}{1.081300in}%
\pgfsys@useobject{currentmarker}{}%
\end{pgfscope}%
\end{pgfscope}%
\begin{pgfscope}%
\definecolor{textcolor}{rgb}{0.000000,0.000000,0.000000}%
\pgfsetstrokecolor{textcolor}%
\pgfsetfillcolor{textcolor}%
\pgftext[x=6.384502in,y=0.784078in,,top]{\color{textcolor}\sffamily\fontsize{36.000000}{12.000000}\selectfont 100}%
\end{pgfscope}%
\begin{pgfscope}%
\pgfsetbuttcap%
\pgfsetroundjoin%
\definecolor{currentfill}{rgb}{0.000000,0.000000,0.000000}%
\pgfsetfillcolor{currentfill}%
\pgfsetlinewidth{0.803000pt}%
\definecolor{currentstroke}{rgb}{0.000000,0.000000,0.000000}%
\pgfsetstrokecolor{currentstroke}%
\pgfsetdash{}{0pt}%
\pgfsys@defobject{currentmarker}{\pgfqpoint{0.000000in}{-0.048611in}}{\pgfqpoint{0.000000in}{0.000000in}}{%
\pgfpathmoveto{\pgfqpoint{0.000000in}{0.000000in}}%
\pgfpathlineto{\pgfqpoint{0.000000in}{-0.048611in}}%
\pgfusepath{stroke,fill}%
}%
\begin{pgfscope}%
\pgfsys@transformshift{8.040227in}{1.081300in}%
\pgfsys@useobject{currentmarker}{}%
\end{pgfscope}%
\end{pgfscope}%
\begin{pgfscope}%
\definecolor{textcolor}{rgb}{0.000000,0.000000,0.000000}%
\pgfsetstrokecolor{textcolor}%
\pgfsetfillcolor{textcolor}%
\pgftext[x=8.040227in,y=0.784078in,,top]{\color{textcolor}\sffamily\fontsize{36.000000}{12.000000}\selectfont 150}%
\end{pgfscope}%
\begin{pgfscope}%
\pgfsetbuttcap%
\pgfsetroundjoin%
\definecolor{currentfill}{rgb}{0.000000,0.000000,0.000000}%
\pgfsetfillcolor{currentfill}%
\pgfsetlinewidth{0.803000pt}%
\definecolor{currentstroke}{rgb}{0.000000,0.000000,0.000000}%
\pgfsetstrokecolor{currentstroke}%
\pgfsetdash{}{0pt}%
\pgfsys@defobject{currentmarker}{\pgfqpoint{0.000000in}{-0.048611in}}{\pgfqpoint{0.000000in}{0.000000in}}{%
\pgfpathmoveto{\pgfqpoint{0.000000in}{0.000000in}}%
\pgfpathlineto{\pgfqpoint{0.000000in}{-0.048611in}}%
\pgfusepath{stroke,fill}%
}%
\begin{pgfscope}%
\pgfsys@transformshift{9.695952in}{1.081300in}%
\pgfsys@useobject{currentmarker}{}%
\end{pgfscope}%
\end{pgfscope}%
\begin{pgfscope}%
\definecolor{textcolor}{rgb}{0.000000,0.000000,0.000000}%
\pgfsetstrokecolor{textcolor}%
\pgfsetfillcolor{textcolor}%
\pgftext[x=9.695952in,y=0.784078in,,top]{\color{textcolor}\sffamily\fontsize{36.000000}{12.000000}\selectfont 200}%
\end{pgfscope}%
\begin{pgfscope}%
\pgfsetbuttcap%
\pgfsetroundjoin%
\definecolor{currentfill}{rgb}{0.000000,0.000000,0.000000}%
\pgfsetfillcolor{currentfill}%
\pgfsetlinewidth{0.803000pt}%
\definecolor{currentstroke}{rgb}{0.000000,0.000000,0.000000}%
\pgfsetstrokecolor{currentstroke}%
\pgfsetdash{}{0pt}%
\pgfsys@defobject{currentmarker}{\pgfqpoint{0.000000in}{-0.048611in}}{\pgfqpoint{0.000000in}{0.000000in}}{%
\pgfpathmoveto{\pgfqpoint{0.000000in}{0.000000in}}%
\pgfpathlineto{\pgfqpoint{0.000000in}{-0.048611in}}%
\pgfusepath{stroke,fill}%
}%
\begin{pgfscope}%
\pgfsys@transformshift{11.351677in}{1.081300in}%
\pgfsys@useobject{currentmarker}{}%
\end{pgfscope}%
\end{pgfscope}%
\begin{pgfscope}%
\definecolor{textcolor}{rgb}{0.000000,0.000000,0.000000}%
\pgfsetstrokecolor{textcolor}%
\pgfsetfillcolor{textcolor}%
\pgftext[x=11.351677in,y=0.784078in,,top]{\color{textcolor}\sffamily\fontsize{36.000000}{12.000000}\selectfont 250}%
\end{pgfscope}%
\begin{pgfscope}%
\pgfsetbuttcap%
\pgfsetroundjoin%
\definecolor{currentfill}{rgb}{0.000000,0.000000,0.000000}%
\pgfsetfillcolor{currentfill}%
\pgfsetlinewidth{0.803000pt}%
\definecolor{currentstroke}{rgb}{0.000000,0.000000,0.000000}%
\pgfsetstrokecolor{currentstroke}%
\pgfsetdash{}{0pt}%
\pgfsys@defobject{currentmarker}{\pgfqpoint{0.000000in}{-0.048611in}}{\pgfqpoint{0.000000in}{0.000000in}}{%
\pgfpathmoveto{\pgfqpoint{0.000000in}{0.000000in}}%
\pgfpathlineto{\pgfqpoint{0.000000in}{-0.048611in}}%
\pgfusepath{stroke,fill}%
}%
\begin{pgfscope}%
\pgfsys@transformshift{13.007402in}{1.081300in}%
\pgfsys@useobject{currentmarker}{}%
\end{pgfscope}%
\end{pgfscope}%
\begin{pgfscope}%
\definecolor{textcolor}{rgb}{0.000000,0.000000,0.000000}%
\pgfsetstrokecolor{textcolor}%
\pgfsetfillcolor{textcolor}%
\pgftext[x=13.007402in,y=0.784078in,,top]{\color{textcolor}\sffamily\fontsize{36.000000}{12.000000}\selectfont 300}%
\end{pgfscope}%
\begin{pgfscope}%
\pgfsetbuttcap%
\pgfsetroundjoin%
\definecolor{currentfill}{rgb}{0.000000,0.000000,0.000000}%
\pgfsetfillcolor{currentfill}%
\pgfsetlinewidth{0.803000pt}%
\definecolor{currentstroke}{rgb}{0.000000,0.000000,0.000000}%
\pgfsetstrokecolor{currentstroke}%
\pgfsetdash{}{0pt}%
\pgfsys@defobject{currentmarker}{\pgfqpoint{0.000000in}{-0.048611in}}{\pgfqpoint{0.000000in}{0.000000in}}{%
\pgfpathmoveto{\pgfqpoint{0.000000in}{0.000000in}}%
\pgfpathlineto{\pgfqpoint{0.000000in}{-0.048611in}}%
\pgfusepath{stroke,fill}%
}%
\begin{pgfscope}%
\pgfsys@transformshift{14.663127in}{1.081300in}%
\pgfsys@useobject{currentmarker}{}%
\end{pgfscope}%
\end{pgfscope}%
\begin{pgfscope}%
\definecolor{textcolor}{rgb}{0.000000,0.000000,0.000000}%
\pgfsetstrokecolor{textcolor}%
\pgfsetfillcolor{textcolor}%
\pgftext[x=14.663127in,y=0.784078in,,top]{\color{textcolor}\sffamily\fontsize{36.000000}{12.000000}\selectfont 350}%
\end{pgfscope}%
\begin{pgfscope}%
\pgfsetbuttcap%
\pgfsetroundjoin%
\definecolor{currentfill}{rgb}{0.000000,0.000000,0.000000}%
\pgfsetfillcolor{currentfill}%
\pgfsetlinewidth{0.803000pt}%
\definecolor{currentstroke}{rgb}{0.000000,0.000000,0.000000}%
\pgfsetstrokecolor{currentstroke}%
\pgfsetdash{}{0pt}%
\pgfsys@defobject{currentmarker}{\pgfqpoint{0.000000in}{-0.048611in}}{\pgfqpoint{0.000000in}{0.000000in}}{%
\pgfpathmoveto{\pgfqpoint{0.000000in}{0.000000in}}%
\pgfpathlineto{\pgfqpoint{0.000000in}{-0.048611in}}%
\pgfusepath{stroke,fill}%
}%
\begin{pgfscope}%
\pgfsys@transformshift{16.318852in}{1.081300in}%
\pgfsys@useobject{currentmarker}{}%
\end{pgfscope}%
\end{pgfscope}%
\begin{pgfscope}%
\definecolor{textcolor}{rgb}{0.000000,0.000000,0.000000}%
\pgfsetstrokecolor{textcolor}%
\pgfsetfillcolor{textcolor}%
\pgftext[x=16.318852in,y=0.784078in,,top]{\color{textcolor}\sffamily\fontsize{36.000000}{12.000000}\selectfont 400}%
\end{pgfscope}%
\begin{pgfscope}%
\pgfsetbuttcap%
\pgfsetroundjoin%
\definecolor{currentfill}{rgb}{0.000000,0.000000,0.000000}%
\pgfsetfillcolor{currentfill}%
\pgfsetlinewidth{0.803000pt}%
\definecolor{currentstroke}{rgb}{0.000000,0.000000,0.000000}%
\pgfsetstrokecolor{currentstroke}%
\pgfsetdash{}{0pt}%
\pgfsys@defobject{currentmarker}{\pgfqpoint{-0.048611in}{0.000000in}}{\pgfqpoint{-0.000000in}{0.000000in}}{%
\pgfpathmoveto{\pgfqpoint{-0.000000in}{0.000000in}}%
\pgfpathlineto{\pgfqpoint{-0.048611in}{0.000000in}}%
\pgfusepath{stroke,fill}%
}%
\begin{pgfscope}%
\pgfsys@transformshift{2.400000in}{1.425350in}%
\pgfsys@useobject{currentmarker}{}%
\end{pgfscope}%
\end{pgfscope}%
\begin{pgfscope}%
\definecolor{textcolor}{rgb}{0.000000,0.000000,0.000000}%
\pgfsetstrokecolor{textcolor}%
\pgfsetfillcolor{textcolor}%
\pgftext[x=1.105168in, y=1.372588in, left, base]{\color{textcolor}\sffamily\fontsize{36.000000}{12.000000}\selectfont 0.000}%
\end{pgfscope}%
\begin{pgfscope}%
\pgfsetbuttcap%
\pgfsetroundjoin%
\definecolor{currentfill}{rgb}{0.000000,0.000000,0.000000}%
\pgfsetfillcolor{currentfill}%
\pgfsetlinewidth{0.803000pt}%
\definecolor{currentstroke}{rgb}{0.000000,0.000000,0.000000}%
\pgfsetstrokecolor{currentstroke}%
\pgfsetdash{}{0pt}%
\pgfsys@defobject{currentmarker}{\pgfqpoint{-0.048611in}{0.000000in}}{\pgfqpoint{-0.000000in}{0.000000in}}{%
\pgfpathmoveto{\pgfqpoint{-0.000000in}{0.000000in}}%
\pgfpathlineto{\pgfqpoint{-0.048611in}{0.000000in}}%
\pgfusepath{stroke,fill}%
}%
\begin{pgfscope}%
\pgfsys@transformshift{2.400000in}{2.421604in}%
\pgfsys@useobject{currentmarker}{}%
\end{pgfscope}%
\end{pgfscope}%
\begin{pgfscope}%
\definecolor{textcolor}{rgb}{0.000000,0.000000,0.000000}%
\pgfsetstrokecolor{textcolor}%
\pgfsetfillcolor{textcolor}%
\pgftext[x=1.105168in, y=2.368843in, left, base]{\color{textcolor}\sffamily\fontsize{36.000000}{12.000000}\selectfont 0.005}%
\end{pgfscope}%
\begin{pgfscope}%
\pgfsetbuttcap%
\pgfsetroundjoin%
\definecolor{currentfill}{rgb}{0.000000,0.000000,0.000000}%
\pgfsetfillcolor{currentfill}%
\pgfsetlinewidth{0.803000pt}%
\definecolor{currentstroke}{rgb}{0.000000,0.000000,0.000000}%
\pgfsetstrokecolor{currentstroke}%
\pgfsetdash{}{0pt}%
\pgfsys@defobject{currentmarker}{\pgfqpoint{-0.048611in}{0.000000in}}{\pgfqpoint{-0.000000in}{0.000000in}}{%
\pgfpathmoveto{\pgfqpoint{-0.000000in}{0.000000in}}%
\pgfpathlineto{\pgfqpoint{-0.048611in}{0.000000in}}%
\pgfusepath{stroke,fill}%
}%
\begin{pgfscope}%
\pgfsys@transformshift{2.400000in}{3.417859in}%
\pgfsys@useobject{currentmarker}{}%
\end{pgfscope}%
\end{pgfscope}%
\begin{pgfscope}%
\definecolor{textcolor}{rgb}{0.000000,0.000000,0.000000}%
\pgfsetstrokecolor{textcolor}%
\pgfsetfillcolor{textcolor}%
\pgftext[x=1.105168in, y=3.365097in, left, base]{\color{textcolor}\sffamily\fontsize{36.000000}{12.000000}\selectfont 0.010}%
\end{pgfscope}%
\begin{pgfscope}%
\pgfsetbuttcap%
\pgfsetroundjoin%
\definecolor{currentfill}{rgb}{0.000000,0.000000,0.000000}%
\pgfsetfillcolor{currentfill}%
\pgfsetlinewidth{0.803000pt}%
\definecolor{currentstroke}{rgb}{0.000000,0.000000,0.000000}%
\pgfsetstrokecolor{currentstroke}%
\pgfsetdash{}{0pt}%
\pgfsys@defobject{currentmarker}{\pgfqpoint{-0.048611in}{0.000000in}}{\pgfqpoint{-0.000000in}{0.000000in}}{%
\pgfpathmoveto{\pgfqpoint{-0.000000in}{0.000000in}}%
\pgfpathlineto{\pgfqpoint{-0.048611in}{0.000000in}}%
\pgfusepath{stroke,fill}%
}%
\begin{pgfscope}%
\pgfsys@transformshift{2.400000in}{4.414113in}%
\pgfsys@useobject{currentmarker}{}%
\end{pgfscope}%
\end{pgfscope}%
\begin{pgfscope}%
\definecolor{textcolor}{rgb}{0.000000,0.000000,0.000000}%
\pgfsetstrokecolor{textcolor}%
\pgfsetfillcolor{textcolor}%
\pgftext[x=1.105168in, y=4.361352in, left, base]{\color{textcolor}\sffamily\fontsize{36.000000}{12.000000}\selectfont 0.015}%
\end{pgfscope}%
\begin{pgfscope}%
\pgfsetbuttcap%
\pgfsetroundjoin%
\definecolor{currentfill}{rgb}{0.000000,0.000000,0.000000}%
\pgfsetfillcolor{currentfill}%
\pgfsetlinewidth{0.803000pt}%
\definecolor{currentstroke}{rgb}{0.000000,0.000000,0.000000}%
\pgfsetstrokecolor{currentstroke}%
\pgfsetdash{}{0pt}%
\pgfsys@defobject{currentmarker}{\pgfqpoint{-0.048611in}{0.000000in}}{\pgfqpoint{-0.000000in}{0.000000in}}{%
\pgfpathmoveto{\pgfqpoint{-0.000000in}{0.000000in}}%
\pgfpathlineto{\pgfqpoint{-0.048611in}{0.000000in}}%
\pgfusepath{stroke,fill}%
}%
\begin{pgfscope}%
\pgfsys@transformshift{2.400000in}{5.410368in}%
\pgfsys@useobject{currentmarker}{}%
\end{pgfscope}%
\end{pgfscope}%
\begin{pgfscope}%
\definecolor{textcolor}{rgb}{0.000000,0.000000,0.000000}%
\pgfsetstrokecolor{textcolor}%
\pgfsetfillcolor{textcolor}%
\pgftext[x=1.105168in, y=5.357606in, left, base]{\color{textcolor}\sffamily\fontsize{36.000000}{12.000000}\selectfont 0.020}%
\end{pgfscope}%
\begin{pgfscope}%
\pgfsetbuttcap%
\pgfsetroundjoin%
\definecolor{currentfill}{rgb}{0.000000,0.000000,0.000000}%
\pgfsetfillcolor{currentfill}%
\pgfsetlinewidth{0.803000pt}%
\definecolor{currentstroke}{rgb}{0.000000,0.000000,0.000000}%
\pgfsetstrokecolor{currentstroke}%
\pgfsetdash{}{0pt}%
\pgfsys@defobject{currentmarker}{\pgfqpoint{-0.048611in}{0.000000in}}{\pgfqpoint{-0.000000in}{0.000000in}}{%
\pgfpathmoveto{\pgfqpoint{-0.000000in}{0.000000in}}%
\pgfpathlineto{\pgfqpoint{-0.048611in}{0.000000in}}%
\pgfusepath{stroke,fill}%
}%
\begin{pgfscope}%
\pgfsys@transformshift{2.400000in}{6.406622in}%
\pgfsys@useobject{currentmarker}{}%
\end{pgfscope}%
\end{pgfscope}%
\begin{pgfscope}%
\definecolor{textcolor}{rgb}{0.000000,0.000000,0.000000}%
\pgfsetstrokecolor{textcolor}%
\pgfsetfillcolor{textcolor}%
\pgftext[x=1.105168in, y=6.353861in, left, base]{\color{textcolor}\sffamily\fontsize{36.000000}{12.000000}\selectfont 0.025}%
\end{pgfscope}%
\begin{pgfscope}%
\pgfsetbuttcap%
\pgfsetroundjoin%
\definecolor{currentfill}{rgb}{0.000000,0.000000,0.000000}%
\pgfsetfillcolor{currentfill}%
\pgfsetlinewidth{0.803000pt}%
\definecolor{currentstroke}{rgb}{0.000000,0.000000,0.000000}%
\pgfsetstrokecolor{currentstroke}%
\pgfsetdash{}{0pt}%
\pgfsys@defobject{currentmarker}{\pgfqpoint{-0.048611in}{0.000000in}}{\pgfqpoint{-0.000000in}{0.000000in}}{%
\pgfpathmoveto{\pgfqpoint{-0.000000in}{0.000000in}}%
\pgfpathlineto{\pgfqpoint{-0.048611in}{0.000000in}}%
\pgfusepath{stroke,fill}%
}%
\begin{pgfscope}%
\pgfsys@transformshift{2.400000in}{7.402877in}%
\pgfsys@useobject{currentmarker}{}%
\end{pgfscope}%
\end{pgfscope}%
\begin{pgfscope}%
\definecolor{textcolor}{rgb}{0.000000,0.000000,0.000000}%
\pgfsetstrokecolor{textcolor}%
\pgfsetfillcolor{textcolor}%
\pgftext[x=1.105168in, y=7.350115in, left, base]{\color{textcolor}\sffamily\fontsize{36.000000}{12.000000}\selectfont 0.030}%
\end{pgfscope}%
\begin{pgfscope}%
\pgfsetbuttcap%
\pgfsetroundjoin%
\definecolor{currentfill}{rgb}{0.000000,0.000000,0.000000}%
\pgfsetfillcolor{currentfill}%
\pgfsetlinewidth{0.803000pt}%
\definecolor{currentstroke}{rgb}{0.000000,0.000000,0.000000}%
\pgfsetstrokecolor{currentstroke}%
\pgfsetdash{}{0pt}%
\pgfsys@defobject{currentmarker}{\pgfqpoint{-0.048611in}{0.000000in}}{\pgfqpoint{-0.000000in}{0.000000in}}{%
\pgfpathmoveto{\pgfqpoint{-0.000000in}{0.000000in}}%
\pgfpathlineto{\pgfqpoint{-0.048611in}{0.000000in}}%
\pgfusepath{stroke,fill}%
}%
\begin{pgfscope}%
\pgfsys@transformshift{2.400000in}{8.399131in}%
\pgfsys@useobject{currentmarker}{}%
\end{pgfscope}%
\end{pgfscope}%
\begin{pgfscope}%
\definecolor{textcolor}{rgb}{0.000000,0.000000,0.000000}%
\pgfsetstrokecolor{textcolor}%
\pgfsetfillcolor{textcolor}%
\pgftext[x=1.105168in, y=8.346369in, left, base]{\color{textcolor}\sffamily\fontsize{36.000000}{12.000000}\selectfont 0.035}%
\end{pgfscope}%
\begin{pgfscope}%
\pgfpathrectangle{\pgfqpoint{2.400000in}{1.081300in}}{\pgfqpoint{14.880000in}{7.569100in}}%
\pgfusepath{clip}%
\pgfsetrectcap%
\pgfsetroundjoin%
\pgfsetlinewidth{1.505625pt}%
\definecolor{currentstroke}{rgb}{0.121569,0.466667,0.705882}%
\pgfsetstrokecolor{currentstroke}%
\pgfsetdash{}{0pt}%
\pgfpathmoveto{\pgfqpoint{3.076364in}{1.425350in}}%
\pgfpathlineto{\pgfqpoint{3.135970in}{1.957877in}}%
\pgfpathlineto{\pgfqpoint{3.155838in}{2.117054in}}%
\pgfpathlineto{\pgfqpoint{3.169084in}{2.209333in}}%
\pgfpathlineto{\pgfqpoint{3.182330in}{2.285436in}}%
\pgfpathlineto{\pgfqpoint{3.192264in}{2.329811in}}%
\pgfpathlineto{\pgfqpoint{3.202199in}{2.362604in}}%
\pgfpathlineto{\pgfqpoint{3.208822in}{2.378150in}}%
\pgfpathlineto{\pgfqpoint{3.215445in}{2.388950in}}%
\pgfpathlineto{\pgfqpoint{3.222067in}{2.395392in}}%
\pgfpathlineto{\pgfqpoint{3.228690in}{2.397933in}}%
\pgfpathlineto{\pgfqpoint{3.235313in}{2.397065in}}%
\pgfpathlineto{\pgfqpoint{3.241936in}{2.393283in}}%
\pgfpathlineto{\pgfqpoint{3.248559in}{2.387073in}}%
\pgfpathlineto{\pgfqpoint{3.258493in}{2.374204in}}%
\pgfpathlineto{\pgfqpoint{3.271739in}{2.352612in}}%
\pgfpathlineto{\pgfqpoint{3.311477in}{2.284434in}}%
\pgfpathlineto{\pgfqpoint{3.321411in}{2.271998in}}%
\pgfpathlineto{\pgfqpoint{3.331345in}{2.263072in}}%
\pgfpathlineto{\pgfqpoint{3.337968in}{2.259348in}}%
\pgfpathlineto{\pgfqpoint{3.344591in}{2.257528in}}%
\pgfpathlineto{\pgfqpoint{3.351214in}{2.257659in}}%
\pgfpathlineto{\pgfqpoint{3.357837in}{2.259734in}}%
\pgfpathlineto{\pgfqpoint{3.364460in}{2.263699in}}%
\pgfpathlineto{\pgfqpoint{3.374394in}{2.272975in}}%
\pgfpathlineto{\pgfqpoint{3.384328in}{2.285839in}}%
\pgfpathlineto{\pgfqpoint{3.397574in}{2.307604in}}%
\pgfpathlineto{\pgfqpoint{3.414132in}{2.340026in}}%
\pgfpathlineto{\pgfqpoint{3.457180in}{2.428216in}}%
\pgfpathlineto{\pgfqpoint{3.470426in}{2.450059in}}%
\pgfpathlineto{\pgfqpoint{3.480361in}{2.463305in}}%
\pgfpathlineto{\pgfqpoint{3.490295in}{2.473384in}}%
\pgfpathlineto{\pgfqpoint{3.500229in}{2.479960in}}%
\pgfpathlineto{\pgfqpoint{3.506852in}{2.482271in}}%
\pgfpathlineto{\pgfqpoint{3.513475in}{2.482856in}}%
\pgfpathlineto{\pgfqpoint{3.520098in}{2.481672in}}%
\pgfpathlineto{\pgfqpoint{3.526721in}{2.478687in}}%
\pgfpathlineto{\pgfqpoint{3.533344in}{2.473878in}}%
\pgfpathlineto{\pgfqpoint{3.543278in}{2.463213in}}%
\pgfpathlineto{\pgfqpoint{3.553212in}{2.448394in}}%
\pgfpathlineto{\pgfqpoint{3.563147in}{2.429448in}}%
\pgfpathlineto{\pgfqpoint{3.576393in}{2.397900in}}%
\pgfpathlineto{\pgfqpoint{3.589638in}{2.359463in}}%
\pgfpathlineto{\pgfqpoint{3.602884in}{2.314609in}}%
\pgfpathlineto{\pgfqpoint{3.619441in}{2.250583in}}%
\pgfpathlineto{\pgfqpoint{3.642622in}{2.150352in}}%
\pgfpathlineto{\pgfqpoint{3.665802in}{2.050163in}}%
\pgfpathlineto{\pgfqpoint{3.675736in}{2.013629in}}%
\pgfpathlineto{\pgfqpoint{3.685670in}{1.985363in}}%
\pgfpathlineto{\pgfqpoint{3.692293in}{1.972777in}}%
\pgfpathlineto{\pgfqpoint{3.698916in}{1.966174in}}%
\pgfpathlineto{\pgfqpoint{3.702228in}{1.965288in}}%
\pgfpathlineto{\pgfqpoint{3.705539in}{1.966052in}}%
\pgfpathlineto{\pgfqpoint{3.708851in}{1.968461in}}%
\pgfpathlineto{\pgfqpoint{3.715473in}{1.978047in}}%
\pgfpathlineto{\pgfqpoint{3.722096in}{1.993454in}}%
\pgfpathlineto{\pgfqpoint{3.732031in}{2.025427in}}%
\pgfpathlineto{\pgfqpoint{3.745277in}{2.078796in}}%
\pgfpathlineto{\pgfqpoint{3.785014in}{2.246788in}}%
\pgfpathlineto{\pgfqpoint{3.801571in}{2.305324in}}%
\pgfpathlineto{\pgfqpoint{3.824751in}{2.377348in}}%
\pgfpathlineto{\pgfqpoint{3.851243in}{2.460361in}}%
\pgfpathlineto{\pgfqpoint{3.867800in}{2.520515in}}%
\pgfpathlineto{\pgfqpoint{3.887669in}{2.602537in}}%
\pgfpathlineto{\pgfqpoint{3.917472in}{2.726774in}}%
\pgfpathlineto{\pgfqpoint{3.930718in}{2.771408in}}%
\pgfpathlineto{\pgfqpoint{3.940652in}{2.796872in}}%
\pgfpathlineto{\pgfqpoint{3.947275in}{2.809204in}}%
\pgfpathlineto{\pgfqpoint{3.953898in}{2.817384in}}%
\pgfpathlineto{\pgfqpoint{3.960521in}{2.821094in}}%
\pgfpathlineto{\pgfqpoint{3.963832in}{2.821187in}}%
\pgfpathlineto{\pgfqpoint{3.967144in}{2.820064in}}%
\pgfpathlineto{\pgfqpoint{3.973767in}{2.814074in}}%
\pgfpathlineto{\pgfqpoint{3.980389in}{2.802965in}}%
\pgfpathlineto{\pgfqpoint{3.987012in}{2.786652in}}%
\pgfpathlineto{\pgfqpoint{3.993635in}{2.765148in}}%
\pgfpathlineto{\pgfqpoint{4.003570in}{2.723517in}}%
\pgfpathlineto{\pgfqpoint{4.013504in}{2.671828in}}%
\pgfpathlineto{\pgfqpoint{4.030061in}{2.571100in}}%
\pgfpathlineto{\pgfqpoint{4.046618in}{2.474170in}}%
\pgfpathlineto{\pgfqpoint{4.053241in}{2.445000in}}%
\pgfpathlineto{\pgfqpoint{4.059864in}{2.425701in}}%
\pgfpathlineto{\pgfqpoint{4.063176in}{2.420609in}}%
\pgfpathlineto{\pgfqpoint{4.066487in}{2.418932in}}%
\pgfpathlineto{\pgfqpoint{4.069799in}{2.420875in}}%
\pgfpathlineto{\pgfqpoint{4.073110in}{2.426566in}}%
\pgfpathlineto{\pgfqpoint{4.076421in}{2.436054in}}%
\pgfpathlineto{\pgfqpoint{4.083044in}{2.466183in}}%
\pgfpathlineto{\pgfqpoint{4.089667in}{2.510046in}}%
\pgfpathlineto{\pgfqpoint{4.099602in}{2.596697in}}%
\pgfpathlineto{\pgfqpoint{4.112847in}{2.736724in}}%
\pgfpathlineto{\pgfqpoint{4.142650in}{3.058237in}}%
\pgfpathlineto{\pgfqpoint{4.155896in}{3.171766in}}%
\pgfpathlineto{\pgfqpoint{4.165831in}{3.237131in}}%
\pgfpathlineto{\pgfqpoint{4.172454in}{3.269916in}}%
\pgfpathlineto{\pgfqpoint{4.179076in}{3.293432in}}%
\pgfpathlineto{\pgfqpoint{4.185699in}{3.307278in}}%
\pgfpathlineto{\pgfqpoint{4.189011in}{3.310485in}}%
\pgfpathlineto{\pgfqpoint{4.192322in}{3.311185in}}%
\pgfpathlineto{\pgfqpoint{4.195634in}{3.309368in}}%
\pgfpathlineto{\pgfqpoint{4.198945in}{3.305032in}}%
\pgfpathlineto{\pgfqpoint{4.205568in}{3.288855in}}%
\pgfpathlineto{\pgfqpoint{4.212191in}{3.262873in}}%
\pgfpathlineto{\pgfqpoint{4.218814in}{3.227500in}}%
\pgfpathlineto{\pgfqpoint{4.228748in}{3.158275in}}%
\pgfpathlineto{\pgfqpoint{4.238683in}{3.072464in}}%
\pgfpathlineto{\pgfqpoint{4.255240in}{2.904419in}}%
\pgfpathlineto{\pgfqpoint{4.281731in}{2.628960in}}%
\pgfpathlineto{\pgfqpoint{4.291666in}{2.544042in}}%
\pgfpathlineto{\pgfqpoint{4.301600in}{2.477888in}}%
\pgfpathlineto{\pgfqpoint{4.308223in}{2.445960in}}%
\pgfpathlineto{\pgfqpoint{4.314846in}{2.424228in}}%
\pgfpathlineto{\pgfqpoint{4.321469in}{2.412401in}}%
\pgfpathlineto{\pgfqpoint{4.324780in}{2.409951in}}%
\pgfpathlineto{\pgfqpoint{4.328092in}{2.409610in}}%
\pgfpathlineto{\pgfqpoint{4.331403in}{2.411201in}}%
\pgfpathlineto{\pgfqpoint{4.338026in}{2.419369in}}%
\pgfpathlineto{\pgfqpoint{4.344649in}{2.432756in}}%
\pgfpathlineto{\pgfqpoint{4.354583in}{2.458856in}}%
\pgfpathlineto{\pgfqpoint{4.377763in}{2.522582in}}%
\pgfpathlineto{\pgfqpoint{4.387698in}{2.543210in}}%
\pgfpathlineto{\pgfqpoint{4.397632in}{2.557913in}}%
\pgfpathlineto{\pgfqpoint{4.407566in}{2.567378in}}%
\pgfpathlineto{\pgfqpoint{4.434058in}{2.586861in}}%
\pgfpathlineto{\pgfqpoint{4.440681in}{2.596639in}}%
\pgfpathlineto{\pgfqpoint{4.447304in}{2.610860in}}%
\pgfpathlineto{\pgfqpoint{4.453927in}{2.630549in}}%
\pgfpathlineto{\pgfqpoint{4.460550in}{2.656448in}}%
\pgfpathlineto{\pgfqpoint{4.470484in}{2.707714in}}%
\pgfpathlineto{\pgfqpoint{4.480418in}{2.773597in}}%
\pgfpathlineto{\pgfqpoint{4.493664in}{2.880974in}}%
\pgfpathlineto{\pgfqpoint{4.510221in}{3.036377in}}%
\pgfpathlineto{\pgfqpoint{4.546647in}{3.384861in}}%
\pgfpathlineto{\pgfqpoint{4.559893in}{3.490354in}}%
\pgfpathlineto{\pgfqpoint{4.569828in}{3.556415in}}%
\pgfpathlineto{\pgfqpoint{4.579762in}{3.610045in}}%
\pgfpathlineto{\pgfqpoint{4.589696in}{3.651716in}}%
\pgfpathlineto{\pgfqpoint{4.599631in}{3.683981in}}%
\pgfpathlineto{\pgfqpoint{4.612876in}{3.719153in}}%
\pgfpathlineto{\pgfqpoint{4.622811in}{3.741179in}}%
\pgfpathlineto{\pgfqpoint{4.629434in}{3.751226in}}%
\pgfpathlineto{\pgfqpoint{4.632745in}{3.754007in}}%
\pgfpathlineto{\pgfqpoint{4.636057in}{3.754969in}}%
\pgfpathlineto{\pgfqpoint{4.639368in}{3.753947in}}%
\pgfpathlineto{\pgfqpoint{4.642679in}{3.750858in}}%
\pgfpathlineto{\pgfqpoint{4.649302in}{3.738521in}}%
\pgfpathlineto{\pgfqpoint{4.655925in}{3.718685in}}%
\pgfpathlineto{\pgfqpoint{4.665860in}{3.678307in}}%
\pgfpathlineto{\pgfqpoint{4.698974in}{3.529817in}}%
\pgfpathlineto{\pgfqpoint{4.715531in}{3.473251in}}%
\pgfpathlineto{\pgfqpoint{4.735400in}{3.404719in}}%
\pgfpathlineto{\pgfqpoint{4.781760in}{3.235460in}}%
\pgfpathlineto{\pgfqpoint{4.795006in}{3.197608in}}%
\pgfpathlineto{\pgfqpoint{4.814875in}{3.148347in}}%
\pgfpathlineto{\pgfqpoint{4.834744in}{3.097289in}}%
\pgfpathlineto{\pgfqpoint{4.847989in}{3.056617in}}%
\pgfpathlineto{\pgfqpoint{4.861235in}{3.007751in}}%
\pgfpathlineto{\pgfqpoint{4.874481in}{2.949261in}}%
\pgfpathlineto{\pgfqpoint{4.891038in}{2.864215in}}%
\pgfpathlineto{\pgfqpoint{4.907595in}{2.778592in}}%
\pgfpathlineto{\pgfqpoint{4.917530in}{2.737732in}}%
\pgfpathlineto{\pgfqpoint{4.924153in}{2.718231in}}%
\pgfpathlineto{\pgfqpoint{4.930776in}{2.705658in}}%
\pgfpathlineto{\pgfqpoint{4.937398in}{2.699710in}}%
\pgfpathlineto{\pgfqpoint{4.940710in}{2.698966in}}%
\pgfpathlineto{\pgfqpoint{4.944021in}{2.699540in}}%
\pgfpathlineto{\pgfqpoint{4.950644in}{2.704155in}}%
\pgfpathlineto{\pgfqpoint{4.957267in}{2.712709in}}%
\pgfpathlineto{\pgfqpoint{4.963890in}{2.724635in}}%
\pgfpathlineto{\pgfqpoint{4.973824in}{2.748310in}}%
\pgfpathlineto{\pgfqpoint{4.983759in}{2.778993in}}%
\pgfpathlineto{\pgfqpoint{4.993693in}{2.817315in}}%
\pgfpathlineto{\pgfqpoint{5.003627in}{2.864054in}}%
\pgfpathlineto{\pgfqpoint{5.013562in}{2.919815in}}%
\pgfpathlineto{\pgfqpoint{5.026808in}{3.008670in}}%
\pgfpathlineto{\pgfqpoint{5.040053in}{3.113779in}}%
\pgfpathlineto{\pgfqpoint{5.056611in}{3.265431in}}%
\pgfpathlineto{\pgfqpoint{5.079791in}{3.502855in}}%
\pgfpathlineto{\pgfqpoint{5.112905in}{3.841854in}}%
\pgfpathlineto{\pgfqpoint{5.129463in}{3.988714in}}%
\pgfpathlineto{\pgfqpoint{5.142708in}{4.088374in}}%
\pgfpathlineto{\pgfqpoint{5.155954in}{4.171955in}}%
\pgfpathlineto{\pgfqpoint{5.169200in}{4.242046in}}%
\pgfpathlineto{\pgfqpoint{5.182446in}{4.301479in}}%
\pgfpathlineto{\pgfqpoint{5.195692in}{4.351788in}}%
\pgfpathlineto{\pgfqpoint{5.208937in}{4.393398in}}%
\pgfpathlineto{\pgfqpoint{5.222183in}{4.426556in}}%
\pgfpathlineto{\pgfqpoint{5.232118in}{4.446199in}}%
\pgfpathlineto{\pgfqpoint{5.242052in}{4.461705in}}%
\pgfpathlineto{\pgfqpoint{5.251986in}{4.473363in}}%
\pgfpathlineto{\pgfqpoint{5.261921in}{4.481382in}}%
\pgfpathlineto{\pgfqpoint{5.271855in}{4.485870in}}%
\pgfpathlineto{\pgfqpoint{5.278478in}{4.486919in}}%
\pgfpathlineto{\pgfqpoint{5.285101in}{4.486413in}}%
\pgfpathlineto{\pgfqpoint{5.291724in}{4.484350in}}%
\pgfpathlineto{\pgfqpoint{5.301658in}{4.478348in}}%
\pgfpathlineto{\pgfqpoint{5.311592in}{4.468909in}}%
\pgfpathlineto{\pgfqpoint{5.321527in}{4.456146in}}%
\pgfpathlineto{\pgfqpoint{5.331461in}{4.440235in}}%
\pgfpathlineto{\pgfqpoint{5.344707in}{4.414517in}}%
\pgfpathlineto{\pgfqpoint{5.361264in}{4.376042in}}%
\pgfpathlineto{\pgfqpoint{5.381133in}{4.322613in}}%
\pgfpathlineto{\pgfqpoint{5.444050in}{4.146111in}}%
\pgfpathlineto{\pgfqpoint{5.460608in}{4.109516in}}%
\pgfpathlineto{\pgfqpoint{5.473853in}{4.085345in}}%
\pgfpathlineto{\pgfqpoint{5.487099in}{4.065652in}}%
\pgfpathlineto{\pgfqpoint{5.500345in}{4.049762in}}%
\pgfpathlineto{\pgfqpoint{5.516902in}{4.033536in}}%
\pgfpathlineto{\pgfqpoint{5.569885in}{3.987466in}}%
\pgfpathlineto{\pgfqpoint{5.586443in}{3.975181in}}%
\pgfpathlineto{\pgfqpoint{5.596377in}{3.970185in}}%
\pgfpathlineto{\pgfqpoint{5.606311in}{3.967921in}}%
\pgfpathlineto{\pgfqpoint{5.612934in}{3.968266in}}%
\pgfpathlineto{\pgfqpoint{5.619557in}{3.970278in}}%
\pgfpathlineto{\pgfqpoint{5.626180in}{3.974072in}}%
\pgfpathlineto{\pgfqpoint{5.632803in}{3.979721in}}%
\pgfpathlineto{\pgfqpoint{5.642737in}{3.991743in}}%
\pgfpathlineto{\pgfqpoint{5.652672in}{4.007993in}}%
\pgfpathlineto{\pgfqpoint{5.662606in}{4.028278in}}%
\pgfpathlineto{\pgfqpoint{5.675852in}{4.060900in}}%
\pgfpathlineto{\pgfqpoint{5.692409in}{4.108333in}}%
\pgfpathlineto{\pgfqpoint{5.728835in}{4.216151in}}%
\pgfpathlineto{\pgfqpoint{5.742081in}{4.248618in}}%
\pgfpathlineto{\pgfqpoint{5.755327in}{4.275028in}}%
\pgfpathlineto{\pgfqpoint{5.768572in}{4.295590in}}%
\pgfpathlineto{\pgfqpoint{5.781818in}{4.311375in}}%
\pgfpathlineto{\pgfqpoint{5.795064in}{4.323399in}}%
\pgfpathlineto{\pgfqpoint{5.808310in}{4.332005in}}%
\pgfpathlineto{\pgfqpoint{5.818244in}{4.335995in}}%
\pgfpathlineto{\pgfqpoint{5.828178in}{4.337511in}}%
\pgfpathlineto{\pgfqpoint{5.838113in}{4.336267in}}%
\pgfpathlineto{\pgfqpoint{5.848047in}{4.332187in}}%
\pgfpathlineto{\pgfqpoint{5.857982in}{4.325441in}}%
\pgfpathlineto{\pgfqpoint{5.867916in}{4.316308in}}%
\pgfpathlineto{\pgfqpoint{5.881162in}{4.300735in}}%
\pgfpathlineto{\pgfqpoint{5.894407in}{4.281249in}}%
\pgfpathlineto{\pgfqpoint{5.907653in}{4.257664in}}%
\pgfpathlineto{\pgfqpoint{5.924211in}{4.223097in}}%
\pgfpathlineto{\pgfqpoint{5.963948in}{4.136808in}}%
\pgfpathlineto{\pgfqpoint{5.977194in}{4.115299in}}%
\pgfpathlineto{\pgfqpoint{5.987128in}{4.103500in}}%
\pgfpathlineto{\pgfqpoint{5.997062in}{4.095711in}}%
\pgfpathlineto{\pgfqpoint{6.003685in}{4.092728in}}%
\pgfpathlineto{\pgfqpoint{6.010308in}{4.091426in}}%
\pgfpathlineto{\pgfqpoint{6.020243in}{4.092357in}}%
\pgfpathlineto{\pgfqpoint{6.030177in}{4.096270in}}%
\pgfpathlineto{\pgfqpoint{6.040111in}{4.102527in}}%
\pgfpathlineto{\pgfqpoint{6.056669in}{4.116223in}}%
\pgfpathlineto{\pgfqpoint{6.083160in}{4.138608in}}%
\pgfpathlineto{\pgfqpoint{6.093094in}{4.144534in}}%
\pgfpathlineto{\pgfqpoint{6.103029in}{4.148051in}}%
\pgfpathlineto{\pgfqpoint{6.112963in}{4.148823in}}%
\pgfpathlineto{\pgfqpoint{6.122898in}{4.146888in}}%
\pgfpathlineto{\pgfqpoint{6.132832in}{4.142645in}}%
\pgfpathlineto{\pgfqpoint{6.149389in}{4.132371in}}%
\pgfpathlineto{\pgfqpoint{6.172569in}{4.117929in}}%
\pgfpathlineto{\pgfqpoint{6.185815in}{4.112697in}}%
\pgfpathlineto{\pgfqpoint{6.195749in}{4.110879in}}%
\pgfpathlineto{\pgfqpoint{6.205684in}{4.110974in}}%
\pgfpathlineto{\pgfqpoint{6.215618in}{4.112977in}}%
\pgfpathlineto{\pgfqpoint{6.225552in}{4.116853in}}%
\pgfpathlineto{\pgfqpoint{6.238798in}{4.124858in}}%
\pgfpathlineto{\pgfqpoint{6.252044in}{4.135971in}}%
\pgfpathlineto{\pgfqpoint{6.265290in}{4.149945in}}%
\pgfpathlineto{\pgfqpoint{6.281847in}{4.171103in}}%
\pgfpathlineto{\pgfqpoint{6.298404in}{4.196556in}}%
\pgfpathlineto{\pgfqpoint{6.311650in}{4.220579in}}%
\pgfpathlineto{\pgfqpoint{6.324896in}{4.248384in}}%
\pgfpathlineto{\pgfqpoint{6.341453in}{4.288961in}}%
\pgfpathlineto{\pgfqpoint{6.358010in}{4.336225in}}%
\pgfpathlineto{\pgfqpoint{6.377879in}{4.400570in}}%
\pgfpathlineto{\pgfqpoint{6.410994in}{4.509478in}}%
\pgfpathlineto{\pgfqpoint{6.420928in}{4.535818in}}%
\pgfpathlineto{\pgfqpoint{6.430862in}{4.555590in}}%
\pgfpathlineto{\pgfqpoint{6.437485in}{4.563842in}}%
\pgfpathlineto{\pgfqpoint{6.440797in}{4.566160in}}%
\pgfpathlineto{\pgfqpoint{6.444108in}{4.567110in}}%
\pgfpathlineto{\pgfqpoint{6.447420in}{4.566558in}}%
\pgfpathlineto{\pgfqpoint{6.450731in}{4.564359in}}%
\pgfpathlineto{\pgfqpoint{6.454043in}{4.560361in}}%
\pgfpathlineto{\pgfqpoint{6.460665in}{4.546290in}}%
\pgfpathlineto{\pgfqpoint{6.467288in}{4.522861in}}%
\pgfpathlineto{\pgfqpoint{6.473911in}{4.488372in}}%
\pgfpathlineto{\pgfqpoint{6.480534in}{4.440889in}}%
\pgfpathlineto{\pgfqpoint{6.487157in}{4.378284in}}%
\pgfpathlineto{\pgfqpoint{6.493780in}{4.298346in}}%
\pgfpathlineto{\pgfqpoint{6.503714in}{4.141815in}}%
\pgfpathlineto{\pgfqpoint{6.513649in}{3.940571in}}%
\pgfpathlineto{\pgfqpoint{6.533517in}{3.503372in}}%
\pgfpathlineto{\pgfqpoint{6.540140in}{3.413731in}}%
\pgfpathlineto{\pgfqpoint{6.543452in}{3.388375in}}%
\pgfpathlineto{\pgfqpoint{6.546763in}{3.375013in}}%
\pgfpathlineto{\pgfqpoint{6.550075in}{3.371094in}}%
\pgfpathlineto{\pgfqpoint{6.553386in}{3.373163in}}%
\pgfpathlineto{\pgfqpoint{6.560009in}{3.380745in}}%
\pgfpathlineto{\pgfqpoint{6.563320in}{3.380232in}}%
\pgfpathlineto{\pgfqpoint{6.566632in}{3.374205in}}%
\pgfpathlineto{\pgfqpoint{6.569943in}{3.361753in}}%
\pgfpathlineto{\pgfqpoint{6.576566in}{3.317190in}}%
\pgfpathlineto{\pgfqpoint{6.583189in}{3.249688in}}%
\pgfpathlineto{\pgfqpoint{6.593123in}{3.119282in}}%
\pgfpathlineto{\pgfqpoint{6.619615in}{2.752061in}}%
\pgfpathlineto{\pgfqpoint{6.626238in}{2.684633in}}%
\pgfpathlineto{\pgfqpoint{6.632861in}{2.634764in}}%
\pgfpathlineto{\pgfqpoint{6.639484in}{2.604837in}}%
\pgfpathlineto{\pgfqpoint{6.642795in}{2.597886in}}%
\pgfpathlineto{\pgfqpoint{6.646107in}{2.596486in}}%
\pgfpathlineto{\pgfqpoint{6.649418in}{2.600759in}}%
\pgfpathlineto{\pgfqpoint{6.652729in}{2.610803in}}%
\pgfpathlineto{\pgfqpoint{6.656041in}{2.626699in}}%
\pgfpathlineto{\pgfqpoint{6.662664in}{2.676281in}}%
\pgfpathlineto{\pgfqpoint{6.669287in}{2.749759in}}%
\pgfpathlineto{\pgfqpoint{6.675910in}{2.846972in}}%
\pgfpathlineto{\pgfqpoint{6.685844in}{3.035606in}}%
\pgfpathlineto{\pgfqpoint{6.695778in}{3.271014in}}%
\pgfpathlineto{\pgfqpoint{6.709024in}{3.643595in}}%
\pgfpathlineto{\pgfqpoint{6.735516in}{4.484024in}}%
\pgfpathlineto{\pgfqpoint{6.752073in}{4.972896in}}%
\pgfpathlineto{\pgfqpoint{6.765319in}{5.295574in}}%
\pgfpathlineto{\pgfqpoint{6.775253in}{5.486330in}}%
\pgfpathlineto{\pgfqpoint{6.781876in}{5.585613in}}%
\pgfpathlineto{\pgfqpoint{6.788499in}{5.660997in}}%
\pgfpathlineto{\pgfqpoint{6.795122in}{5.712273in}}%
\pgfpathlineto{\pgfqpoint{6.801745in}{5.740434in}}%
\pgfpathlineto{\pgfqpoint{6.805056in}{5.746417in}}%
\pgfpathlineto{\pgfqpoint{6.808368in}{5.747381in}}%
\pgfpathlineto{\pgfqpoint{6.811679in}{5.743634in}}%
\pgfpathlineto{\pgfqpoint{6.814991in}{5.735492in}}%
\pgfpathlineto{\pgfqpoint{6.821613in}{5.707290in}}%
\pgfpathlineto{\pgfqpoint{6.828236in}{5.665290in}}%
\pgfpathlineto{\pgfqpoint{6.838171in}{5.581877in}}%
\pgfpathlineto{\pgfqpoint{6.851416in}{5.445788in}}%
\pgfpathlineto{\pgfqpoint{6.881220in}{5.128524in}}%
\pgfpathlineto{\pgfqpoint{6.894465in}{5.011899in}}%
\pgfpathlineto{\pgfqpoint{6.907711in}{4.913440in}}%
\pgfpathlineto{\pgfqpoint{6.920957in}{4.829138in}}%
\pgfpathlineto{\pgfqpoint{6.934203in}{4.756774in}}%
\pgfpathlineto{\pgfqpoint{6.947449in}{4.697726in}}%
\pgfpathlineto{\pgfqpoint{6.957383in}{4.664039in}}%
\pgfpathlineto{\pgfqpoint{6.964006in}{4.647121in}}%
\pgfpathlineto{\pgfqpoint{6.970629in}{4.634527in}}%
\pgfpathlineto{\pgfqpoint{6.977252in}{4.625721in}}%
\pgfpathlineto{\pgfqpoint{6.987186in}{4.617274in}}%
\pgfpathlineto{\pgfqpoint{6.997120in}{4.609599in}}%
\pgfpathlineto{\pgfqpoint{7.003743in}{4.601538in}}%
\pgfpathlineto{\pgfqpoint{7.010366in}{4.588826in}}%
\pgfpathlineto{\pgfqpoint{7.016989in}{4.570061in}}%
\pgfpathlineto{\pgfqpoint{7.023612in}{4.544626in}}%
\pgfpathlineto{\pgfqpoint{7.033546in}{4.494270in}}%
\pgfpathlineto{\pgfqpoint{7.043481in}{4.430505in}}%
\pgfpathlineto{\pgfqpoint{7.053415in}{4.353578in}}%
\pgfpathlineto{\pgfqpoint{7.063349in}{4.262072in}}%
\pgfpathlineto{\pgfqpoint{7.076595in}{4.118168in}}%
\pgfpathlineto{\pgfqpoint{7.103087in}{3.815838in}}%
\pgfpathlineto{\pgfqpoint{7.113021in}{3.724929in}}%
\pgfpathlineto{\pgfqpoint{7.119644in}{3.677184in}}%
\pgfpathlineto{\pgfqpoint{7.126267in}{3.641784in}}%
\pgfpathlineto{\pgfqpoint{7.132890in}{3.620393in}}%
\pgfpathlineto{\pgfqpoint{7.136201in}{3.615473in}}%
\pgfpathlineto{\pgfqpoint{7.139513in}{3.614682in}}%
\pgfpathlineto{\pgfqpoint{7.142824in}{3.618221in}}%
\pgfpathlineto{\pgfqpoint{7.146136in}{3.626280in}}%
\pgfpathlineto{\pgfqpoint{7.149447in}{3.639018in}}%
\pgfpathlineto{\pgfqpoint{7.156070in}{3.678963in}}%
\pgfpathlineto{\pgfqpoint{7.162693in}{3.738296in}}%
\pgfpathlineto{\pgfqpoint{7.169316in}{3.816041in}}%
\pgfpathlineto{\pgfqpoint{7.179250in}{3.961846in}}%
\pgfpathlineto{\pgfqpoint{7.195807in}{4.251762in}}%
\pgfpathlineto{\pgfqpoint{7.218987in}{4.660599in}}%
\pgfpathlineto{\pgfqpoint{7.232233in}{4.862925in}}%
\pgfpathlineto{\pgfqpoint{7.245479in}{5.031385in}}%
\pgfpathlineto{\pgfqpoint{7.255413in}{5.126371in}}%
\pgfpathlineto{\pgfqpoint{7.262036in}{5.170080in}}%
\pgfpathlineto{\pgfqpoint{7.268659in}{5.196484in}}%
\pgfpathlineto{\pgfqpoint{7.271971in}{5.203331in}}%
\pgfpathlineto{\pgfqpoint{7.275282in}{5.206264in}}%
\pgfpathlineto{\pgfqpoint{7.278594in}{5.205649in}}%
\pgfpathlineto{\pgfqpoint{7.281905in}{5.201916in}}%
\pgfpathlineto{\pgfqpoint{7.288528in}{5.186854in}}%
\pgfpathlineto{\pgfqpoint{7.295151in}{5.164132in}}%
\pgfpathlineto{\pgfqpoint{7.305085in}{5.119041in}}%
\pgfpathlineto{\pgfqpoint{7.315019in}{5.060781in}}%
\pgfpathlineto{\pgfqpoint{7.324954in}{4.989003in}}%
\pgfpathlineto{\pgfqpoint{7.351445in}{4.787041in}}%
\pgfpathlineto{\pgfqpoint{7.358068in}{4.755093in}}%
\pgfpathlineto{\pgfqpoint{7.364691in}{4.736670in}}%
\pgfpathlineto{\pgfqpoint{7.368003in}{4.732575in}}%
\pgfpathlineto{\pgfqpoint{7.371314in}{4.731643in}}%
\pgfpathlineto{\pgfqpoint{7.374626in}{4.733555in}}%
\pgfpathlineto{\pgfqpoint{7.377937in}{4.737924in}}%
\pgfpathlineto{\pgfqpoint{7.384560in}{4.752328in}}%
\pgfpathlineto{\pgfqpoint{7.394494in}{4.781895in}}%
\pgfpathlineto{\pgfqpoint{7.407740in}{4.822047in}}%
\pgfpathlineto{\pgfqpoint{7.414363in}{4.838035in}}%
\pgfpathlineto{\pgfqpoint{7.420986in}{4.849466in}}%
\pgfpathlineto{\pgfqpoint{7.427609in}{4.855380in}}%
\pgfpathlineto{\pgfqpoint{7.430920in}{4.856051in}}%
\pgfpathlineto{\pgfqpoint{7.434232in}{4.855115in}}%
\pgfpathlineto{\pgfqpoint{7.437543in}{4.852532in}}%
\pgfpathlineto{\pgfqpoint{7.444166in}{4.842317in}}%
\pgfpathlineto{\pgfqpoint{7.450789in}{4.825306in}}%
\pgfpathlineto{\pgfqpoint{7.457412in}{4.801539in}}%
\pgfpathlineto{\pgfqpoint{7.467346in}{4.753652in}}%
\pgfpathlineto{\pgfqpoint{7.477281in}{4.692296in}}%
\pgfpathlineto{\pgfqpoint{7.490526in}{4.593571in}}%
\pgfpathlineto{\pgfqpoint{7.510395in}{4.423971in}}%
\pgfpathlineto{\pgfqpoint{7.540198in}{4.150069in}}%
\pgfpathlineto{\pgfqpoint{7.563378in}{3.916232in}}%
\pgfpathlineto{\pgfqpoint{7.586558in}{3.656791in}}%
\pgfpathlineto{\pgfqpoint{7.616361in}{3.294924in}}%
\pgfpathlineto{\pgfqpoint{7.656099in}{2.812537in}}%
\pgfpathlineto{\pgfqpoint{7.672656in}{2.634208in}}%
\pgfpathlineto{\pgfqpoint{7.685902in}{2.510865in}}%
\pgfpathlineto{\pgfqpoint{7.699148in}{2.408927in}}%
\pgfpathlineto{\pgfqpoint{7.709082in}{2.347356in}}%
\pgfpathlineto{\pgfqpoint{7.719016in}{2.298068in}}%
\pgfpathlineto{\pgfqpoint{7.728951in}{2.259976in}}%
\pgfpathlineto{\pgfqpoint{7.738885in}{2.231702in}}%
\pgfpathlineto{\pgfqpoint{7.748819in}{2.211792in}}%
\pgfpathlineto{\pgfqpoint{7.755442in}{2.202454in}}%
\pgfpathlineto{\pgfqpoint{7.762065in}{2.195779in}}%
\pgfpathlineto{\pgfqpoint{7.768688in}{2.191349in}}%
\pgfpathlineto{\pgfqpoint{7.778622in}{2.188006in}}%
\pgfpathlineto{\pgfqpoint{7.788557in}{2.187380in}}%
\pgfpathlineto{\pgfqpoint{7.834917in}{2.188690in}}%
\pgfpathlineto{\pgfqpoint{7.858097in}{2.187017in}}%
\pgfpathlineto{\pgfqpoint{7.871343in}{2.188821in}}%
\pgfpathlineto{\pgfqpoint{7.881277in}{2.192241in}}%
\pgfpathlineto{\pgfqpoint{7.894523in}{2.199181in}}%
\pgfpathlineto{\pgfqpoint{7.924326in}{2.216078in}}%
\pgfpathlineto{\pgfqpoint{7.934261in}{2.219258in}}%
\pgfpathlineto{\pgfqpoint{7.944195in}{2.220506in}}%
\pgfpathlineto{\pgfqpoint{7.957441in}{2.219440in}}%
\pgfpathlineto{\pgfqpoint{7.993867in}{2.213500in}}%
\pgfpathlineto{\pgfqpoint{8.003801in}{2.216008in}}%
\pgfpathlineto{\pgfqpoint{8.010424in}{2.219741in}}%
\pgfpathlineto{\pgfqpoint{8.017047in}{2.225417in}}%
\pgfpathlineto{\pgfqpoint{8.023670in}{2.233218in}}%
\pgfpathlineto{\pgfqpoint{8.033604in}{2.249127in}}%
\pgfpathlineto{\pgfqpoint{8.043538in}{2.270152in}}%
\pgfpathlineto{\pgfqpoint{8.053473in}{2.296111in}}%
\pgfpathlineto{\pgfqpoint{8.066719in}{2.337669in}}%
\pgfpathlineto{\pgfqpoint{8.083276in}{2.398390in}}%
\pgfpathlineto{\pgfqpoint{8.129636in}{2.575647in}}%
\pgfpathlineto{\pgfqpoint{8.146193in}{2.628572in}}%
\pgfpathlineto{\pgfqpoint{8.166062in}{2.684812in}}%
\pgfpathlineto{\pgfqpoint{8.185931in}{2.740304in}}%
\pgfpathlineto{\pgfqpoint{8.195865in}{2.776186in}}%
\pgfpathlineto{\pgfqpoint{8.202488in}{2.809767in}}%
\pgfpathlineto{\pgfqpoint{8.209111in}{2.856453in}}%
\pgfpathlineto{\pgfqpoint{8.215734in}{2.919350in}}%
\pgfpathlineto{\pgfqpoint{8.225668in}{3.039013in}}%
\pgfpathlineto{\pgfqpoint{8.238914in}{3.202019in}}%
\pgfpathlineto{\pgfqpoint{8.245537in}{3.265804in}}%
\pgfpathlineto{\pgfqpoint{8.252160in}{3.311457in}}%
\pgfpathlineto{\pgfqpoint{8.258783in}{3.337466in}}%
\pgfpathlineto{\pgfqpoint{8.262094in}{3.343024in}}%
\pgfpathlineto{\pgfqpoint{8.265406in}{3.343670in}}%
\pgfpathlineto{\pgfqpoint{8.268717in}{3.339471in}}%
\pgfpathlineto{\pgfqpoint{8.272028in}{3.330509in}}%
\pgfpathlineto{\pgfqpoint{8.278651in}{3.298646in}}%
\pgfpathlineto{\pgfqpoint{8.285274in}{3.248803in}}%
\pgfpathlineto{\pgfqpoint{8.291897in}{3.181695in}}%
\pgfpathlineto{\pgfqpoint{8.301832in}{3.050172in}}%
\pgfpathlineto{\pgfqpoint{8.311766in}{2.883645in}}%
\pgfpathlineto{\pgfqpoint{8.325012in}{2.611509in}}%
\pgfpathlineto{\pgfqpoint{8.338257in}{2.290058in}}%
\pgfpathlineto{\pgfqpoint{8.361438in}{1.708757in}}%
\pgfpathlineto{\pgfqpoint{8.364749in}{1.671859in}}%
\pgfpathlineto{\pgfqpoint{8.368061in}{1.677637in}}%
\pgfpathlineto{\pgfqpoint{8.371372in}{1.724634in}}%
\pgfpathlineto{\pgfqpoint{8.377995in}{1.884925in}}%
\pgfpathlineto{\pgfqpoint{8.404486in}{2.616831in}}%
\pgfpathlineto{\pgfqpoint{8.414421in}{2.812396in}}%
\pgfpathlineto{\pgfqpoint{8.421044in}{2.902465in}}%
\pgfpathlineto{\pgfqpoint{8.427667in}{2.959480in}}%
\pgfpathlineto{\pgfqpoint{8.430978in}{2.976239in}}%
\pgfpathlineto{\pgfqpoint{8.434290in}{2.985708in}}%
\pgfpathlineto{\pgfqpoint{8.437601in}{2.988333in}}%
\pgfpathlineto{\pgfqpoint{8.440912in}{2.984545in}}%
\pgfpathlineto{\pgfqpoint{8.444224in}{2.974745in}}%
\pgfpathlineto{\pgfqpoint{8.447535in}{2.959291in}}%
\pgfpathlineto{\pgfqpoint{8.454158in}{2.912685in}}%
\pgfpathlineto{\pgfqpoint{8.460781in}{2.847051in}}%
\pgfpathlineto{\pgfqpoint{8.470715in}{2.717762in}}%
\pgfpathlineto{\pgfqpoint{8.480650in}{2.558491in}}%
\pgfpathlineto{\pgfqpoint{8.497207in}{2.250150in}}%
\pgfpathlineto{\pgfqpoint{8.520387in}{1.815050in}}%
\pgfpathlineto{\pgfqpoint{8.527010in}{1.720165in}}%
\pgfpathlineto{\pgfqpoint{8.530322in}{1.685724in}}%
\pgfpathlineto{\pgfqpoint{8.533633in}{1.663749in}}%
\pgfpathlineto{\pgfqpoint{8.536944in}{1.656394in}}%
\pgfpathlineto{\pgfqpoint{8.540256in}{1.663539in}}%
\pgfpathlineto{\pgfqpoint{8.543567in}{1.682555in}}%
\pgfpathlineto{\pgfqpoint{8.550190in}{1.741829in}}%
\pgfpathlineto{\pgfqpoint{8.573370in}{1.972162in}}%
\pgfpathlineto{\pgfqpoint{8.583305in}{2.042144in}}%
\pgfpathlineto{\pgfqpoint{8.589928in}{2.075372in}}%
\pgfpathlineto{\pgfqpoint{8.596551in}{2.098106in}}%
\pgfpathlineto{\pgfqpoint{8.603173in}{2.111844in}}%
\pgfpathlineto{\pgfqpoint{8.609796in}{2.119321in}}%
\pgfpathlineto{\pgfqpoint{8.626354in}{2.132676in}}%
\pgfpathlineto{\pgfqpoint{8.632977in}{2.141596in}}%
\pgfpathlineto{\pgfqpoint{8.652845in}{2.172895in}}%
\pgfpathlineto{\pgfqpoint{8.656157in}{2.175371in}}%
\pgfpathlineto{\pgfqpoint{8.659468in}{2.176039in}}%
\pgfpathlineto{\pgfqpoint{8.662780in}{2.174591in}}%
\pgfpathlineto{\pgfqpoint{8.666091in}{2.170803in}}%
\pgfpathlineto{\pgfqpoint{8.669402in}{2.164528in}}%
\pgfpathlineto{\pgfqpoint{8.676025in}{2.144296in}}%
\pgfpathlineto{\pgfqpoint{8.682648in}{2.113997in}}%
\pgfpathlineto{\pgfqpoint{8.689271in}{2.074343in}}%
\pgfpathlineto{\pgfqpoint{8.699206in}{1.999783in}}%
\pgfpathlineto{\pgfqpoint{8.712451in}{1.880713in}}%
\pgfpathlineto{\pgfqpoint{8.725697in}{1.760358in}}%
\pgfpathlineto{\pgfqpoint{8.732320in}{1.714352in}}%
\pgfpathlineto{\pgfqpoint{8.735631in}{1.699219in}}%
\pgfpathlineto{\pgfqpoint{8.738943in}{1.691137in}}%
\pgfpathlineto{\pgfqpoint{8.742254in}{1.690946in}}%
\pgfpathlineto{\pgfqpoint{8.745566in}{1.698689in}}%
\pgfpathlineto{\pgfqpoint{8.748877in}{1.713602in}}%
\pgfpathlineto{\pgfqpoint{8.755500in}{1.759567in}}%
\pgfpathlineto{\pgfqpoint{8.768746in}{1.880113in}}%
\pgfpathlineto{\pgfqpoint{8.781992in}{1.997244in}}%
\pgfpathlineto{\pgfqpoint{8.791926in}{2.069007in}}%
\pgfpathlineto{\pgfqpoint{8.801860in}{2.122766in}}%
\pgfpathlineto{\pgfqpoint{8.808483in}{2.147533in}}%
\pgfpathlineto{\pgfqpoint{8.815106in}{2.162424in}}%
\pgfpathlineto{\pgfqpoint{8.818418in}{2.165763in}}%
\pgfpathlineto{\pgfqpoint{8.821729in}{2.166118in}}%
\pgfpathlineto{\pgfqpoint{8.825041in}{2.163324in}}%
\pgfpathlineto{\pgfqpoint{8.828352in}{2.157276in}}%
\pgfpathlineto{\pgfqpoint{8.834975in}{2.135523in}}%
\pgfpathlineto{\pgfqpoint{8.841598in}{2.102552in}}%
\pgfpathlineto{\pgfqpoint{8.861467in}{1.987229in}}%
\pgfpathlineto{\pgfqpoint{8.868089in}{1.963897in}}%
\pgfpathlineto{\pgfqpoint{8.871401in}{1.957479in}}%
\pgfpathlineto{\pgfqpoint{8.874712in}{1.954701in}}%
\pgfpathlineto{\pgfqpoint{8.878024in}{1.955449in}}%
\pgfpathlineto{\pgfqpoint{8.881335in}{1.959471in}}%
\pgfpathlineto{\pgfqpoint{8.887958in}{1.975879in}}%
\pgfpathlineto{\pgfqpoint{8.894581in}{2.000632in}}%
\pgfpathlineto{\pgfqpoint{8.907827in}{2.062468in}}%
\pgfpathlineto{\pgfqpoint{8.924384in}{2.138444in}}%
\pgfpathlineto{\pgfqpoint{8.934318in}{2.173606in}}%
\pgfpathlineto{\pgfqpoint{8.940941in}{2.190324in}}%
\pgfpathlineto{\pgfqpoint{8.947564in}{2.200639in}}%
\pgfpathlineto{\pgfqpoint{8.950876in}{2.203165in}}%
\pgfpathlineto{\pgfqpoint{8.954187in}{2.203835in}}%
\pgfpathlineto{\pgfqpoint{8.957499in}{2.202582in}}%
\pgfpathlineto{\pgfqpoint{8.960810in}{2.199353in}}%
\pgfpathlineto{\pgfqpoint{8.967433in}{2.186809in}}%
\pgfpathlineto{\pgfqpoint{8.974056in}{2.166033in}}%
\pgfpathlineto{\pgfqpoint{8.980679in}{2.137140in}}%
\pgfpathlineto{\pgfqpoint{8.990613in}{2.079804in}}%
\pgfpathlineto{\pgfqpoint{9.003859in}{1.984935in}}%
\pgfpathlineto{\pgfqpoint{9.013793in}{1.914439in}}%
\pgfpathlineto{\pgfqpoint{9.020416in}{1.879953in}}%
\pgfpathlineto{\pgfqpoint{9.023728in}{1.870132in}}%
\pgfpathlineto{\pgfqpoint{9.027039in}{1.867025in}}%
\pgfpathlineto{\pgfqpoint{9.030351in}{1.871708in}}%
\pgfpathlineto{\pgfqpoint{9.033662in}{1.884815in}}%
\pgfpathlineto{\pgfqpoint{9.036973in}{1.906422in}}%
\pgfpathlineto{\pgfqpoint{9.043596in}{1.972793in}}%
\pgfpathlineto{\pgfqpoint{9.053531in}{2.113161in}}%
\pgfpathlineto{\pgfqpoint{9.073399in}{2.414837in}}%
\pgfpathlineto{\pgfqpoint{9.083334in}{2.531393in}}%
\pgfpathlineto{\pgfqpoint{9.093268in}{2.619885in}}%
\pgfpathlineto{\pgfqpoint{9.103202in}{2.684137in}}%
\pgfpathlineto{\pgfqpoint{9.109825in}{2.714579in}}%
\pgfpathlineto{\pgfqpoint{9.116448in}{2.735415in}}%
\pgfpathlineto{\pgfqpoint{9.123071in}{2.746987in}}%
\pgfpathlineto{\pgfqpoint{9.126383in}{2.749490in}}%
\pgfpathlineto{\pgfqpoint{9.129694in}{2.749952in}}%
\pgfpathlineto{\pgfqpoint{9.133005in}{2.748515in}}%
\pgfpathlineto{\pgfqpoint{9.136317in}{2.745347in}}%
\pgfpathlineto{\pgfqpoint{9.142940in}{2.734580in}}%
\pgfpathlineto{\pgfqpoint{9.152874in}{2.710651in}}%
\pgfpathlineto{\pgfqpoint{9.172743in}{2.658824in}}%
\pgfpathlineto{\pgfqpoint{9.179366in}{2.647305in}}%
\pgfpathlineto{\pgfqpoint{9.185989in}{2.641220in}}%
\pgfpathlineto{\pgfqpoint{9.189300in}{2.640518in}}%
\pgfpathlineto{\pgfqpoint{9.192612in}{2.641472in}}%
\pgfpathlineto{\pgfqpoint{9.195923in}{2.644107in}}%
\pgfpathlineto{\pgfqpoint{9.202546in}{2.654345in}}%
\pgfpathlineto{\pgfqpoint{9.209169in}{2.670749in}}%
\pgfpathlineto{\pgfqpoint{9.219103in}{2.704727in}}%
\pgfpathlineto{\pgfqpoint{9.235660in}{2.774333in}}%
\pgfpathlineto{\pgfqpoint{9.248906in}{2.827424in}}%
\pgfpathlineto{\pgfqpoint{9.258841in}{2.858712in}}%
\pgfpathlineto{\pgfqpoint{9.265463in}{2.873913in}}%
\pgfpathlineto{\pgfqpoint{9.272086in}{2.884092in}}%
\pgfpathlineto{\pgfqpoint{9.278709in}{2.889146in}}%
\pgfpathlineto{\pgfqpoint{9.282021in}{2.889784in}}%
\pgfpathlineto{\pgfqpoint{9.285332in}{2.889205in}}%
\pgfpathlineto{\pgfqpoint{9.291955in}{2.884575in}}%
\pgfpathlineto{\pgfqpoint{9.298578in}{2.875658in}}%
\pgfpathlineto{\pgfqpoint{9.305201in}{2.862883in}}%
\pgfpathlineto{\pgfqpoint{9.315135in}{2.837400in}}%
\pgfpathlineto{\pgfqpoint{9.325070in}{2.805553in}}%
\pgfpathlineto{\pgfqpoint{9.341627in}{2.742865in}}%
\pgfpathlineto{\pgfqpoint{9.361495in}{2.668177in}}%
\pgfpathlineto{\pgfqpoint{9.368118in}{2.649320in}}%
\pgfpathlineto{\pgfqpoint{9.374741in}{2.636563in}}%
\pgfpathlineto{\pgfqpoint{9.378053in}{2.633118in}}%
\pgfpathlineto{\pgfqpoint{9.381364in}{2.631969in}}%
\pgfpathlineto{\pgfqpoint{9.384676in}{2.633358in}}%
\pgfpathlineto{\pgfqpoint{9.387987in}{2.637501in}}%
\pgfpathlineto{\pgfqpoint{9.391299in}{2.644584in}}%
\pgfpathlineto{\pgfqpoint{9.397921in}{2.668074in}}%
\pgfpathlineto{\pgfqpoint{9.404544in}{2.704241in}}%
\pgfpathlineto{\pgfqpoint{9.411167in}{2.752388in}}%
\pgfpathlineto{\pgfqpoint{9.421102in}{2.842730in}}%
\pgfpathlineto{\pgfqpoint{9.457528in}{3.204321in}}%
\pgfpathlineto{\pgfqpoint{9.467462in}{3.275706in}}%
\pgfpathlineto{\pgfqpoint{9.477396in}{3.329090in}}%
\pgfpathlineto{\pgfqpoint{9.484019in}{3.355096in}}%
\pgfpathlineto{\pgfqpoint{9.490642in}{3.374303in}}%
\pgfpathlineto{\pgfqpoint{9.497265in}{3.387734in}}%
\pgfpathlineto{\pgfqpoint{9.503888in}{3.396482in}}%
\pgfpathlineto{\pgfqpoint{9.510511in}{3.401551in}}%
\pgfpathlineto{\pgfqpoint{9.517134in}{3.403801in}}%
\pgfpathlineto{\pgfqpoint{9.527068in}{3.403700in}}%
\pgfpathlineto{\pgfqpoint{9.540314in}{3.402586in}}%
\pgfpathlineto{\pgfqpoint{9.546937in}{3.404190in}}%
\pgfpathlineto{\pgfqpoint{9.553560in}{3.408191in}}%
\pgfpathlineto{\pgfqpoint{9.560182in}{3.414582in}}%
\pgfpathlineto{\pgfqpoint{9.570117in}{3.427802in}}%
\pgfpathlineto{\pgfqpoint{9.580051in}{3.445268in}}%
\pgfpathlineto{\pgfqpoint{9.586674in}{3.461514in}}%
\pgfpathlineto{\pgfqpoint{9.593297in}{3.486194in}}%
\pgfpathlineto{\pgfqpoint{9.599920in}{3.527403in}}%
\pgfpathlineto{\pgfqpoint{9.606543in}{3.597173in}}%
\pgfpathlineto{\pgfqpoint{9.613166in}{3.707222in}}%
\pgfpathlineto{\pgfqpoint{9.623100in}{3.944123in}}%
\pgfpathlineto{\pgfqpoint{9.633034in}{4.192520in}}%
\pgfpathlineto{\pgfqpoint{9.639657in}{4.311069in}}%
\pgfpathlineto{\pgfqpoint{9.642969in}{4.349478in}}%
\pgfpathlineto{\pgfqpoint{9.646280in}{4.373777in}}%
\pgfpathlineto{\pgfqpoint{9.649592in}{4.384974in}}%
\pgfpathlineto{\pgfqpoint{9.652903in}{4.384648in}}%
\pgfpathlineto{\pgfqpoint{9.656215in}{4.374663in}}%
\pgfpathlineto{\pgfqpoint{9.662837in}{4.333393in}}%
\pgfpathlineto{\pgfqpoint{9.672772in}{4.243501in}}%
\pgfpathlineto{\pgfqpoint{9.686018in}{4.121191in}}%
\pgfpathlineto{\pgfqpoint{9.695952in}{4.050864in}}%
\pgfpathlineto{\pgfqpoint{9.702575in}{4.018805in}}%
\pgfpathlineto{\pgfqpoint{9.709198in}{3.999655in}}%
\pgfpathlineto{\pgfqpoint{9.712509in}{3.994982in}}%
\pgfpathlineto{\pgfqpoint{9.715821in}{3.993548in}}%
\pgfpathlineto{\pgfqpoint{9.719132in}{3.995305in}}%
\pgfpathlineto{\pgfqpoint{9.722444in}{4.000183in}}%
\pgfpathlineto{\pgfqpoint{9.725755in}{4.008097in}}%
\pgfpathlineto{\pgfqpoint{9.732378in}{4.032630in}}%
\pgfpathlineto{\pgfqpoint{9.739001in}{4.067970in}}%
\pgfpathlineto{\pgfqpoint{9.748935in}{4.138754in}}%
\pgfpathlineto{\pgfqpoint{9.758869in}{4.226720in}}%
\pgfpathlineto{\pgfqpoint{9.778738in}{4.429922in}}%
\pgfpathlineto{\pgfqpoint{9.795295in}{4.591710in}}%
\pgfpathlineto{\pgfqpoint{9.811853in}{4.727399in}}%
\pgfpathlineto{\pgfqpoint{9.821787in}{4.815011in}}%
\pgfpathlineto{\pgfqpoint{9.831721in}{4.928526in}}%
\pgfpathlineto{\pgfqpoint{9.844967in}{5.119901in}}%
\pgfpathlineto{\pgfqpoint{9.858213in}{5.305406in}}%
\pgfpathlineto{\pgfqpoint{9.864836in}{5.376904in}}%
\pgfpathlineto{\pgfqpoint{9.871459in}{5.430544in}}%
\pgfpathlineto{\pgfqpoint{9.878082in}{5.467351in}}%
\pgfpathlineto{\pgfqpoint{9.884705in}{5.489444in}}%
\pgfpathlineto{\pgfqpoint{9.888016in}{5.495644in}}%
\pgfpathlineto{\pgfqpoint{9.891327in}{5.498925in}}%
\pgfpathlineto{\pgfqpoint{9.894639in}{5.499483in}}%
\pgfpathlineto{\pgfqpoint{9.897950in}{5.497494in}}%
\pgfpathlineto{\pgfqpoint{9.901262in}{5.493112in}}%
\pgfpathlineto{\pgfqpoint{9.907885in}{5.477727in}}%
\pgfpathlineto{\pgfqpoint{9.914508in}{5.454382in}}%
\pgfpathlineto{\pgfqpoint{9.924442in}{5.406744in}}%
\pgfpathlineto{\pgfqpoint{9.937688in}{5.326217in}}%
\pgfpathlineto{\pgfqpoint{9.970802in}{5.100205in}}%
\pgfpathlineto{\pgfqpoint{9.980737in}{5.021370in}}%
\pgfpathlineto{\pgfqpoint{9.990671in}{4.925284in}}%
\pgfpathlineto{\pgfqpoint{10.013851in}{4.680397in}}%
\pgfpathlineto{\pgfqpoint{10.020474in}{4.636519in}}%
\pgfpathlineto{\pgfqpoint{10.027097in}{4.611496in}}%
\pgfpathlineto{\pgfqpoint{10.030408in}{4.604901in}}%
\pgfpathlineto{\pgfqpoint{10.033720in}{4.601153in}}%
\pgfpathlineto{\pgfqpoint{10.040343in}{4.598658in}}%
\pgfpathlineto{\pgfqpoint{10.056900in}{4.598131in}}%
\pgfpathlineto{\pgfqpoint{10.063523in}{4.600636in}}%
\pgfpathlineto{\pgfqpoint{10.066834in}{4.603830in}}%
\pgfpathlineto{\pgfqpoint{10.070146in}{4.608822in}}%
\pgfpathlineto{\pgfqpoint{10.076769in}{4.625356in}}%
\pgfpathlineto{\pgfqpoint{10.083392in}{4.651782in}}%
\pgfpathlineto{\pgfqpoint{10.090014in}{4.688422in}}%
\pgfpathlineto{\pgfqpoint{10.099949in}{4.760728in}}%
\pgfpathlineto{\pgfqpoint{10.113195in}{4.880835in}}%
\pgfpathlineto{\pgfqpoint{10.146309in}{5.217681in}}%
\pgfpathlineto{\pgfqpoint{10.195981in}{5.724866in}}%
\pgfpathlineto{\pgfqpoint{10.205915in}{5.805278in}}%
\pgfpathlineto{\pgfqpoint{10.212538in}{5.848260in}}%
\pgfpathlineto{\pgfqpoint{10.219161in}{5.880260in}}%
\pgfpathlineto{\pgfqpoint{10.225784in}{5.900346in}}%
\pgfpathlineto{\pgfqpoint{10.229095in}{5.905933in}}%
\pgfpathlineto{\pgfqpoint{10.232407in}{5.908654in}}%
\pgfpathlineto{\pgfqpoint{10.235718in}{5.908625in}}%
\pgfpathlineto{\pgfqpoint{10.239030in}{5.905977in}}%
\pgfpathlineto{\pgfqpoint{10.242341in}{5.900844in}}%
\pgfpathlineto{\pgfqpoint{10.248964in}{5.883662in}}%
\pgfpathlineto{\pgfqpoint{10.255587in}{5.858129in}}%
\pgfpathlineto{\pgfqpoint{10.265521in}{5.806374in}}%
\pgfpathlineto{\pgfqpoint{10.275456in}{5.741354in}}%
\pgfpathlineto{\pgfqpoint{10.288701in}{5.639118in}}%
\pgfpathlineto{\pgfqpoint{10.308570in}{5.463569in}}%
\pgfpathlineto{\pgfqpoint{10.328439in}{5.267266in}}%
\pgfpathlineto{\pgfqpoint{10.348308in}{5.044099in}}%
\pgfpathlineto{\pgfqpoint{10.368176in}{4.826124in}}%
\pgfpathlineto{\pgfqpoint{10.381422in}{4.704327in}}%
\pgfpathlineto{\pgfqpoint{10.394668in}{4.605027in}}%
\pgfpathlineto{\pgfqpoint{10.404602in}{4.546257in}}%
\pgfpathlineto{\pgfqpoint{10.414537in}{4.502359in}}%
\pgfpathlineto{\pgfqpoint{10.421159in}{4.482114in}}%
\pgfpathlineto{\pgfqpoint{10.427782in}{4.469511in}}%
\pgfpathlineto{\pgfqpoint{10.431094in}{4.466128in}}%
\pgfpathlineto{\pgfqpoint{10.434405in}{4.464685in}}%
\pgfpathlineto{\pgfqpoint{10.437717in}{4.465152in}}%
\pgfpathlineto{\pgfqpoint{10.441028in}{4.467479in}}%
\pgfpathlineto{\pgfqpoint{10.447651in}{4.477413in}}%
\pgfpathlineto{\pgfqpoint{10.454274in}{4.493737in}}%
\pgfpathlineto{\pgfqpoint{10.460897in}{4.515563in}}%
\pgfpathlineto{\pgfqpoint{10.470831in}{4.556850in}}%
\pgfpathlineto{\pgfqpoint{10.480766in}{4.607113in}}%
\pgfpathlineto{\pgfqpoint{10.490700in}{4.666738in}}%
\pgfpathlineto{\pgfqpoint{10.503946in}{4.762852in}}%
\pgfpathlineto{\pgfqpoint{10.520503in}{4.906024in}}%
\pgfpathlineto{\pgfqpoint{10.546995in}{5.139930in}}%
\pgfpathlineto{\pgfqpoint{10.560240in}{5.236379in}}%
\pgfpathlineto{\pgfqpoint{10.573486in}{5.314633in}}%
\pgfpathlineto{\pgfqpoint{10.593355in}{5.413223in}}%
\pgfpathlineto{\pgfqpoint{10.619846in}{5.534732in}}%
\pgfpathlineto{\pgfqpoint{10.629781in}{5.573518in}}%
\pgfpathlineto{\pgfqpoint{10.639715in}{5.603142in}}%
\pgfpathlineto{\pgfqpoint{10.646338in}{5.616124in}}%
\pgfpathlineto{\pgfqpoint{10.652961in}{5.623490in}}%
\pgfpathlineto{\pgfqpoint{10.659584in}{5.626133in}}%
\pgfpathlineto{\pgfqpoint{10.666207in}{5.625515in}}%
\pgfpathlineto{\pgfqpoint{10.672830in}{5.622892in}}%
\pgfpathlineto{\pgfqpoint{10.679453in}{5.618643in}}%
\pgfpathlineto{\pgfqpoint{10.686075in}{5.612134in}}%
\pgfpathlineto{\pgfqpoint{10.692698in}{5.602127in}}%
\pgfpathlineto{\pgfqpoint{10.699321in}{5.587372in}}%
\pgfpathlineto{\pgfqpoint{10.705944in}{5.567012in}}%
\pgfpathlineto{\pgfqpoint{10.712567in}{5.540650in}}%
\pgfpathlineto{\pgfqpoint{10.722501in}{5.489773in}}%
\pgfpathlineto{\pgfqpoint{10.732436in}{5.425997in}}%
\pgfpathlineto{\pgfqpoint{10.745682in}{5.324709in}}%
\pgfpathlineto{\pgfqpoint{10.805288in}{4.840207in}}%
\pgfpathlineto{\pgfqpoint{10.828468in}{4.680366in}}%
\pgfpathlineto{\pgfqpoint{10.841714in}{4.603341in}}%
\pgfpathlineto{\pgfqpoint{10.851648in}{4.561051in}}%
\pgfpathlineto{\pgfqpoint{10.858271in}{4.542239in}}%
\pgfpathlineto{\pgfqpoint{10.864894in}{4.530718in}}%
\pgfpathlineto{\pgfqpoint{10.871517in}{4.525405in}}%
\pgfpathlineto{\pgfqpoint{10.878140in}{4.524947in}}%
\pgfpathlineto{\pgfqpoint{10.884762in}{4.528188in}}%
\pgfpathlineto{\pgfqpoint{10.891385in}{4.534401in}}%
\pgfpathlineto{\pgfqpoint{10.898008in}{4.543275in}}%
\pgfpathlineto{\pgfqpoint{10.907943in}{4.561538in}}%
\pgfpathlineto{\pgfqpoint{10.917877in}{4.585908in}}%
\pgfpathlineto{\pgfqpoint{10.931123in}{4.627027in}}%
\pgfpathlineto{\pgfqpoint{10.964237in}{4.737965in}}%
\pgfpathlineto{\pgfqpoint{10.974172in}{4.763165in}}%
\pgfpathlineto{\pgfqpoint{10.984106in}{4.782355in}}%
\pgfpathlineto{\pgfqpoint{10.994040in}{4.796055in}}%
\pgfpathlineto{\pgfqpoint{11.017220in}{4.823884in}}%
\pgfpathlineto{\pgfqpoint{11.030466in}{4.840501in}}%
\pgfpathlineto{\pgfqpoint{11.037089in}{4.844346in}}%
\pgfpathlineto{\pgfqpoint{11.040401in}{4.844365in}}%
\pgfpathlineto{\pgfqpoint{11.043712in}{4.842959in}}%
\pgfpathlineto{\pgfqpoint{11.050335in}{4.835828in}}%
\pgfpathlineto{\pgfqpoint{11.056958in}{4.823450in}}%
\pgfpathlineto{\pgfqpoint{11.066892in}{4.797560in}}%
\pgfpathlineto{\pgfqpoint{11.083449in}{4.745078in}}%
\pgfpathlineto{\pgfqpoint{11.119875in}{4.622082in}}%
\pgfpathlineto{\pgfqpoint{11.129810in}{4.579007in}}%
\pgfpathlineto{\pgfqpoint{11.139744in}{4.525330in}}%
\pgfpathlineto{\pgfqpoint{11.152990in}{4.437036in}}%
\pgfpathlineto{\pgfqpoint{11.186104in}{4.202625in}}%
\pgfpathlineto{\pgfqpoint{11.205973in}{4.084946in}}%
\pgfpathlineto{\pgfqpoint{11.222530in}{3.997626in}}%
\pgfpathlineto{\pgfqpoint{11.235776in}{3.938147in}}%
\pgfpathlineto{\pgfqpoint{11.245710in}{3.902071in}}%
\pgfpathlineto{\pgfqpoint{11.255645in}{3.873783in}}%
\pgfpathlineto{\pgfqpoint{11.265579in}{3.852214in}}%
\pgfpathlineto{\pgfqpoint{11.278825in}{3.830194in}}%
\pgfpathlineto{\pgfqpoint{11.298694in}{3.799466in}}%
\pgfpathlineto{\pgfqpoint{11.308628in}{3.780554in}}%
\pgfpathlineto{\pgfqpoint{11.318562in}{3.757589in}}%
\pgfpathlineto{\pgfqpoint{11.331808in}{3.720084in}}%
\pgfpathlineto{\pgfqpoint{11.345054in}{3.675506in}}%
\pgfpathlineto{\pgfqpoint{11.371546in}{3.575105in}}%
\pgfpathlineto{\pgfqpoint{11.384791in}{3.528235in}}%
\pgfpathlineto{\pgfqpoint{11.394726in}{3.499268in}}%
\pgfpathlineto{\pgfqpoint{11.404660in}{3.478073in}}%
\pgfpathlineto{\pgfqpoint{11.411283in}{3.469017in}}%
\pgfpathlineto{\pgfqpoint{11.417906in}{3.464249in}}%
\pgfpathlineto{\pgfqpoint{11.424529in}{3.463687in}}%
\pgfpathlineto{\pgfqpoint{11.431152in}{3.466991in}}%
\pgfpathlineto{\pgfqpoint{11.437775in}{3.473597in}}%
\pgfpathlineto{\pgfqpoint{11.447709in}{3.488124in}}%
\pgfpathlineto{\pgfqpoint{11.464266in}{3.518409in}}%
\pgfpathlineto{\pgfqpoint{11.484135in}{3.553973in}}%
\pgfpathlineto{\pgfqpoint{11.497381in}{3.573608in}}%
\pgfpathlineto{\pgfqpoint{11.507315in}{3.584547in}}%
\pgfpathlineto{\pgfqpoint{11.513938in}{3.588809in}}%
\pgfpathlineto{\pgfqpoint{11.520561in}{3.589637in}}%
\pgfpathlineto{\pgfqpoint{11.527184in}{3.586343in}}%
\pgfpathlineto{\pgfqpoint{11.533807in}{3.578685in}}%
\pgfpathlineto{\pgfqpoint{11.540430in}{3.566922in}}%
\pgfpathlineto{\pgfqpoint{11.550364in}{3.543002in}}%
\pgfpathlineto{\pgfqpoint{11.563610in}{3.503294in}}%
\pgfpathlineto{\pgfqpoint{11.583478in}{3.435549in}}%
\pgfpathlineto{\pgfqpoint{11.606659in}{3.348689in}}%
\pgfpathlineto{\pgfqpoint{11.619904in}{3.291726in}}%
\pgfpathlineto{\pgfqpoint{11.633150in}{3.226052in}}%
\pgfpathlineto{\pgfqpoint{11.653019in}{3.114349in}}%
\pgfpathlineto{\pgfqpoint{11.739117in}{2.620277in}}%
\pgfpathlineto{\pgfqpoint{11.749051in}{2.577914in}}%
\pgfpathlineto{\pgfqpoint{11.755674in}{2.556917in}}%
\pgfpathlineto{\pgfqpoint{11.762297in}{2.543001in}}%
\pgfpathlineto{\pgfqpoint{11.765608in}{2.538976in}}%
\pgfpathlineto{\pgfqpoint{11.768920in}{2.537014in}}%
\pgfpathlineto{\pgfqpoint{11.772231in}{2.537161in}}%
\pgfpathlineto{\pgfqpoint{11.775542in}{2.539434in}}%
\pgfpathlineto{\pgfqpoint{11.778854in}{2.543814in}}%
\pgfpathlineto{\pgfqpoint{11.785477in}{2.558644in}}%
\pgfpathlineto{\pgfqpoint{11.792100in}{2.580824in}}%
\pgfpathlineto{\pgfqpoint{11.802034in}{2.625130in}}%
\pgfpathlineto{\pgfqpoint{11.815280in}{2.698398in}}%
\pgfpathlineto{\pgfqpoint{11.831837in}{2.806587in}}%
\pgfpathlineto{\pgfqpoint{11.851706in}{2.939030in}}%
\pgfpathlineto{\pgfqpoint{11.861640in}{2.988283in}}%
\pgfpathlineto{\pgfqpoint{11.868263in}{3.012001in}}%
\pgfpathlineto{\pgfqpoint{11.874886in}{3.028954in}}%
\pgfpathlineto{\pgfqpoint{11.881509in}{3.040170in}}%
\pgfpathlineto{\pgfqpoint{11.888132in}{3.046771in}}%
\pgfpathlineto{\pgfqpoint{11.894755in}{3.049762in}}%
\pgfpathlineto{\pgfqpoint{11.901378in}{3.049929in}}%
\pgfpathlineto{\pgfqpoint{11.908000in}{3.047798in}}%
\pgfpathlineto{\pgfqpoint{11.914623in}{3.043627in}}%
\pgfpathlineto{\pgfqpoint{11.921246in}{3.037451in}}%
\pgfpathlineto{\pgfqpoint{11.931181in}{3.024203in}}%
\pgfpathlineto{\pgfqpoint{11.941115in}{3.005819in}}%
\pgfpathlineto{\pgfqpoint{11.951049in}{2.982014in}}%
\pgfpathlineto{\pgfqpoint{11.960984in}{2.952412in}}%
\pgfpathlineto{\pgfqpoint{11.970918in}{2.916529in}}%
\pgfpathlineto{\pgfqpoint{11.984164in}{2.859180in}}%
\pgfpathlineto{\pgfqpoint{12.004032in}{2.760677in}}%
\pgfpathlineto{\pgfqpoint{12.033836in}{2.602961in}}%
\pgfpathlineto{\pgfqpoint{12.057016in}{2.468189in}}%
\pgfpathlineto{\pgfqpoint{12.086819in}{2.279909in}}%
\pgfpathlineto{\pgfqpoint{12.103376in}{2.179162in}}%
\pgfpathlineto{\pgfqpoint{12.113310in}{2.130246in}}%
\pgfpathlineto{\pgfqpoint{12.119933in}{2.106559in}}%
\pgfpathlineto{\pgfqpoint{12.126556in}{2.092290in}}%
\pgfpathlineto{\pgfqpoint{12.129868in}{2.089114in}}%
\pgfpathlineto{\pgfqpoint{12.133179in}{2.088676in}}%
\pgfpathlineto{\pgfqpoint{12.136490in}{2.090960in}}%
\pgfpathlineto{\pgfqpoint{12.139802in}{2.095880in}}%
\pgfpathlineto{\pgfqpoint{12.146425in}{2.113018in}}%
\pgfpathlineto{\pgfqpoint{12.153048in}{2.138614in}}%
\pgfpathlineto{\pgfqpoint{12.162982in}{2.189403in}}%
\pgfpathlineto{\pgfqpoint{12.176228in}{2.272361in}}%
\pgfpathlineto{\pgfqpoint{12.209342in}{2.487998in}}%
\pgfpathlineto{\pgfqpoint{12.222588in}{2.559423in}}%
\pgfpathlineto{\pgfqpoint{12.235834in}{2.617834in}}%
\pgfpathlineto{\pgfqpoint{12.245768in}{2.651929in}}%
\pgfpathlineto{\pgfqpoint{12.255703in}{2.676795in}}%
\pgfpathlineto{\pgfqpoint{12.262326in}{2.687869in}}%
\pgfpathlineto{\pgfqpoint{12.268948in}{2.694334in}}%
\pgfpathlineto{\pgfqpoint{12.275571in}{2.696125in}}%
\pgfpathlineto{\pgfqpoint{12.282194in}{2.693361in}}%
\pgfpathlineto{\pgfqpoint{12.288817in}{2.686419in}}%
\pgfpathlineto{\pgfqpoint{12.295440in}{2.675951in}}%
\pgfpathlineto{\pgfqpoint{12.305374in}{2.655491in}}%
\pgfpathlineto{\pgfqpoint{12.321932in}{2.615103in}}%
\pgfpathlineto{\pgfqpoint{12.341800in}{2.566560in}}%
\pgfpathlineto{\pgfqpoint{12.355046in}{2.540393in}}%
\pgfpathlineto{\pgfqpoint{12.368292in}{2.519714in}}%
\pgfpathlineto{\pgfqpoint{12.381538in}{2.499109in}}%
\pgfpathlineto{\pgfqpoint{12.391472in}{2.478764in}}%
\pgfpathlineto{\pgfqpoint{12.398095in}{2.460869in}}%
\pgfpathlineto{\pgfqpoint{12.404718in}{2.438434in}}%
\pgfpathlineto{\pgfqpoint{12.414652in}{2.394864in}}%
\pgfpathlineto{\pgfqpoint{12.424587in}{2.338728in}}%
\pgfpathlineto{\pgfqpoint{12.437832in}{2.246448in}}%
\pgfpathlineto{\pgfqpoint{12.454390in}{2.113234in}}%
\pgfpathlineto{\pgfqpoint{12.454390in}{2.113234in}}%
\pgfusepath{stroke}%
\end{pgfscope}%
\begin{pgfscope}%
\pgfpathrectangle{\pgfqpoint{2.400000in}{1.081300in}}{\pgfqpoint{14.880000in}{7.569100in}}%
\pgfusepath{clip}%
\pgfsetrectcap%
\pgfsetroundjoin%
\pgfsetlinewidth{1.505625pt}%
\definecolor{currentstroke}{rgb}{1.000000,0.498039,0.054902}%
\pgfsetstrokecolor{currentstroke}%
\pgfsetdash{}{0pt}%
\pgfpathmoveto{\pgfqpoint{3.076364in}{1.425350in}}%
\pgfpathlineto{\pgfqpoint{3.255182in}{1.486120in}}%
\pgfpathlineto{\pgfqpoint{3.278362in}{1.490405in}}%
\pgfpathlineto{\pgfqpoint{3.298231in}{1.492019in}}%
\pgfpathlineto{\pgfqpoint{3.318099in}{1.491438in}}%
\pgfpathlineto{\pgfqpoint{3.337968in}{1.488599in}}%
\pgfpathlineto{\pgfqpoint{3.361148in}{1.482980in}}%
\pgfpathlineto{\pgfqpoint{3.434000in}{1.463358in}}%
\pgfpathlineto{\pgfqpoint{3.457180in}{1.459724in}}%
\pgfpathlineto{\pgfqpoint{3.477049in}{1.458573in}}%
\pgfpathlineto{\pgfqpoint{3.496918in}{1.459737in}}%
\pgfpathlineto{\pgfqpoint{3.520098in}{1.463780in}}%
\pgfpathlineto{\pgfqpoint{3.583015in}{1.476692in}}%
\pgfpathlineto{\pgfqpoint{3.602884in}{1.477933in}}%
\pgfpathlineto{\pgfqpoint{3.622753in}{1.477033in}}%
\pgfpathlineto{\pgfqpoint{3.645933in}{1.473507in}}%
\pgfpathlineto{\pgfqpoint{3.695605in}{1.464530in}}%
\pgfpathlineto{\pgfqpoint{3.712162in}{1.464168in}}%
\pgfpathlineto{\pgfqpoint{3.732031in}{1.466097in}}%
\pgfpathlineto{\pgfqpoint{3.761834in}{1.471777in}}%
\pgfpathlineto{\pgfqpoint{3.811506in}{1.481406in}}%
\pgfpathlineto{\pgfqpoint{3.844620in}{1.485482in}}%
\pgfpathlineto{\pgfqpoint{3.874423in}{1.487087in}}%
\pgfpathlineto{\pgfqpoint{3.904226in}{1.486645in}}%
\pgfpathlineto{\pgfqpoint{3.937341in}{1.483962in}}%
\pgfpathlineto{\pgfqpoint{4.006881in}{1.477247in}}%
\pgfpathlineto{\pgfqpoint{4.033373in}{1.477781in}}%
\pgfpathlineto{\pgfqpoint{4.092979in}{1.480329in}}%
\pgfpathlineto{\pgfqpoint{4.116159in}{1.478590in}}%
\pgfpathlineto{\pgfqpoint{4.142650in}{1.474209in}}%
\pgfpathlineto{\pgfqpoint{4.258551in}{1.451292in}}%
\pgfpathlineto{\pgfqpoint{4.321469in}{1.439973in}}%
\pgfpathlineto{\pgfqpoint{4.334715in}{1.440837in}}%
\pgfpathlineto{\pgfqpoint{4.347960in}{1.443839in}}%
\pgfpathlineto{\pgfqpoint{4.371141in}{1.451925in}}%
\pgfpathlineto{\pgfqpoint{4.417501in}{1.468700in}}%
\pgfpathlineto{\pgfqpoint{4.443992in}{1.475948in}}%
\pgfpathlineto{\pgfqpoint{4.473795in}{1.481707in}}%
\pgfpathlineto{\pgfqpoint{4.506910in}{1.485593in}}%
\pgfpathlineto{\pgfqpoint{4.540024in}{1.487095in}}%
\pgfpathlineto{\pgfqpoint{4.569828in}{1.486035in}}%
\pgfpathlineto{\pgfqpoint{4.596319in}{1.482821in}}%
\pgfpathlineto{\pgfqpoint{4.622811in}{1.477462in}}%
\pgfpathlineto{\pgfqpoint{4.655925in}{1.468326in}}%
\pgfpathlineto{\pgfqpoint{4.725466in}{1.448045in}}%
\pgfpathlineto{\pgfqpoint{4.751957in}{1.442954in}}%
\pgfpathlineto{\pgfqpoint{4.778449in}{1.440031in}}%
\pgfpathlineto{\pgfqpoint{4.804940in}{1.439355in}}%
\pgfpathlineto{\pgfqpoint{4.828121in}{1.440765in}}%
\pgfpathlineto{\pgfqpoint{4.851301in}{1.444179in}}%
\pgfpathlineto{\pgfqpoint{4.877792in}{1.450327in}}%
\pgfpathlineto{\pgfqpoint{4.914218in}{1.461447in}}%
\pgfpathlineto{\pgfqpoint{5.020185in}{1.495719in}}%
\pgfpathlineto{\pgfqpoint{5.046676in}{1.500975in}}%
\pgfpathlineto{\pgfqpoint{5.069856in}{1.503241in}}%
\pgfpathlineto{\pgfqpoint{5.093037in}{1.503060in}}%
\pgfpathlineto{\pgfqpoint{5.116217in}{1.500346in}}%
\pgfpathlineto{\pgfqpoint{5.139397in}{1.495149in}}%
\pgfpathlineto{\pgfqpoint{5.162577in}{1.487718in}}%
\pgfpathlineto{\pgfqpoint{5.195692in}{1.474510in}}%
\pgfpathlineto{\pgfqpoint{5.235429in}{1.458804in}}%
\pgfpathlineto{\pgfqpoint{5.258609in}{1.452061in}}%
\pgfpathlineto{\pgfqpoint{5.278478in}{1.448692in}}%
\pgfpathlineto{\pgfqpoint{5.298347in}{1.447829in}}%
\pgfpathlineto{\pgfqpoint{5.321527in}{1.449408in}}%
\pgfpathlineto{\pgfqpoint{5.430804in}{1.460771in}}%
\pgfpathlineto{\pgfqpoint{5.483788in}{1.464618in}}%
\pgfpathlineto{\pgfqpoint{5.513591in}{1.469070in}}%
\pgfpathlineto{\pgfqpoint{5.553328in}{1.477435in}}%
\pgfpathlineto{\pgfqpoint{5.606311in}{1.488351in}}%
\pgfpathlineto{\pgfqpoint{5.636114in}{1.492033in}}%
\pgfpathlineto{\pgfqpoint{5.662606in}{1.492863in}}%
\pgfpathlineto{\pgfqpoint{5.685786in}{1.491451in}}%
\pgfpathlineto{\pgfqpoint{5.712278in}{1.487543in}}%
\pgfpathlineto{\pgfqpoint{5.745392in}{1.480196in}}%
\pgfpathlineto{\pgfqpoint{5.811621in}{1.464753in}}%
\pgfpathlineto{\pgfqpoint{5.841424in}{1.460531in}}%
\pgfpathlineto{\pgfqpoint{5.874539in}{1.458293in}}%
\pgfpathlineto{\pgfqpoint{5.924211in}{1.457498in}}%
\pgfpathlineto{\pgfqpoint{5.957325in}{1.458962in}}%
\pgfpathlineto{\pgfqpoint{5.983817in}{1.462434in}}%
\pgfpathlineto{\pgfqpoint{6.010308in}{1.468352in}}%
\pgfpathlineto{\pgfqpoint{6.046734in}{1.479231in}}%
\pgfpathlineto{\pgfqpoint{6.089783in}{1.491759in}}%
\pgfpathlineto{\pgfqpoint{6.116275in}{1.497251in}}%
\pgfpathlineto{\pgfqpoint{6.142766in}{1.500337in}}%
\pgfpathlineto{\pgfqpoint{6.165946in}{1.500877in}}%
\pgfpathlineto{\pgfqpoint{6.189127in}{1.499340in}}%
\pgfpathlineto{\pgfqpoint{6.215618in}{1.495129in}}%
\pgfpathlineto{\pgfqpoint{6.245421in}{1.487759in}}%
\pgfpathlineto{\pgfqpoint{6.288470in}{1.474296in}}%
\pgfpathlineto{\pgfqpoint{6.334830in}{1.460213in}}%
\pgfpathlineto{\pgfqpoint{6.364633in}{1.453577in}}%
\pgfpathlineto{\pgfqpoint{6.394436in}{1.449469in}}%
\pgfpathlineto{\pgfqpoint{6.434174in}{1.446695in}}%
\pgfpathlineto{\pgfqpoint{6.473911in}{1.445959in}}%
\pgfpathlineto{\pgfqpoint{6.503714in}{1.447513in}}%
\pgfpathlineto{\pgfqpoint{6.589812in}{1.454326in}}%
\pgfpathlineto{\pgfqpoint{6.612992in}{1.452907in}}%
\pgfpathlineto{\pgfqpoint{6.639484in}{1.449065in}}%
\pgfpathlineto{\pgfqpoint{6.709024in}{1.436880in}}%
\pgfpathlineto{\pgfqpoint{6.728893in}{1.436796in}}%
\pgfpathlineto{\pgfqpoint{6.775253in}{1.437729in}}%
\pgfpathlineto{\pgfqpoint{6.798433in}{1.435580in}}%
\pgfpathlineto{\pgfqpoint{6.838171in}{1.430859in}}%
\pgfpathlineto{\pgfqpoint{6.851416in}{1.432807in}}%
\pgfpathlineto{\pgfqpoint{6.877908in}{1.440058in}}%
\pgfpathlineto{\pgfqpoint{6.911023in}{1.448450in}}%
\pgfpathlineto{\pgfqpoint{6.934203in}{1.452131in}}%
\pgfpathlineto{\pgfqpoint{6.960694in}{1.453919in}}%
\pgfpathlineto{\pgfqpoint{6.997120in}{1.453749in}}%
\pgfpathlineto{\pgfqpoint{7.053415in}{1.451523in}}%
\pgfpathlineto{\pgfqpoint{7.083218in}{1.447983in}}%
\pgfpathlineto{\pgfqpoint{7.126267in}{1.441774in}}%
\pgfpathlineto{\pgfqpoint{7.139513in}{1.442436in}}%
\pgfpathlineto{\pgfqpoint{7.152758in}{1.445310in}}%
\pgfpathlineto{\pgfqpoint{7.172627in}{1.452493in}}%
\pgfpathlineto{\pgfqpoint{7.225610in}{1.473424in}}%
\pgfpathlineto{\pgfqpoint{7.245479in}{1.478406in}}%
\pgfpathlineto{\pgfqpoint{7.265348in}{1.480966in}}%
\pgfpathlineto{\pgfqpoint{7.285216in}{1.481093in}}%
\pgfpathlineto{\pgfqpoint{7.305085in}{1.479090in}}%
\pgfpathlineto{\pgfqpoint{7.331577in}{1.474097in}}%
\pgfpathlineto{\pgfqpoint{7.387871in}{1.462651in}}%
\pgfpathlineto{\pgfqpoint{7.411052in}{1.460483in}}%
\pgfpathlineto{\pgfqpoint{7.434232in}{1.460534in}}%
\pgfpathlineto{\pgfqpoint{7.460723in}{1.462971in}}%
\pgfpathlineto{\pgfqpoint{7.550132in}{1.473758in}}%
\pgfpathlineto{\pgfqpoint{7.576624in}{1.473336in}}%
\pgfpathlineto{\pgfqpoint{7.646164in}{1.470402in}}%
\pgfpathlineto{\pgfqpoint{7.669345in}{1.472525in}}%
\pgfpathlineto{\pgfqpoint{7.702459in}{1.478173in}}%
\pgfpathlineto{\pgfqpoint{7.738885in}{1.484021in}}%
\pgfpathlineto{\pgfqpoint{7.765377in}{1.486030in}}%
\pgfpathlineto{\pgfqpoint{7.798491in}{1.485949in}}%
\pgfpathlineto{\pgfqpoint{7.848163in}{1.485434in}}%
\pgfpathlineto{\pgfqpoint{7.874655in}{1.487418in}}%
\pgfpathlineto{\pgfqpoint{7.904458in}{1.492028in}}%
\pgfpathlineto{\pgfqpoint{7.990555in}{1.507514in}}%
\pgfpathlineto{\pgfqpoint{8.013735in}{1.508755in}}%
\pgfpathlineto{\pgfqpoint{8.036916in}{1.507874in}}%
\pgfpathlineto{\pgfqpoint{8.060096in}{1.504872in}}%
\pgfpathlineto{\pgfqpoint{8.089899in}{1.498559in}}%
\pgfpathlineto{\pgfqpoint{8.162751in}{1.481813in}}%
\pgfpathlineto{\pgfqpoint{8.195865in}{1.477309in}}%
\pgfpathlineto{\pgfqpoint{8.242225in}{1.473525in}}%
\pgfpathlineto{\pgfqpoint{8.291897in}{1.469195in}}%
\pgfpathlineto{\pgfqpoint{8.315077in}{1.465163in}}%
\pgfpathlineto{\pgfqpoint{8.334946in}{1.459768in}}%
\pgfpathlineto{\pgfqpoint{8.358126in}{1.450957in}}%
\pgfpathlineto{\pgfqpoint{8.387929in}{1.438980in}}%
\pgfpathlineto{\pgfqpoint{8.397864in}{1.438186in}}%
\pgfpathlineto{\pgfqpoint{8.407798in}{1.440807in}}%
\pgfpathlineto{\pgfqpoint{8.421044in}{1.447699in}}%
\pgfpathlineto{\pgfqpoint{8.487273in}{1.486918in}}%
\pgfpathlineto{\pgfqpoint{8.503830in}{1.493227in}}%
\pgfpathlineto{\pgfqpoint{8.520387in}{1.497286in}}%
\pgfpathlineto{\pgfqpoint{8.536944in}{1.498961in}}%
\pgfpathlineto{\pgfqpoint{8.553502in}{1.498297in}}%
\pgfpathlineto{\pgfqpoint{8.570059in}{1.495536in}}%
\pgfpathlineto{\pgfqpoint{8.593239in}{1.489014in}}%
\pgfpathlineto{\pgfqpoint{8.652845in}{1.470258in}}%
\pgfpathlineto{\pgfqpoint{8.672714in}{1.467054in}}%
\pgfpathlineto{\pgfqpoint{8.695894in}{1.465754in}}%
\pgfpathlineto{\pgfqpoint{8.729009in}{1.466460in}}%
\pgfpathlineto{\pgfqpoint{8.858155in}{1.471114in}}%
\pgfpathlineto{\pgfqpoint{8.887958in}{1.468892in}}%
\pgfpathlineto{\pgfqpoint{8.921073in}{1.463889in}}%
\pgfpathlineto{\pgfqpoint{8.970744in}{1.456196in}}%
\pgfpathlineto{\pgfqpoint{9.000547in}{1.454390in}}%
\pgfpathlineto{\pgfqpoint{9.139628in}{1.452219in}}%
\pgfpathlineto{\pgfqpoint{9.166120in}{1.454447in}}%
\pgfpathlineto{\pgfqpoint{9.189300in}{1.458603in}}%
\pgfpathlineto{\pgfqpoint{9.222415in}{1.467197in}}%
\pgfpathlineto{\pgfqpoint{9.275398in}{1.481222in}}%
\pgfpathlineto{\pgfqpoint{9.301889in}{1.485701in}}%
\pgfpathlineto{\pgfqpoint{9.325070in}{1.487512in}}%
\pgfpathlineto{\pgfqpoint{9.351561in}{1.487259in}}%
\pgfpathlineto{\pgfqpoint{9.384676in}{1.484497in}}%
\pgfpathlineto{\pgfqpoint{9.460839in}{1.477264in}}%
\pgfpathlineto{\pgfqpoint{9.563494in}{1.469934in}}%
\pgfpathlineto{\pgfqpoint{9.599920in}{1.464393in}}%
\pgfpathlineto{\pgfqpoint{9.676083in}{1.450997in}}%
\pgfpathlineto{\pgfqpoint{9.692640in}{1.451148in}}%
\pgfpathlineto{\pgfqpoint{9.712509in}{1.453793in}}%
\pgfpathlineto{\pgfqpoint{9.745624in}{1.461256in}}%
\pgfpathlineto{\pgfqpoint{9.782050in}{1.468663in}}%
\pgfpathlineto{\pgfqpoint{9.808541in}{1.471740in}}%
\pgfpathlineto{\pgfqpoint{9.835033in}{1.472699in}}%
\pgfpathlineto{\pgfqpoint{9.868147in}{1.471658in}}%
\pgfpathlineto{\pgfqpoint{10.076769in}{1.460634in}}%
\pgfpathlineto{\pgfqpoint{10.152932in}{1.458400in}}%
\pgfpathlineto{\pgfqpoint{10.209227in}{1.457070in}}%
\pgfpathlineto{\pgfqpoint{10.301947in}{1.456579in}}%
\pgfpathlineto{\pgfqpoint{10.341685in}{1.453008in}}%
\pgfpathlineto{\pgfqpoint{10.401291in}{1.447481in}}%
\pgfpathlineto{\pgfqpoint{10.450963in}{1.445616in}}%
\pgfpathlineto{\pgfqpoint{10.490700in}{1.445916in}}%
\pgfpathlineto{\pgfqpoint{10.517191in}{1.448344in}}%
\pgfpathlineto{\pgfqpoint{10.546995in}{1.453629in}}%
\pgfpathlineto{\pgfqpoint{10.643027in}{1.473131in}}%
\pgfpathlineto{\pgfqpoint{10.676141in}{1.476300in}}%
\pgfpathlineto{\pgfqpoint{10.715878in}{1.477514in}}%
\pgfpathlineto{\pgfqpoint{10.772173in}{1.479237in}}%
\pgfpathlineto{\pgfqpoint{10.864894in}{1.484288in}}%
\pgfpathlineto{\pgfqpoint{10.888074in}{1.482428in}}%
\pgfpathlineto{\pgfqpoint{10.911254in}{1.478260in}}%
\pgfpathlineto{\pgfqpoint{10.934434in}{1.471794in}}%
\pgfpathlineto{\pgfqpoint{10.960926in}{1.462092in}}%
\pgfpathlineto{\pgfqpoint{11.040401in}{1.430773in}}%
\pgfpathlineto{\pgfqpoint{11.050335in}{1.430432in}}%
\pgfpathlineto{\pgfqpoint{11.063581in}{1.432935in}}%
\pgfpathlineto{\pgfqpoint{11.103318in}{1.441458in}}%
\pgfpathlineto{\pgfqpoint{11.133121in}{1.445139in}}%
\pgfpathlineto{\pgfqpoint{11.176170in}{1.447772in}}%
\pgfpathlineto{\pgfqpoint{11.245710in}{1.449528in}}%
\pgfpathlineto{\pgfqpoint{11.298694in}{1.448744in}}%
\pgfpathlineto{\pgfqpoint{11.351677in}{1.448219in}}%
\pgfpathlineto{\pgfqpoint{11.381480in}{1.450459in}}%
\pgfpathlineto{\pgfqpoint{11.427840in}{1.456875in}}%
\pgfpathlineto{\pgfqpoint{11.467578in}{1.461574in}}%
\pgfpathlineto{\pgfqpoint{11.500692in}{1.463273in}}%
\pgfpathlineto{\pgfqpoint{11.533807in}{1.462782in}}%
\pgfpathlineto{\pgfqpoint{11.573544in}{1.459946in}}%
\pgfpathlineto{\pgfqpoint{11.649707in}{1.451662in}}%
\pgfpathlineto{\pgfqpoint{11.712625in}{1.445696in}}%
\pgfpathlineto{\pgfqpoint{11.739117in}{1.445213in}}%
\pgfpathlineto{\pgfqpoint{11.762297in}{1.447059in}}%
\pgfpathlineto{\pgfqpoint{11.788788in}{1.451829in}}%
\pgfpathlineto{\pgfqpoint{11.874886in}{1.470058in}}%
\pgfpathlineto{\pgfqpoint{11.908000in}{1.473393in}}%
\pgfpathlineto{\pgfqpoint{11.980852in}{1.479761in}}%
\pgfpathlineto{\pgfqpoint{12.116622in}{1.496505in}}%
\pgfpathlineto{\pgfqpoint{12.146425in}{1.496982in}}%
\pgfpathlineto{\pgfqpoint{12.172916in}{1.495226in}}%
\pgfpathlineto{\pgfqpoint{12.199408in}{1.491001in}}%
\pgfpathlineto{\pgfqpoint{12.229211in}{1.483718in}}%
\pgfpathlineto{\pgfqpoint{12.268948in}{1.473787in}}%
\pgfpathlineto{\pgfqpoint{12.288817in}{1.471327in}}%
\pgfpathlineto{\pgfqpoint{12.305374in}{1.471427in}}%
\pgfpathlineto{\pgfqpoint{12.325243in}{1.473800in}}%
\pgfpathlineto{\pgfqpoint{12.374915in}{1.481321in}}%
\pgfpathlineto{\pgfqpoint{12.394784in}{1.481492in}}%
\pgfpathlineto{\pgfqpoint{12.414652in}{1.479209in}}%
\pgfpathlineto{\pgfqpoint{12.437832in}{1.473870in}}%
\pgfpathlineto{\pgfqpoint{12.523930in}{1.450678in}}%
\pgfpathlineto{\pgfqpoint{12.547110in}{1.448445in}}%
\pgfpathlineto{\pgfqpoint{12.566979in}{1.448879in}}%
\pgfpathlineto{\pgfqpoint{12.583536in}{1.451277in}}%
\pgfpathlineto{\pgfqpoint{12.603405in}{1.456482in}}%
\pgfpathlineto{\pgfqpoint{12.672945in}{1.477869in}}%
\pgfpathlineto{\pgfqpoint{12.689503in}{1.479404in}}%
\pgfpathlineto{\pgfqpoint{12.706060in}{1.478518in}}%
\pgfpathlineto{\pgfqpoint{12.722617in}{1.475146in}}%
\pgfpathlineto{\pgfqpoint{12.739174in}{1.469523in}}%
\pgfpathlineto{\pgfqpoint{12.762355in}{1.459031in}}%
\pgfpathlineto{\pgfqpoint{12.785535in}{1.448575in}}%
\pgfpathlineto{\pgfqpoint{12.798780in}{1.445132in}}%
\pgfpathlineto{\pgfqpoint{12.808715in}{1.444903in}}%
\pgfpathlineto{\pgfqpoint{12.821961in}{1.447563in}}%
\pgfpathlineto{\pgfqpoint{12.848452in}{1.456777in}}%
\pgfpathlineto{\pgfqpoint{12.871632in}{1.463467in}}%
\pgfpathlineto{\pgfqpoint{12.891501in}{1.466523in}}%
\pgfpathlineto{\pgfqpoint{12.911370in}{1.466956in}}%
\pgfpathlineto{\pgfqpoint{12.931238in}{1.465140in}}%
\pgfpathlineto{\pgfqpoint{12.957730in}{1.460360in}}%
\pgfpathlineto{\pgfqpoint{13.020648in}{1.447847in}}%
\pgfpathlineto{\pgfqpoint{13.043828in}{1.445743in}}%
\pgfpathlineto{\pgfqpoint{13.063696in}{1.446234in}}%
\pgfpathlineto{\pgfqpoint{13.083565in}{1.449311in}}%
\pgfpathlineto{\pgfqpoint{13.106745in}{1.455498in}}%
\pgfpathlineto{\pgfqpoint{13.163040in}{1.471921in}}%
\pgfpathlineto{\pgfqpoint{13.186220in}{1.475724in}}%
\pgfpathlineto{\pgfqpoint{13.206089in}{1.476808in}}%
\pgfpathlineto{\pgfqpoint{13.229269in}{1.475584in}}%
\pgfpathlineto{\pgfqpoint{13.255761in}{1.471542in}}%
\pgfpathlineto{\pgfqpoint{13.345170in}{1.454820in}}%
\pgfpathlineto{\pgfqpoint{13.371661in}{1.453890in}}%
\pgfpathlineto{\pgfqpoint{13.408087in}{1.455237in}}%
\pgfpathlineto{\pgfqpoint{13.461070in}{1.457100in}}%
\pgfpathlineto{\pgfqpoint{13.494185in}{1.456033in}}%
\pgfpathlineto{\pgfqpoint{13.533922in}{1.452301in}}%
\pgfpathlineto{\pgfqpoint{13.576971in}{1.448670in}}%
\pgfpathlineto{\pgfqpoint{13.603463in}{1.448849in}}%
\pgfpathlineto{\pgfqpoint{13.626643in}{1.451369in}}%
\pgfpathlineto{\pgfqpoint{13.649823in}{1.456418in}}%
\pgfpathlineto{\pgfqpoint{13.676315in}{1.464875in}}%
\pgfpathlineto{\pgfqpoint{13.745855in}{1.489116in}}%
\pgfpathlineto{\pgfqpoint{13.769035in}{1.494171in}}%
\pgfpathlineto{\pgfqpoint{13.788904in}{1.496449in}}%
\pgfpathlineto{\pgfqpoint{13.808773in}{1.496644in}}%
\pgfpathlineto{\pgfqpoint{13.828641in}{1.494719in}}%
\pgfpathlineto{\pgfqpoint{13.851822in}{1.489985in}}%
\pgfpathlineto{\pgfqpoint{13.878313in}{1.482049in}}%
\pgfpathlineto{\pgfqpoint{13.941231in}{1.461742in}}%
\pgfpathlineto{\pgfqpoint{13.961099in}{1.457957in}}%
\pgfpathlineto{\pgfqpoint{13.980968in}{1.456507in}}%
\pgfpathlineto{\pgfqpoint{14.000837in}{1.457360in}}%
\pgfpathlineto{\pgfqpoint{14.027328in}{1.460999in}}%
\pgfpathlineto{\pgfqpoint{14.222704in}{1.493834in}}%
\pgfpathlineto{\pgfqpoint{14.249196in}{1.495103in}}%
\pgfpathlineto{\pgfqpoint{14.272376in}{1.494150in}}%
\pgfpathlineto{\pgfqpoint{14.298867in}{1.490727in}}%
\pgfpathlineto{\pgfqpoint{14.331982in}{1.483853in}}%
\pgfpathlineto{\pgfqpoint{14.391588in}{1.470822in}}%
\pgfpathlineto{\pgfqpoint{14.414768in}{1.468205in}}%
\pgfpathlineto{\pgfqpoint{14.437948in}{1.468040in}}%
\pgfpathlineto{\pgfqpoint{14.461128in}{1.470219in}}%
\pgfpathlineto{\pgfqpoint{14.504177in}{1.477284in}}%
\pgfpathlineto{\pgfqpoint{14.533980in}{1.481180in}}%
\pgfpathlineto{\pgfqpoint{14.557160in}{1.482154in}}%
\pgfpathlineto{\pgfqpoint{14.580340in}{1.480824in}}%
\pgfpathlineto{\pgfqpoint{14.603521in}{1.477288in}}%
\pgfpathlineto{\pgfqpoint{14.639947in}{1.469033in}}%
\pgfpathlineto{\pgfqpoint{14.673061in}{1.462306in}}%
\pgfpathlineto{\pgfqpoint{14.692930in}{1.460396in}}%
\pgfpathlineto{\pgfqpoint{14.712798in}{1.460583in}}%
\pgfpathlineto{\pgfqpoint{14.739290in}{1.463335in}}%
\pgfpathlineto{\pgfqpoint{14.818765in}{1.473276in}}%
\pgfpathlineto{\pgfqpoint{14.868437in}{1.475993in}}%
\pgfpathlineto{\pgfqpoint{14.908174in}{1.478779in}}%
\pgfpathlineto{\pgfqpoint{14.931354in}{1.482526in}}%
\pgfpathlineto{\pgfqpoint{14.954534in}{1.488669in}}%
\pgfpathlineto{\pgfqpoint{14.981026in}{1.498252in}}%
\pgfpathlineto{\pgfqpoint{15.043943in}{1.522375in}}%
\pgfpathlineto{\pgfqpoint{15.063812in}{1.527332in}}%
\pgfpathlineto{\pgfqpoint{15.083681in}{1.529845in}}%
\pgfpathlineto{\pgfqpoint{15.100238in}{1.529763in}}%
\pgfpathlineto{\pgfqpoint{15.116795in}{1.527587in}}%
\pgfpathlineto{\pgfqpoint{15.133353in}{1.523363in}}%
\pgfpathlineto{\pgfqpoint{15.153221in}{1.515881in}}%
\pgfpathlineto{\pgfqpoint{15.176401in}{1.504591in}}%
\pgfpathlineto{\pgfqpoint{15.245942in}{1.468522in}}%
\pgfpathlineto{\pgfqpoint{15.262499in}{1.463334in}}%
\pgfpathlineto{\pgfqpoint{15.279056in}{1.460693in}}%
\pgfpathlineto{\pgfqpoint{15.295614in}{1.460485in}}%
\pgfpathlineto{\pgfqpoint{15.315482in}{1.462497in}}%
\pgfpathlineto{\pgfqpoint{15.358531in}{1.469823in}}%
\pgfpathlineto{\pgfqpoint{15.411514in}{1.477918in}}%
\pgfpathlineto{\pgfqpoint{15.451252in}{1.481837in}}%
\pgfpathlineto{\pgfqpoint{15.490989in}{1.483525in}}%
\pgfpathlineto{\pgfqpoint{15.543972in}{1.485802in}}%
\pgfpathlineto{\pgfqpoint{15.570464in}{1.489399in}}%
\pgfpathlineto{\pgfqpoint{15.600267in}{1.495764in}}%
\pgfpathlineto{\pgfqpoint{15.656562in}{1.508551in}}%
\pgfpathlineto{\pgfqpoint{15.679742in}{1.511400in}}%
\pgfpathlineto{\pgfqpoint{15.702922in}{1.511875in}}%
\pgfpathlineto{\pgfqpoint{15.726102in}{1.510002in}}%
\pgfpathlineto{\pgfqpoint{15.752594in}{1.505628in}}%
\pgfpathlineto{\pgfqpoint{15.795643in}{1.495948in}}%
\pgfpathlineto{\pgfqpoint{15.861872in}{1.478925in}}%
\pgfpathlineto{\pgfqpoint{15.898298in}{1.467308in}}%
\pgfpathlineto{\pgfqpoint{15.934723in}{1.453198in}}%
\pgfpathlineto{\pgfqpoint{15.977772in}{1.436022in}}%
\pgfpathlineto{\pgfqpoint{15.987707in}{1.433805in}}%
\pgfpathlineto{\pgfqpoint{15.997641in}{1.433827in}}%
\pgfpathlineto{\pgfqpoint{16.010887in}{1.436922in}}%
\pgfpathlineto{\pgfqpoint{16.077116in}{1.456940in}}%
\pgfpathlineto{\pgfqpoint{16.103607in}{1.461339in}}%
\pgfpathlineto{\pgfqpoint{16.130099in}{1.463348in}}%
\pgfpathlineto{\pgfqpoint{16.159902in}{1.463280in}}%
\pgfpathlineto{\pgfqpoint{16.239377in}{1.461223in}}%
\pgfpathlineto{\pgfqpoint{16.262557in}{1.463634in}}%
\pgfpathlineto{\pgfqpoint{16.289049in}{1.468880in}}%
\pgfpathlineto{\pgfqpoint{16.322163in}{1.477987in}}%
\pgfpathlineto{\pgfqpoint{16.375146in}{1.492811in}}%
\pgfpathlineto{\pgfqpoint{16.401638in}{1.497874in}}%
\pgfpathlineto{\pgfqpoint{16.424818in}{1.500075in}}%
\pgfpathlineto{\pgfqpoint{16.447998in}{1.499831in}}%
\pgfpathlineto{\pgfqpoint{16.471178in}{1.497078in}}%
\pgfpathlineto{\pgfqpoint{16.494359in}{1.492093in}}%
\pgfpathlineto{\pgfqpoint{16.500981in}{1.490341in}}%
\pgfpathlineto{\pgfqpoint{16.500981in}{1.490341in}}%
\pgfusepath{stroke}%
\end{pgfscope}%
\begin{pgfscope}%
\pgfpathrectangle{\pgfqpoint{2.400000in}{1.081300in}}{\pgfqpoint{14.880000in}{7.569100in}}%
\pgfusepath{clip}%
\pgfsetrectcap%
\pgfsetroundjoin%
\pgfsetlinewidth{1.505625pt}%
\definecolor{currentstroke}{rgb}{0.172549,0.627451,0.172549}%
\pgfsetstrokecolor{currentstroke}%
\pgfsetdash{}{0pt}%
\pgfpathmoveto{\pgfqpoint{3.076364in}{1.425350in}}%
\pgfpathlineto{\pgfqpoint{3.135970in}{2.115721in}}%
\pgfpathlineto{\pgfqpoint{3.152527in}{2.285819in}}%
\pgfpathlineto{\pgfqpoint{3.165773in}{2.401194in}}%
\pgfpathlineto{\pgfqpoint{3.175707in}{2.469932in}}%
\pgfpathlineto{\pgfqpoint{3.182330in}{2.505367in}}%
\pgfpathlineto{\pgfqpoint{3.188953in}{2.531611in}}%
\pgfpathlineto{\pgfqpoint{3.195576in}{2.548247in}}%
\pgfpathlineto{\pgfqpoint{3.198887in}{2.552924in}}%
\pgfpathlineto{\pgfqpoint{3.202199in}{2.555193in}}%
\pgfpathlineto{\pgfqpoint{3.205510in}{2.555093in}}%
\pgfpathlineto{\pgfqpoint{3.208822in}{2.552677in}}%
\pgfpathlineto{\pgfqpoint{3.212133in}{2.548011in}}%
\pgfpathlineto{\pgfqpoint{3.218756in}{2.532248in}}%
\pgfpathlineto{\pgfqpoint{3.225379in}{2.508502in}}%
\pgfpathlineto{\pgfqpoint{3.232002in}{2.477527in}}%
\pgfpathlineto{\pgfqpoint{3.241936in}{2.419154in}}%
\pgfpathlineto{\pgfqpoint{3.255182in}{2.322754in}}%
\pgfpathlineto{\pgfqpoint{3.271739in}{2.179435in}}%
\pgfpathlineto{\pgfqpoint{3.314788in}{1.780678in}}%
\pgfpathlineto{\pgfqpoint{3.321411in}{1.744331in}}%
\pgfpathlineto{\pgfqpoint{3.324722in}{1.733949in}}%
\pgfpathlineto{\pgfqpoint{3.328034in}{1.729806in}}%
\pgfpathlineto{\pgfqpoint{3.331345in}{1.732284in}}%
\pgfpathlineto{\pgfqpoint{3.334657in}{1.741304in}}%
\pgfpathlineto{\pgfqpoint{3.337968in}{1.756352in}}%
\pgfpathlineto{\pgfqpoint{3.344591in}{1.801149in}}%
\pgfpathlineto{\pgfqpoint{3.354525in}{1.891545in}}%
\pgfpathlineto{\pgfqpoint{3.384328in}{2.181127in}}%
\pgfpathlineto{\pgfqpoint{3.394263in}{2.254528in}}%
\pgfpathlineto{\pgfqpoint{3.404197in}{2.308787in}}%
\pgfpathlineto{\pgfqpoint{3.410820in}{2.333454in}}%
\pgfpathlineto{\pgfqpoint{3.417443in}{2.348751in}}%
\pgfpathlineto{\pgfqpoint{3.420754in}{2.352922in}}%
\pgfpathlineto{\pgfqpoint{3.424066in}{2.354822in}}%
\pgfpathlineto{\pgfqpoint{3.427377in}{2.354501in}}%
\pgfpathlineto{\pgfqpoint{3.430689in}{2.352020in}}%
\pgfpathlineto{\pgfqpoint{3.434000in}{2.347453in}}%
\pgfpathlineto{\pgfqpoint{3.440623in}{2.332396in}}%
\pgfpathlineto{\pgfqpoint{3.447246in}{2.310109in}}%
\pgfpathlineto{\pgfqpoint{3.457180in}{2.265244in}}%
\pgfpathlineto{\pgfqpoint{3.470426in}{2.190616in}}%
\pgfpathlineto{\pgfqpoint{3.490295in}{2.075031in}}%
\pgfpathlineto{\pgfqpoint{3.500229in}{2.032080in}}%
\pgfpathlineto{\pgfqpoint{3.506852in}{2.014363in}}%
\pgfpathlineto{\pgfqpoint{3.510164in}{2.009426in}}%
\pgfpathlineto{\pgfqpoint{3.513475in}{2.007248in}}%
\pgfpathlineto{\pgfqpoint{3.516786in}{2.007839in}}%
\pgfpathlineto{\pgfqpoint{3.520098in}{2.011130in}}%
\pgfpathlineto{\pgfqpoint{3.523409in}{2.016977in}}%
\pgfpathlineto{\pgfqpoint{3.530032in}{2.035452in}}%
\pgfpathlineto{\pgfqpoint{3.539967in}{2.075835in}}%
\pgfpathlineto{\pgfqpoint{3.573081in}{2.229922in}}%
\pgfpathlineto{\pgfqpoint{3.583015in}{2.262252in}}%
\pgfpathlineto{\pgfqpoint{3.589638in}{2.277758in}}%
\pgfpathlineto{\pgfqpoint{3.596261in}{2.288065in}}%
\pgfpathlineto{\pgfqpoint{3.602884in}{2.293025in}}%
\pgfpathlineto{\pgfqpoint{3.606196in}{2.293483in}}%
\pgfpathlineto{\pgfqpoint{3.609507in}{2.292597in}}%
\pgfpathlineto{\pgfqpoint{3.616130in}{2.286830in}}%
\pgfpathlineto{\pgfqpoint{3.622753in}{2.275846in}}%
\pgfpathlineto{\pgfqpoint{3.629376in}{2.259840in}}%
\pgfpathlineto{\pgfqpoint{3.639310in}{2.227012in}}%
\pgfpathlineto{\pgfqpoint{3.649244in}{2.184726in}}%
\pgfpathlineto{\pgfqpoint{3.662490in}{2.116697in}}%
\pgfpathlineto{\pgfqpoint{3.702228in}{1.900438in}}%
\pgfpathlineto{\pgfqpoint{3.712162in}{1.864243in}}%
\pgfpathlineto{\pgfqpoint{3.718785in}{1.847332in}}%
\pgfpathlineto{\pgfqpoint{3.725408in}{1.836056in}}%
\pgfpathlineto{\pgfqpoint{3.732031in}{1.829596in}}%
\pgfpathlineto{\pgfqpoint{3.738654in}{1.826656in}}%
\pgfpathlineto{\pgfqpoint{3.751899in}{1.825293in}}%
\pgfpathlineto{\pgfqpoint{3.761834in}{1.822923in}}%
\pgfpathlineto{\pgfqpoint{3.768457in}{1.818987in}}%
\pgfpathlineto{\pgfqpoint{3.775080in}{1.812773in}}%
\pgfpathlineto{\pgfqpoint{3.785014in}{1.799832in}}%
\pgfpathlineto{\pgfqpoint{3.798260in}{1.781833in}}%
\pgfpathlineto{\pgfqpoint{3.804883in}{1.777265in}}%
\pgfpathlineto{\pgfqpoint{3.808194in}{1.777364in}}%
\pgfpathlineto{\pgfqpoint{3.811506in}{1.779583in}}%
\pgfpathlineto{\pgfqpoint{3.814817in}{1.784298in}}%
\pgfpathlineto{\pgfqpoint{3.818128in}{1.791828in}}%
\pgfpathlineto{\pgfqpoint{3.824751in}{1.816214in}}%
\pgfpathlineto{\pgfqpoint{3.831374in}{1.853532in}}%
\pgfpathlineto{\pgfqpoint{3.837997in}{1.903135in}}%
\pgfpathlineto{\pgfqpoint{3.847931in}{1.996861in}}%
\pgfpathlineto{\pgfqpoint{3.861177in}{2.146505in}}%
\pgfpathlineto{\pgfqpoint{3.894292in}{2.536160in}}%
\pgfpathlineto{\pgfqpoint{3.904226in}{2.630449in}}%
\pgfpathlineto{\pgfqpoint{3.914160in}{2.705068in}}%
\pgfpathlineto{\pgfqpoint{3.920783in}{2.742213in}}%
\pgfpathlineto{\pgfqpoint{3.927406in}{2.768491in}}%
\pgfpathlineto{\pgfqpoint{3.934029in}{2.783510in}}%
\pgfpathlineto{\pgfqpoint{3.937341in}{2.786737in}}%
\pgfpathlineto{\pgfqpoint{3.940652in}{2.787102in}}%
\pgfpathlineto{\pgfqpoint{3.943964in}{2.784614in}}%
\pgfpathlineto{\pgfqpoint{3.947275in}{2.779291in}}%
\pgfpathlineto{\pgfqpoint{3.953898in}{2.760265in}}%
\pgfpathlineto{\pgfqpoint{3.960521in}{2.730358in}}%
\pgfpathlineto{\pgfqpoint{3.967144in}{2.690035in}}%
\pgfpathlineto{\pgfqpoint{3.977078in}{2.611376in}}%
\pgfpathlineto{\pgfqpoint{3.987012in}{2.513416in}}%
\pgfpathlineto{\pgfqpoint{4.000258in}{2.360084in}}%
\pgfpathlineto{\pgfqpoint{4.023438in}{2.087757in}}%
\pgfpathlineto{\pgfqpoint{4.030061in}{2.034387in}}%
\pgfpathlineto{\pgfqpoint{4.033373in}{2.016920in}}%
\pgfpathlineto{\pgfqpoint{4.036684in}{2.007044in}}%
\pgfpathlineto{\pgfqpoint{4.039996in}{2.005542in}}%
\pgfpathlineto{\pgfqpoint{4.043307in}{2.012787in}}%
\pgfpathlineto{\pgfqpoint{4.046618in}{2.028674in}}%
\pgfpathlineto{\pgfqpoint{4.053241in}{2.083857in}}%
\pgfpathlineto{\pgfqpoint{4.059864in}{2.163631in}}%
\pgfpathlineto{\pgfqpoint{4.073110in}{2.363678in}}%
\pgfpathlineto{\pgfqpoint{4.096290in}{2.722100in}}%
\pgfpathlineto{\pgfqpoint{4.106225in}{2.847384in}}%
\pgfpathlineto{\pgfqpoint{4.116159in}{2.945000in}}%
\pgfpathlineto{\pgfqpoint{4.122782in}{2.992403in}}%
\pgfpathlineto{\pgfqpoint{4.129405in}{3.024541in}}%
\pgfpathlineto{\pgfqpoint{4.132716in}{3.034678in}}%
\pgfpathlineto{\pgfqpoint{4.136028in}{3.040785in}}%
\pgfpathlineto{\pgfqpoint{4.139339in}{3.042826in}}%
\pgfpathlineto{\pgfqpoint{4.142650in}{3.040784in}}%
\pgfpathlineto{\pgfqpoint{4.145962in}{3.034656in}}%
\pgfpathlineto{\pgfqpoint{4.149273in}{3.024458in}}%
\pgfpathlineto{\pgfqpoint{4.155896in}{2.992008in}}%
\pgfpathlineto{\pgfqpoint{4.162519in}{2.943928in}}%
\pgfpathlineto{\pgfqpoint{4.169142in}{2.881040in}}%
\pgfpathlineto{\pgfqpoint{4.179076in}{2.761695in}}%
\pgfpathlineto{\pgfqpoint{4.192322in}{2.566102in}}%
\pgfpathlineto{\pgfqpoint{4.215502in}{2.204450in}}%
\pgfpathlineto{\pgfqpoint{4.222125in}{2.127601in}}%
\pgfpathlineto{\pgfqpoint{4.228748in}{2.079629in}}%
\pgfpathlineto{\pgfqpoint{4.232060in}{2.069112in}}%
\pgfpathlineto{\pgfqpoint{4.235371in}{2.068245in}}%
\pgfpathlineto{\pgfqpoint{4.238683in}{2.076861in}}%
\pgfpathlineto{\pgfqpoint{4.241994in}{2.094271in}}%
\pgfpathlineto{\pgfqpoint{4.248617in}{2.150903in}}%
\pgfpathlineto{\pgfqpoint{4.258551in}{2.270761in}}%
\pgfpathlineto{\pgfqpoint{4.285043in}{2.618199in}}%
\pgfpathlineto{\pgfqpoint{4.294977in}{2.722629in}}%
\pgfpathlineto{\pgfqpoint{4.304912in}{2.804990in}}%
\pgfpathlineto{\pgfqpoint{4.311534in}{2.846866in}}%
\pgfpathlineto{\pgfqpoint{4.318157in}{2.878172in}}%
\pgfpathlineto{\pgfqpoint{4.324780in}{2.898979in}}%
\pgfpathlineto{\pgfqpoint{4.328092in}{2.905504in}}%
\pgfpathlineto{\pgfqpoint{4.331403in}{2.909485in}}%
\pgfpathlineto{\pgfqpoint{4.334715in}{2.910960in}}%
\pgfpathlineto{\pgfqpoint{4.338026in}{2.909972in}}%
\pgfpathlineto{\pgfqpoint{4.341337in}{2.906566in}}%
\pgfpathlineto{\pgfqpoint{4.344649in}{2.900792in}}%
\pgfpathlineto{\pgfqpoint{4.351272in}{2.882347in}}%
\pgfpathlineto{\pgfqpoint{4.357895in}{2.855090in}}%
\pgfpathlineto{\pgfqpoint{4.364518in}{2.819527in}}%
\pgfpathlineto{\pgfqpoint{4.374452in}{2.751853in}}%
\pgfpathlineto{\pgfqpoint{4.384386in}{2.668984in}}%
\pgfpathlineto{\pgfqpoint{4.397632in}{2.539603in}}%
\pgfpathlineto{\pgfqpoint{4.447304in}{2.026588in}}%
\pgfpathlineto{\pgfqpoint{4.457238in}{1.960912in}}%
\pgfpathlineto{\pgfqpoint{4.463861in}{1.930819in}}%
\pgfpathlineto{\pgfqpoint{4.470484in}{1.911928in}}%
\pgfpathlineto{\pgfqpoint{4.473795in}{1.906405in}}%
\pgfpathlineto{\pgfqpoint{4.477107in}{1.903222in}}%
\pgfpathlineto{\pgfqpoint{4.480418in}{1.902110in}}%
\pgfpathlineto{\pgfqpoint{4.483730in}{1.902774in}}%
\pgfpathlineto{\pgfqpoint{4.490353in}{1.908210in}}%
\pgfpathlineto{\pgfqpoint{4.500287in}{1.922461in}}%
\pgfpathlineto{\pgfqpoint{4.530090in}{1.973648in}}%
\pgfpathlineto{\pgfqpoint{4.540024in}{1.996596in}}%
\pgfpathlineto{\pgfqpoint{4.546647in}{2.017561in}}%
\pgfpathlineto{\pgfqpoint{4.553270in}{2.045073in}}%
\pgfpathlineto{\pgfqpoint{4.559893in}{2.080462in}}%
\pgfpathlineto{\pgfqpoint{4.569828in}{2.149607in}}%
\pgfpathlineto{\pgfqpoint{4.579762in}{2.236920in}}%
\pgfpathlineto{\pgfqpoint{4.596319in}{2.410140in}}%
\pgfpathlineto{\pgfqpoint{4.619499in}{2.657656in}}%
\pgfpathlineto{\pgfqpoint{4.629434in}{2.746600in}}%
\pgfpathlineto{\pgfqpoint{4.639368in}{2.817322in}}%
\pgfpathlineto{\pgfqpoint{4.645991in}{2.852416in}}%
\pgfpathlineto{\pgfqpoint{4.652614in}{2.876905in}}%
\pgfpathlineto{\pgfqpoint{4.659237in}{2.890256in}}%
\pgfpathlineto{\pgfqpoint{4.662548in}{2.892656in}}%
\pgfpathlineto{\pgfqpoint{4.665860in}{2.892180in}}%
\pgfpathlineto{\pgfqpoint{4.669171in}{2.888828in}}%
\pgfpathlineto{\pgfqpoint{4.672482in}{2.882611in}}%
\pgfpathlineto{\pgfqpoint{4.679105in}{2.861687in}}%
\pgfpathlineto{\pgfqpoint{4.685728in}{2.829731in}}%
\pgfpathlineto{\pgfqpoint{4.692351in}{2.787238in}}%
\pgfpathlineto{\pgfqpoint{4.702286in}{2.705199in}}%
\pgfpathlineto{\pgfqpoint{4.712220in}{2.603753in}}%
\pgfpathlineto{\pgfqpoint{4.725466in}{2.444518in}}%
\pgfpathlineto{\pgfqpoint{4.768515in}{1.890548in}}%
\pgfpathlineto{\pgfqpoint{4.775137in}{1.833698in}}%
\pgfpathlineto{\pgfqpoint{4.781760in}{1.797332in}}%
\pgfpathlineto{\pgfqpoint{4.785072in}{1.788053in}}%
\pgfpathlineto{\pgfqpoint{4.788383in}{1.784827in}}%
\pgfpathlineto{\pgfqpoint{4.791695in}{1.787299in}}%
\pgfpathlineto{\pgfqpoint{4.795006in}{1.794799in}}%
\pgfpathlineto{\pgfqpoint{4.801629in}{1.821306in}}%
\pgfpathlineto{\pgfqpoint{4.814875in}{1.895638in}}%
\pgfpathlineto{\pgfqpoint{4.828121in}{1.965353in}}%
\pgfpathlineto{\pgfqpoint{4.834744in}{1.991450in}}%
\pgfpathlineto{\pgfqpoint{4.841366in}{2.009828in}}%
\pgfpathlineto{\pgfqpoint{4.847989in}{2.019849in}}%
\pgfpathlineto{\pgfqpoint{4.851301in}{2.021669in}}%
\pgfpathlineto{\pgfqpoint{4.854612in}{2.021399in}}%
\pgfpathlineto{\pgfqpoint{4.857924in}{2.019111in}}%
\pgfpathlineto{\pgfqpoint{4.864547in}{2.008979in}}%
\pgfpathlineto{\pgfqpoint{4.871169in}{1.992765in}}%
\pgfpathlineto{\pgfqpoint{4.887727in}{1.944798in}}%
\pgfpathlineto{\pgfqpoint{4.891038in}{1.938473in}}%
\pgfpathlineto{\pgfqpoint{4.894350in}{1.934966in}}%
\pgfpathlineto{\pgfqpoint{4.897661in}{1.935079in}}%
\pgfpathlineto{\pgfqpoint{4.900973in}{1.939555in}}%
\pgfpathlineto{\pgfqpoint{4.904284in}{1.949008in}}%
\pgfpathlineto{\pgfqpoint{4.907595in}{1.963866in}}%
\pgfpathlineto{\pgfqpoint{4.914218in}{2.010387in}}%
\pgfpathlineto{\pgfqpoint{4.920841in}{2.078242in}}%
\pgfpathlineto{\pgfqpoint{4.930776in}{2.212687in}}%
\pgfpathlineto{\pgfqpoint{4.944021in}{2.431411in}}%
\pgfpathlineto{\pgfqpoint{4.980447in}{3.063662in}}%
\pgfpathlineto{\pgfqpoint{4.993693in}{3.251144in}}%
\pgfpathlineto{\pgfqpoint{5.003627in}{3.365194in}}%
\pgfpathlineto{\pgfqpoint{5.013562in}{3.453120in}}%
\pgfpathlineto{\pgfqpoint{5.020185in}{3.496204in}}%
\pgfpathlineto{\pgfqpoint{5.026808in}{3.526392in}}%
\pgfpathlineto{\pgfqpoint{5.033431in}{3.543460in}}%
\pgfpathlineto{\pgfqpoint{5.036742in}{3.547044in}}%
\pgfpathlineto{\pgfqpoint{5.040053in}{3.547330in}}%
\pgfpathlineto{\pgfqpoint{5.043365in}{3.544330in}}%
\pgfpathlineto{\pgfqpoint{5.046676in}{3.538065in}}%
\pgfpathlineto{\pgfqpoint{5.053299in}{3.515870in}}%
\pgfpathlineto{\pgfqpoint{5.059922in}{3.481092in}}%
\pgfpathlineto{\pgfqpoint{5.066545in}{3.434221in}}%
\pgfpathlineto{\pgfqpoint{5.076479in}{3.342674in}}%
\pgfpathlineto{\pgfqpoint{5.086414in}{3.228181in}}%
\pgfpathlineto{\pgfqpoint{5.099660in}{3.046105in}}%
\pgfpathlineto{\pgfqpoint{5.119528in}{2.732217in}}%
\pgfpathlineto{\pgfqpoint{5.146020in}{2.310232in}}%
\pgfpathlineto{\pgfqpoint{5.159266in}{2.136517in}}%
\pgfpathlineto{\pgfqpoint{5.169200in}{2.039766in}}%
\pgfpathlineto{\pgfqpoint{5.175823in}{1.995934in}}%
\pgfpathlineto{\pgfqpoint{5.182446in}{1.970059in}}%
\pgfpathlineto{\pgfqpoint{5.185757in}{1.963684in}}%
\pgfpathlineto{\pgfqpoint{5.189069in}{1.961338in}}%
\pgfpathlineto{\pgfqpoint{5.192380in}{1.962624in}}%
\pgfpathlineto{\pgfqpoint{5.195692in}{1.967073in}}%
\pgfpathlineto{\pgfqpoint{5.202314in}{1.983403in}}%
\pgfpathlineto{\pgfqpoint{5.212249in}{2.019022in}}%
\pgfpathlineto{\pgfqpoint{5.232118in}{2.093525in}}%
\pgfpathlineto{\pgfqpoint{5.242052in}{2.120779in}}%
\pgfpathlineto{\pgfqpoint{5.248675in}{2.133495in}}%
\pgfpathlineto{\pgfqpoint{5.255298in}{2.141674in}}%
\pgfpathlineto{\pgfqpoint{5.261921in}{2.145470in}}%
\pgfpathlineto{\pgfqpoint{5.268543in}{2.145253in}}%
\pgfpathlineto{\pgfqpoint{5.275166in}{2.141569in}}%
\pgfpathlineto{\pgfqpoint{5.281789in}{2.135099in}}%
\pgfpathlineto{\pgfqpoint{5.291724in}{2.121900in}}%
\pgfpathlineto{\pgfqpoint{5.314904in}{2.089387in}}%
\pgfpathlineto{\pgfqpoint{5.321527in}{2.083076in}}%
\pgfpathlineto{\pgfqpoint{5.328150in}{2.079169in}}%
\pgfpathlineto{\pgfqpoint{5.334772in}{2.078056in}}%
\pgfpathlineto{\pgfqpoint{5.341395in}{2.080011in}}%
\pgfpathlineto{\pgfqpoint{5.348018in}{2.085227in}}%
\pgfpathlineto{\pgfqpoint{5.354641in}{2.093843in}}%
\pgfpathlineto{\pgfqpoint{5.361264in}{2.105984in}}%
\pgfpathlineto{\pgfqpoint{5.367887in}{2.121778in}}%
\pgfpathlineto{\pgfqpoint{5.377821in}{2.152635in}}%
\pgfpathlineto{\pgfqpoint{5.387756in}{2.192541in}}%
\pgfpathlineto{\pgfqpoint{5.397690in}{2.241861in}}%
\pgfpathlineto{\pgfqpoint{5.410936in}{2.322155in}}%
\pgfpathlineto{\pgfqpoint{5.424182in}{2.417309in}}%
\pgfpathlineto{\pgfqpoint{5.444050in}{2.578251in}}%
\pgfpathlineto{\pgfqpoint{5.467230in}{2.765433in}}%
\pgfpathlineto{\pgfqpoint{5.480476in}{2.855853in}}%
\pgfpathlineto{\pgfqpoint{5.490411in}{2.909457in}}%
\pgfpathlineto{\pgfqpoint{5.497033in}{2.936731in}}%
\pgfpathlineto{\pgfqpoint{5.503656in}{2.956346in}}%
\pgfpathlineto{\pgfqpoint{5.510279in}{2.967670in}}%
\pgfpathlineto{\pgfqpoint{5.513591in}{2.970063in}}%
\pgfpathlineto{\pgfqpoint{5.516902in}{2.970207in}}%
\pgfpathlineto{\pgfqpoint{5.520214in}{2.968060in}}%
\pgfpathlineto{\pgfqpoint{5.523525in}{2.963590in}}%
\pgfpathlineto{\pgfqpoint{5.530148in}{2.947590in}}%
\pgfpathlineto{\pgfqpoint{5.536771in}{2.922113in}}%
\pgfpathlineto{\pgfqpoint{5.543394in}{2.887198in}}%
\pgfpathlineto{\pgfqpoint{5.553328in}{2.817529in}}%
\pgfpathlineto{\pgfqpoint{5.563262in}{2.728122in}}%
\pgfpathlineto{\pgfqpoint{5.576508in}{2.581258in}}%
\pgfpathlineto{\pgfqpoint{5.589754in}{2.408190in}}%
\pgfpathlineto{\pgfqpoint{5.612934in}{2.067330in}}%
\pgfpathlineto{\pgfqpoint{5.632803in}{1.782051in}}%
\pgfpathlineto{\pgfqpoint{5.642737in}{1.673429in}}%
\pgfpathlineto{\pgfqpoint{5.646049in}{1.649481in}}%
\pgfpathlineto{\pgfqpoint{5.649360in}{1.634510in}}%
\pgfpathlineto{\pgfqpoint{5.652672in}{1.629613in}}%
\pgfpathlineto{\pgfqpoint{5.655983in}{1.634531in}}%
\pgfpathlineto{\pgfqpoint{5.659295in}{1.647628in}}%
\pgfpathlineto{\pgfqpoint{5.665917in}{1.689273in}}%
\pgfpathlineto{\pgfqpoint{5.685786in}{1.832214in}}%
\pgfpathlineto{\pgfqpoint{5.692409in}{1.868304in}}%
\pgfpathlineto{\pgfqpoint{5.699032in}{1.895197in}}%
\pgfpathlineto{\pgfqpoint{5.705655in}{1.912112in}}%
\pgfpathlineto{\pgfqpoint{5.708966in}{1.916710in}}%
\pgfpathlineto{\pgfqpoint{5.712278in}{1.918726in}}%
\pgfpathlineto{\pgfqpoint{5.715589in}{1.918182in}}%
\pgfpathlineto{\pgfqpoint{5.718901in}{1.915132in}}%
\pgfpathlineto{\pgfqpoint{5.722212in}{1.909660in}}%
\pgfpathlineto{\pgfqpoint{5.728835in}{1.891982in}}%
\pgfpathlineto{\pgfqpoint{5.735458in}{1.866719in}}%
\pgfpathlineto{\pgfqpoint{5.755327in}{1.781574in}}%
\pgfpathlineto{\pgfqpoint{5.758638in}{1.774646in}}%
\pgfpathlineto{\pgfqpoint{5.761949in}{1.772860in}}%
\pgfpathlineto{\pgfqpoint{5.765261in}{1.777304in}}%
\pgfpathlineto{\pgfqpoint{5.768572in}{1.788716in}}%
\pgfpathlineto{\pgfqpoint{5.771884in}{1.807371in}}%
\pgfpathlineto{\pgfqpoint{5.778507in}{1.865323in}}%
\pgfpathlineto{\pgfqpoint{5.785130in}{1.946457in}}%
\pgfpathlineto{\pgfqpoint{5.795064in}{2.098746in}}%
\pgfpathlineto{\pgfqpoint{5.811621in}{2.396590in}}%
\pgfpathlineto{\pgfqpoint{5.841424in}{2.943581in}}%
\pgfpathlineto{\pgfqpoint{5.854670in}{3.151696in}}%
\pgfpathlineto{\pgfqpoint{5.864604in}{3.283098in}}%
\pgfpathlineto{\pgfqpoint{5.874539in}{3.389329in}}%
\pgfpathlineto{\pgfqpoint{5.884473in}{3.467799in}}%
\pgfpathlineto{\pgfqpoint{5.891096in}{3.503829in}}%
\pgfpathlineto{\pgfqpoint{5.897719in}{3.526486in}}%
\pgfpathlineto{\pgfqpoint{5.901030in}{3.532758in}}%
\pgfpathlineto{\pgfqpoint{5.904342in}{3.535657in}}%
\pgfpathlineto{\pgfqpoint{5.907653in}{3.535193in}}%
\pgfpathlineto{\pgfqpoint{5.910965in}{3.531386in}}%
\pgfpathlineto{\pgfqpoint{5.914276in}{3.524265in}}%
\pgfpathlineto{\pgfqpoint{5.920899in}{3.500239in}}%
\pgfpathlineto{\pgfqpoint{5.927522in}{3.463517in}}%
\pgfpathlineto{\pgfqpoint{5.934145in}{3.414636in}}%
\pgfpathlineto{\pgfqpoint{5.944079in}{3.320006in}}%
\pgfpathlineto{\pgfqpoint{5.954014in}{3.202319in}}%
\pgfpathlineto{\pgfqpoint{5.967259in}{3.015491in}}%
\pgfpathlineto{\pgfqpoint{5.983817in}{2.747874in}}%
\pgfpathlineto{\pgfqpoint{6.020243in}{2.141449in}}%
\pgfpathlineto{\pgfqpoint{6.033488in}{1.964775in}}%
\pgfpathlineto{\pgfqpoint{6.043423in}{1.867625in}}%
\pgfpathlineto{\pgfqpoint{6.050046in}{1.824281in}}%
\pgfpathlineto{\pgfqpoint{6.056669in}{1.798757in}}%
\pgfpathlineto{\pgfqpoint{6.059980in}{1.792061in}}%
\pgfpathlineto{\pgfqpoint{6.063291in}{1.788749in}}%
\pgfpathlineto{\pgfqpoint{6.066603in}{1.788183in}}%
\pgfpathlineto{\pgfqpoint{6.069914in}{1.789681in}}%
\pgfpathlineto{\pgfqpoint{6.076537in}{1.796204in}}%
\pgfpathlineto{\pgfqpoint{6.086472in}{1.806415in}}%
\pgfpathlineto{\pgfqpoint{6.089783in}{1.808239in}}%
\pgfpathlineto{\pgfqpoint{6.093094in}{1.808787in}}%
\pgfpathlineto{\pgfqpoint{6.096406in}{1.807867in}}%
\pgfpathlineto{\pgfqpoint{6.099717in}{1.805344in}}%
\pgfpathlineto{\pgfqpoint{6.106340in}{1.795217in}}%
\pgfpathlineto{\pgfqpoint{6.112963in}{1.778377in}}%
\pgfpathlineto{\pgfqpoint{6.122898in}{1.742709in}}%
\pgfpathlineto{\pgfqpoint{6.136143in}{1.690742in}}%
\pgfpathlineto{\pgfqpoint{6.139455in}{1.681495in}}%
\pgfpathlineto{\pgfqpoint{6.142766in}{1.675767in}}%
\pgfpathlineto{\pgfqpoint{6.146078in}{1.674646in}}%
\pgfpathlineto{\pgfqpoint{6.149389in}{1.679005in}}%
\pgfpathlineto{\pgfqpoint{6.152701in}{1.689330in}}%
\pgfpathlineto{\pgfqpoint{6.156012in}{1.705633in}}%
\pgfpathlineto{\pgfqpoint{6.162635in}{1.754377in}}%
\pgfpathlineto{\pgfqpoint{6.172569in}{1.857570in}}%
\pgfpathlineto{\pgfqpoint{6.185815in}{2.026672in}}%
\pgfpathlineto{\pgfqpoint{6.232175in}{2.644075in}}%
\pgfpathlineto{\pgfqpoint{6.245421in}{2.788789in}}%
\pgfpathlineto{\pgfqpoint{6.258667in}{2.910673in}}%
\pgfpathlineto{\pgfqpoint{6.268601in}{2.985426in}}%
\pgfpathlineto{\pgfqpoint{6.278536in}{3.044585in}}%
\pgfpathlineto{\pgfqpoint{6.285159in}{3.074860in}}%
\pgfpathlineto{\pgfqpoint{6.291781in}{3.097656in}}%
\pgfpathlineto{\pgfqpoint{6.298404in}{3.113080in}}%
\pgfpathlineto{\pgfqpoint{6.305027in}{3.121431in}}%
\pgfpathlineto{\pgfqpoint{6.308339in}{3.123086in}}%
\pgfpathlineto{\pgfqpoint{6.311650in}{3.123148in}}%
\pgfpathlineto{\pgfqpoint{6.314962in}{3.121696in}}%
\pgfpathlineto{\pgfqpoint{6.321585in}{3.114634in}}%
\pgfpathlineto{\pgfqpoint{6.328207in}{3.102870in}}%
\pgfpathlineto{\pgfqpoint{6.338142in}{3.079269in}}%
\pgfpathlineto{\pgfqpoint{6.358010in}{3.029030in}}%
\pgfpathlineto{\pgfqpoint{6.367945in}{3.010347in}}%
\pgfpathlineto{\pgfqpoint{6.374568in}{3.001697in}}%
\pgfpathlineto{\pgfqpoint{6.381191in}{2.996303in}}%
\pgfpathlineto{\pgfqpoint{6.387814in}{2.994172in}}%
\pgfpathlineto{\pgfqpoint{6.394436in}{2.995197in}}%
\pgfpathlineto{\pgfqpoint{6.401059in}{2.999159in}}%
\pgfpathlineto{\pgfqpoint{6.407682in}{3.005702in}}%
\pgfpathlineto{\pgfqpoint{6.417617in}{3.019292in}}%
\pgfpathlineto{\pgfqpoint{6.434174in}{3.047392in}}%
\pgfpathlineto{\pgfqpoint{6.460665in}{3.092295in}}%
\pgfpathlineto{\pgfqpoint{6.477223in}{3.116041in}}%
\pgfpathlineto{\pgfqpoint{6.493780in}{3.136096in}}%
\pgfpathlineto{\pgfqpoint{6.513649in}{3.156303in}}%
\pgfpathlineto{\pgfqpoint{6.566632in}{3.207253in}}%
\pgfpathlineto{\pgfqpoint{6.593123in}{3.234826in}}%
\pgfpathlineto{\pgfqpoint{6.599746in}{3.239400in}}%
\pgfpathlineto{\pgfqpoint{6.606369in}{3.241197in}}%
\pgfpathlineto{\pgfqpoint{6.609681in}{3.240689in}}%
\pgfpathlineto{\pgfqpoint{6.612992in}{3.239056in}}%
\pgfpathlineto{\pgfqpoint{6.619615in}{3.231925in}}%
\pgfpathlineto{\pgfqpoint{6.626238in}{3.218959in}}%
\pgfpathlineto{\pgfqpoint{6.632861in}{3.199556in}}%
\pgfpathlineto{\pgfqpoint{6.639484in}{3.173343in}}%
\pgfpathlineto{\pgfqpoint{6.649418in}{3.120961in}}%
\pgfpathlineto{\pgfqpoint{6.659352in}{3.053270in}}%
\pgfpathlineto{\pgfqpoint{6.672598in}{2.941704in}}%
\pgfpathlineto{\pgfqpoint{6.689155in}{2.777131in}}%
\pgfpathlineto{\pgfqpoint{6.718958in}{2.472783in}}%
\pgfpathlineto{\pgfqpoint{6.728893in}{2.390425in}}%
\pgfpathlineto{\pgfqpoint{6.738827in}{2.326318in}}%
\pgfpathlineto{\pgfqpoint{6.745450in}{2.294846in}}%
\pgfpathlineto{\pgfqpoint{6.752073in}{2.272119in}}%
\pgfpathlineto{\pgfqpoint{6.758696in}{2.257105in}}%
\pgfpathlineto{\pgfqpoint{6.765319in}{2.248271in}}%
\pgfpathlineto{\pgfqpoint{6.771942in}{2.243888in}}%
\pgfpathlineto{\pgfqpoint{6.778565in}{2.242319in}}%
\pgfpathlineto{\pgfqpoint{6.791810in}{2.242647in}}%
\pgfpathlineto{\pgfqpoint{6.814991in}{2.243073in}}%
\pgfpathlineto{\pgfqpoint{6.828236in}{2.243979in}}%
\pgfpathlineto{\pgfqpoint{6.838171in}{2.247211in}}%
\pgfpathlineto{\pgfqpoint{6.844794in}{2.251318in}}%
\pgfpathlineto{\pgfqpoint{6.851416in}{2.257362in}}%
\pgfpathlineto{\pgfqpoint{6.858039in}{2.265598in}}%
\pgfpathlineto{\pgfqpoint{6.867974in}{2.282528in}}%
\pgfpathlineto{\pgfqpoint{6.877908in}{2.305469in}}%
\pgfpathlineto{\pgfqpoint{6.887842in}{2.334745in}}%
\pgfpathlineto{\pgfqpoint{6.901088in}{2.383263in}}%
\pgfpathlineto{\pgfqpoint{6.917645in}{2.455924in}}%
\pgfpathlineto{\pgfqpoint{6.964006in}{2.679507in}}%
\pgfpathlineto{\pgfqpoint{6.993809in}{2.827154in}}%
\pgfpathlineto{\pgfqpoint{7.016989in}{2.953007in}}%
\pgfpathlineto{\pgfqpoint{7.046792in}{3.129025in}}%
\pgfpathlineto{\pgfqpoint{7.103087in}{3.477261in}}%
\pgfpathlineto{\pgfqpoint{7.132890in}{3.662752in}}%
\pgfpathlineto{\pgfqpoint{7.146136in}{3.733220in}}%
\pgfpathlineto{\pgfqpoint{7.156070in}{3.776820in}}%
\pgfpathlineto{\pgfqpoint{7.166004in}{3.810141in}}%
\pgfpathlineto{\pgfqpoint{7.172627in}{3.825702in}}%
\pgfpathlineto{\pgfqpoint{7.179250in}{3.835408in}}%
\pgfpathlineto{\pgfqpoint{7.182561in}{3.837967in}}%
\pgfpathlineto{\pgfqpoint{7.185873in}{3.838965in}}%
\pgfpathlineto{\pgfqpoint{7.189184in}{3.838398in}}%
\pgfpathlineto{\pgfqpoint{7.192496in}{3.836271in}}%
\pgfpathlineto{\pgfqpoint{7.199119in}{3.827450in}}%
\pgfpathlineto{\pgfqpoint{7.205742in}{3.812873in}}%
\pgfpathlineto{\pgfqpoint{7.212365in}{3.793163in}}%
\pgfpathlineto{\pgfqpoint{7.222299in}{3.755934in}}%
\pgfpathlineto{\pgfqpoint{7.258725in}{3.607205in}}%
\pgfpathlineto{\pgfqpoint{7.265348in}{3.591044in}}%
\pgfpathlineto{\pgfqpoint{7.271971in}{3.581891in}}%
\pgfpathlineto{\pgfqpoint{7.275282in}{3.580333in}}%
\pgfpathlineto{\pgfqpoint{7.278594in}{3.580974in}}%
\pgfpathlineto{\pgfqpoint{7.281905in}{3.583936in}}%
\pgfpathlineto{\pgfqpoint{7.285216in}{3.589328in}}%
\pgfpathlineto{\pgfqpoint{7.291839in}{3.607747in}}%
\pgfpathlineto{\pgfqpoint{7.298462in}{3.636743in}}%
\pgfpathlineto{\pgfqpoint{7.305085in}{3.676492in}}%
\pgfpathlineto{\pgfqpoint{7.315019in}{3.755756in}}%
\pgfpathlineto{\pgfqpoint{7.324954in}{3.856448in}}%
\pgfpathlineto{\pgfqpoint{7.338200in}{4.017087in}}%
\pgfpathlineto{\pgfqpoint{7.361380in}{4.336444in}}%
\pgfpathlineto{\pgfqpoint{7.387871in}{4.694146in}}%
\pgfpathlineto{\pgfqpoint{7.407740in}{4.928468in}}%
\pgfpathlineto{\pgfqpoint{7.437543in}{5.242720in}}%
\pgfpathlineto{\pgfqpoint{7.454100in}{5.429261in}}%
\pgfpathlineto{\pgfqpoint{7.467346in}{5.602701in}}%
\pgfpathlineto{\pgfqpoint{7.480592in}{5.808280in}}%
\pgfpathlineto{\pgfqpoint{7.493838in}{6.053510in}}%
\pgfpathlineto{\pgfqpoint{7.507084in}{6.342528in}}%
\pgfpathlineto{\pgfqpoint{7.523641in}{6.764045in}}%
\pgfpathlineto{\pgfqpoint{7.546821in}{7.429951in}}%
\pgfpathlineto{\pgfqpoint{7.560067in}{7.794668in}}%
\pgfpathlineto{\pgfqpoint{7.566690in}{7.945178in}}%
\pgfpathlineto{\pgfqpoint{7.573313in}{8.053418in}}%
\pgfpathlineto{\pgfqpoint{7.576624in}{8.084781in}}%
\pgfpathlineto{\pgfqpoint{7.579935in}{8.096554in}}%
\pgfpathlineto{\pgfqpoint{7.583247in}{8.085051in}}%
\pgfpathlineto{\pgfqpoint{7.586558in}{8.046445in}}%
\pgfpathlineto{\pgfqpoint{7.589870in}{7.976999in}}%
\pgfpathlineto{\pgfqpoint{7.593181in}{7.873395in}}%
\pgfpathlineto{\pgfqpoint{7.599804in}{7.555100in}}%
\pgfpathlineto{\pgfqpoint{7.606427in}{7.090326in}}%
\pgfpathlineto{\pgfqpoint{7.619673in}{5.882342in}}%
\pgfpathlineto{\pgfqpoint{7.629607in}{5.011003in}}%
\pgfpathlineto{\pgfqpoint{7.639542in}{4.355707in}}%
\pgfpathlineto{\pgfqpoint{7.649476in}{3.877275in}}%
\pgfpathlineto{\pgfqpoint{7.662722in}{3.345100in}}%
\pgfpathlineto{\pgfqpoint{7.666033in}{3.253597in}}%
\pgfpathlineto{\pgfqpoint{7.669345in}{3.206430in}}%
\pgfpathlineto{\pgfqpoint{7.672656in}{3.219861in}}%
\pgfpathlineto{\pgfqpoint{7.675968in}{3.302184in}}%
\pgfpathlineto{\pgfqpoint{7.679279in}{3.449741in}}%
\pgfpathlineto{\pgfqpoint{7.685902in}{3.883960in}}%
\pgfpathlineto{\pgfqpoint{7.705771in}{5.358938in}}%
\pgfpathlineto{\pgfqpoint{7.715705in}{5.931443in}}%
\pgfpathlineto{\pgfqpoint{7.725639in}{6.388330in}}%
\pgfpathlineto{\pgfqpoint{7.735574in}{6.756138in}}%
\pgfpathlineto{\pgfqpoint{7.745508in}{7.050498in}}%
\pgfpathlineto{\pgfqpoint{7.755442in}{7.277611in}}%
\pgfpathlineto{\pgfqpoint{7.765377in}{7.441750in}}%
\pgfpathlineto{\pgfqpoint{7.772000in}{7.519569in}}%
\pgfpathlineto{\pgfqpoint{7.778622in}{7.575488in}}%
\pgfpathlineto{\pgfqpoint{7.785245in}{7.612883in}}%
\pgfpathlineto{\pgfqpoint{7.791868in}{7.635209in}}%
\pgfpathlineto{\pgfqpoint{7.795180in}{7.641756in}}%
\pgfpathlineto{\pgfqpoint{7.798491in}{7.645739in}}%
\pgfpathlineto{\pgfqpoint{7.801803in}{7.647508in}}%
\pgfpathlineto{\pgfqpoint{7.805114in}{7.647381in}}%
\pgfpathlineto{\pgfqpoint{7.808426in}{7.645645in}}%
\pgfpathlineto{\pgfqpoint{7.815048in}{7.638329in}}%
\pgfpathlineto{\pgfqpoint{7.821671in}{7.627194in}}%
\pgfpathlineto{\pgfqpoint{7.831606in}{7.605754in}}%
\pgfpathlineto{\pgfqpoint{7.848163in}{7.562722in}}%
\pgfpathlineto{\pgfqpoint{7.871343in}{7.500193in}}%
\pgfpathlineto{\pgfqpoint{7.881277in}{7.479253in}}%
\pgfpathlineto{\pgfqpoint{7.887900in}{7.469357in}}%
\pgfpathlineto{\pgfqpoint{7.894523in}{7.463367in}}%
\pgfpathlineto{\pgfqpoint{7.901146in}{7.461409in}}%
\pgfpathlineto{\pgfqpoint{7.907769in}{7.463183in}}%
\pgfpathlineto{\pgfqpoint{7.914392in}{7.468022in}}%
\pgfpathlineto{\pgfqpoint{7.924326in}{7.479049in}}%
\pgfpathlineto{\pgfqpoint{7.944195in}{7.503310in}}%
\pgfpathlineto{\pgfqpoint{7.954129in}{7.511501in}}%
\pgfpathlineto{\pgfqpoint{7.960752in}{7.514200in}}%
\pgfpathlineto{\pgfqpoint{7.967375in}{7.514161in}}%
\pgfpathlineto{\pgfqpoint{7.973998in}{7.511005in}}%
\pgfpathlineto{\pgfqpoint{7.980621in}{7.504438in}}%
\pgfpathlineto{\pgfqpoint{7.987244in}{7.494298in}}%
\pgfpathlineto{\pgfqpoint{7.993867in}{7.480597in}}%
\pgfpathlineto{\pgfqpoint{8.003801in}{7.453973in}}%
\pgfpathlineto{\pgfqpoint{8.020358in}{7.399308in}}%
\pgfpathlineto{\pgfqpoint{8.033604in}{7.357278in}}%
\pgfpathlineto{\pgfqpoint{8.040227in}{7.341133in}}%
\pgfpathlineto{\pgfqpoint{8.046850in}{7.330308in}}%
\pgfpathlineto{\pgfqpoint{8.050161in}{7.327285in}}%
\pgfpathlineto{\pgfqpoint{8.053473in}{7.326011in}}%
\pgfpathlineto{\pgfqpoint{8.056784in}{7.326558in}}%
\pgfpathlineto{\pgfqpoint{8.060096in}{7.328962in}}%
\pgfpathlineto{\pgfqpoint{8.066719in}{7.339277in}}%
\pgfpathlineto{\pgfqpoint{8.073341in}{7.356437in}}%
\pgfpathlineto{\pgfqpoint{8.083276in}{7.392541in}}%
\pgfpathlineto{\pgfqpoint{8.099833in}{7.467253in}}%
\pgfpathlineto{\pgfqpoint{8.116390in}{7.539671in}}%
\pgfpathlineto{\pgfqpoint{8.126325in}{7.575433in}}%
\pgfpathlineto{\pgfqpoint{8.136259in}{7.603568in}}%
\pgfpathlineto{\pgfqpoint{8.146193in}{7.623587in}}%
\pgfpathlineto{\pgfqpoint{8.152816in}{7.632319in}}%
\pgfpathlineto{\pgfqpoint{8.159439in}{7.637266in}}%
\pgfpathlineto{\pgfqpoint{8.166062in}{7.638291in}}%
\pgfpathlineto{\pgfqpoint{8.172685in}{7.635199in}}%
\pgfpathlineto{\pgfqpoint{8.179308in}{7.627752in}}%
\pgfpathlineto{\pgfqpoint{8.185931in}{7.615704in}}%
\pgfpathlineto{\pgfqpoint{8.192554in}{7.598866in}}%
\pgfpathlineto{\pgfqpoint{8.202488in}{7.564590in}}%
\pgfpathlineto{\pgfqpoint{8.212422in}{7.520472in}}%
\pgfpathlineto{\pgfqpoint{8.228980in}{7.433060in}}%
\pgfpathlineto{\pgfqpoint{8.248848in}{7.319769in}}%
\pgfpathlineto{\pgfqpoint{8.258783in}{7.250828in}}%
\pgfpathlineto{\pgfqpoint{8.268717in}{7.163885in}}%
\pgfpathlineto{\pgfqpoint{8.285274in}{7.011371in}}%
\pgfpathlineto{\pgfqpoint{8.291897in}{6.966710in}}%
\pgfpathlineto{\pgfqpoint{8.298520in}{6.935815in}}%
\pgfpathlineto{\pgfqpoint{8.305143in}{6.917137in}}%
\pgfpathlineto{\pgfqpoint{8.311766in}{6.907748in}}%
\pgfpathlineto{\pgfqpoint{8.318389in}{6.904629in}}%
\pgfpathlineto{\pgfqpoint{8.325012in}{6.905264in}}%
\pgfpathlineto{\pgfqpoint{8.338257in}{6.910837in}}%
\pgfpathlineto{\pgfqpoint{8.351503in}{6.915510in}}%
\pgfpathlineto{\pgfqpoint{8.361438in}{6.916250in}}%
\pgfpathlineto{\pgfqpoint{8.371372in}{6.914400in}}%
\pgfpathlineto{\pgfqpoint{8.387929in}{6.907912in}}%
\pgfpathlineto{\pgfqpoint{8.397864in}{6.904558in}}%
\pgfpathlineto{\pgfqpoint{8.404486in}{6.903722in}}%
\pgfpathlineto{\pgfqpoint{8.411109in}{6.904645in}}%
\pgfpathlineto{\pgfqpoint{8.417732in}{6.907803in}}%
\pgfpathlineto{\pgfqpoint{8.424355in}{6.913572in}}%
\pgfpathlineto{\pgfqpoint{8.430978in}{6.922201in}}%
\pgfpathlineto{\pgfqpoint{8.437601in}{6.933773in}}%
\pgfpathlineto{\pgfqpoint{8.447535in}{6.956281in}}%
\pgfpathlineto{\pgfqpoint{8.460781in}{6.993299in}}%
\pgfpathlineto{\pgfqpoint{8.483961in}{7.065680in}}%
\pgfpathlineto{\pgfqpoint{8.497207in}{7.113094in}}%
\pgfpathlineto{\pgfqpoint{8.510453in}{7.169399in}}%
\pgfpathlineto{\pgfqpoint{8.527010in}{7.251380in}}%
\pgfpathlineto{\pgfqpoint{8.560125in}{7.420020in}}%
\pgfpathlineto{\pgfqpoint{8.570059in}{7.460962in}}%
\pgfpathlineto{\pgfqpoint{8.579993in}{7.493637in}}%
\pgfpathlineto{\pgfqpoint{8.586616in}{7.510283in}}%
\pgfpathlineto{\pgfqpoint{8.593239in}{7.522668in}}%
\pgfpathlineto{\pgfqpoint{8.599862in}{7.530812in}}%
\pgfpathlineto{\pgfqpoint{8.606485in}{7.534848in}}%
\pgfpathlineto{\pgfqpoint{8.613108in}{7.534998in}}%
\pgfpathlineto{\pgfqpoint{8.619731in}{7.531560in}}%
\pgfpathlineto{\pgfqpoint{8.626354in}{7.524885in}}%
\pgfpathlineto{\pgfqpoint{8.632977in}{7.515348in}}%
\pgfpathlineto{\pgfqpoint{8.642911in}{7.496449in}}%
\pgfpathlineto{\pgfqpoint{8.652845in}{7.472778in}}%
\pgfpathlineto{\pgfqpoint{8.662780in}{7.444420in}}%
\pgfpathlineto{\pgfqpoint{8.672714in}{7.410463in}}%
\pgfpathlineto{\pgfqpoint{8.682648in}{7.369526in}}%
\pgfpathlineto{\pgfqpoint{8.695894in}{7.303272in}}%
\pgfpathlineto{\pgfqpoint{8.712451in}{7.206829in}}%
\pgfpathlineto{\pgfqpoint{8.755500in}{6.950369in}}%
\pgfpathlineto{\pgfqpoint{8.778680in}{6.826839in}}%
\pgfpathlineto{\pgfqpoint{8.795238in}{6.748939in}}%
\pgfpathlineto{\pgfqpoint{8.808483in}{6.695119in}}%
\pgfpathlineto{\pgfqpoint{8.821729in}{6.650893in}}%
\pgfpathlineto{\pgfqpoint{8.831664in}{6.625081in}}%
\pgfpathlineto{\pgfqpoint{8.841598in}{6.606175in}}%
\pgfpathlineto{\pgfqpoint{8.848221in}{6.597531in}}%
\pgfpathlineto{\pgfqpoint{8.854844in}{6.592027in}}%
\pgfpathlineto{\pgfqpoint{8.861467in}{6.589513in}}%
\pgfpathlineto{\pgfqpoint{8.868089in}{6.589685in}}%
\pgfpathlineto{\pgfqpoint{8.874712in}{6.592055in}}%
\pgfpathlineto{\pgfqpoint{8.884647in}{6.598306in}}%
\pgfpathlineto{\pgfqpoint{8.924384in}{6.628888in}}%
\pgfpathlineto{\pgfqpoint{8.934318in}{6.639346in}}%
\pgfpathlineto{\pgfqpoint{8.944253in}{6.652531in}}%
\pgfpathlineto{\pgfqpoint{8.957499in}{6.674604in}}%
\pgfpathlineto{\pgfqpoint{8.970744in}{6.701219in}}%
\pgfpathlineto{\pgfqpoint{9.010482in}{6.786048in}}%
\pgfpathlineto{\pgfqpoint{9.020416in}{6.800562in}}%
\pgfpathlineto{\pgfqpoint{9.027039in}{6.807072in}}%
\pgfpathlineto{\pgfqpoint{9.033662in}{6.810610in}}%
\pgfpathlineto{\pgfqpoint{9.040285in}{6.810882in}}%
\pgfpathlineto{\pgfqpoint{9.046908in}{6.807648in}}%
\pgfpathlineto{\pgfqpoint{9.053531in}{6.800703in}}%
\pgfpathlineto{\pgfqpoint{9.060154in}{6.789824in}}%
\pgfpathlineto{\pgfqpoint{9.066776in}{6.774707in}}%
\pgfpathlineto{\pgfqpoint{9.073399in}{6.754862in}}%
\pgfpathlineto{\pgfqpoint{9.080022in}{6.729487in}}%
\pgfpathlineto{\pgfqpoint{9.086645in}{6.697298in}}%
\pgfpathlineto{\pgfqpoint{9.093268in}{6.656337in}}%
\pgfpathlineto{\pgfqpoint{9.099891in}{6.603791in}}%
\pgfpathlineto{\pgfqpoint{9.106514in}{6.535971in}}%
\pgfpathlineto{\pgfqpoint{9.113137in}{6.448710in}}%
\pgfpathlineto{\pgfqpoint{9.119760in}{6.338690in}}%
\pgfpathlineto{\pgfqpoint{9.129694in}{6.133238in}}%
\pgfpathlineto{\pgfqpoint{9.146251in}{5.782393in}}%
\pgfpathlineto{\pgfqpoint{9.152874in}{5.690075in}}%
\pgfpathlineto{\pgfqpoint{9.159497in}{5.637143in}}%
\pgfpathlineto{\pgfqpoint{9.162809in}{5.623471in}}%
\pgfpathlineto{\pgfqpoint{9.166120in}{5.616350in}}%
\pgfpathlineto{\pgfqpoint{9.169431in}{5.614096in}}%
\pgfpathlineto{\pgfqpoint{9.172743in}{5.615066in}}%
\pgfpathlineto{\pgfqpoint{9.185989in}{5.624832in}}%
\pgfpathlineto{\pgfqpoint{9.189300in}{5.624156in}}%
\pgfpathlineto{\pgfqpoint{9.192612in}{5.621148in}}%
\pgfpathlineto{\pgfqpoint{9.195923in}{5.615566in}}%
\pgfpathlineto{\pgfqpoint{9.202546in}{5.596319in}}%
\pgfpathlineto{\pgfqpoint{9.209169in}{5.566531in}}%
\pgfpathlineto{\pgfqpoint{9.215792in}{5.527198in}}%
\pgfpathlineto{\pgfqpoint{9.225726in}{5.453414in}}%
\pgfpathlineto{\pgfqpoint{9.238972in}{5.335309in}}%
\pgfpathlineto{\pgfqpoint{9.265463in}{5.069985in}}%
\pgfpathlineto{\pgfqpoint{9.288644in}{4.844549in}}%
\pgfpathlineto{\pgfqpoint{9.305201in}{4.702472in}}%
\pgfpathlineto{\pgfqpoint{9.318447in}{4.605928in}}%
\pgfpathlineto{\pgfqpoint{9.328381in}{4.546648in}}%
\pgfpathlineto{\pgfqpoint{9.338315in}{4.499749in}}%
\pgfpathlineto{\pgfqpoint{9.348250in}{4.462722in}}%
\pgfpathlineto{\pgfqpoint{9.361495in}{4.421430in}}%
\pgfpathlineto{\pgfqpoint{9.371430in}{4.396156in}}%
\pgfpathlineto{\pgfqpoint{9.378053in}{4.383397in}}%
\pgfpathlineto{\pgfqpoint{9.384676in}{4.373925in}}%
\pgfpathlineto{\pgfqpoint{9.394610in}{4.363956in}}%
\pgfpathlineto{\pgfqpoint{9.417790in}{4.346142in}}%
\pgfpathlineto{\pgfqpoint{9.427724in}{4.340774in}}%
\pgfpathlineto{\pgfqpoint{9.434347in}{4.340354in}}%
\pgfpathlineto{\pgfqpoint{9.437659in}{4.341677in}}%
\pgfpathlineto{\pgfqpoint{9.444282in}{4.348256in}}%
\pgfpathlineto{\pgfqpoint{9.450905in}{4.360719in}}%
\pgfpathlineto{\pgfqpoint{9.457528in}{4.378922in}}%
\pgfpathlineto{\pgfqpoint{9.467462in}{4.414349in}}%
\pgfpathlineto{\pgfqpoint{9.503888in}{4.554420in}}%
\pgfpathlineto{\pgfqpoint{9.523757in}{4.626011in}}%
\pgfpathlineto{\pgfqpoint{9.533691in}{4.671823in}}%
\pgfpathlineto{\pgfqpoint{9.543625in}{4.729748in}}%
\pgfpathlineto{\pgfqpoint{9.556871in}{4.824730in}}%
\pgfpathlineto{\pgfqpoint{9.603231in}{5.181337in}}%
\pgfpathlineto{\pgfqpoint{9.613166in}{5.236576in}}%
\pgfpathlineto{\pgfqpoint{9.623100in}{5.281855in}}%
\pgfpathlineto{\pgfqpoint{9.636346in}{5.330148in}}%
\pgfpathlineto{\pgfqpoint{9.652903in}{5.379895in}}%
\pgfpathlineto{\pgfqpoint{9.699263in}{5.510163in}}%
\pgfpathlineto{\pgfqpoint{9.705886in}{5.522752in}}%
\pgfpathlineto{\pgfqpoint{9.712509in}{5.529922in}}%
\pgfpathlineto{\pgfqpoint{9.715821in}{5.530685in}}%
\pgfpathlineto{\pgfqpoint{9.719132in}{5.529030in}}%
\pgfpathlineto{\pgfqpoint{9.722444in}{5.524464in}}%
\pgfpathlineto{\pgfqpoint{9.725755in}{5.516405in}}%
\pgfpathlineto{\pgfqpoint{9.729066in}{5.504168in}}%
\pgfpathlineto{\pgfqpoint{9.732378in}{5.486953in}}%
\pgfpathlineto{\pgfqpoint{9.739001in}{5.433764in}}%
\pgfpathlineto{\pgfqpoint{9.745624in}{5.348059in}}%
\pgfpathlineto{\pgfqpoint{9.752247in}{5.220084in}}%
\pgfpathlineto{\pgfqpoint{9.758869in}{5.041680in}}%
\pgfpathlineto{\pgfqpoint{9.765492in}{4.810555in}}%
\pgfpathlineto{\pgfqpoint{9.778738in}{4.232464in}}%
\pgfpathlineto{\pgfqpoint{9.791984in}{3.653196in}}%
\pgfpathlineto{\pgfqpoint{9.801918in}{3.318675in}}%
\pgfpathlineto{\pgfqpoint{9.808541in}{3.155152in}}%
\pgfpathlineto{\pgfqpoint{9.815164in}{3.032660in}}%
\pgfpathlineto{\pgfqpoint{9.821787in}{2.941133in}}%
\pgfpathlineto{\pgfqpoint{9.831721in}{2.839727in}}%
\pgfpathlineto{\pgfqpoint{9.848279in}{2.706206in}}%
\pgfpathlineto{\pgfqpoint{9.874770in}{2.490826in}}%
\pgfpathlineto{\pgfqpoint{9.911196in}{2.196101in}}%
\pgfpathlineto{\pgfqpoint{9.954245in}{1.867596in}}%
\pgfpathlineto{\pgfqpoint{9.960868in}{1.831192in}}%
\pgfpathlineto{\pgfqpoint{9.967491in}{1.805075in}}%
\pgfpathlineto{\pgfqpoint{9.974114in}{1.790741in}}%
\pgfpathlineto{\pgfqpoint{9.977425in}{1.788121in}}%
\pgfpathlineto{\pgfqpoint{9.980737in}{1.788436in}}%
\pgfpathlineto{\pgfqpoint{9.984048in}{1.791527in}}%
\pgfpathlineto{\pgfqpoint{9.987360in}{1.797181in}}%
\pgfpathlineto{\pgfqpoint{9.993982in}{1.815165in}}%
\pgfpathlineto{\pgfqpoint{10.000605in}{1.840241in}}%
\pgfpathlineto{\pgfqpoint{10.010540in}{1.886648in}}%
\pgfpathlineto{\pgfqpoint{10.056900in}{2.118299in}}%
\pgfpathlineto{\pgfqpoint{10.070146in}{2.167489in}}%
\pgfpathlineto{\pgfqpoint{10.080080in}{2.196615in}}%
\pgfpathlineto{\pgfqpoint{10.090014in}{2.218611in}}%
\pgfpathlineto{\pgfqpoint{10.096637in}{2.229148in}}%
\pgfpathlineto{\pgfqpoint{10.103260in}{2.236276in}}%
\pgfpathlineto{\pgfqpoint{10.109883in}{2.239925in}}%
\pgfpathlineto{\pgfqpoint{10.116506in}{2.240087in}}%
\pgfpathlineto{\pgfqpoint{10.123129in}{2.236890in}}%
\pgfpathlineto{\pgfqpoint{10.129752in}{2.230688in}}%
\pgfpathlineto{\pgfqpoint{10.139686in}{2.217241in}}%
\pgfpathlineto{\pgfqpoint{10.166178in}{2.178041in}}%
\pgfpathlineto{\pgfqpoint{10.182735in}{2.154858in}}%
\pgfpathlineto{\pgfqpoint{10.192669in}{2.136892in}}%
\pgfpathlineto{\pgfqpoint{10.205915in}{2.107251in}}%
\pgfpathlineto{\pgfqpoint{10.232407in}{2.045212in}}%
\pgfpathlineto{\pgfqpoint{10.242341in}{2.027572in}}%
\pgfpathlineto{\pgfqpoint{10.248964in}{2.018876in}}%
\pgfpathlineto{\pgfqpoint{10.255587in}{2.013067in}}%
\pgfpathlineto{\pgfqpoint{10.262210in}{2.010439in}}%
\pgfpathlineto{\pgfqpoint{10.268833in}{2.011170in}}%
\pgfpathlineto{\pgfqpoint{10.275456in}{2.015273in}}%
\pgfpathlineto{\pgfqpoint{10.282079in}{2.022546in}}%
\pgfpathlineto{\pgfqpoint{10.292013in}{2.038358in}}%
\pgfpathlineto{\pgfqpoint{10.325127in}{2.098932in}}%
\pgfpathlineto{\pgfqpoint{10.331750in}{2.105301in}}%
\pgfpathlineto{\pgfqpoint{10.338373in}{2.108184in}}%
\pgfpathlineto{\pgfqpoint{10.344996in}{2.107705in}}%
\pgfpathlineto{\pgfqpoint{10.351619in}{2.104571in}}%
\pgfpathlineto{\pgfqpoint{10.364865in}{2.094714in}}%
\pgfpathlineto{\pgfqpoint{10.381422in}{2.080500in}}%
\pgfpathlineto{\pgfqpoint{10.388045in}{2.071905in}}%
\pgfpathlineto{\pgfqpoint{10.394668in}{2.059789in}}%
\pgfpathlineto{\pgfqpoint{10.401291in}{2.043511in}}%
\pgfpathlineto{\pgfqpoint{10.411225in}{2.011667in}}%
\pgfpathlineto{\pgfqpoint{10.427782in}{1.946457in}}%
\pgfpathlineto{\pgfqpoint{10.444340in}{1.882970in}}%
\pgfpathlineto{\pgfqpoint{10.454274in}{1.853580in}}%
\pgfpathlineto{\pgfqpoint{10.460897in}{1.839383in}}%
\pgfpathlineto{\pgfqpoint{10.467520in}{1.829771in}}%
\pgfpathlineto{\pgfqpoint{10.474143in}{1.824410in}}%
\pgfpathlineto{\pgfqpoint{10.480766in}{1.822594in}}%
\pgfpathlineto{\pgfqpoint{10.487388in}{1.823452in}}%
\pgfpathlineto{\pgfqpoint{10.497323in}{1.828034in}}%
\pgfpathlineto{\pgfqpoint{10.507257in}{1.835143in}}%
\pgfpathlineto{\pgfqpoint{10.517191in}{1.844506in}}%
\pgfpathlineto{\pgfqpoint{10.527126in}{1.856495in}}%
\pgfpathlineto{\pgfqpoint{10.550306in}{1.886522in}}%
\pgfpathlineto{\pgfqpoint{10.560240in}{1.894416in}}%
\pgfpathlineto{\pgfqpoint{10.570175in}{1.898703in}}%
\pgfpathlineto{\pgfqpoint{10.576798in}{1.899676in}}%
\pgfpathlineto{\pgfqpoint{10.583420in}{1.898945in}}%
\pgfpathlineto{\pgfqpoint{10.590043in}{1.896267in}}%
\pgfpathlineto{\pgfqpoint{10.596666in}{1.891555in}}%
\pgfpathlineto{\pgfqpoint{10.606601in}{1.881333in}}%
\pgfpathlineto{\pgfqpoint{10.619846in}{1.866708in}}%
\pgfpathlineto{\pgfqpoint{10.626469in}{1.862142in}}%
\pgfpathlineto{\pgfqpoint{10.633092in}{1.861174in}}%
\pgfpathlineto{\pgfqpoint{10.639715in}{1.864326in}}%
\pgfpathlineto{\pgfqpoint{10.646338in}{1.871245in}}%
\pgfpathlineto{\pgfqpoint{10.656272in}{1.886765in}}%
\pgfpathlineto{\pgfqpoint{10.669518in}{1.913058in}}%
\pgfpathlineto{\pgfqpoint{10.689387in}{1.953899in}}%
\pgfpathlineto{\pgfqpoint{10.696010in}{1.964107in}}%
\pgfpathlineto{\pgfqpoint{10.702633in}{1.970860in}}%
\pgfpathlineto{\pgfqpoint{10.709256in}{1.973595in}}%
\pgfpathlineto{\pgfqpoint{10.715878in}{1.972472in}}%
\pgfpathlineto{\pgfqpoint{10.722501in}{1.968461in}}%
\pgfpathlineto{\pgfqpoint{10.742370in}{1.954082in}}%
\pgfpathlineto{\pgfqpoint{10.748993in}{1.951978in}}%
\pgfpathlineto{\pgfqpoint{10.755616in}{1.951708in}}%
\pgfpathlineto{\pgfqpoint{10.762239in}{1.953292in}}%
\pgfpathlineto{\pgfqpoint{10.768862in}{1.956875in}}%
\pgfpathlineto{\pgfqpoint{10.775485in}{1.962704in}}%
\pgfpathlineto{\pgfqpoint{10.782107in}{1.970997in}}%
\pgfpathlineto{\pgfqpoint{10.792042in}{1.988190in}}%
\pgfpathlineto{\pgfqpoint{10.801976in}{2.010589in}}%
\pgfpathlineto{\pgfqpoint{10.815222in}{2.046400in}}%
\pgfpathlineto{\pgfqpoint{10.851648in}{2.150053in}}%
\pgfpathlineto{\pgfqpoint{10.861582in}{2.171400in}}%
\pgfpathlineto{\pgfqpoint{10.871517in}{2.187525in}}%
\pgfpathlineto{\pgfqpoint{10.881451in}{2.199071in}}%
\pgfpathlineto{\pgfqpoint{10.891385in}{2.207193in}}%
\pgfpathlineto{\pgfqpoint{10.901320in}{2.212848in}}%
\pgfpathlineto{\pgfqpoint{10.914565in}{2.217537in}}%
\pgfpathlineto{\pgfqpoint{10.927811in}{2.219615in}}%
\pgfpathlineto{\pgfqpoint{10.944369in}{2.219457in}}%
\pgfpathlineto{\pgfqpoint{10.960926in}{2.217067in}}%
\pgfpathlineto{\pgfqpoint{10.977483in}{2.212229in}}%
\pgfpathlineto{\pgfqpoint{10.994040in}{2.204705in}}%
\pgfpathlineto{\pgfqpoint{11.027155in}{2.188349in}}%
\pgfpathlineto{\pgfqpoint{11.037089in}{2.186243in}}%
\pgfpathlineto{\pgfqpoint{11.043712in}{2.186401in}}%
\pgfpathlineto{\pgfqpoint{11.050335in}{2.188157in}}%
\pgfpathlineto{\pgfqpoint{11.056958in}{2.191841in}}%
\pgfpathlineto{\pgfqpoint{11.063581in}{2.197824in}}%
\pgfpathlineto{\pgfqpoint{11.070204in}{2.206522in}}%
\pgfpathlineto{\pgfqpoint{11.076827in}{2.218349in}}%
\pgfpathlineto{\pgfqpoint{11.083449in}{2.233630in}}%
\pgfpathlineto{\pgfqpoint{11.093384in}{2.263320in}}%
\pgfpathlineto{\pgfqpoint{11.103318in}{2.300870in}}%
\pgfpathlineto{\pgfqpoint{11.116564in}{2.361698in}}%
\pgfpathlineto{\pgfqpoint{11.133121in}{2.450926in}}%
\pgfpathlineto{\pgfqpoint{11.172859in}{2.675436in}}%
\pgfpathlineto{\pgfqpoint{11.182793in}{2.720392in}}%
\pgfpathlineto{\pgfqpoint{11.192727in}{2.756420in}}%
\pgfpathlineto{\pgfqpoint{11.202662in}{2.782624in}}%
\pgfpathlineto{\pgfqpoint{11.209285in}{2.794692in}}%
\pgfpathlineto{\pgfqpoint{11.215907in}{2.802756in}}%
\pgfpathlineto{\pgfqpoint{11.222530in}{2.807237in}}%
\pgfpathlineto{\pgfqpoint{11.229153in}{2.808640in}}%
\pgfpathlineto{\pgfqpoint{11.235776in}{2.807506in}}%
\pgfpathlineto{\pgfqpoint{11.242399in}{2.804374in}}%
\pgfpathlineto{\pgfqpoint{11.252333in}{2.797048in}}%
\pgfpathlineto{\pgfqpoint{11.311939in}{2.746130in}}%
\pgfpathlineto{\pgfqpoint{11.331808in}{2.728975in}}%
\pgfpathlineto{\pgfqpoint{11.361611in}{2.701983in}}%
\pgfpathlineto{\pgfqpoint{11.374857in}{2.693751in}}%
\pgfpathlineto{\pgfqpoint{11.384791in}{2.690307in}}%
\pgfpathlineto{\pgfqpoint{11.394726in}{2.689702in}}%
\pgfpathlineto{\pgfqpoint{11.401349in}{2.691125in}}%
\pgfpathlineto{\pgfqpoint{11.407972in}{2.694234in}}%
\pgfpathlineto{\pgfqpoint{11.414594in}{2.699282in}}%
\pgfpathlineto{\pgfqpoint{11.421217in}{2.706576in}}%
\pgfpathlineto{\pgfqpoint{11.427840in}{2.716456in}}%
\pgfpathlineto{\pgfqpoint{11.434463in}{2.729253in}}%
\pgfpathlineto{\pgfqpoint{11.444397in}{2.754477in}}%
\pgfpathlineto{\pgfqpoint{11.454332in}{2.787318in}}%
\pgfpathlineto{\pgfqpoint{11.464266in}{2.827729in}}%
\pgfpathlineto{\pgfqpoint{11.477512in}{2.892466in}}%
\pgfpathlineto{\pgfqpoint{11.494069in}{2.986742in}}%
\pgfpathlineto{\pgfqpoint{11.523872in}{3.160992in}}%
\pgfpathlineto{\pgfqpoint{11.537118in}{3.224453in}}%
\pgfpathlineto{\pgfqpoint{11.547052in}{3.262265in}}%
\pgfpathlineto{\pgfqpoint{11.556987in}{3.290760in}}%
\pgfpathlineto{\pgfqpoint{11.563610in}{3.304437in}}%
\pgfpathlineto{\pgfqpoint{11.570233in}{3.313848in}}%
\pgfpathlineto{\pgfqpoint{11.576855in}{3.319017in}}%
\pgfpathlineto{\pgfqpoint{11.583478in}{3.319999in}}%
\pgfpathlineto{\pgfqpoint{11.590101in}{3.316888in}}%
\pgfpathlineto{\pgfqpoint{11.596724in}{3.309850in}}%
\pgfpathlineto{\pgfqpoint{11.603347in}{3.299151in}}%
\pgfpathlineto{\pgfqpoint{11.613281in}{3.277146in}}%
\pgfpathlineto{\pgfqpoint{11.626527in}{3.239791in}}%
\pgfpathlineto{\pgfqpoint{11.666265in}{3.120486in}}%
\pgfpathlineto{\pgfqpoint{11.682822in}{3.080489in}}%
\pgfpathlineto{\pgfqpoint{11.696068in}{3.053876in}}%
\pgfpathlineto{\pgfqpoint{11.709313in}{3.032195in}}%
\pgfpathlineto{\pgfqpoint{11.719248in}{3.019394in}}%
\pgfpathlineto{\pgfqpoint{11.729182in}{3.009898in}}%
\pgfpathlineto{\pgfqpoint{11.739117in}{3.004202in}}%
\pgfpathlineto{\pgfqpoint{11.745739in}{3.002801in}}%
\pgfpathlineto{\pgfqpoint{11.752362in}{3.003492in}}%
\pgfpathlineto{\pgfqpoint{11.758985in}{3.006364in}}%
\pgfpathlineto{\pgfqpoint{11.765608in}{3.011424in}}%
\pgfpathlineto{\pgfqpoint{11.775542in}{3.022922in}}%
\pgfpathlineto{\pgfqpoint{11.785477in}{3.038577in}}%
\pgfpathlineto{\pgfqpoint{11.798723in}{3.064607in}}%
\pgfpathlineto{\pgfqpoint{11.815280in}{3.103318in}}%
\pgfpathlineto{\pgfqpoint{11.831837in}{3.147749in}}%
\pgfpathlineto{\pgfqpoint{11.848394in}{3.197964in}}%
\pgfpathlineto{\pgfqpoint{11.898066in}{3.356455in}}%
\pgfpathlineto{\pgfqpoint{11.908000in}{3.378656in}}%
\pgfpathlineto{\pgfqpoint{11.914623in}{3.389677in}}%
\pgfpathlineto{\pgfqpoint{11.921246in}{3.397343in}}%
\pgfpathlineto{\pgfqpoint{11.927869in}{3.401543in}}%
\pgfpathlineto{\pgfqpoint{11.934492in}{3.402318in}}%
\pgfpathlineto{\pgfqpoint{11.941115in}{3.399833in}}%
\pgfpathlineto{\pgfqpoint{11.947738in}{3.394336in}}%
\pgfpathlineto{\pgfqpoint{11.954361in}{3.386112in}}%
\pgfpathlineto{\pgfqpoint{11.964295in}{3.369282in}}%
\pgfpathlineto{\pgfqpoint{11.974229in}{3.347820in}}%
\pgfpathlineto{\pgfqpoint{11.987475in}{3.313338in}}%
\pgfpathlineto{\pgfqpoint{12.004032in}{3.263591in}}%
\pgfpathlineto{\pgfqpoint{12.043770in}{3.141281in}}%
\pgfpathlineto{\pgfqpoint{12.060327in}{3.099409in}}%
\pgfpathlineto{\pgfqpoint{12.076884in}{3.064634in}}%
\pgfpathlineto{\pgfqpoint{12.096753in}{3.029386in}}%
\pgfpathlineto{\pgfqpoint{12.169605in}{2.908640in}}%
\pgfpathlineto{\pgfqpoint{12.186162in}{2.886591in}}%
\pgfpathlineto{\pgfqpoint{12.202719in}{2.867874in}}%
\pgfpathlineto{\pgfqpoint{12.215965in}{2.855919in}}%
\pgfpathlineto{\pgfqpoint{12.225900in}{2.849258in}}%
\pgfpathlineto{\pgfqpoint{12.235834in}{2.844886in}}%
\pgfpathlineto{\pgfqpoint{12.245768in}{2.842816in}}%
\pgfpathlineto{\pgfqpoint{12.259014in}{2.843031in}}%
\pgfpathlineto{\pgfqpoint{12.292129in}{2.846870in}}%
\pgfpathlineto{\pgfqpoint{12.302063in}{2.844944in}}%
\pgfpathlineto{\pgfqpoint{12.311997in}{2.840174in}}%
\pgfpathlineto{\pgfqpoint{12.321932in}{2.832313in}}%
\pgfpathlineto{\pgfqpoint{12.331866in}{2.821556in}}%
\pgfpathlineto{\pgfqpoint{12.345112in}{2.803705in}}%
\pgfpathlineto{\pgfqpoint{12.404718in}{2.717001in}}%
\pgfpathlineto{\pgfqpoint{12.421275in}{2.699407in}}%
\pgfpathlineto{\pgfqpoint{12.434521in}{2.688745in}}%
\pgfpathlineto{\pgfqpoint{12.444455in}{2.683074in}}%
\pgfpathlineto{\pgfqpoint{12.454390in}{2.679630in}}%
\pgfpathlineto{\pgfqpoint{12.464324in}{2.678614in}}%
\pgfpathlineto{\pgfqpoint{12.474258in}{2.680210in}}%
\pgfpathlineto{\pgfqpoint{12.484193in}{2.684575in}}%
\pgfpathlineto{\pgfqpoint{12.494127in}{2.691816in}}%
\pgfpathlineto{\pgfqpoint{12.504061in}{2.701975in}}%
\pgfpathlineto{\pgfqpoint{12.513996in}{2.715002in}}%
\pgfpathlineto{\pgfqpoint{12.527242in}{2.736525in}}%
\pgfpathlineto{\pgfqpoint{12.543799in}{2.769052in}}%
\pgfpathlineto{\pgfqpoint{12.563668in}{2.814191in}}%
\pgfpathlineto{\pgfqpoint{12.583536in}{2.865057in}}%
\pgfpathlineto{\pgfqpoint{12.603405in}{2.922956in}}%
\pgfpathlineto{\pgfqpoint{12.626585in}{2.999237in}}%
\pgfpathlineto{\pgfqpoint{12.663011in}{3.121544in}}%
\pgfpathlineto{\pgfqpoint{12.676257in}{3.159816in}}%
\pgfpathlineto{\pgfqpoint{12.686191in}{3.183903in}}%
\pgfpathlineto{\pgfqpoint{12.696126in}{3.202674in}}%
\pgfpathlineto{\pgfqpoint{12.702748in}{3.211642in}}%
\pgfpathlineto{\pgfqpoint{12.709371in}{3.217419in}}%
\pgfpathlineto{\pgfqpoint{12.715994in}{3.219780in}}%
\pgfpathlineto{\pgfqpoint{12.722617in}{3.218603in}}%
\pgfpathlineto{\pgfqpoint{12.729240in}{3.213877in}}%
\pgfpathlineto{\pgfqpoint{12.735863in}{3.205710in}}%
\pgfpathlineto{\pgfqpoint{12.742486in}{3.194316in}}%
\pgfpathlineto{\pgfqpoint{12.752420in}{3.171868in}}%
\pgfpathlineto{\pgfqpoint{12.765666in}{3.134006in}}%
\pgfpathlineto{\pgfqpoint{12.785535in}{3.067188in}}%
\pgfpathlineto{\pgfqpoint{12.815338in}{2.965627in}}%
\pgfpathlineto{\pgfqpoint{12.828584in}{2.927382in}}%
\pgfpathlineto{\pgfqpoint{12.838518in}{2.903976in}}%
\pgfpathlineto{\pgfqpoint{12.848452in}{2.886324in}}%
\pgfpathlineto{\pgfqpoint{12.855075in}{2.878161in}}%
\pgfpathlineto{\pgfqpoint{12.861698in}{2.873027in}}%
\pgfpathlineto{\pgfqpoint{12.868321in}{2.870913in}}%
\pgfpathlineto{\pgfqpoint{12.874944in}{2.871687in}}%
\pgfpathlineto{\pgfqpoint{12.881567in}{2.875101in}}%
\pgfpathlineto{\pgfqpoint{12.888190in}{2.880819in}}%
\pgfpathlineto{\pgfqpoint{12.898124in}{2.892863in}}%
\pgfpathlineto{\pgfqpoint{12.911370in}{2.913328in}}%
\pgfpathlineto{\pgfqpoint{12.934550in}{2.954468in}}%
\pgfpathlineto{\pgfqpoint{12.970976in}{3.023241in}}%
\pgfpathlineto{\pgfqpoint{12.990845in}{3.065504in}}%
\pgfpathlineto{\pgfqpoint{13.014025in}{3.121234in}}%
\pgfpathlineto{\pgfqpoint{13.047139in}{3.201021in}}%
\pgfpathlineto{\pgfqpoint{13.067008in}{3.242468in}}%
\pgfpathlineto{\pgfqpoint{13.080254in}{3.265455in}}%
\pgfpathlineto{\pgfqpoint{13.090188in}{3.278162in}}%
\pgfpathlineto{\pgfqpoint{13.096811in}{3.283408in}}%
\pgfpathlineto{\pgfqpoint{13.103434in}{3.285398in}}%
\pgfpathlineto{\pgfqpoint{13.110057in}{3.283675in}}%
\pgfpathlineto{\pgfqpoint{13.116680in}{3.277975in}}%
\pgfpathlineto{\pgfqpoint{13.123303in}{3.268278in}}%
\pgfpathlineto{\pgfqpoint{13.129925in}{3.254827in}}%
\pgfpathlineto{\pgfqpoint{13.139860in}{3.228703in}}%
\pgfpathlineto{\pgfqpoint{13.153106in}{3.186508in}}%
\pgfpathlineto{\pgfqpoint{13.179597in}{3.099865in}}%
\pgfpathlineto{\pgfqpoint{13.192843in}{3.065164in}}%
\pgfpathlineto{\pgfqpoint{13.202777in}{3.045452in}}%
\pgfpathlineto{\pgfqpoint{13.209400in}{3.035711in}}%
\pgfpathlineto{\pgfqpoint{13.216023in}{3.028833in}}%
\pgfpathlineto{\pgfqpoint{13.222646in}{3.024896in}}%
\pgfpathlineto{\pgfqpoint{13.229269in}{3.023952in}}%
\pgfpathlineto{\pgfqpoint{13.235892in}{3.026044in}}%
\pgfpathlineto{\pgfqpoint{13.242515in}{3.031199in}}%
\pgfpathlineto{\pgfqpoint{13.249138in}{3.039411in}}%
\pgfpathlineto{\pgfqpoint{13.255761in}{3.050614in}}%
\pgfpathlineto{\pgfqpoint{13.265695in}{3.072661in}}%
\pgfpathlineto{\pgfqpoint{13.278941in}{3.110181in}}%
\pgfpathlineto{\pgfqpoint{13.298809in}{3.175971in}}%
\pgfpathlineto{\pgfqpoint{13.318678in}{3.240205in}}%
\pgfpathlineto{\pgfqpoint{13.331924in}{3.277132in}}%
\pgfpathlineto{\pgfqpoint{13.348481in}{3.316033in}}%
\pgfpathlineto{\pgfqpoint{13.371661in}{3.363115in}}%
\pgfpathlineto{\pgfqpoint{13.398153in}{3.413233in}}%
\pgfpathlineto{\pgfqpoint{13.411399in}{3.434081in}}%
\pgfpathlineto{\pgfqpoint{13.421333in}{3.445782in}}%
\pgfpathlineto{\pgfqpoint{13.427956in}{3.451120in}}%
\pgfpathlineto{\pgfqpoint{13.434579in}{3.454207in}}%
\pgfpathlineto{\pgfqpoint{13.441202in}{3.454884in}}%
\pgfpathlineto{\pgfqpoint{13.447825in}{3.453061in}}%
\pgfpathlineto{\pgfqpoint{13.454448in}{3.448694in}}%
\pgfpathlineto{\pgfqpoint{13.461070in}{3.441764in}}%
\pgfpathlineto{\pgfqpoint{13.467693in}{3.432248in}}%
\pgfpathlineto{\pgfqpoint{13.477628in}{3.413045in}}%
\pgfpathlineto{\pgfqpoint{13.487562in}{3.387796in}}%
\pgfpathlineto{\pgfqpoint{13.497496in}{3.356486in}}%
\pgfpathlineto{\pgfqpoint{13.510742in}{3.306021in}}%
\pgfpathlineto{\pgfqpoint{13.527299in}{3.232813in}}%
\pgfpathlineto{\pgfqpoint{13.553791in}{3.113856in}}%
\pgfpathlineto{\pgfqpoint{13.567037in}{3.063715in}}%
\pgfpathlineto{\pgfqpoint{13.576971in}{3.032959in}}%
\pgfpathlineto{\pgfqpoint{13.586906in}{3.009030in}}%
\pgfpathlineto{\pgfqpoint{13.593528in}{2.997138in}}%
\pgfpathlineto{\pgfqpoint{13.600151in}{2.988565in}}%
\pgfpathlineto{\pgfqpoint{13.606774in}{2.983257in}}%
\pgfpathlineto{\pgfqpoint{13.613397in}{2.981051in}}%
\pgfpathlineto{\pgfqpoint{13.620020in}{2.981686in}}%
\pgfpathlineto{\pgfqpoint{13.626643in}{2.984826in}}%
\pgfpathlineto{\pgfqpoint{13.633266in}{2.990100in}}%
\pgfpathlineto{\pgfqpoint{13.643200in}{3.001189in}}%
\pgfpathlineto{\pgfqpoint{13.656446in}{3.020138in}}%
\pgfpathlineto{\pgfqpoint{13.676315in}{3.053275in}}%
\pgfpathlineto{\pgfqpoint{13.709429in}{3.109219in}}%
\pgfpathlineto{\pgfqpoint{13.725986in}{3.132768in}}%
\pgfpathlineto{\pgfqpoint{13.739232in}{3.147671in}}%
\pgfpathlineto{\pgfqpoint{13.749167in}{3.155992in}}%
\pgfpathlineto{\pgfqpoint{13.759101in}{3.161500in}}%
\pgfpathlineto{\pgfqpoint{13.769035in}{3.163865in}}%
\pgfpathlineto{\pgfqpoint{13.775658in}{3.163528in}}%
\pgfpathlineto{\pgfqpoint{13.782281in}{3.161553in}}%
\pgfpathlineto{\pgfqpoint{13.788904in}{3.157864in}}%
\pgfpathlineto{\pgfqpoint{13.798838in}{3.149008in}}%
\pgfpathlineto{\pgfqpoint{13.808773in}{3.136105in}}%
\pgfpathlineto{\pgfqpoint{13.818707in}{3.119166in}}%
\pgfpathlineto{\pgfqpoint{13.828641in}{3.098249in}}%
\pgfpathlineto{\pgfqpoint{13.841887in}{3.064328in}}%
\pgfpathlineto{\pgfqpoint{13.855133in}{3.023878in}}%
\pgfpathlineto{\pgfqpoint{13.871690in}{2.965336in}}%
\pgfpathlineto{\pgfqpoint{13.898182in}{2.860474in}}%
\pgfpathlineto{\pgfqpoint{13.921362in}{2.771951in}}%
\pgfpathlineto{\pgfqpoint{13.934608in}{2.730472in}}%
\pgfpathlineto{\pgfqpoint{13.944542in}{2.706392in}}%
\pgfpathlineto{\pgfqpoint{13.951165in}{2.694243in}}%
\pgfpathlineto{\pgfqpoint{13.957788in}{2.685372in}}%
\pgfpathlineto{\pgfqpoint{13.964411in}{2.679775in}}%
\pgfpathlineto{\pgfqpoint{13.971034in}{2.677352in}}%
\pgfpathlineto{\pgfqpoint{13.977657in}{2.677955in}}%
\pgfpathlineto{\pgfqpoint{13.984280in}{2.681429in}}%
\pgfpathlineto{\pgfqpoint{13.990902in}{2.687635in}}%
\pgfpathlineto{\pgfqpoint{13.997525in}{2.696438in}}%
\pgfpathlineto{\pgfqpoint{14.007460in}{2.714165in}}%
\pgfpathlineto{\pgfqpoint{14.017394in}{2.736637in}}%
\pgfpathlineto{\pgfqpoint{14.030640in}{2.772151in}}%
\pgfpathlineto{\pgfqpoint{14.060443in}{2.861556in}}%
\pgfpathlineto{\pgfqpoint{14.080312in}{2.918283in}}%
\pgfpathlineto{\pgfqpoint{14.096869in}{2.958706in}}%
\pgfpathlineto{\pgfqpoint{14.106803in}{2.978379in}}%
\pgfpathlineto{\pgfqpoint{14.116738in}{2.993554in}}%
\pgfpathlineto{\pgfqpoint{14.123360in}{3.000724in}}%
\pgfpathlineto{\pgfqpoint{14.129983in}{3.005232in}}%
\pgfpathlineto{\pgfqpoint{14.136606in}{3.006843in}}%
\pgfpathlineto{\pgfqpoint{14.143229in}{3.005367in}}%
\pgfpathlineto{\pgfqpoint{14.149852in}{3.000673in}}%
\pgfpathlineto{\pgfqpoint{14.156475in}{2.992718in}}%
\pgfpathlineto{\pgfqpoint{14.163098in}{2.981554in}}%
\pgfpathlineto{\pgfqpoint{14.173032in}{2.959168in}}%
\pgfpathlineto{\pgfqpoint{14.182967in}{2.930845in}}%
\pgfpathlineto{\pgfqpoint{14.196212in}{2.885644in}}%
\pgfpathlineto{\pgfqpoint{14.212770in}{2.819373in}}%
\pgfpathlineto{\pgfqpoint{14.229327in}{2.742410in}}%
\pgfpathlineto{\pgfqpoint{14.249196in}{2.636435in}}%
\pgfpathlineto{\pgfqpoint{14.282310in}{2.453978in}}%
\pgfpathlineto{\pgfqpoint{14.295556in}{2.393587in}}%
\pgfpathlineto{\pgfqpoint{14.305490in}{2.356972in}}%
\pgfpathlineto{\pgfqpoint{14.315425in}{2.328916in}}%
\pgfpathlineto{\pgfqpoint{14.322047in}{2.315260in}}%
\pgfpathlineto{\pgfqpoint{14.328670in}{2.305759in}}%
\pgfpathlineto{\pgfqpoint{14.335293in}{2.300455in}}%
\pgfpathlineto{\pgfqpoint{14.341916in}{2.299339in}}%
\pgfpathlineto{\pgfqpoint{14.348539in}{2.302350in}}%
\pgfpathlineto{\pgfqpoint{14.355162in}{2.309383in}}%
\pgfpathlineto{\pgfqpoint{14.361785in}{2.320312in}}%
\pgfpathlineto{\pgfqpoint{14.368408in}{2.335020in}}%
\pgfpathlineto{\pgfqpoint{14.378342in}{2.364009in}}%
\pgfpathlineto{\pgfqpoint{14.388276in}{2.401254in}}%
\pgfpathlineto{\pgfqpoint{14.398211in}{2.446690in}}%
\pgfpathlineto{\pgfqpoint{14.411457in}{2.519130in}}%
\pgfpathlineto{\pgfqpoint{14.428014in}{2.623807in}}%
\pgfpathlineto{\pgfqpoint{14.461128in}{2.837880in}}%
\pgfpathlineto{\pgfqpoint{14.477686in}{2.929378in}}%
\pgfpathlineto{\pgfqpoint{14.494243in}{3.007688in}}%
\pgfpathlineto{\pgfqpoint{14.510800in}{3.075545in}}%
\pgfpathlineto{\pgfqpoint{14.527357in}{3.135090in}}%
\pgfpathlineto{\pgfqpoint{14.540603in}{3.175838in}}%
\pgfpathlineto{\pgfqpoint{14.550537in}{3.200670in}}%
\pgfpathlineto{\pgfqpoint{14.560472in}{3.218963in}}%
\pgfpathlineto{\pgfqpoint{14.567095in}{3.226886in}}%
\pgfpathlineto{\pgfqpoint{14.573718in}{3.231091in}}%
\pgfpathlineto{\pgfqpoint{14.580340in}{3.231468in}}%
\pgfpathlineto{\pgfqpoint{14.586963in}{3.228045in}}%
\pgfpathlineto{\pgfqpoint{14.593586in}{3.220973in}}%
\pgfpathlineto{\pgfqpoint{14.600209in}{3.210516in}}%
\pgfpathlineto{\pgfqpoint{14.610144in}{3.189255in}}%
\pgfpathlineto{\pgfqpoint{14.620078in}{3.162505in}}%
\pgfpathlineto{\pgfqpoint{14.633324in}{3.120559in}}%
\pgfpathlineto{\pgfqpoint{14.653192in}{3.049164in}}%
\pgfpathlineto{\pgfqpoint{14.686307in}{2.927185in}}%
\pgfpathlineto{\pgfqpoint{14.699553in}{2.887296in}}%
\pgfpathlineto{\pgfqpoint{14.709487in}{2.864594in}}%
\pgfpathlineto{\pgfqpoint{14.716110in}{2.853771in}}%
\pgfpathlineto{\pgfqpoint{14.722733in}{2.846823in}}%
\pgfpathlineto{\pgfqpoint{14.729356in}{2.844001in}}%
\pgfpathlineto{\pgfqpoint{14.735979in}{2.845433in}}%
\pgfpathlineto{\pgfqpoint{14.742602in}{2.851119in}}%
\pgfpathlineto{\pgfqpoint{14.749224in}{2.860928in}}%
\pgfpathlineto{\pgfqpoint{14.755847in}{2.874618in}}%
\pgfpathlineto{\pgfqpoint{14.765782in}{2.901705in}}%
\pgfpathlineto{\pgfqpoint{14.775716in}{2.935531in}}%
\pgfpathlineto{\pgfqpoint{14.788962in}{2.988931in}}%
\pgfpathlineto{\pgfqpoint{14.805519in}{3.065204in}}%
\pgfpathlineto{\pgfqpoint{14.832011in}{3.199090in}}%
\pgfpathlineto{\pgfqpoint{14.888305in}{3.486802in}}%
\pgfpathlineto{\pgfqpoint{14.908174in}{3.577306in}}%
\pgfpathlineto{\pgfqpoint{14.924731in}{3.643303in}}%
\pgfpathlineto{\pgfqpoint{14.937977in}{3.688153in}}%
\pgfpathlineto{\pgfqpoint{14.951223in}{3.724551in}}%
\pgfpathlineto{\pgfqpoint{14.961157in}{3.745262in}}%
\pgfpathlineto{\pgfqpoint{14.967780in}{3.755284in}}%
\pgfpathlineto{\pgfqpoint{14.974403in}{3.761576in}}%
\pgfpathlineto{\pgfqpoint{14.977714in}{3.763018in}}%
\pgfpathlineto{\pgfqpoint{14.981026in}{3.763087in}}%
\pgfpathlineto{\pgfqpoint{14.984337in}{3.761540in}}%
\pgfpathlineto{\pgfqpoint{14.987649in}{3.758079in}}%
\pgfpathlineto{\pgfqpoint{14.990960in}{3.752340in}}%
\pgfpathlineto{\pgfqpoint{14.994272in}{3.743907in}}%
\pgfpathlineto{\pgfqpoint{15.000895in}{3.717187in}}%
\pgfpathlineto{\pgfqpoint{15.007518in}{3.674904in}}%
\pgfpathlineto{\pgfqpoint{15.014140in}{3.616567in}}%
\pgfpathlineto{\pgfqpoint{15.027386in}{3.470932in}}%
\pgfpathlineto{\pgfqpoint{15.040632in}{3.336183in}}%
\pgfpathlineto{\pgfqpoint{15.050566in}{3.262033in}}%
\pgfpathlineto{\pgfqpoint{15.057189in}{3.225523in}}%
\pgfpathlineto{\pgfqpoint{15.063812in}{3.197959in}}%
\pgfpathlineto{\pgfqpoint{15.070435in}{3.178199in}}%
\pgfpathlineto{\pgfqpoint{15.077058in}{3.165356in}}%
\pgfpathlineto{\pgfqpoint{15.083681in}{3.158768in}}%
\pgfpathlineto{\pgfqpoint{15.086992in}{3.157651in}}%
\pgfpathlineto{\pgfqpoint{15.090304in}{3.157905in}}%
\pgfpathlineto{\pgfqpoint{15.093615in}{3.159470in}}%
\pgfpathlineto{\pgfqpoint{15.100238in}{3.166290in}}%
\pgfpathlineto{\pgfqpoint{15.106861in}{3.177599in}}%
\pgfpathlineto{\pgfqpoint{15.113484in}{3.192844in}}%
\pgfpathlineto{\pgfqpoint{15.123418in}{3.221811in}}%
\pgfpathlineto{\pgfqpoint{15.136664in}{3.268862in}}%
\pgfpathlineto{\pgfqpoint{15.153221in}{3.336420in}}%
\pgfpathlineto{\pgfqpoint{15.176401in}{3.441885in}}%
\pgfpathlineto{\pgfqpoint{15.189647in}{3.501167in}}%
\pgfpathlineto{\pgfqpoint{15.196270in}{3.525234in}}%
\pgfpathlineto{\pgfqpoint{15.202893in}{3.542404in}}%
\pgfpathlineto{\pgfqpoint{15.209516in}{3.551394in}}%
\pgfpathlineto{\pgfqpoint{15.212827in}{3.552910in}}%
\pgfpathlineto{\pgfqpoint{15.216139in}{3.552718in}}%
\pgfpathlineto{\pgfqpoint{15.222762in}{3.548494in}}%
\pgfpathlineto{\pgfqpoint{15.236008in}{3.533711in}}%
\pgfpathlineto{\pgfqpoint{15.249253in}{3.520078in}}%
\pgfpathlineto{\pgfqpoint{15.265811in}{3.506592in}}%
\pgfpathlineto{\pgfqpoint{15.288991in}{3.487945in}}%
\pgfpathlineto{\pgfqpoint{15.302237in}{3.474290in}}%
\pgfpathlineto{\pgfqpoint{15.312171in}{3.461349in}}%
\pgfpathlineto{\pgfqpoint{15.322105in}{3.445208in}}%
\pgfpathlineto{\pgfqpoint{15.332040in}{3.425151in}}%
\pgfpathlineto{\pgfqpoint{15.345285in}{3.392318in}}%
\pgfpathlineto{\pgfqpoint{15.394957in}{3.253919in}}%
\pgfpathlineto{\pgfqpoint{15.404892in}{3.219228in}}%
\pgfpathlineto{\pgfqpoint{15.414826in}{3.176495in}}%
\pgfpathlineto{\pgfqpoint{15.447940in}{3.022497in}}%
\pgfpathlineto{\pgfqpoint{15.457875in}{2.989997in}}%
\pgfpathlineto{\pgfqpoint{15.464498in}{2.973805in}}%
\pgfpathlineto{\pgfqpoint{15.467809in}{2.967385in}}%
\pgfpathlineto{\pgfqpoint{15.467809in}{2.967385in}}%
\pgfusepath{stroke}%
\end{pgfscope}%
\begin{pgfscope}%
\pgfpathrectangle{\pgfqpoint{2.400000in}{1.081300in}}{\pgfqpoint{14.880000in}{7.569100in}}%
\pgfusepath{clip}%
\pgfsetrectcap%
\pgfsetroundjoin%
\pgfsetlinewidth{1.505625pt}%
\definecolor{currentstroke}{rgb}{0.839216,0.152941,0.156863}%
\pgfsetstrokecolor{currentstroke}%
\pgfsetdash{}{0pt}%
\pgfpathmoveto{\pgfqpoint{3.076364in}{1.425350in}}%
\pgfpathlineto{\pgfqpoint{3.149216in}{1.436120in}}%
\pgfpathlineto{\pgfqpoint{3.188953in}{1.439823in}}%
\pgfpathlineto{\pgfqpoint{3.265116in}{1.445805in}}%
\pgfpathlineto{\pgfqpoint{3.298231in}{1.451226in}}%
\pgfpathlineto{\pgfqpoint{3.334657in}{1.459755in}}%
\pgfpathlineto{\pgfqpoint{3.400886in}{1.476138in}}%
\pgfpathlineto{\pgfqpoint{3.440623in}{1.483009in}}%
\pgfpathlineto{\pgfqpoint{3.486983in}{1.488625in}}%
\pgfpathlineto{\pgfqpoint{3.533344in}{1.491969in}}%
\pgfpathlineto{\pgfqpoint{3.579704in}{1.492875in}}%
\pgfpathlineto{\pgfqpoint{3.629376in}{1.491552in}}%
\pgfpathlineto{\pgfqpoint{3.695605in}{1.487253in}}%
\pgfpathlineto{\pgfqpoint{3.930718in}{1.468193in}}%
\pgfpathlineto{\pgfqpoint{3.970455in}{1.462645in}}%
\pgfpathlineto{\pgfqpoint{4.006881in}{1.455254in}}%
\pgfpathlineto{\pgfqpoint{4.056553in}{1.444751in}}%
\pgfpathlineto{\pgfqpoint{4.076421in}{1.443214in}}%
\pgfpathlineto{\pgfqpoint{4.092979in}{1.444043in}}%
\pgfpathlineto{\pgfqpoint{4.116159in}{1.447648in}}%
\pgfpathlineto{\pgfqpoint{4.192322in}{1.461467in}}%
\pgfpathlineto{\pgfqpoint{4.222125in}{1.463602in}}%
\pgfpathlineto{\pgfqpoint{4.255240in}{1.463513in}}%
\pgfpathlineto{\pgfqpoint{4.304912in}{1.460703in}}%
\pgfpathlineto{\pgfqpoint{4.361206in}{1.458007in}}%
\pgfpathlineto{\pgfqpoint{4.404255in}{1.458371in}}%
\pgfpathlineto{\pgfqpoint{4.477107in}{1.459393in}}%
\pgfpathlineto{\pgfqpoint{4.513533in}{1.457390in}}%
\pgfpathlineto{\pgfqpoint{4.559893in}{1.452255in}}%
\pgfpathlineto{\pgfqpoint{4.612876in}{1.446888in}}%
\pgfpathlineto{\pgfqpoint{4.655925in}{1.445058in}}%
\pgfpathlineto{\pgfqpoint{4.708908in}{1.445297in}}%
\pgfpathlineto{\pgfqpoint{4.808252in}{1.448376in}}%
\pgfpathlineto{\pgfqpoint{4.960579in}{1.454655in}}%
\pgfpathlineto{\pgfqpoint{5.049988in}{1.453506in}}%
\pgfpathlineto{\pgfqpoint{5.149331in}{1.457082in}}%
\pgfpathlineto{\pgfqpoint{5.192380in}{1.454521in}}%
\pgfpathlineto{\pgfqpoint{5.308281in}{1.446590in}}%
\pgfpathlineto{\pgfqpoint{5.338084in}{1.446015in}}%
\pgfpathlineto{\pgfqpoint{5.361264in}{1.447596in}}%
\pgfpathlineto{\pgfqpoint{5.391067in}{1.452024in}}%
\pgfpathlineto{\pgfqpoint{5.444050in}{1.460062in}}%
\pgfpathlineto{\pgfqpoint{5.480476in}{1.462850in}}%
\pgfpathlineto{\pgfqpoint{5.536771in}{1.464340in}}%
\pgfpathlineto{\pgfqpoint{5.589754in}{1.463903in}}%
\pgfpathlineto{\pgfqpoint{5.646049in}{1.460772in}}%
\pgfpathlineto{\pgfqpoint{5.685786in}{1.459433in}}%
\pgfpathlineto{\pgfqpoint{5.735458in}{1.460549in}}%
\pgfpathlineto{\pgfqpoint{5.768572in}{1.460087in}}%
\pgfpathlineto{\pgfqpoint{5.795064in}{1.457390in}}%
\pgfpathlineto{\pgfqpoint{5.824867in}{1.451805in}}%
\pgfpathlineto{\pgfqpoint{5.877850in}{1.440802in}}%
\pgfpathlineto{\pgfqpoint{5.897719in}{1.439103in}}%
\pgfpathlineto{\pgfqpoint{5.924211in}{1.439405in}}%
\pgfpathlineto{\pgfqpoint{5.980505in}{1.443146in}}%
\pgfpathlineto{\pgfqpoint{6.020243in}{1.447561in}}%
\pgfpathlineto{\pgfqpoint{6.146078in}{1.464380in}}%
\pgfpathlineto{\pgfqpoint{6.192438in}{1.466189in}}%
\pgfpathlineto{\pgfqpoint{6.338142in}{1.467690in}}%
\pgfpathlineto{\pgfqpoint{6.364633in}{1.465584in}}%
\pgfpathlineto{\pgfqpoint{6.387814in}{1.461725in}}%
\pgfpathlineto{\pgfqpoint{6.414305in}{1.454926in}}%
\pgfpathlineto{\pgfqpoint{6.454043in}{1.441744in}}%
\pgfpathlineto{\pgfqpoint{6.477223in}{1.435071in}}%
\pgfpathlineto{\pgfqpoint{6.490468in}{1.433697in}}%
\pgfpathlineto{\pgfqpoint{6.503714in}{1.435023in}}%
\pgfpathlineto{\pgfqpoint{6.576566in}{1.447827in}}%
\pgfpathlineto{\pgfqpoint{6.603058in}{1.448357in}}%
\pgfpathlineto{\pgfqpoint{6.632861in}{1.446523in}}%
\pgfpathlineto{\pgfqpoint{6.685844in}{1.440311in}}%
\pgfpathlineto{\pgfqpoint{6.725581in}{1.436617in}}%
\pgfpathlineto{\pgfqpoint{6.752073in}{1.436227in}}%
\pgfpathlineto{\pgfqpoint{6.778565in}{1.438175in}}%
\pgfpathlineto{\pgfqpoint{6.811679in}{1.443129in}}%
\pgfpathlineto{\pgfqpoint{6.884531in}{1.454903in}}%
\pgfpathlineto{\pgfqpoint{6.920957in}{1.458098in}}%
\pgfpathlineto{\pgfqpoint{6.970629in}{1.459890in}}%
\pgfpathlineto{\pgfqpoint{7.086529in}{1.461490in}}%
\pgfpathlineto{\pgfqpoint{7.132890in}{1.459242in}}%
\pgfpathlineto{\pgfqpoint{7.232233in}{1.453164in}}%
\pgfpathlineto{\pgfqpoint{7.271971in}{1.453828in}}%
\pgfpathlineto{\pgfqpoint{7.341511in}{1.455518in}}%
\pgfpathlineto{\pgfqpoint{7.374626in}{1.453699in}}%
\pgfpathlineto{\pgfqpoint{7.407740in}{1.449641in}}%
\pgfpathlineto{\pgfqpoint{7.483903in}{1.439272in}}%
\pgfpathlineto{\pgfqpoint{7.510395in}{1.438979in}}%
\pgfpathlineto{\pgfqpoint{7.550132in}{1.441450in}}%
\pgfpathlineto{\pgfqpoint{7.619673in}{1.445687in}}%
\pgfpathlineto{\pgfqpoint{7.679279in}{1.447159in}}%
\pgfpathlineto{\pgfqpoint{7.758754in}{1.448539in}}%
\pgfpathlineto{\pgfqpoint{7.811737in}{1.453414in}}%
\pgfpathlineto{\pgfqpoint{7.848163in}{1.455661in}}%
\pgfpathlineto{\pgfqpoint{7.877966in}{1.455415in}}%
\pgfpathlineto{\pgfqpoint{7.914392in}{1.452690in}}%
\pgfpathlineto{\pgfqpoint{8.003801in}{1.444553in}}%
\pgfpathlineto{\pgfqpoint{8.106456in}{1.440560in}}%
\pgfpathlineto{\pgfqpoint{8.146193in}{1.439916in}}%
\pgfpathlineto{\pgfqpoint{8.169374in}{1.441532in}}%
\pgfpathlineto{\pgfqpoint{8.192554in}{1.445291in}}%
\pgfpathlineto{\pgfqpoint{8.228980in}{1.453970in}}%
\pgfpathlineto{\pgfqpoint{8.268717in}{1.462889in}}%
\pgfpathlineto{\pgfqpoint{8.295209in}{1.466424in}}%
\pgfpathlineto{\pgfqpoint{8.318389in}{1.467330in}}%
\pgfpathlineto{\pgfqpoint{8.341569in}{1.466129in}}%
\pgfpathlineto{\pgfqpoint{8.371372in}{1.462045in}}%
\pgfpathlineto{\pgfqpoint{8.427667in}{1.453418in}}%
\pgfpathlineto{\pgfqpoint{8.454158in}{1.452313in}}%
\pgfpathlineto{\pgfqpoint{8.507141in}{1.453523in}}%
\pgfpathlineto{\pgfqpoint{8.546879in}{1.453037in}}%
\pgfpathlineto{\pgfqpoint{8.682648in}{1.448888in}}%
\pgfpathlineto{\pgfqpoint{8.795238in}{1.449967in}}%
\pgfpathlineto{\pgfqpoint{8.844909in}{1.446449in}}%
\pgfpathlineto{\pgfqpoint{8.891270in}{1.443762in}}%
\pgfpathlineto{\pgfqpoint{8.924384in}{1.444310in}}%
\pgfpathlineto{\pgfqpoint{8.960810in}{1.447278in}}%
\pgfpathlineto{\pgfqpoint{9.046908in}{1.455436in}}%
\pgfpathlineto{\pgfqpoint{9.076711in}{1.455440in}}%
\pgfpathlineto{\pgfqpoint{9.109825in}{1.452959in}}%
\pgfpathlineto{\pgfqpoint{9.159497in}{1.446373in}}%
\pgfpathlineto{\pgfqpoint{9.222415in}{1.438500in}}%
\pgfpathlineto{\pgfqpoint{9.252218in}{1.436852in}}%
\pgfpathlineto{\pgfqpoint{9.272086in}{1.437787in}}%
\pgfpathlineto{\pgfqpoint{9.295266in}{1.441044in}}%
\pgfpathlineto{\pgfqpoint{9.371430in}{1.453863in}}%
\pgfpathlineto{\pgfqpoint{9.397921in}{1.455170in}}%
\pgfpathlineto{\pgfqpoint{9.424413in}{1.454182in}}%
\pgfpathlineto{\pgfqpoint{9.460839in}{1.450294in}}%
\pgfpathlineto{\pgfqpoint{9.497265in}{1.446846in}}%
\pgfpathlineto{\pgfqpoint{9.523757in}{1.446686in}}%
\pgfpathlineto{\pgfqpoint{9.556871in}{1.449076in}}%
\pgfpathlineto{\pgfqpoint{9.652903in}{1.457364in}}%
\pgfpathlineto{\pgfqpoint{9.692640in}{1.457947in}}%
\pgfpathlineto{\pgfqpoint{9.745624in}{1.456127in}}%
\pgfpathlineto{\pgfqpoint{9.801918in}{1.454864in}}%
\pgfpathlineto{\pgfqpoint{9.854902in}{1.456279in}}%
\pgfpathlineto{\pgfqpoint{9.904573in}{1.456950in}}%
\pgfpathlineto{\pgfqpoint{9.940999in}{1.455035in}}%
\pgfpathlineto{\pgfqpoint{10.007228in}{1.450519in}}%
\pgfpathlineto{\pgfqpoint{10.030408in}{1.451666in}}%
\pgfpathlineto{\pgfqpoint{10.056900in}{1.455473in}}%
\pgfpathlineto{\pgfqpoint{10.159555in}{1.473683in}}%
\pgfpathlineto{\pgfqpoint{10.192669in}{1.475839in}}%
\pgfpathlineto{\pgfqpoint{10.242341in}{1.476433in}}%
\pgfpathlineto{\pgfqpoint{10.298636in}{1.477465in}}%
\pgfpathlineto{\pgfqpoint{10.344996in}{1.480770in}}%
\pgfpathlineto{\pgfqpoint{10.470831in}{1.491232in}}%
\pgfpathlineto{\pgfqpoint{10.517191in}{1.492572in}}%
\pgfpathlineto{\pgfqpoint{10.563552in}{1.491602in}}%
\pgfpathlineto{\pgfqpoint{10.636404in}{1.487346in}}%
\pgfpathlineto{\pgfqpoint{10.696010in}{1.482230in}}%
\pgfpathlineto{\pgfqpoint{10.732436in}{1.476794in}}%
\pgfpathlineto{\pgfqpoint{10.768862in}{1.468829in}}%
\pgfpathlineto{\pgfqpoint{10.848336in}{1.450049in}}%
\pgfpathlineto{\pgfqpoint{10.871517in}{1.447268in}}%
\pgfpathlineto{\pgfqpoint{10.898008in}{1.446545in}}%
\pgfpathlineto{\pgfqpoint{10.931123in}{1.448163in}}%
\pgfpathlineto{\pgfqpoint{11.047023in}{1.457607in}}%
\pgfpathlineto{\pgfqpoint{11.100007in}{1.463926in}}%
\pgfpathlineto{\pgfqpoint{11.212596in}{1.478836in}}%
\pgfpathlineto{\pgfqpoint{11.242399in}{1.479808in}}%
\pgfpathlineto{\pgfqpoint{11.272202in}{1.478595in}}%
\pgfpathlineto{\pgfqpoint{11.305317in}{1.474918in}}%
\pgfpathlineto{\pgfqpoint{11.358300in}{1.466252in}}%
\pgfpathlineto{\pgfqpoint{11.424529in}{1.455926in}}%
\pgfpathlineto{\pgfqpoint{11.470889in}{1.451068in}}%
\pgfpathlineto{\pgfqpoint{11.623216in}{1.439501in}}%
\pgfpathlineto{\pgfqpoint{11.653019in}{1.440656in}}%
\pgfpathlineto{\pgfqpoint{11.702691in}{1.445652in}}%
\pgfpathlineto{\pgfqpoint{11.749051in}{1.449366in}}%
\pgfpathlineto{\pgfqpoint{11.792100in}{1.450450in}}%
\pgfpathlineto{\pgfqpoint{11.908000in}{1.451578in}}%
\pgfpathlineto{\pgfqpoint{11.960984in}{1.455597in}}%
\pgfpathlineto{\pgfqpoint{12.043770in}{1.462234in}}%
\pgfpathlineto{\pgfqpoint{12.080196in}{1.462689in}}%
\pgfpathlineto{\pgfqpoint{12.119933in}{1.460927in}}%
\pgfpathlineto{\pgfqpoint{12.259014in}{1.452511in}}%
\pgfpathlineto{\pgfqpoint{12.341800in}{1.451744in}}%
\pgfpathlineto{\pgfqpoint{12.388161in}{1.453338in}}%
\pgfpathlineto{\pgfqpoint{12.523930in}{1.460375in}}%
\pgfpathlineto{\pgfqpoint{12.563668in}{1.458302in}}%
\pgfpathlineto{\pgfqpoint{12.619962in}{1.452718in}}%
\pgfpathlineto{\pgfqpoint{12.706060in}{1.444377in}}%
\pgfpathlineto{\pgfqpoint{12.775600in}{1.439946in}}%
\pgfpathlineto{\pgfqpoint{12.828584in}{1.438651in}}%
\pgfpathlineto{\pgfqpoint{12.891501in}{1.439686in}}%
\pgfpathlineto{\pgfqpoint{13.000779in}{1.441549in}}%
\pgfpathlineto{\pgfqpoint{13.043828in}{1.439863in}}%
\pgfpathlineto{\pgfqpoint{13.100122in}{1.435077in}}%
\pgfpathlineto{\pgfqpoint{13.156417in}{1.430710in}}%
\pgfpathlineto{\pgfqpoint{13.202777in}{1.429746in}}%
\pgfpathlineto{\pgfqpoint{13.305432in}{1.429759in}}%
\pgfpathlineto{\pgfqpoint{13.391530in}{1.433338in}}%
\pgfpathlineto{\pgfqpoint{13.444513in}{1.435906in}}%
\pgfpathlineto{\pgfqpoint{13.500808in}{1.441305in}}%
\pgfpathlineto{\pgfqpoint{13.560414in}{1.449103in}}%
\pgfpathlineto{\pgfqpoint{13.643200in}{1.460441in}}%
\pgfpathlineto{\pgfqpoint{13.679626in}{1.462569in}}%
\pgfpathlineto{\pgfqpoint{13.729298in}{1.462726in}}%
\pgfpathlineto{\pgfqpoint{13.792215in}{1.463338in}}%
\pgfpathlineto{\pgfqpoint{13.908116in}{1.467774in}}%
\pgfpathlineto{\pgfqpoint{13.947854in}{1.472069in}}%
\pgfpathlineto{\pgfqpoint{14.007460in}{1.478925in}}%
\pgfpathlineto{\pgfqpoint{14.037263in}{1.479889in}}%
\pgfpathlineto{\pgfqpoint{14.070377in}{1.478711in}}%
\pgfpathlineto{\pgfqpoint{14.126672in}{1.474011in}}%
\pgfpathlineto{\pgfqpoint{14.182967in}{1.467462in}}%
\pgfpathlineto{\pgfqpoint{14.249196in}{1.459192in}}%
\pgfpathlineto{\pgfqpoint{14.272376in}{1.458967in}}%
\pgfpathlineto{\pgfqpoint{14.298867in}{1.461322in}}%
\pgfpathlineto{\pgfqpoint{14.368408in}{1.469517in}}%
\pgfpathlineto{\pgfqpoint{14.394899in}{1.469853in}}%
\pgfpathlineto{\pgfqpoint{14.424702in}{1.468005in}}%
\pgfpathlineto{\pgfqpoint{14.471063in}{1.462363in}}%
\pgfpathlineto{\pgfqpoint{14.510800in}{1.458246in}}%
\pgfpathlineto{\pgfqpoint{14.540603in}{1.457447in}}%
\pgfpathlineto{\pgfqpoint{14.577029in}{1.458987in}}%
\pgfpathlineto{\pgfqpoint{14.639947in}{1.462030in}}%
\pgfpathlineto{\pgfqpoint{14.696241in}{1.461668in}}%
\pgfpathlineto{\pgfqpoint{14.732667in}{1.462620in}}%
\pgfpathlineto{\pgfqpoint{14.762470in}{1.465617in}}%
\pgfpathlineto{\pgfqpoint{14.808831in}{1.472980in}}%
\pgfpathlineto{\pgfqpoint{14.861814in}{1.480750in}}%
\pgfpathlineto{\pgfqpoint{14.904863in}{1.484584in}}%
\pgfpathlineto{\pgfqpoint{14.951223in}{1.486332in}}%
\pgfpathlineto{\pgfqpoint{15.093615in}{1.487574in}}%
\pgfpathlineto{\pgfqpoint{15.169779in}{1.487367in}}%
\pgfpathlineto{\pgfqpoint{15.328728in}{1.484798in}}%
\pgfpathlineto{\pgfqpoint{15.388334in}{1.484822in}}%
\pgfpathlineto{\pgfqpoint{15.421449in}{1.482409in}}%
\pgfpathlineto{\pgfqpoint{15.454563in}{1.477564in}}%
\pgfpathlineto{\pgfqpoint{15.497612in}{1.468614in}}%
\pgfpathlineto{\pgfqpoint{15.543972in}{1.459246in}}%
\pgfpathlineto{\pgfqpoint{15.573775in}{1.455608in}}%
\pgfpathlineto{\pgfqpoint{15.606890in}{1.454109in}}%
\pgfpathlineto{\pgfqpoint{15.689676in}{1.451639in}}%
\pgfpathlineto{\pgfqpoint{15.749282in}{1.446369in}}%
\pgfpathlineto{\pgfqpoint{15.785708in}{1.444181in}}%
\pgfpathlineto{\pgfqpoint{15.812200in}{1.444687in}}%
\pgfpathlineto{\pgfqpoint{15.851937in}{1.448157in}}%
\pgfpathlineto{\pgfqpoint{15.901609in}{1.452204in}}%
\pgfpathlineto{\pgfqpoint{15.947969in}{1.453390in}}%
\pgfpathlineto{\pgfqpoint{15.997641in}{1.455001in}}%
\pgfpathlineto{\pgfqpoint{16.030756in}{1.458521in}}%
\pgfpathlineto{\pgfqpoint{16.073804in}{1.465708in}}%
\pgfpathlineto{\pgfqpoint{16.133410in}{1.475651in}}%
\pgfpathlineto{\pgfqpoint{16.163214in}{1.478341in}}%
\pgfpathlineto{\pgfqpoint{16.189705in}{1.478636in}}%
\pgfpathlineto{\pgfqpoint{16.219508in}{1.476592in}}%
\pgfpathlineto{\pgfqpoint{16.259246in}{1.471214in}}%
\pgfpathlineto{\pgfqpoint{16.318852in}{1.463155in}}%
\pgfpathlineto{\pgfqpoint{16.358589in}{1.460309in}}%
\pgfpathlineto{\pgfqpoint{16.441375in}{1.457579in}}%
\pgfpathlineto{\pgfqpoint{16.487736in}{1.455058in}}%
\pgfpathlineto{\pgfqpoint{16.487736in}{1.455058in}}%
\pgfusepath{stroke}%
\end{pgfscope}%
\begin{pgfscope}%
\pgfpathrectangle{\pgfqpoint{2.400000in}{1.081300in}}{\pgfqpoint{14.880000in}{7.569100in}}%
\pgfusepath{clip}%
\pgfsetrectcap%
\pgfsetroundjoin%
\pgfsetlinewidth{1.505625pt}%
\definecolor{currentstroke}{rgb}{0.580392,0.403922,0.741176}%
\pgfsetstrokecolor{currentstroke}%
\pgfsetdash{}{0pt}%
\pgfpathmoveto{\pgfqpoint{3.076364in}{1.425350in}}%
\pgfpathlineto{\pgfqpoint{3.132658in}{1.867941in}}%
\pgfpathlineto{\pgfqpoint{3.165773in}{2.117534in}}%
\pgfpathlineto{\pgfqpoint{3.182330in}{2.226951in}}%
\pgfpathlineto{\pgfqpoint{3.195576in}{2.299070in}}%
\pgfpathlineto{\pgfqpoint{3.205510in}{2.341415in}}%
\pgfpathlineto{\pgfqpoint{3.215445in}{2.372461in}}%
\pgfpathlineto{\pgfqpoint{3.222067in}{2.386636in}}%
\pgfpathlineto{\pgfqpoint{3.228690in}{2.395570in}}%
\pgfpathlineto{\pgfqpoint{3.235313in}{2.399325in}}%
\pgfpathlineto{\pgfqpoint{3.238625in}{2.399293in}}%
\pgfpathlineto{\pgfqpoint{3.241936in}{2.398010in}}%
\pgfpathlineto{\pgfqpoint{3.248559in}{2.391756in}}%
\pgfpathlineto{\pgfqpoint{3.255182in}{2.380706in}}%
\pgfpathlineto{\pgfqpoint{3.261805in}{2.365003in}}%
\pgfpathlineto{\pgfqpoint{3.271739in}{2.333019in}}%
\pgfpathlineto{\pgfqpoint{3.281674in}{2.291312in}}%
\pgfpathlineto{\pgfqpoint{3.291608in}{2.240348in}}%
\pgfpathlineto{\pgfqpoint{3.304854in}{2.159098in}}%
\pgfpathlineto{\pgfqpoint{3.321411in}{2.040133in}}%
\pgfpathlineto{\pgfqpoint{3.347903in}{1.845638in}}%
\pgfpathlineto{\pgfqpoint{3.354525in}{1.810129in}}%
\pgfpathlineto{\pgfqpoint{3.361148in}{1.787262in}}%
\pgfpathlineto{\pgfqpoint{3.364460in}{1.781708in}}%
\pgfpathlineto{\pgfqpoint{3.367771in}{1.780436in}}%
\pgfpathlineto{\pgfqpoint{3.371083in}{1.783476in}}%
\pgfpathlineto{\pgfqpoint{3.374394in}{1.790653in}}%
\pgfpathlineto{\pgfqpoint{3.381017in}{1.815875in}}%
\pgfpathlineto{\pgfqpoint{3.387640in}{1.852131in}}%
\pgfpathlineto{\pgfqpoint{3.400886in}{1.941388in}}%
\pgfpathlineto{\pgfqpoint{3.420754in}{2.076021in}}%
\pgfpathlineto{\pgfqpoint{3.434000in}{2.149505in}}%
\pgfpathlineto{\pgfqpoint{3.443935in}{2.192325in}}%
\pgfpathlineto{\pgfqpoint{3.453869in}{2.223403in}}%
\pgfpathlineto{\pgfqpoint{3.460492in}{2.237260in}}%
\pgfpathlineto{\pgfqpoint{3.467115in}{2.245459in}}%
\pgfpathlineto{\pgfqpoint{3.470426in}{2.247403in}}%
\pgfpathlineto{\pgfqpoint{3.473738in}{2.247897in}}%
\pgfpathlineto{\pgfqpoint{3.477049in}{2.246935in}}%
\pgfpathlineto{\pgfqpoint{3.480361in}{2.244514in}}%
\pgfpathlineto{\pgfqpoint{3.486983in}{2.235298in}}%
\pgfpathlineto{\pgfqpoint{3.493606in}{2.220299in}}%
\pgfpathlineto{\pgfqpoint{3.500229in}{2.199649in}}%
\pgfpathlineto{\pgfqpoint{3.510164in}{2.158589in}}%
\pgfpathlineto{\pgfqpoint{3.520098in}{2.106547in}}%
\pgfpathlineto{\pgfqpoint{3.533344in}{2.023300in}}%
\pgfpathlineto{\pgfqpoint{3.573081in}{1.754253in}}%
\pgfpathlineto{\pgfqpoint{3.579704in}{1.725412in}}%
\pgfpathlineto{\pgfqpoint{3.583015in}{1.715484in}}%
\pgfpathlineto{\pgfqpoint{3.586327in}{1.709108in}}%
\pgfpathlineto{\pgfqpoint{3.589638in}{1.706582in}}%
\pgfpathlineto{\pgfqpoint{3.592950in}{1.708052in}}%
\pgfpathlineto{\pgfqpoint{3.596261in}{1.713488in}}%
\pgfpathlineto{\pgfqpoint{3.599573in}{1.722689in}}%
\pgfpathlineto{\pgfqpoint{3.606196in}{1.751001in}}%
\pgfpathlineto{\pgfqpoint{3.612819in}{1.789777in}}%
\pgfpathlineto{\pgfqpoint{3.622753in}{1.861010in}}%
\pgfpathlineto{\pgfqpoint{3.639310in}{1.997938in}}%
\pgfpathlineto{\pgfqpoint{3.685670in}{2.390114in}}%
\pgfpathlineto{\pgfqpoint{3.702228in}{2.506092in}}%
\pgfpathlineto{\pgfqpoint{3.715473in}{2.580905in}}%
\pgfpathlineto{\pgfqpoint{3.725408in}{2.624304in}}%
\pgfpathlineto{\pgfqpoint{3.732031in}{2.646594in}}%
\pgfpathlineto{\pgfqpoint{3.738654in}{2.663324in}}%
\pgfpathlineto{\pgfqpoint{3.745277in}{2.674369in}}%
\pgfpathlineto{\pgfqpoint{3.751899in}{2.679676in}}%
\pgfpathlineto{\pgfqpoint{3.755211in}{2.680180in}}%
\pgfpathlineto{\pgfqpoint{3.758522in}{2.679258in}}%
\pgfpathlineto{\pgfqpoint{3.761834in}{2.676920in}}%
\pgfpathlineto{\pgfqpoint{3.768457in}{2.668051in}}%
\pgfpathlineto{\pgfqpoint{3.775080in}{2.653706in}}%
\pgfpathlineto{\pgfqpoint{3.781702in}{2.634054in}}%
\pgfpathlineto{\pgfqpoint{3.791637in}{2.595063in}}%
\pgfpathlineto{\pgfqpoint{3.801571in}{2.545363in}}%
\pgfpathlineto{\pgfqpoint{3.814817in}{2.464053in}}%
\pgfpathlineto{\pgfqpoint{3.828063in}{2.368006in}}%
\pgfpathlineto{\pgfqpoint{3.847931in}{2.203755in}}%
\pgfpathlineto{\pgfqpoint{3.890980in}{1.835563in}}%
\pgfpathlineto{\pgfqpoint{3.904226in}{1.743357in}}%
\pgfpathlineto{\pgfqpoint{3.914160in}{1.688607in}}%
\pgfpathlineto{\pgfqpoint{3.920783in}{1.660952in}}%
\pgfpathlineto{\pgfqpoint{3.927406in}{1.641463in}}%
\pgfpathlineto{\pgfqpoint{3.934029in}{1.630553in}}%
\pgfpathlineto{\pgfqpoint{3.937341in}{1.628223in}}%
\pgfpathlineto{\pgfqpoint{3.940652in}{1.627815in}}%
\pgfpathlineto{\pgfqpoint{3.943964in}{1.629152in}}%
\pgfpathlineto{\pgfqpoint{3.950586in}{1.636218in}}%
\pgfpathlineto{\pgfqpoint{3.957209in}{1.647721in}}%
\pgfpathlineto{\pgfqpoint{3.967144in}{1.670142in}}%
\pgfpathlineto{\pgfqpoint{3.987012in}{1.722890in}}%
\pgfpathlineto{\pgfqpoint{4.033373in}{1.848395in}}%
\pgfpathlineto{\pgfqpoint{4.046618in}{1.878857in}}%
\pgfpathlineto{\pgfqpoint{4.056553in}{1.897784in}}%
\pgfpathlineto{\pgfqpoint{4.066487in}{1.912172in}}%
\pgfpathlineto{\pgfqpoint{4.073110in}{1.918767in}}%
\pgfpathlineto{\pgfqpoint{4.079733in}{1.922693in}}%
\pgfpathlineto{\pgfqpoint{4.086356in}{1.923781in}}%
\pgfpathlineto{\pgfqpoint{4.092979in}{1.921924in}}%
\pgfpathlineto{\pgfqpoint{4.099602in}{1.917082in}}%
\pgfpathlineto{\pgfqpoint{4.106225in}{1.909270in}}%
\pgfpathlineto{\pgfqpoint{4.112847in}{1.898560in}}%
\pgfpathlineto{\pgfqpoint{4.122782in}{1.877330in}}%
\pgfpathlineto{\pgfqpoint{4.132716in}{1.850398in}}%
\pgfpathlineto{\pgfqpoint{4.145962in}{1.806813in}}%
\pgfpathlineto{\pgfqpoint{4.162519in}{1.742660in}}%
\pgfpathlineto{\pgfqpoint{4.189011in}{1.627634in}}%
\pgfpathlineto{\pgfqpoint{4.208879in}{1.542947in}}%
\pgfpathlineto{\pgfqpoint{4.218814in}{1.507840in}}%
\pgfpathlineto{\pgfqpoint{4.225437in}{1.492138in}}%
\pgfpathlineto{\pgfqpoint{4.228748in}{1.488290in}}%
\pgfpathlineto{\pgfqpoint{4.232060in}{1.487773in}}%
\pgfpathlineto{\pgfqpoint{4.235371in}{1.490589in}}%
\pgfpathlineto{\pgfqpoint{4.238683in}{1.496281in}}%
\pgfpathlineto{\pgfqpoint{4.245305in}{1.513689in}}%
\pgfpathlineto{\pgfqpoint{4.255240in}{1.547844in}}%
\pgfpathlineto{\pgfqpoint{4.271797in}{1.614713in}}%
\pgfpathlineto{\pgfqpoint{4.288354in}{1.692283in}}%
\pgfpathlineto{\pgfqpoint{4.301600in}{1.764391in}}%
\pgfpathlineto{\pgfqpoint{4.314846in}{1.847420in}}%
\pgfpathlineto{\pgfqpoint{4.331403in}{1.967843in}}%
\pgfpathlineto{\pgfqpoint{4.347960in}{2.105823in}}%
\pgfpathlineto{\pgfqpoint{4.371141in}{2.320105in}}%
\pgfpathlineto{\pgfqpoint{4.404255in}{2.625569in}}%
\pgfpathlineto{\pgfqpoint{4.417501in}{2.729480in}}%
\pgfpathlineto{\pgfqpoint{4.427435in}{2.794975in}}%
\pgfpathlineto{\pgfqpoint{4.437370in}{2.847483in}}%
\pgfpathlineto{\pgfqpoint{4.443992in}{2.874458in}}%
\pgfpathlineto{\pgfqpoint{4.450615in}{2.894580in}}%
\pgfpathlineto{\pgfqpoint{4.457238in}{2.907590in}}%
\pgfpathlineto{\pgfqpoint{4.460550in}{2.911373in}}%
\pgfpathlineto{\pgfqpoint{4.463861in}{2.913323in}}%
\pgfpathlineto{\pgfqpoint{4.467173in}{2.913435in}}%
\pgfpathlineto{\pgfqpoint{4.470484in}{2.911704in}}%
\pgfpathlineto{\pgfqpoint{4.473795in}{2.908136in}}%
\pgfpathlineto{\pgfqpoint{4.480418in}{2.895522in}}%
\pgfpathlineto{\pgfqpoint{4.487041in}{2.875707in}}%
\pgfpathlineto{\pgfqpoint{4.493664in}{2.848875in}}%
\pgfpathlineto{\pgfqpoint{4.503599in}{2.796031in}}%
\pgfpathlineto{\pgfqpoint{4.513533in}{2.729082in}}%
\pgfpathlineto{\pgfqpoint{4.526779in}{2.620404in}}%
\pgfpathlineto{\pgfqpoint{4.543336in}{2.459690in}}%
\pgfpathlineto{\pgfqpoint{4.569828in}{2.172352in}}%
\pgfpathlineto{\pgfqpoint{4.589696in}{1.963607in}}%
\pgfpathlineto{\pgfqpoint{4.602942in}{1.844677in}}%
\pgfpathlineto{\pgfqpoint{4.612876in}{1.773835in}}%
\pgfpathlineto{\pgfqpoint{4.619499in}{1.738067in}}%
\pgfpathlineto{\pgfqpoint{4.626122in}{1.712781in}}%
\pgfpathlineto{\pgfqpoint{4.632745in}{1.698249in}}%
\pgfpathlineto{\pgfqpoint{4.636057in}{1.694792in}}%
\pgfpathlineto{\pgfqpoint{4.639368in}{1.693612in}}%
\pgfpathlineto{\pgfqpoint{4.642679in}{1.694446in}}%
\pgfpathlineto{\pgfqpoint{4.645991in}{1.697004in}}%
\pgfpathlineto{\pgfqpoint{4.652614in}{1.706128in}}%
\pgfpathlineto{\pgfqpoint{4.662548in}{1.726047in}}%
\pgfpathlineto{\pgfqpoint{4.682417in}{1.773374in}}%
\pgfpathlineto{\pgfqpoint{4.705597in}{1.832745in}}%
\pgfpathlineto{\pgfqpoint{4.722154in}{1.882162in}}%
\pgfpathlineto{\pgfqpoint{4.738711in}{1.939654in}}%
\pgfpathlineto{\pgfqpoint{4.775137in}{2.071315in}}%
\pgfpathlineto{\pgfqpoint{4.785072in}{2.099773in}}%
\pgfpathlineto{\pgfqpoint{4.795006in}{2.122031in}}%
\pgfpathlineto{\pgfqpoint{4.801629in}{2.133004in}}%
\pgfpathlineto{\pgfqpoint{4.808252in}{2.140792in}}%
\pgfpathlineto{\pgfqpoint{4.814875in}{2.145491in}}%
\pgfpathlineto{\pgfqpoint{4.821498in}{2.147360in}}%
\pgfpathlineto{\pgfqpoint{4.828121in}{2.146834in}}%
\pgfpathlineto{\pgfqpoint{4.838055in}{2.142918in}}%
\pgfpathlineto{\pgfqpoint{4.854612in}{2.135123in}}%
\pgfpathlineto{\pgfqpoint{4.861235in}{2.134461in}}%
\pgfpathlineto{\pgfqpoint{4.867858in}{2.136705in}}%
\pgfpathlineto{\pgfqpoint{4.874481in}{2.142690in}}%
\pgfpathlineto{\pgfqpoint{4.881104in}{2.152995in}}%
\pgfpathlineto{\pgfqpoint{4.887727in}{2.167862in}}%
\pgfpathlineto{\pgfqpoint{4.897661in}{2.198323in}}%
\pgfpathlineto{\pgfqpoint{4.910907in}{2.250740in}}%
\pgfpathlineto{\pgfqpoint{4.937398in}{2.363349in}}%
\pgfpathlineto{\pgfqpoint{4.947333in}{2.395930in}}%
\pgfpathlineto{\pgfqpoint{4.953956in}{2.411885in}}%
\pgfpathlineto{\pgfqpoint{4.960579in}{2.422278in}}%
\pgfpathlineto{\pgfqpoint{4.963890in}{2.425182in}}%
\pgfpathlineto{\pgfqpoint{4.967202in}{2.426464in}}%
\pgfpathlineto{\pgfqpoint{4.970513in}{2.426066in}}%
\pgfpathlineto{\pgfqpoint{4.973824in}{2.423939in}}%
\pgfpathlineto{\pgfqpoint{4.977136in}{2.420042in}}%
\pgfpathlineto{\pgfqpoint{4.983759in}{2.406820in}}%
\pgfpathlineto{\pgfqpoint{4.990382in}{2.386243in}}%
\pgfpathlineto{\pgfqpoint{4.997005in}{2.358281in}}%
\pgfpathlineto{\pgfqpoint{5.006939in}{2.302746in}}%
\pgfpathlineto{\pgfqpoint{5.016873in}{2.231759in}}%
\pgfpathlineto{\pgfqpoint{5.030119in}{2.116209in}}%
\pgfpathlineto{\pgfqpoint{5.066545in}{1.775742in}}%
\pgfpathlineto{\pgfqpoint{5.069856in}{1.759981in}}%
\pgfpathlineto{\pgfqpoint{5.073168in}{1.751097in}}%
\pgfpathlineto{\pgfqpoint{5.076479in}{1.749955in}}%
\pgfpathlineto{\pgfqpoint{5.079791in}{1.756859in}}%
\pgfpathlineto{\pgfqpoint{5.083102in}{1.771469in}}%
\pgfpathlineto{\pgfqpoint{5.089725in}{1.820121in}}%
\pgfpathlineto{\pgfqpoint{5.099660in}{1.924939in}}%
\pgfpathlineto{\pgfqpoint{5.142708in}{2.432262in}}%
\pgfpathlineto{\pgfqpoint{5.152643in}{2.519329in}}%
\pgfpathlineto{\pgfqpoint{5.162577in}{2.587297in}}%
\pgfpathlineto{\pgfqpoint{5.169200in}{2.621021in}}%
\pgfpathlineto{\pgfqpoint{5.175823in}{2.645002in}}%
\pgfpathlineto{\pgfqpoint{5.182446in}{2.658988in}}%
\pgfpathlineto{\pgfqpoint{5.185757in}{2.662193in}}%
\pgfpathlineto{\pgfqpoint{5.189069in}{2.662869in}}%
\pgfpathlineto{\pgfqpoint{5.192380in}{2.661023in}}%
\pgfpathlineto{\pgfqpoint{5.195692in}{2.656673in}}%
\pgfpathlineto{\pgfqpoint{5.202314in}{2.640574in}}%
\pgfpathlineto{\pgfqpoint{5.208937in}{2.614907in}}%
\pgfpathlineto{\pgfqpoint{5.215560in}{2.580181in}}%
\pgfpathlineto{\pgfqpoint{5.225495in}{2.512726in}}%
\pgfpathlineto{\pgfqpoint{5.238740in}{2.399876in}}%
\pgfpathlineto{\pgfqpoint{5.268543in}{2.130930in}}%
\pgfpathlineto{\pgfqpoint{5.275166in}{2.092699in}}%
\pgfpathlineto{\pgfqpoint{5.278478in}{2.080142in}}%
\pgfpathlineto{\pgfqpoint{5.281789in}{2.072708in}}%
\pgfpathlineto{\pgfqpoint{5.285101in}{2.070808in}}%
\pgfpathlineto{\pgfqpoint{5.288412in}{2.074676in}}%
\pgfpathlineto{\pgfqpoint{5.291724in}{2.084345in}}%
\pgfpathlineto{\pgfqpoint{5.295035in}{2.099655in}}%
\pgfpathlineto{\pgfqpoint{5.301658in}{2.145735in}}%
\pgfpathlineto{\pgfqpoint{5.308281in}{2.209012in}}%
\pgfpathlineto{\pgfqpoint{5.318215in}{2.326380in}}%
\pgfpathlineto{\pgfqpoint{5.338084in}{2.596794in}}%
\pgfpathlineto{\pgfqpoint{5.361264in}{2.904270in}}%
\pgfpathlineto{\pgfqpoint{5.377821in}{3.092553in}}%
\pgfpathlineto{\pgfqpoint{5.391067in}{3.219866in}}%
\pgfpathlineto{\pgfqpoint{5.407624in}{3.356876in}}%
\pgfpathlineto{\pgfqpoint{5.447362in}{3.665332in}}%
\pgfpathlineto{\pgfqpoint{5.453985in}{3.693402in}}%
\pgfpathlineto{\pgfqpoint{5.457296in}{3.701529in}}%
\pgfpathlineto{\pgfqpoint{5.460608in}{3.705471in}}%
\pgfpathlineto{\pgfqpoint{5.463919in}{3.705180in}}%
\pgfpathlineto{\pgfqpoint{5.467230in}{3.700706in}}%
\pgfpathlineto{\pgfqpoint{5.470542in}{3.692179in}}%
\pgfpathlineto{\pgfqpoint{5.477165in}{3.663791in}}%
\pgfpathlineto{\pgfqpoint{5.483788in}{3.622074in}}%
\pgfpathlineto{\pgfqpoint{5.493722in}{3.539720in}}%
\pgfpathlineto{\pgfqpoint{5.506968in}{3.404054in}}%
\pgfpathlineto{\pgfqpoint{5.523525in}{3.205276in}}%
\pgfpathlineto{\pgfqpoint{5.536771in}{3.021526in}}%
\pgfpathlineto{\pgfqpoint{5.550017in}{2.809722in}}%
\pgfpathlineto{\pgfqpoint{5.566574in}{2.506576in}}%
\pgfpathlineto{\pgfqpoint{5.586443in}{2.144183in}}%
\pgfpathlineto{\pgfqpoint{5.593066in}{2.058701in}}%
\pgfpathlineto{\pgfqpoint{5.596377in}{2.031044in}}%
\pgfpathlineto{\pgfqpoint{5.599688in}{2.016193in}}%
\pgfpathlineto{\pgfqpoint{5.603000in}{2.015425in}}%
\pgfpathlineto{\pgfqpoint{5.606311in}{2.028908in}}%
\pgfpathlineto{\pgfqpoint{5.609623in}{2.055650in}}%
\pgfpathlineto{\pgfqpoint{5.616246in}{2.141100in}}%
\pgfpathlineto{\pgfqpoint{5.626180in}{2.316519in}}%
\pgfpathlineto{\pgfqpoint{5.646049in}{2.686717in}}%
\pgfpathlineto{\pgfqpoint{5.655983in}{2.834812in}}%
\pgfpathlineto{\pgfqpoint{5.662606in}{2.911200in}}%
\pgfpathlineto{\pgfqpoint{5.669229in}{2.967992in}}%
\pgfpathlineto{\pgfqpoint{5.675852in}{3.005352in}}%
\pgfpathlineto{\pgfqpoint{5.679163in}{3.017190in}}%
\pgfpathlineto{\pgfqpoint{5.682475in}{3.024872in}}%
\pgfpathlineto{\pgfqpoint{5.685786in}{3.028788in}}%
\pgfpathlineto{\pgfqpoint{5.689098in}{3.029385in}}%
\pgfpathlineto{\pgfqpoint{5.692409in}{3.027154in}}%
\pgfpathlineto{\pgfqpoint{5.699032in}{3.016274in}}%
\pgfpathlineto{\pgfqpoint{5.708966in}{2.991423in}}%
\pgfpathlineto{\pgfqpoint{5.735458in}{2.917158in}}%
\pgfpathlineto{\pgfqpoint{5.742081in}{2.888858in}}%
\pgfpathlineto{\pgfqpoint{5.748704in}{2.851441in}}%
\pgfpathlineto{\pgfqpoint{5.758638in}{2.777130in}}%
\pgfpathlineto{\pgfqpoint{5.771884in}{2.651098in}}%
\pgfpathlineto{\pgfqpoint{5.811621in}{2.248012in}}%
\pgfpathlineto{\pgfqpoint{5.824867in}{2.142235in}}%
\pgfpathlineto{\pgfqpoint{5.834801in}{2.077273in}}%
\pgfpathlineto{\pgfqpoint{5.844736in}{2.025423in}}%
\pgfpathlineto{\pgfqpoint{5.854670in}{1.986744in}}%
\pgfpathlineto{\pgfqpoint{5.861293in}{1.967912in}}%
\pgfpathlineto{\pgfqpoint{5.867916in}{1.954122in}}%
\pgfpathlineto{\pgfqpoint{5.874539in}{1.944789in}}%
\pgfpathlineto{\pgfqpoint{5.881162in}{1.939288in}}%
\pgfpathlineto{\pgfqpoint{5.887785in}{1.937045in}}%
\pgfpathlineto{\pgfqpoint{5.894407in}{1.937581in}}%
\pgfpathlineto{\pgfqpoint{5.901030in}{1.940524in}}%
\pgfpathlineto{\pgfqpoint{5.907653in}{1.945590in}}%
\pgfpathlineto{\pgfqpoint{5.917588in}{1.956684in}}%
\pgfpathlineto{\pgfqpoint{5.927522in}{1.971408in}}%
\pgfpathlineto{\pgfqpoint{5.940768in}{1.995670in}}%
\pgfpathlineto{\pgfqpoint{5.957325in}{2.031506in}}%
\pgfpathlineto{\pgfqpoint{6.013620in}{2.159360in}}%
\pgfpathlineto{\pgfqpoint{6.023554in}{2.175429in}}%
\pgfpathlineto{\pgfqpoint{6.033488in}{2.187476in}}%
\pgfpathlineto{\pgfqpoint{6.043423in}{2.195134in}}%
\pgfpathlineto{\pgfqpoint{6.050046in}{2.197820in}}%
\pgfpathlineto{\pgfqpoint{6.056669in}{2.198697in}}%
\pgfpathlineto{\pgfqpoint{6.063291in}{2.197925in}}%
\pgfpathlineto{\pgfqpoint{6.073226in}{2.194034in}}%
\pgfpathlineto{\pgfqpoint{6.083160in}{2.187303in}}%
\pgfpathlineto{\pgfqpoint{6.096406in}{2.174720in}}%
\pgfpathlineto{\pgfqpoint{6.112963in}{2.155105in}}%
\pgfpathlineto{\pgfqpoint{6.139455in}{2.122893in}}%
\pgfpathlineto{\pgfqpoint{6.152701in}{2.110363in}}%
\pgfpathlineto{\pgfqpoint{6.162635in}{2.103803in}}%
\pgfpathlineto{\pgfqpoint{6.172569in}{2.100119in}}%
\pgfpathlineto{\pgfqpoint{6.182504in}{2.099480in}}%
\pgfpathlineto{\pgfqpoint{6.192438in}{2.101858in}}%
\pgfpathlineto{\pgfqpoint{6.202372in}{2.107077in}}%
\pgfpathlineto{\pgfqpoint{6.212307in}{2.114867in}}%
\pgfpathlineto{\pgfqpoint{6.225552in}{2.128663in}}%
\pgfpathlineto{\pgfqpoint{6.242110in}{2.149892in}}%
\pgfpathlineto{\pgfqpoint{6.275224in}{2.193749in}}%
\pgfpathlineto{\pgfqpoint{6.288470in}{2.206678in}}%
\pgfpathlineto{\pgfqpoint{6.298404in}{2.213236in}}%
\pgfpathlineto{\pgfqpoint{6.308339in}{2.216647in}}%
\pgfpathlineto{\pgfqpoint{6.314962in}{2.217037in}}%
\pgfpathlineto{\pgfqpoint{6.321585in}{2.215867in}}%
\pgfpathlineto{\pgfqpoint{6.331519in}{2.211189in}}%
\pgfpathlineto{\pgfqpoint{6.341453in}{2.203194in}}%
\pgfpathlineto{\pgfqpoint{6.351388in}{2.192363in}}%
\pgfpathlineto{\pgfqpoint{6.367945in}{2.170284in}}%
\pgfpathlineto{\pgfqpoint{6.387814in}{2.143670in}}%
\pgfpathlineto{\pgfqpoint{6.397748in}{2.133166in}}%
\pgfpathlineto{\pgfqpoint{6.407682in}{2.125948in}}%
\pgfpathlineto{\pgfqpoint{6.414305in}{2.123360in}}%
\pgfpathlineto{\pgfqpoint{6.420928in}{2.122764in}}%
\pgfpathlineto{\pgfqpoint{6.427551in}{2.124296in}}%
\pgfpathlineto{\pgfqpoint{6.434174in}{2.128053in}}%
\pgfpathlineto{\pgfqpoint{6.440797in}{2.134100in}}%
\pgfpathlineto{\pgfqpoint{6.447420in}{2.142461in}}%
\pgfpathlineto{\pgfqpoint{6.457354in}{2.159284in}}%
\pgfpathlineto{\pgfqpoint{6.467288in}{2.180953in}}%
\pgfpathlineto{\pgfqpoint{6.480534in}{2.216228in}}%
\pgfpathlineto{\pgfqpoint{6.500403in}{2.277410in}}%
\pgfpathlineto{\pgfqpoint{6.523583in}{2.347905in}}%
\pgfpathlineto{\pgfqpoint{6.536829in}{2.381906in}}%
\pgfpathlineto{\pgfqpoint{6.546763in}{2.402757in}}%
\pgfpathlineto{\pgfqpoint{6.556697in}{2.418962in}}%
\pgfpathlineto{\pgfqpoint{6.566632in}{2.430145in}}%
\pgfpathlineto{\pgfqpoint{6.573255in}{2.434714in}}%
\pgfpathlineto{\pgfqpoint{6.579878in}{2.436965in}}%
\pgfpathlineto{\pgfqpoint{6.586501in}{2.436941in}}%
\pgfpathlineto{\pgfqpoint{6.593123in}{2.434727in}}%
\pgfpathlineto{\pgfqpoint{6.599746in}{2.430455in}}%
\pgfpathlineto{\pgfqpoint{6.609681in}{2.420587in}}%
\pgfpathlineto{\pgfqpoint{6.619615in}{2.407273in}}%
\pgfpathlineto{\pgfqpoint{6.636172in}{2.380144in}}%
\pgfpathlineto{\pgfqpoint{6.662664in}{2.335853in}}%
\pgfpathlineto{\pgfqpoint{6.675910in}{2.318871in}}%
\pgfpathlineto{\pgfqpoint{6.685844in}{2.309928in}}%
\pgfpathlineto{\pgfqpoint{6.695778in}{2.304591in}}%
\pgfpathlineto{\pgfqpoint{6.705713in}{2.302788in}}%
\pgfpathlineto{\pgfqpoint{6.715647in}{2.304128in}}%
\pgfpathlineto{\pgfqpoint{6.725581in}{2.307995in}}%
\pgfpathlineto{\pgfqpoint{6.738827in}{2.315848in}}%
\pgfpathlineto{\pgfqpoint{6.765319in}{2.335386in}}%
\pgfpathlineto{\pgfqpoint{6.821613in}{2.379282in}}%
\pgfpathlineto{\pgfqpoint{6.838171in}{2.395624in}}%
\pgfpathlineto{\pgfqpoint{6.854728in}{2.415409in}}%
\pgfpathlineto{\pgfqpoint{6.871285in}{2.439182in}}%
\pgfpathlineto{\pgfqpoint{6.887842in}{2.467019in}}%
\pgfpathlineto{\pgfqpoint{6.907711in}{2.505381in}}%
\pgfpathlineto{\pgfqpoint{6.930891in}{2.555986in}}%
\pgfpathlineto{\pgfqpoint{6.960694in}{2.627622in}}%
\pgfpathlineto{\pgfqpoint{7.000432in}{2.723587in}}%
\pgfpathlineto{\pgfqpoint{7.020300in}{2.765796in}}%
\pgfpathlineto{\pgfqpoint{7.036858in}{2.795785in}}%
\pgfpathlineto{\pgfqpoint{7.050103in}{2.815921in}}%
\pgfpathlineto{\pgfqpoint{7.063349in}{2.832575in}}%
\pgfpathlineto{\pgfqpoint{7.076595in}{2.845757in}}%
\pgfpathlineto{\pgfqpoint{7.089841in}{2.855327in}}%
\pgfpathlineto{\pgfqpoint{7.099775in}{2.859903in}}%
\pgfpathlineto{\pgfqpoint{7.109710in}{2.861964in}}%
\pgfpathlineto{\pgfqpoint{7.119644in}{2.861191in}}%
\pgfpathlineto{\pgfqpoint{7.129578in}{2.857253in}}%
\pgfpathlineto{\pgfqpoint{7.139513in}{2.849848in}}%
\pgfpathlineto{\pgfqpoint{7.149447in}{2.838764in}}%
\pgfpathlineto{\pgfqpoint{7.159381in}{2.823951in}}%
\pgfpathlineto{\pgfqpoint{7.172627in}{2.798927in}}%
\pgfpathlineto{\pgfqpoint{7.199119in}{2.744880in}}%
\pgfpathlineto{\pgfqpoint{7.205742in}{2.737205in}}%
\pgfpathlineto{\pgfqpoint{7.209053in}{2.735345in}}%
\pgfpathlineto{\pgfqpoint{7.212365in}{2.734934in}}%
\pgfpathlineto{\pgfqpoint{7.215676in}{2.735921in}}%
\pgfpathlineto{\pgfqpoint{7.222299in}{2.741117in}}%
\pgfpathlineto{\pgfqpoint{7.232233in}{2.750513in}}%
\pgfpathlineto{\pgfqpoint{7.238856in}{2.752771in}}%
\pgfpathlineto{\pgfqpoint{7.242168in}{2.751915in}}%
\pgfpathlineto{\pgfqpoint{7.245479in}{2.749655in}}%
\pgfpathlineto{\pgfqpoint{7.252102in}{2.741168in}}%
\pgfpathlineto{\pgfqpoint{7.258725in}{2.728221in}}%
\pgfpathlineto{\pgfqpoint{7.268659in}{2.702963in}}%
\pgfpathlineto{\pgfqpoint{7.285216in}{2.653088in}}%
\pgfpathlineto{\pgfqpoint{7.331577in}{2.510294in}}%
\pgfpathlineto{\pgfqpoint{7.354757in}{2.445426in}}%
\pgfpathlineto{\pgfqpoint{7.364691in}{2.422322in}}%
\pgfpathlineto{\pgfqpoint{7.374626in}{2.404050in}}%
\pgfpathlineto{\pgfqpoint{7.384560in}{2.390175in}}%
\pgfpathlineto{\pgfqpoint{7.401117in}{2.371992in}}%
\pgfpathlineto{\pgfqpoint{7.414363in}{2.359938in}}%
\pgfpathlineto{\pgfqpoint{7.424297in}{2.353421in}}%
\pgfpathlineto{\pgfqpoint{7.430920in}{2.350899in}}%
\pgfpathlineto{\pgfqpoint{7.437543in}{2.350157in}}%
\pgfpathlineto{\pgfqpoint{7.444166in}{2.351381in}}%
\pgfpathlineto{\pgfqpoint{7.450789in}{2.354664in}}%
\pgfpathlineto{\pgfqpoint{7.457412in}{2.360015in}}%
\pgfpathlineto{\pgfqpoint{7.467346in}{2.371771in}}%
\pgfpathlineto{\pgfqpoint{7.477281in}{2.387626in}}%
\pgfpathlineto{\pgfqpoint{7.490526in}{2.414097in}}%
\pgfpathlineto{\pgfqpoint{7.507084in}{2.453042in}}%
\pgfpathlineto{\pgfqpoint{7.540198in}{2.532311in}}%
\pgfpathlineto{\pgfqpoint{7.553444in}{2.558584in}}%
\pgfpathlineto{\pgfqpoint{7.566690in}{2.580303in}}%
\pgfpathlineto{\pgfqpoint{7.579935in}{2.597615in}}%
\pgfpathlineto{\pgfqpoint{7.593181in}{2.610710in}}%
\pgfpathlineto{\pgfqpoint{7.603116in}{2.617712in}}%
\pgfpathlineto{\pgfqpoint{7.613050in}{2.622240in}}%
\pgfpathlineto{\pgfqpoint{7.622984in}{2.624322in}}%
\pgfpathlineto{\pgfqpoint{7.632919in}{2.624066in}}%
\pgfpathlineto{\pgfqpoint{7.642853in}{2.621635in}}%
\pgfpathlineto{\pgfqpoint{7.652787in}{2.617226in}}%
\pgfpathlineto{\pgfqpoint{7.666033in}{2.608642in}}%
\pgfpathlineto{\pgfqpoint{7.679279in}{2.597294in}}%
\pgfpathlineto{\pgfqpoint{7.692525in}{2.583099in}}%
\pgfpathlineto{\pgfqpoint{7.705771in}{2.565476in}}%
\pgfpathlineto{\pgfqpoint{7.719016in}{2.543706in}}%
\pgfpathlineto{\pgfqpoint{7.732262in}{2.517260in}}%
\pgfpathlineto{\pgfqpoint{7.745508in}{2.485773in}}%
\pgfpathlineto{\pgfqpoint{7.758754in}{2.448746in}}%
\pgfpathlineto{\pgfqpoint{7.772000in}{2.405284in}}%
\pgfpathlineto{\pgfqpoint{7.785245in}{2.354349in}}%
\pgfpathlineto{\pgfqpoint{7.801803in}{2.279769in}}%
\pgfpathlineto{\pgfqpoint{7.821671in}{2.177202in}}%
\pgfpathlineto{\pgfqpoint{7.851474in}{2.008276in}}%
\pgfpathlineto{\pgfqpoint{7.877966in}{1.859867in}}%
\pgfpathlineto{\pgfqpoint{7.887900in}{1.817822in}}%
\pgfpathlineto{\pgfqpoint{7.894523in}{1.801577in}}%
\pgfpathlineto{\pgfqpoint{7.897835in}{1.798632in}}%
\pgfpathlineto{\pgfqpoint{7.901146in}{1.799829in}}%
\pgfpathlineto{\pgfqpoint{7.904458in}{1.805538in}}%
\pgfpathlineto{\pgfqpoint{7.907769in}{1.815949in}}%
\pgfpathlineto{\pgfqpoint{7.914392in}{1.850630in}}%
\pgfpathlineto{\pgfqpoint{7.921015in}{1.901960in}}%
\pgfpathlineto{\pgfqpoint{7.930949in}{2.005204in}}%
\pgfpathlineto{\pgfqpoint{7.940883in}{2.136763in}}%
\pgfpathlineto{\pgfqpoint{7.957441in}{2.371872in}}%
\pgfpathlineto{\pgfqpoint{7.964064in}{2.444665in}}%
\pgfpathlineto{\pgfqpoint{7.970687in}{2.498758in}}%
\pgfpathlineto{\pgfqpoint{7.977309in}{2.536287in}}%
\pgfpathlineto{\pgfqpoint{7.983932in}{2.561280in}}%
\pgfpathlineto{\pgfqpoint{7.990555in}{2.577786in}}%
\pgfpathlineto{\pgfqpoint{7.997178in}{2.589066in}}%
\pgfpathlineto{\pgfqpoint{8.007112in}{2.601101in}}%
\pgfpathlineto{\pgfqpoint{8.043538in}{2.640161in}}%
\pgfpathlineto{\pgfqpoint{8.060096in}{2.661055in}}%
\pgfpathlineto{\pgfqpoint{8.076653in}{2.685605in}}%
\pgfpathlineto{\pgfqpoint{8.103145in}{2.727429in}}%
\pgfpathlineto{\pgfqpoint{8.109767in}{2.735030in}}%
\pgfpathlineto{\pgfqpoint{8.116390in}{2.739601in}}%
\pgfpathlineto{\pgfqpoint{8.119702in}{2.740370in}}%
\pgfpathlineto{\pgfqpoint{8.123013in}{2.739916in}}%
\pgfpathlineto{\pgfqpoint{8.126325in}{2.738077in}}%
\pgfpathlineto{\pgfqpoint{8.129636in}{2.734703in}}%
\pgfpathlineto{\pgfqpoint{8.136259in}{2.722844in}}%
\pgfpathlineto{\pgfqpoint{8.142882in}{2.703717in}}%
\pgfpathlineto{\pgfqpoint{8.149505in}{2.677699in}}%
\pgfpathlineto{\pgfqpoint{8.172685in}{2.574640in}}%
\pgfpathlineto{\pgfqpoint{8.175996in}{2.566158in}}%
\pgfpathlineto{\pgfqpoint{8.179308in}{2.560775in}}%
\pgfpathlineto{\pgfqpoint{8.182619in}{2.558649in}}%
\pgfpathlineto{\pgfqpoint{8.185931in}{2.559728in}}%
\pgfpathlineto{\pgfqpoint{8.189242in}{2.563760in}}%
\pgfpathlineto{\pgfqpoint{8.195865in}{2.578866in}}%
\pgfpathlineto{\pgfqpoint{8.219045in}{2.646002in}}%
\pgfpathlineto{\pgfqpoint{8.225668in}{2.656607in}}%
\pgfpathlineto{\pgfqpoint{8.232291in}{2.662145in}}%
\pgfpathlineto{\pgfqpoint{8.238914in}{2.663438in}}%
\pgfpathlineto{\pgfqpoint{8.245537in}{2.661526in}}%
\pgfpathlineto{\pgfqpoint{8.255471in}{2.654844in}}%
\pgfpathlineto{\pgfqpoint{8.291897in}{2.625095in}}%
\pgfpathlineto{\pgfqpoint{8.308454in}{2.616160in}}%
\pgfpathlineto{\pgfqpoint{8.331635in}{2.604006in}}%
\pgfpathlineto{\pgfqpoint{8.341569in}{2.595997in}}%
\pgfpathlineto{\pgfqpoint{8.351503in}{2.584168in}}%
\pgfpathlineto{\pgfqpoint{8.361438in}{2.567172in}}%
\pgfpathlineto{\pgfqpoint{8.371372in}{2.544790in}}%
\pgfpathlineto{\pgfqpoint{8.397864in}{2.479324in}}%
\pgfpathlineto{\pgfqpoint{8.407798in}{2.462028in}}%
\pgfpathlineto{\pgfqpoint{8.434290in}{2.420945in}}%
\pgfpathlineto{\pgfqpoint{8.447535in}{2.392992in}}%
\pgfpathlineto{\pgfqpoint{8.470715in}{2.342284in}}%
\pgfpathlineto{\pgfqpoint{8.480650in}{2.326404in}}%
\pgfpathlineto{\pgfqpoint{8.487273in}{2.319470in}}%
\pgfpathlineto{\pgfqpoint{8.493896in}{2.316001in}}%
\pgfpathlineto{\pgfqpoint{8.500519in}{2.316248in}}%
\pgfpathlineto{\pgfqpoint{8.507141in}{2.320254in}}%
\pgfpathlineto{\pgfqpoint{8.513764in}{2.327860in}}%
\pgfpathlineto{\pgfqpoint{8.520387in}{2.338710in}}%
\pgfpathlineto{\pgfqpoint{8.530322in}{2.359854in}}%
\pgfpathlineto{\pgfqpoint{8.563436in}{2.437661in}}%
\pgfpathlineto{\pgfqpoint{8.570059in}{2.447232in}}%
\pgfpathlineto{\pgfqpoint{8.576682in}{2.453407in}}%
\pgfpathlineto{\pgfqpoint{8.586616in}{2.458291in}}%
\pgfpathlineto{\pgfqpoint{8.593239in}{2.461784in}}%
\pgfpathlineto{\pgfqpoint{8.599862in}{2.468594in}}%
\pgfpathlineto{\pgfqpoint{8.606485in}{2.480256in}}%
\pgfpathlineto{\pgfqpoint{8.629665in}{2.531200in}}%
\pgfpathlineto{\pgfqpoint{8.632977in}{2.533990in}}%
\pgfpathlineto{\pgfqpoint{8.636288in}{2.534712in}}%
\pgfpathlineto{\pgfqpoint{8.639599in}{2.533284in}}%
\pgfpathlineto{\pgfqpoint{8.642911in}{2.529704in}}%
\pgfpathlineto{\pgfqpoint{8.649534in}{2.516381in}}%
\pgfpathlineto{\pgfqpoint{8.656157in}{2.495697in}}%
\pgfpathlineto{\pgfqpoint{8.666091in}{2.453711in}}%
\pgfpathlineto{\pgfqpoint{8.679337in}{2.384628in}}%
\pgfpathlineto{\pgfqpoint{8.702517in}{2.262111in}}%
\pgfpathlineto{\pgfqpoint{8.712451in}{2.222662in}}%
\pgfpathlineto{\pgfqpoint{8.719074in}{2.203260in}}%
\pgfpathlineto{\pgfqpoint{8.725697in}{2.189581in}}%
\pgfpathlineto{\pgfqpoint{8.732320in}{2.181385in}}%
\pgfpathlineto{\pgfqpoint{8.738943in}{2.178196in}}%
\pgfpathlineto{\pgfqpoint{8.745566in}{2.179349in}}%
\pgfpathlineto{\pgfqpoint{8.752189in}{2.184031in}}%
\pgfpathlineto{\pgfqpoint{8.762123in}{2.195654in}}%
\pgfpathlineto{\pgfqpoint{8.781992in}{2.225172in}}%
\pgfpathlineto{\pgfqpoint{8.818418in}{2.279674in}}%
\pgfpathlineto{\pgfqpoint{8.825041in}{2.286527in}}%
\pgfpathlineto{\pgfqpoint{8.831664in}{2.290820in}}%
\pgfpathlineto{\pgfqpoint{8.838286in}{2.292251in}}%
\pgfpathlineto{\pgfqpoint{8.844909in}{2.290811in}}%
\pgfpathlineto{\pgfqpoint{8.851532in}{2.286700in}}%
\pgfpathlineto{\pgfqpoint{8.858155in}{2.280250in}}%
\pgfpathlineto{\pgfqpoint{8.868089in}{2.267067in}}%
\pgfpathlineto{\pgfqpoint{8.881335in}{2.245225in}}%
\pgfpathlineto{\pgfqpoint{8.907827in}{2.199931in}}%
\pgfpathlineto{\pgfqpoint{8.917761in}{2.186663in}}%
\pgfpathlineto{\pgfqpoint{8.927696in}{2.177358in}}%
\pgfpathlineto{\pgfqpoint{8.934318in}{2.173795in}}%
\pgfpathlineto{\pgfqpoint{8.940941in}{2.172499in}}%
\pgfpathlineto{\pgfqpoint{8.947564in}{2.173451in}}%
\pgfpathlineto{\pgfqpoint{8.954187in}{2.176469in}}%
\pgfpathlineto{\pgfqpoint{8.964122in}{2.184100in}}%
\pgfpathlineto{\pgfqpoint{8.983990in}{2.204403in}}%
\pgfpathlineto{\pgfqpoint{9.000547in}{2.219636in}}%
\pgfpathlineto{\pgfqpoint{9.017105in}{2.231573in}}%
\pgfpathlineto{\pgfqpoint{9.030351in}{2.239252in}}%
\pgfpathlineto{\pgfqpoint{9.040285in}{2.242813in}}%
\pgfpathlineto{\pgfqpoint{9.046908in}{2.243383in}}%
\pgfpathlineto{\pgfqpoint{9.053531in}{2.242146in}}%
\pgfpathlineto{\pgfqpoint{9.060154in}{2.239021in}}%
\pgfpathlineto{\pgfqpoint{9.070088in}{2.231182in}}%
\pgfpathlineto{\pgfqpoint{9.086645in}{2.213372in}}%
\pgfpathlineto{\pgfqpoint{9.099891in}{2.200182in}}%
\pgfpathlineto{\pgfqpoint{9.106514in}{2.195642in}}%
\pgfpathlineto{\pgfqpoint{9.113137in}{2.193187in}}%
\pgfpathlineto{\pgfqpoint{9.119760in}{2.193201in}}%
\pgfpathlineto{\pgfqpoint{9.126383in}{2.195915in}}%
\pgfpathlineto{\pgfqpoint{9.133005in}{2.201419in}}%
\pgfpathlineto{\pgfqpoint{9.139628in}{2.209691in}}%
\pgfpathlineto{\pgfqpoint{9.149563in}{2.227090in}}%
\pgfpathlineto{\pgfqpoint{9.159497in}{2.250050in}}%
\pgfpathlineto{\pgfqpoint{9.172743in}{2.288166in}}%
\pgfpathlineto{\pgfqpoint{9.189300in}{2.344773in}}%
\pgfpathlineto{\pgfqpoint{9.232349in}{2.499459in}}%
\pgfpathlineto{\pgfqpoint{9.245595in}{2.537964in}}%
\pgfpathlineto{\pgfqpoint{9.255529in}{2.561394in}}%
\pgfpathlineto{\pgfqpoint{9.265463in}{2.579420in}}%
\pgfpathlineto{\pgfqpoint{9.275398in}{2.591926in}}%
\pgfpathlineto{\pgfqpoint{9.282021in}{2.597364in}}%
\pgfpathlineto{\pgfqpoint{9.288644in}{2.600685in}}%
\pgfpathlineto{\pgfqpoint{9.295266in}{2.602084in}}%
\pgfpathlineto{\pgfqpoint{9.301889in}{2.601761in}}%
\pgfpathlineto{\pgfqpoint{9.311824in}{2.598505in}}%
\pgfpathlineto{\pgfqpoint{9.321758in}{2.592603in}}%
\pgfpathlineto{\pgfqpoint{9.338315in}{2.579303in}}%
\pgfpathlineto{\pgfqpoint{9.358184in}{2.563519in}}%
\pgfpathlineto{\pgfqpoint{9.368118in}{2.558269in}}%
\pgfpathlineto{\pgfqpoint{9.378053in}{2.556142in}}%
\pgfpathlineto{\pgfqpoint{9.384676in}{2.556805in}}%
\pgfpathlineto{\pgfqpoint{9.391299in}{2.559231in}}%
\pgfpathlineto{\pgfqpoint{9.401233in}{2.566087in}}%
\pgfpathlineto{\pgfqpoint{9.411167in}{2.576471in}}%
\pgfpathlineto{\pgfqpoint{9.421102in}{2.589933in}}%
\pgfpathlineto{\pgfqpoint{9.434347in}{2.611578in}}%
\pgfpathlineto{\pgfqpoint{9.460839in}{2.656418in}}%
\pgfpathlineto{\pgfqpoint{9.470773in}{2.669444in}}%
\pgfpathlineto{\pgfqpoint{9.480708in}{2.679258in}}%
\pgfpathlineto{\pgfqpoint{9.490642in}{2.686048in}}%
\pgfpathlineto{\pgfqpoint{9.503888in}{2.691657in}}%
\pgfpathlineto{\pgfqpoint{9.530379in}{2.700910in}}%
\pgfpathlineto{\pgfqpoint{9.540314in}{2.706832in}}%
\pgfpathlineto{\pgfqpoint{9.550248in}{2.715137in}}%
\pgfpathlineto{\pgfqpoint{9.560182in}{2.726326in}}%
\pgfpathlineto{\pgfqpoint{9.570117in}{2.740882in}}%
\pgfpathlineto{\pgfqpoint{9.580051in}{2.759184in}}%
\pgfpathlineto{\pgfqpoint{9.589986in}{2.781423in}}%
\pgfpathlineto{\pgfqpoint{9.603231in}{2.817213in}}%
\pgfpathlineto{\pgfqpoint{9.616477in}{2.859881in}}%
\pgfpathlineto{\pgfqpoint{9.629723in}{2.909373in}}%
\pgfpathlineto{\pgfqpoint{9.646280in}{2.980983in}}%
\pgfpathlineto{\pgfqpoint{9.662837in}{3.063968in}}%
\pgfpathlineto{\pgfqpoint{9.679395in}{3.158580in}}%
\pgfpathlineto{\pgfqpoint{9.709198in}{3.335257in}}%
\pgfpathlineto{\pgfqpoint{9.719132in}{3.380493in}}%
\pgfpathlineto{\pgfqpoint{9.729066in}{3.415405in}}%
\pgfpathlineto{\pgfqpoint{9.735689in}{3.433779in}}%
\pgfpathlineto{\pgfqpoint{9.739001in}{3.440226in}}%
\pgfpathlineto{\pgfqpoint{9.742312in}{3.443284in}}%
\pgfpathlineto{\pgfqpoint{9.745624in}{3.441448in}}%
\pgfpathlineto{\pgfqpoint{9.748935in}{3.433365in}}%
\pgfpathlineto{\pgfqpoint{9.752247in}{3.418213in}}%
\pgfpathlineto{\pgfqpoint{9.758869in}{3.367430in}}%
\pgfpathlineto{\pgfqpoint{9.785361in}{3.111755in}}%
\pgfpathlineto{\pgfqpoint{9.791984in}{3.080261in}}%
\pgfpathlineto{\pgfqpoint{9.798607in}{3.063259in}}%
\pgfpathlineto{\pgfqpoint{9.801918in}{3.059103in}}%
\pgfpathlineto{\pgfqpoint{9.805230in}{3.057275in}}%
\pgfpathlineto{\pgfqpoint{9.808541in}{3.057402in}}%
\pgfpathlineto{\pgfqpoint{9.811853in}{3.059169in}}%
\pgfpathlineto{\pgfqpoint{9.818476in}{3.066643in}}%
\pgfpathlineto{\pgfqpoint{9.825098in}{3.078220in}}%
\pgfpathlineto{\pgfqpoint{9.835033in}{3.101486in}}%
\pgfpathlineto{\pgfqpoint{9.844967in}{3.130590in}}%
\pgfpathlineto{\pgfqpoint{9.858213in}{3.177240in}}%
\pgfpathlineto{\pgfqpoint{9.871459in}{3.231857in}}%
\pgfpathlineto{\pgfqpoint{9.888016in}{3.310419in}}%
\pgfpathlineto{\pgfqpoint{9.904573in}{3.400909in}}%
\pgfpathlineto{\pgfqpoint{9.921131in}{3.505781in}}%
\pgfpathlineto{\pgfqpoint{9.937688in}{3.628275in}}%
\pgfpathlineto{\pgfqpoint{9.954245in}{3.768616in}}%
\pgfpathlineto{\pgfqpoint{9.974114in}{3.935120in}}%
\pgfpathlineto{\pgfqpoint{9.980737in}{3.973938in}}%
\pgfpathlineto{\pgfqpoint{9.987360in}{3.996793in}}%
\pgfpathlineto{\pgfqpoint{9.993982in}{4.010290in}}%
\pgfpathlineto{\pgfqpoint{9.997294in}{4.021657in}}%
\pgfpathlineto{\pgfqpoint{10.000605in}{4.043710in}}%
\pgfpathlineto{\pgfqpoint{10.003917in}{4.083015in}}%
\pgfpathlineto{\pgfqpoint{10.007228in}{4.144777in}}%
\pgfpathlineto{\pgfqpoint{10.013851in}{4.338851in}}%
\pgfpathlineto{\pgfqpoint{10.030408in}{4.944655in}}%
\pgfpathlineto{\pgfqpoint{10.037031in}{5.105862in}}%
\pgfpathlineto{\pgfqpoint{10.043654in}{5.207035in}}%
\pgfpathlineto{\pgfqpoint{10.050277in}{5.266170in}}%
\pgfpathlineto{\pgfqpoint{10.056900in}{5.303050in}}%
\pgfpathlineto{\pgfqpoint{10.070146in}{5.365380in}}%
\pgfpathlineto{\pgfqpoint{10.080080in}{5.427454in}}%
\pgfpathlineto{\pgfqpoint{10.090014in}{5.506831in}}%
\pgfpathlineto{\pgfqpoint{10.106572in}{5.664063in}}%
\pgfpathlineto{\pgfqpoint{10.133063in}{5.921286in}}%
\pgfpathlineto{\pgfqpoint{10.146309in}{6.028256in}}%
\pgfpathlineto{\pgfqpoint{10.156243in}{6.090135in}}%
\pgfpathlineto{\pgfqpoint{10.162866in}{6.120433in}}%
\pgfpathlineto{\pgfqpoint{10.169489in}{6.141395in}}%
\pgfpathlineto{\pgfqpoint{10.176112in}{6.153734in}}%
\pgfpathlineto{\pgfqpoint{10.182735in}{6.159688in}}%
\pgfpathlineto{\pgfqpoint{10.192669in}{6.165333in}}%
\pgfpathlineto{\pgfqpoint{10.195981in}{6.168769in}}%
\pgfpathlineto{\pgfqpoint{10.199292in}{6.173941in}}%
\pgfpathlineto{\pgfqpoint{10.205915in}{6.190989in}}%
\pgfpathlineto{\pgfqpoint{10.212538in}{6.217471in}}%
\pgfpathlineto{\pgfqpoint{10.235718in}{6.328625in}}%
\pgfpathlineto{\pgfqpoint{10.242341in}{6.344421in}}%
\pgfpathlineto{\pgfqpoint{10.245653in}{6.347951in}}%
\pgfpathlineto{\pgfqpoint{10.248964in}{6.348692in}}%
\pgfpathlineto{\pgfqpoint{10.252276in}{6.346888in}}%
\pgfpathlineto{\pgfqpoint{10.255587in}{6.342877in}}%
\pgfpathlineto{\pgfqpoint{10.262210in}{6.329920in}}%
\pgfpathlineto{\pgfqpoint{10.278767in}{6.292705in}}%
\pgfpathlineto{\pgfqpoint{10.282079in}{6.289014in}}%
\pgfpathlineto{\pgfqpoint{10.285390in}{6.287905in}}%
\pgfpathlineto{\pgfqpoint{10.288701in}{6.289842in}}%
\pgfpathlineto{\pgfqpoint{10.292013in}{6.295176in}}%
\pgfpathlineto{\pgfqpoint{10.295324in}{6.304115in}}%
\pgfpathlineto{\pgfqpoint{10.301947in}{6.332880in}}%
\pgfpathlineto{\pgfqpoint{10.308570in}{6.374969in}}%
\pgfpathlineto{\pgfqpoint{10.318505in}{6.457269in}}%
\pgfpathlineto{\pgfqpoint{10.335062in}{6.621039in}}%
\pgfpathlineto{\pgfqpoint{10.364865in}{6.914504in}}%
\pgfpathlineto{\pgfqpoint{10.374799in}{7.021301in}}%
\pgfpathlineto{\pgfqpoint{10.384734in}{7.158639in}}%
\pgfpathlineto{\pgfqpoint{10.394668in}{7.343025in}}%
\pgfpathlineto{\pgfqpoint{10.404602in}{7.575028in}}%
\pgfpathlineto{\pgfqpoint{10.427782in}{8.146977in}}%
\pgfpathlineto{\pgfqpoint{10.434405in}{8.255777in}}%
\pgfpathlineto{\pgfqpoint{10.437717in}{8.289796in}}%
\pgfpathlineto{\pgfqpoint{10.441028in}{8.306350in}}%
\pgfpathlineto{\pgfqpoint{10.444340in}{8.302106in}}%
\pgfpathlineto{\pgfqpoint{10.447651in}{8.273394in}}%
\pgfpathlineto{\pgfqpoint{10.450963in}{8.216292in}}%
\pgfpathlineto{\pgfqpoint{10.454274in}{8.126806in}}%
\pgfpathlineto{\pgfqpoint{10.460897in}{7.836411in}}%
\pgfpathlineto{\pgfqpoint{10.467520in}{7.384375in}}%
\pgfpathlineto{\pgfqpoint{10.477454in}{6.444848in}}%
\pgfpathlineto{\pgfqpoint{10.490700in}{5.109071in}}%
\pgfpathlineto{\pgfqpoint{10.497323in}{4.620622in}}%
\pgfpathlineto{\pgfqpoint{10.503946in}{4.310734in}}%
\pgfpathlineto{\pgfqpoint{10.507257in}{4.216005in}}%
\pgfpathlineto{\pgfqpoint{10.510569in}{4.153445in}}%
\pgfpathlineto{\pgfqpoint{10.513880in}{4.116495in}}%
\pgfpathlineto{\pgfqpoint{10.517191in}{4.099153in}}%
\pgfpathlineto{\pgfqpoint{10.520503in}{4.096319in}}%
\pgfpathlineto{\pgfqpoint{10.523814in}{4.103824in}}%
\pgfpathlineto{\pgfqpoint{10.530437in}{4.136874in}}%
\pgfpathlineto{\pgfqpoint{10.540372in}{4.194349in}}%
\pgfpathlineto{\pgfqpoint{10.543683in}{4.207076in}}%
\pgfpathlineto{\pgfqpoint{10.546995in}{4.213171in}}%
\pgfpathlineto{\pgfqpoint{10.550306in}{4.210573in}}%
\pgfpathlineto{\pgfqpoint{10.553617in}{4.197260in}}%
\pgfpathlineto{\pgfqpoint{10.556929in}{4.171361in}}%
\pgfpathlineto{\pgfqpoint{10.560240in}{4.131357in}}%
\pgfpathlineto{\pgfqpoint{10.566863in}{4.006459in}}%
\pgfpathlineto{\pgfqpoint{10.576798in}{3.730063in}}%
\pgfpathlineto{\pgfqpoint{10.586732in}{3.464944in}}%
\pgfpathlineto{\pgfqpoint{10.590043in}{3.409158in}}%
\pgfpathlineto{\pgfqpoint{10.593355in}{3.376233in}}%
\pgfpathlineto{\pgfqpoint{10.596666in}{3.365914in}}%
\pgfpathlineto{\pgfqpoint{10.599978in}{3.375374in}}%
\pgfpathlineto{\pgfqpoint{10.603289in}{3.400254in}}%
\pgfpathlineto{\pgfqpoint{10.609912in}{3.477725in}}%
\pgfpathlineto{\pgfqpoint{10.629781in}{3.733501in}}%
\pgfpathlineto{\pgfqpoint{10.639715in}{3.834174in}}%
\pgfpathlineto{\pgfqpoint{10.652961in}{3.943796in}}%
\pgfpathlineto{\pgfqpoint{10.666207in}{4.031542in}}%
\pgfpathlineto{\pgfqpoint{10.676141in}{4.084149in}}%
\pgfpathlineto{\pgfqpoint{10.686075in}{4.125216in}}%
\pgfpathlineto{\pgfqpoint{10.692698in}{4.145866in}}%
\pgfpathlineto{\pgfqpoint{10.699321in}{4.160770in}}%
\pgfpathlineto{\pgfqpoint{10.705944in}{4.169462in}}%
\pgfpathlineto{\pgfqpoint{10.709256in}{4.171287in}}%
\pgfpathlineto{\pgfqpoint{10.712567in}{4.171309in}}%
\pgfpathlineto{\pgfqpoint{10.715878in}{4.169421in}}%
\pgfpathlineto{\pgfqpoint{10.719190in}{4.165511in}}%
\pgfpathlineto{\pgfqpoint{10.725813in}{4.151150in}}%
\pgfpathlineto{\pgfqpoint{10.732436in}{4.127318in}}%
\pgfpathlineto{\pgfqpoint{10.739059in}{4.093388in}}%
\pgfpathlineto{\pgfqpoint{10.748993in}{4.024152in}}%
\pgfpathlineto{\pgfqpoint{10.775485in}{3.807850in}}%
\pgfpathlineto{\pgfqpoint{10.782107in}{3.774410in}}%
\pgfpathlineto{\pgfqpoint{10.788730in}{3.756038in}}%
\pgfpathlineto{\pgfqpoint{10.792042in}{3.752256in}}%
\pgfpathlineto{\pgfqpoint{10.795353in}{3.751599in}}%
\pgfpathlineto{\pgfqpoint{10.798665in}{3.753530in}}%
\pgfpathlineto{\pgfqpoint{10.805288in}{3.762527in}}%
\pgfpathlineto{\pgfqpoint{10.815222in}{3.778448in}}%
\pgfpathlineto{\pgfqpoint{10.821845in}{3.783866in}}%
\pgfpathlineto{\pgfqpoint{10.825156in}{3.783898in}}%
\pgfpathlineto{\pgfqpoint{10.828468in}{3.781929in}}%
\pgfpathlineto{\pgfqpoint{10.831779in}{3.777930in}}%
\pgfpathlineto{\pgfqpoint{10.838402in}{3.764143in}}%
\pgfpathlineto{\pgfqpoint{10.845025in}{3.743723in}}%
\pgfpathlineto{\pgfqpoint{10.854959in}{3.704625in}}%
\pgfpathlineto{\pgfqpoint{10.878140in}{3.608404in}}%
\pgfpathlineto{\pgfqpoint{10.888074in}{3.578033in}}%
\pgfpathlineto{\pgfqpoint{10.894697in}{3.563451in}}%
\pgfpathlineto{\pgfqpoint{10.901320in}{3.553183in}}%
\pgfpathlineto{\pgfqpoint{10.907943in}{3.546386in}}%
\pgfpathlineto{\pgfqpoint{10.931123in}{3.528651in}}%
\pgfpathlineto{\pgfqpoint{10.937746in}{3.518499in}}%
\pgfpathlineto{\pgfqpoint{10.944369in}{3.503858in}}%
\pgfpathlineto{\pgfqpoint{10.954303in}{3.473874in}}%
\pgfpathlineto{\pgfqpoint{10.970860in}{3.419801in}}%
\pgfpathlineto{\pgfqpoint{10.977483in}{3.406591in}}%
\pgfpathlineto{\pgfqpoint{10.980794in}{3.403389in}}%
\pgfpathlineto{\pgfqpoint{10.984106in}{3.402763in}}%
\pgfpathlineto{\pgfqpoint{10.987417in}{3.404800in}}%
\pgfpathlineto{\pgfqpoint{10.990729in}{3.409465in}}%
\pgfpathlineto{\pgfqpoint{10.997352in}{3.425956in}}%
\pgfpathlineto{\pgfqpoint{11.003975in}{3.449852in}}%
\pgfpathlineto{\pgfqpoint{11.030466in}{3.556071in}}%
\pgfpathlineto{\pgfqpoint{11.037089in}{3.574280in}}%
\pgfpathlineto{\pgfqpoint{11.043712in}{3.587562in}}%
\pgfpathlineto{\pgfqpoint{11.050335in}{3.596077in}}%
\pgfpathlineto{\pgfqpoint{11.056958in}{3.600237in}}%
\pgfpathlineto{\pgfqpoint{11.063581in}{3.600610in}}%
\pgfpathlineto{\pgfqpoint{11.070204in}{3.597866in}}%
\pgfpathlineto{\pgfqpoint{11.076827in}{3.592708in}}%
\pgfpathlineto{\pgfqpoint{11.086761in}{3.581789in}}%
\pgfpathlineto{\pgfqpoint{11.100007in}{3.562993in}}%
\pgfpathlineto{\pgfqpoint{11.109941in}{3.545442in}}%
\pgfpathlineto{\pgfqpoint{11.119875in}{3.523618in}}%
\pgfpathlineto{\pgfqpoint{11.129810in}{3.496111in}}%
\pgfpathlineto{\pgfqpoint{11.139744in}{3.461682in}}%
\pgfpathlineto{\pgfqpoint{11.149678in}{3.419406in}}%
\pgfpathlineto{\pgfqpoint{11.162924in}{3.350509in}}%
\pgfpathlineto{\pgfqpoint{11.179481in}{3.249173in}}%
\pgfpathlineto{\pgfqpoint{11.202662in}{3.107867in}}%
\pgfpathlineto{\pgfqpoint{11.215907in}{3.041538in}}%
\pgfpathlineto{\pgfqpoint{11.229153in}{2.988274in}}%
\pgfpathlineto{\pgfqpoint{11.242399in}{2.946189in}}%
\pgfpathlineto{\pgfqpoint{11.252333in}{2.920908in}}%
\pgfpathlineto{\pgfqpoint{11.262268in}{2.900815in}}%
\pgfpathlineto{\pgfqpoint{11.272202in}{2.886071in}}%
\pgfpathlineto{\pgfqpoint{11.278825in}{2.879273in}}%
\pgfpathlineto{\pgfqpoint{11.285448in}{2.874866in}}%
\pgfpathlineto{\pgfqpoint{11.292071in}{2.872732in}}%
\pgfpathlineto{\pgfqpoint{11.298694in}{2.872672in}}%
\pgfpathlineto{\pgfqpoint{11.305317in}{2.874412in}}%
\pgfpathlineto{\pgfqpoint{11.315251in}{2.879666in}}%
\pgfpathlineto{\pgfqpoint{11.328497in}{2.889772in}}%
\pgfpathlineto{\pgfqpoint{11.354988in}{2.911353in}}%
\pgfpathlineto{\pgfqpoint{11.364923in}{2.917055in}}%
\pgfpathlineto{\pgfqpoint{11.374857in}{2.920008in}}%
\pgfpathlineto{\pgfqpoint{11.381480in}{2.920052in}}%
\pgfpathlineto{\pgfqpoint{11.388103in}{2.918375in}}%
\pgfpathlineto{\pgfqpoint{11.394726in}{2.914930in}}%
\pgfpathlineto{\pgfqpoint{11.404660in}{2.906577in}}%
\pgfpathlineto{\pgfqpoint{11.414594in}{2.894760in}}%
\pgfpathlineto{\pgfqpoint{11.424529in}{2.879770in}}%
\pgfpathlineto{\pgfqpoint{11.434463in}{2.861512in}}%
\pgfpathlineto{\pgfqpoint{11.447709in}{2.831664in}}%
\pgfpathlineto{\pgfqpoint{11.464266in}{2.786886in}}%
\pgfpathlineto{\pgfqpoint{11.487446in}{2.723781in}}%
\pgfpathlineto{\pgfqpoint{11.497381in}{2.702460in}}%
\pgfpathlineto{\pgfqpoint{11.507315in}{2.687012in}}%
\pgfpathlineto{\pgfqpoint{11.513938in}{2.680434in}}%
\pgfpathlineto{\pgfqpoint{11.520561in}{2.676945in}}%
\pgfpathlineto{\pgfqpoint{11.527184in}{2.676482in}}%
\pgfpathlineto{\pgfqpoint{11.533807in}{2.678845in}}%
\pgfpathlineto{\pgfqpoint{11.540430in}{2.683710in}}%
\pgfpathlineto{\pgfqpoint{11.550364in}{2.694717in}}%
\pgfpathlineto{\pgfqpoint{11.566921in}{2.718241in}}%
\pgfpathlineto{\pgfqpoint{11.580167in}{2.736103in}}%
\pgfpathlineto{\pgfqpoint{11.590101in}{2.746009in}}%
\pgfpathlineto{\pgfqpoint{11.596724in}{2.750300in}}%
\pgfpathlineto{\pgfqpoint{11.603347in}{2.752655in}}%
\pgfpathlineto{\pgfqpoint{11.609970in}{2.753232in}}%
\pgfpathlineto{\pgfqpoint{11.619904in}{2.751400in}}%
\pgfpathlineto{\pgfqpoint{11.629839in}{2.747121in}}%
\pgfpathlineto{\pgfqpoint{11.639773in}{2.740691in}}%
\pgfpathlineto{\pgfqpoint{11.649707in}{2.731705in}}%
\pgfpathlineto{\pgfqpoint{11.659642in}{2.719340in}}%
\pgfpathlineto{\pgfqpoint{11.669576in}{2.702808in}}%
\pgfpathlineto{\pgfqpoint{11.679510in}{2.681837in}}%
\pgfpathlineto{\pgfqpoint{11.692756in}{2.647912in}}%
\pgfpathlineto{\pgfqpoint{11.739117in}{2.521229in}}%
\pgfpathlineto{\pgfqpoint{11.749051in}{2.502439in}}%
\pgfpathlineto{\pgfqpoint{11.758985in}{2.488846in}}%
\pgfpathlineto{\pgfqpoint{11.765608in}{2.482702in}}%
\pgfpathlineto{\pgfqpoint{11.772231in}{2.478734in}}%
\pgfpathlineto{\pgfqpoint{11.778854in}{2.476689in}}%
\pgfpathlineto{\pgfqpoint{11.788788in}{2.476529in}}%
\pgfpathlineto{\pgfqpoint{11.798723in}{2.478879in}}%
\pgfpathlineto{\pgfqpoint{11.811968in}{2.484556in}}%
\pgfpathlineto{\pgfqpoint{11.825214in}{2.492747in}}%
\pgfpathlineto{\pgfqpoint{11.841771in}{2.506232in}}%
\pgfpathlineto{\pgfqpoint{11.858329in}{2.519075in}}%
\pgfpathlineto{\pgfqpoint{11.874886in}{2.531428in}}%
\pgfpathlineto{\pgfqpoint{11.884820in}{2.542931in}}%
\pgfpathlineto{\pgfqpoint{11.894755in}{2.558919in}}%
\pgfpathlineto{\pgfqpoint{11.911312in}{2.591732in}}%
\pgfpathlineto{\pgfqpoint{11.931181in}{2.629934in}}%
\pgfpathlineto{\pgfqpoint{11.941115in}{2.645329in}}%
\pgfpathlineto{\pgfqpoint{11.951049in}{2.657228in}}%
\pgfpathlineto{\pgfqpoint{11.960984in}{2.665111in}}%
\pgfpathlineto{\pgfqpoint{11.967607in}{2.667902in}}%
\pgfpathlineto{\pgfqpoint{11.974229in}{2.668572in}}%
\pgfpathlineto{\pgfqpoint{11.980852in}{2.667010in}}%
\pgfpathlineto{\pgfqpoint{11.987475in}{2.663135in}}%
\pgfpathlineto{\pgfqpoint{11.994098in}{2.656908in}}%
\pgfpathlineto{\pgfqpoint{12.004032in}{2.643216in}}%
\pgfpathlineto{\pgfqpoint{12.013967in}{2.624654in}}%
\pgfpathlineto{\pgfqpoint{12.027213in}{2.593823in}}%
\pgfpathlineto{\pgfqpoint{12.057016in}{2.519401in}}%
\pgfpathlineto{\pgfqpoint{12.066950in}{2.501739in}}%
\pgfpathlineto{\pgfqpoint{12.073573in}{2.494171in}}%
\pgfpathlineto{\pgfqpoint{12.080196in}{2.490869in}}%
\pgfpathlineto{\pgfqpoint{12.083507in}{2.491083in}}%
\pgfpathlineto{\pgfqpoint{12.086819in}{2.492693in}}%
\pgfpathlineto{\pgfqpoint{12.090130in}{2.495814in}}%
\pgfpathlineto{\pgfqpoint{12.096753in}{2.507021in}}%
\pgfpathlineto{\pgfqpoint{12.103376in}{2.525407in}}%
\pgfpathlineto{\pgfqpoint{12.109999in}{2.551257in}}%
\pgfpathlineto{\pgfqpoint{12.119933in}{2.603152in}}%
\pgfpathlineto{\pgfqpoint{12.133179in}{2.690265in}}%
\pgfpathlineto{\pgfqpoint{12.156359in}{2.848058in}}%
\pgfpathlineto{\pgfqpoint{12.166294in}{2.900768in}}%
\pgfpathlineto{\pgfqpoint{12.172916in}{2.927448in}}%
\pgfpathlineto{\pgfqpoint{12.179539in}{2.946549in}}%
\pgfpathlineto{\pgfqpoint{12.186162in}{2.957845in}}%
\pgfpathlineto{\pgfqpoint{12.189474in}{2.960606in}}%
\pgfpathlineto{\pgfqpoint{12.192785in}{2.961512in}}%
\pgfpathlineto{\pgfqpoint{12.196097in}{2.960642in}}%
\pgfpathlineto{\pgfqpoint{12.199408in}{2.958098in}}%
\pgfpathlineto{\pgfqpoint{12.206031in}{2.948497in}}%
\pgfpathlineto{\pgfqpoint{12.212654in}{2.933903in}}%
\pgfpathlineto{\pgfqpoint{12.222588in}{2.905860in}}%
\pgfpathlineto{\pgfqpoint{12.242457in}{2.847441in}}%
\pgfpathlineto{\pgfqpoint{12.252391in}{2.826109in}}%
\pgfpathlineto{\pgfqpoint{12.259014in}{2.816411in}}%
\pgfpathlineto{\pgfqpoint{12.265637in}{2.810308in}}%
\pgfpathlineto{\pgfqpoint{12.272260in}{2.807417in}}%
\pgfpathlineto{\pgfqpoint{12.278883in}{2.807204in}}%
\pgfpathlineto{\pgfqpoint{12.285506in}{2.809118in}}%
\pgfpathlineto{\pgfqpoint{12.295440in}{2.814905in}}%
\pgfpathlineto{\pgfqpoint{12.325243in}{2.836910in}}%
\pgfpathlineto{\pgfqpoint{12.331866in}{2.838644in}}%
\pgfpathlineto{\pgfqpoint{12.338489in}{2.838111in}}%
\pgfpathlineto{\pgfqpoint{12.345112in}{2.835277in}}%
\pgfpathlineto{\pgfqpoint{12.355046in}{2.827374in}}%
\pgfpathlineto{\pgfqpoint{12.368292in}{2.812834in}}%
\pgfpathlineto{\pgfqpoint{12.388161in}{2.790992in}}%
\pgfpathlineto{\pgfqpoint{12.401406in}{2.780043in}}%
\pgfpathlineto{\pgfqpoint{12.411341in}{2.774509in}}%
\pgfpathlineto{\pgfqpoint{12.421275in}{2.771750in}}%
\pgfpathlineto{\pgfqpoint{12.427898in}{2.771784in}}%
\pgfpathlineto{\pgfqpoint{12.434521in}{2.773608in}}%
\pgfpathlineto{\pgfqpoint{12.441144in}{2.777472in}}%
\pgfpathlineto{\pgfqpoint{12.447767in}{2.783576in}}%
\pgfpathlineto{\pgfqpoint{12.454390in}{2.792031in}}%
\pgfpathlineto{\pgfqpoint{12.464324in}{2.809081in}}%
\pgfpathlineto{\pgfqpoint{12.474258in}{2.830951in}}%
\pgfpathlineto{\pgfqpoint{12.487504in}{2.866317in}}%
\pgfpathlineto{\pgfqpoint{12.504061in}{2.918001in}}%
\pgfpathlineto{\pgfqpoint{12.543799in}{3.049038in}}%
\pgfpathlineto{\pgfqpoint{12.553733in}{3.073643in}}%
\pgfpathlineto{\pgfqpoint{12.563668in}{3.092244in}}%
\pgfpathlineto{\pgfqpoint{12.573602in}{3.105254in}}%
\pgfpathlineto{\pgfqpoint{12.583536in}{3.113695in}}%
\pgfpathlineto{\pgfqpoint{12.593471in}{3.118494in}}%
\pgfpathlineto{\pgfqpoint{12.603405in}{3.120196in}}%
\pgfpathlineto{\pgfqpoint{12.613339in}{3.119114in}}%
\pgfpathlineto{\pgfqpoint{12.623274in}{3.115642in}}%
\pgfpathlineto{\pgfqpoint{12.639831in}{3.106641in}}%
\pgfpathlineto{\pgfqpoint{12.656388in}{3.098181in}}%
\pgfpathlineto{\pgfqpoint{12.666322in}{3.095262in}}%
\pgfpathlineto{\pgfqpoint{12.679568in}{3.094100in}}%
\pgfpathlineto{\pgfqpoint{12.702748in}{3.093566in}}%
\pgfpathlineto{\pgfqpoint{12.712683in}{3.091284in}}%
\pgfpathlineto{\pgfqpoint{12.722617in}{3.086908in}}%
\pgfpathlineto{\pgfqpoint{12.732551in}{3.080345in}}%
\pgfpathlineto{\pgfqpoint{12.745797in}{3.068658in}}%
\pgfpathlineto{\pgfqpoint{12.795469in}{3.020342in}}%
\pgfpathlineto{\pgfqpoint{12.805403in}{3.014566in}}%
\pgfpathlineto{\pgfqpoint{12.815338in}{3.011113in}}%
\pgfpathlineto{\pgfqpoint{12.825272in}{3.010316in}}%
\pgfpathlineto{\pgfqpoint{12.835206in}{3.012482in}}%
\pgfpathlineto{\pgfqpoint{12.845141in}{3.017833in}}%
\pgfpathlineto{\pgfqpoint{12.855075in}{3.026417in}}%
\pgfpathlineto{\pgfqpoint{12.865009in}{3.037999in}}%
\pgfpathlineto{\pgfqpoint{12.878255in}{3.057048in}}%
\pgfpathlineto{\pgfqpoint{12.944484in}{3.162393in}}%
\pgfpathlineto{\pgfqpoint{12.957730in}{3.188928in}}%
\pgfpathlineto{\pgfqpoint{12.974287in}{3.227975in}}%
\pgfpathlineto{\pgfqpoint{13.004090in}{3.306930in}}%
\pgfpathlineto{\pgfqpoint{13.020648in}{3.348375in}}%
\pgfpathlineto{\pgfqpoint{13.033893in}{3.375822in}}%
\pgfpathlineto{\pgfqpoint{13.043828in}{3.391277in}}%
\pgfpathlineto{\pgfqpoint{13.050451in}{3.398662in}}%
\pgfpathlineto{\pgfqpoint{13.057074in}{3.403552in}}%
\pgfpathlineto{\pgfqpoint{13.063696in}{3.405920in}}%
\pgfpathlineto{\pgfqpoint{13.070319in}{3.405823in}}%
\pgfpathlineto{\pgfqpoint{13.076942in}{3.403389in}}%
\pgfpathlineto{\pgfqpoint{13.083565in}{3.398800in}}%
\pgfpathlineto{\pgfqpoint{13.093499in}{3.388352in}}%
\pgfpathlineto{\pgfqpoint{13.103434in}{3.374364in}}%
\pgfpathlineto{\pgfqpoint{13.116680in}{3.351777in}}%
\pgfpathlineto{\pgfqpoint{13.156417in}{3.280246in}}%
\pgfpathlineto{\pgfqpoint{13.169663in}{3.262513in}}%
\pgfpathlineto{\pgfqpoint{13.179597in}{3.252663in}}%
\pgfpathlineto{\pgfqpoint{13.189532in}{3.246064in}}%
\pgfpathlineto{\pgfqpoint{13.199466in}{3.242930in}}%
\pgfpathlineto{\pgfqpoint{13.206089in}{3.242888in}}%
\pgfpathlineto{\pgfqpoint{13.212712in}{3.244588in}}%
\pgfpathlineto{\pgfqpoint{13.219335in}{3.248123in}}%
\pgfpathlineto{\pgfqpoint{13.225957in}{3.253581in}}%
\pgfpathlineto{\pgfqpoint{13.235892in}{3.265482in}}%
\pgfpathlineto{\pgfqpoint{13.245826in}{3.281786in}}%
\pgfpathlineto{\pgfqpoint{13.255761in}{3.302102in}}%
\pgfpathlineto{\pgfqpoint{13.269006in}{3.334133in}}%
\pgfpathlineto{\pgfqpoint{13.292186in}{3.397801in}}%
\pgfpathlineto{\pgfqpoint{13.358415in}{3.584421in}}%
\pgfpathlineto{\pgfqpoint{13.371661in}{3.613562in}}%
\pgfpathlineto{\pgfqpoint{13.384907in}{3.636741in}}%
\pgfpathlineto{\pgfqpoint{13.394841in}{3.649884in}}%
\pgfpathlineto{\pgfqpoint{13.404776in}{3.659025in}}%
\pgfpathlineto{\pgfqpoint{13.411399in}{3.662645in}}%
\pgfpathlineto{\pgfqpoint{13.418022in}{3.664100in}}%
\pgfpathlineto{\pgfqpoint{13.424644in}{3.663286in}}%
\pgfpathlineto{\pgfqpoint{13.431267in}{3.660214in}}%
\pgfpathlineto{\pgfqpoint{13.437890in}{3.655053in}}%
\pgfpathlineto{\pgfqpoint{13.447825in}{3.644174in}}%
\pgfpathlineto{\pgfqpoint{13.467693in}{3.617117in}}%
\pgfpathlineto{\pgfqpoint{13.490873in}{3.583470in}}%
\pgfpathlineto{\pgfqpoint{13.504119in}{3.559939in}}%
\pgfpathlineto{\pgfqpoint{13.520677in}{3.524196in}}%
\pgfpathlineto{\pgfqpoint{13.540545in}{3.481004in}}%
\pgfpathlineto{\pgfqpoint{13.550480in}{3.464239in}}%
\pgfpathlineto{\pgfqpoint{13.557102in}{3.456009in}}%
\pgfpathlineto{\pgfqpoint{13.563725in}{3.450497in}}%
\pgfpathlineto{\pgfqpoint{13.570348in}{3.447891in}}%
\pgfpathlineto{\pgfqpoint{13.576971in}{3.448294in}}%
\pgfpathlineto{\pgfqpoint{13.583594in}{3.451747in}}%
\pgfpathlineto{\pgfqpoint{13.590217in}{3.458253in}}%
\pgfpathlineto{\pgfqpoint{13.596840in}{3.467788in}}%
\pgfpathlineto{\pgfqpoint{13.606774in}{3.487688in}}%
\pgfpathlineto{\pgfqpoint{13.616709in}{3.514152in}}%
\pgfpathlineto{\pgfqpoint{13.626643in}{3.546933in}}%
\pgfpathlineto{\pgfqpoint{13.639889in}{3.599745in}}%
\pgfpathlineto{\pgfqpoint{13.656446in}{3.678178in}}%
\pgfpathlineto{\pgfqpoint{13.673003in}{3.767422in}}%
\pgfpathlineto{\pgfqpoint{13.692872in}{3.887770in}}%
\pgfpathlineto{\pgfqpoint{13.719364in}{4.067465in}}%
\pgfpathlineto{\pgfqpoint{13.739232in}{4.195718in}}%
\pgfpathlineto{\pgfqpoint{13.752478in}{4.267086in}}%
\pgfpathlineto{\pgfqpoint{13.765724in}{4.324731in}}%
\pgfpathlineto{\pgfqpoint{13.775658in}{4.359597in}}%
\pgfpathlineto{\pgfqpoint{13.785593in}{4.388402in}}%
\pgfpathlineto{\pgfqpoint{13.798838in}{4.419212in}}%
\pgfpathlineto{\pgfqpoint{13.812084in}{4.443074in}}%
\pgfpathlineto{\pgfqpoint{13.822018in}{4.456957in}}%
\pgfpathlineto{\pgfqpoint{13.831953in}{4.467312in}}%
\pgfpathlineto{\pgfqpoint{13.838576in}{4.472022in}}%
\pgfpathlineto{\pgfqpoint{13.845199in}{4.474714in}}%
\pgfpathlineto{\pgfqpoint{13.851822in}{4.475104in}}%
\pgfpathlineto{\pgfqpoint{13.858444in}{4.472893in}}%
\pgfpathlineto{\pgfqpoint{13.865067in}{4.467805in}}%
\pgfpathlineto{\pgfqpoint{13.871690in}{4.459654in}}%
\pgfpathlineto{\pgfqpoint{13.878313in}{4.448400in}}%
\pgfpathlineto{\pgfqpoint{13.888247in}{4.426123in}}%
\pgfpathlineto{\pgfqpoint{13.901493in}{4.388835in}}%
\pgfpathlineto{\pgfqpoint{13.947854in}{4.249582in}}%
\pgfpathlineto{\pgfqpoint{13.957788in}{4.226167in}}%
\pgfpathlineto{\pgfqpoint{13.967722in}{4.208493in}}%
\pgfpathlineto{\pgfqpoint{13.974345in}{4.200844in}}%
\pgfpathlineto{\pgfqpoint{13.980968in}{4.197006in}}%
\pgfpathlineto{\pgfqpoint{13.987591in}{4.197205in}}%
\pgfpathlineto{\pgfqpoint{13.994214in}{4.201467in}}%
\pgfpathlineto{\pgfqpoint{14.000837in}{4.209622in}}%
\pgfpathlineto{\pgfqpoint{14.007460in}{4.221340in}}%
\pgfpathlineto{\pgfqpoint{14.017394in}{4.244622in}}%
\pgfpathlineto{\pgfqpoint{14.030640in}{4.283686in}}%
\pgfpathlineto{\pgfqpoint{14.050509in}{4.351868in}}%
\pgfpathlineto{\pgfqpoint{14.083623in}{4.466479in}}%
\pgfpathlineto{\pgfqpoint{14.120049in}{4.579091in}}%
\pgfpathlineto{\pgfqpoint{14.139918in}{4.638923in}}%
\pgfpathlineto{\pgfqpoint{14.149852in}{4.661715in}}%
\pgfpathlineto{\pgfqpoint{14.156475in}{4.672026in}}%
\pgfpathlineto{\pgfqpoint{14.163098in}{4.677783in}}%
\pgfpathlineto{\pgfqpoint{14.169721in}{4.678855in}}%
\pgfpathlineto{\pgfqpoint{14.176344in}{4.675422in}}%
\pgfpathlineto{\pgfqpoint{14.182967in}{4.667875in}}%
\pgfpathlineto{\pgfqpoint{14.189589in}{4.656706in}}%
\pgfpathlineto{\pgfqpoint{14.199524in}{4.634303in}}%
\pgfpathlineto{\pgfqpoint{14.212770in}{4.596584in}}%
\pgfpathlineto{\pgfqpoint{14.235950in}{4.520359in}}%
\pgfpathlineto{\pgfqpoint{14.259130in}{4.447203in}}%
\pgfpathlineto{\pgfqpoint{14.275687in}{4.402342in}}%
\pgfpathlineto{\pgfqpoint{14.295556in}{4.355824in}}%
\pgfpathlineto{\pgfqpoint{14.312113in}{4.321214in}}%
\pgfpathlineto{\pgfqpoint{14.322047in}{4.305642in}}%
\pgfpathlineto{\pgfqpoint{14.328670in}{4.300296in}}%
\pgfpathlineto{\pgfqpoint{14.331982in}{4.299718in}}%
\pgfpathlineto{\pgfqpoint{14.335293in}{4.300725in}}%
\pgfpathlineto{\pgfqpoint{14.338605in}{4.303386in}}%
\pgfpathlineto{\pgfqpoint{14.345228in}{4.313573in}}%
\pgfpathlineto{\pgfqpoint{14.351850in}{4.329349in}}%
\pgfpathlineto{\pgfqpoint{14.365096in}{4.369661in}}%
\pgfpathlineto{\pgfqpoint{14.378342in}{4.407793in}}%
\pgfpathlineto{\pgfqpoint{14.388276in}{4.429795in}}%
\pgfpathlineto{\pgfqpoint{14.401522in}{4.452116in}}%
\pgfpathlineto{\pgfqpoint{14.418079in}{4.479351in}}%
\pgfpathlineto{\pgfqpoint{14.428014in}{4.501070in}}%
\pgfpathlineto{\pgfqpoint{14.437948in}{4.529941in}}%
\pgfpathlineto{\pgfqpoint{14.447882in}{4.567812in}}%
\pgfpathlineto{\pgfqpoint{14.457817in}{4.615857in}}%
\pgfpathlineto{\pgfqpoint{14.467751in}{4.674357in}}%
\pgfpathlineto{\pgfqpoint{14.480997in}{4.767000in}}%
\pgfpathlineto{\pgfqpoint{14.500866in}{4.925535in}}%
\pgfpathlineto{\pgfqpoint{14.524046in}{5.106222in}}%
\pgfpathlineto{\pgfqpoint{14.540603in}{5.214815in}}%
\pgfpathlineto{\pgfqpoint{14.590275in}{5.518143in}}%
\pgfpathlineto{\pgfqpoint{14.610144in}{5.640739in}}%
\pgfpathlineto{\pgfqpoint{14.620078in}{5.688370in}}%
\pgfpathlineto{\pgfqpoint{14.623389in}{5.698346in}}%
\pgfpathlineto{\pgfqpoint{14.626701in}{5.703270in}}%
\pgfpathlineto{\pgfqpoint{14.630012in}{5.701274in}}%
\pgfpathlineto{\pgfqpoint{14.633324in}{5.689988in}}%
\pgfpathlineto{\pgfqpoint{14.636635in}{5.666498in}}%
\pgfpathlineto{\pgfqpoint{14.639947in}{5.627357in}}%
\pgfpathlineto{\pgfqpoint{14.643258in}{5.568686in}}%
\pgfpathlineto{\pgfqpoint{14.649881in}{5.376664in}}%
\pgfpathlineto{\pgfqpoint{14.656504in}{5.064082in}}%
\pgfpathlineto{\pgfqpoint{14.663127in}{4.629275in}}%
\pgfpathlineto{\pgfqpoint{14.682995in}{3.147649in}}%
\pgfpathlineto{\pgfqpoint{14.689618in}{2.827571in}}%
\pgfpathlineto{\pgfqpoint{14.696241in}{2.634692in}}%
\pgfpathlineto{\pgfqpoint{14.699553in}{2.577577in}}%
\pgfpathlineto{\pgfqpoint{14.702864in}{2.540452in}}%
\pgfpathlineto{\pgfqpoint{14.706176in}{2.518773in}}%
\pgfpathlineto{\pgfqpoint{14.709487in}{2.508569in}}%
\pgfpathlineto{\pgfqpoint{14.712798in}{2.506540in}}%
\pgfpathlineto{\pgfqpoint{14.716110in}{2.510024in}}%
\pgfpathlineto{\pgfqpoint{14.722733in}{2.525451in}}%
\pgfpathlineto{\pgfqpoint{14.732667in}{2.549039in}}%
\pgfpathlineto{\pgfqpoint{14.735979in}{2.553333in}}%
\pgfpathlineto{\pgfqpoint{14.739290in}{2.554887in}}%
\pgfpathlineto{\pgfqpoint{14.742602in}{2.553373in}}%
\pgfpathlineto{\pgfqpoint{14.745913in}{2.548597in}}%
\pgfpathlineto{\pgfqpoint{14.749224in}{2.540490in}}%
\pgfpathlineto{\pgfqpoint{14.755847in}{2.514526in}}%
\pgfpathlineto{\pgfqpoint{14.762470in}{2.476778in}}%
\pgfpathlineto{\pgfqpoint{14.772405in}{2.403037in}}%
\pgfpathlineto{\pgfqpoint{14.792273in}{2.227463in}}%
\pgfpathlineto{\pgfqpoint{14.805519in}{2.118646in}}%
\pgfpathlineto{\pgfqpoint{14.815453in}{2.055253in}}%
\pgfpathlineto{\pgfqpoint{14.822076in}{2.025871in}}%
\pgfpathlineto{\pgfqpoint{14.825388in}{2.016182in}}%
\pgfpathlineto{\pgfqpoint{14.828699in}{2.010628in}}%
\pgfpathlineto{\pgfqpoint{14.832011in}{2.010015in}}%
\pgfpathlineto{\pgfqpoint{14.835322in}{2.015240in}}%
\pgfpathlineto{\pgfqpoint{14.838634in}{2.027091in}}%
\pgfpathlineto{\pgfqpoint{14.841945in}{2.045926in}}%
\pgfpathlineto{\pgfqpoint{14.848568in}{2.101644in}}%
\pgfpathlineto{\pgfqpoint{14.858502in}{2.191068in}}%
\pgfpathlineto{\pgfqpoint{14.861814in}{2.207114in}}%
\pgfpathlineto{\pgfqpoint{14.865125in}{2.210393in}}%
\pgfpathlineto{\pgfqpoint{14.868437in}{2.199088in}}%
\pgfpathlineto{\pgfqpoint{14.871748in}{2.173146in}}%
\pgfpathlineto{\pgfqpoint{14.878371in}{2.086657in}}%
\pgfpathlineto{\pgfqpoint{14.888305in}{1.942014in}}%
\pgfpathlineto{\pgfqpoint{14.891617in}{1.910135in}}%
\pgfpathlineto{\pgfqpoint{14.894928in}{1.890575in}}%
\pgfpathlineto{\pgfqpoint{14.898240in}{1.882130in}}%
\pgfpathlineto{\pgfqpoint{14.901551in}{1.881976in}}%
\pgfpathlineto{\pgfqpoint{14.904863in}{1.886957in}}%
\pgfpathlineto{\pgfqpoint{14.931354in}{1.949450in}}%
\pgfpathlineto{\pgfqpoint{14.941289in}{1.978290in}}%
\pgfpathlineto{\pgfqpoint{14.951223in}{2.017876in}}%
\pgfpathlineto{\pgfqpoint{14.961157in}{2.069017in}}%
\pgfpathlineto{\pgfqpoint{14.984337in}{2.198743in}}%
\pgfpathlineto{\pgfqpoint{14.990960in}{2.218965in}}%
\pgfpathlineto{\pgfqpoint{14.994272in}{2.223311in}}%
\pgfpathlineto{\pgfqpoint{14.997583in}{2.223545in}}%
\pgfpathlineto{\pgfqpoint{15.000895in}{2.219766in}}%
\pgfpathlineto{\pgfqpoint{15.004206in}{2.212294in}}%
\pgfpathlineto{\pgfqpoint{15.010829in}{2.188294in}}%
\pgfpathlineto{\pgfqpoint{15.020763in}{2.138455in}}%
\pgfpathlineto{\pgfqpoint{15.037321in}{2.050972in}}%
\pgfpathlineto{\pgfqpoint{15.047255in}{2.011565in}}%
\pgfpathlineto{\pgfqpoint{15.053878in}{1.994423in}}%
\pgfpathlineto{\pgfqpoint{15.060501in}{1.983486in}}%
\pgfpathlineto{\pgfqpoint{15.070435in}{1.970090in}}%
\pgfpathlineto{\pgfqpoint{15.077058in}{1.956921in}}%
\pgfpathlineto{\pgfqpoint{15.083681in}{1.938395in}}%
\pgfpathlineto{\pgfqpoint{15.103550in}{1.875813in}}%
\pgfpathlineto{\pgfqpoint{15.110172in}{1.862637in}}%
\pgfpathlineto{\pgfqpoint{15.116795in}{1.855211in}}%
\pgfpathlineto{\pgfqpoint{15.123418in}{1.852808in}}%
\pgfpathlineto{\pgfqpoint{15.130041in}{1.854277in}}%
\pgfpathlineto{\pgfqpoint{15.136664in}{1.858685in}}%
\pgfpathlineto{\pgfqpoint{15.143287in}{1.865772in}}%
\pgfpathlineto{\pgfqpoint{15.149910in}{1.876103in}}%
\pgfpathlineto{\pgfqpoint{15.156533in}{1.890843in}}%
\pgfpathlineto{\pgfqpoint{15.163156in}{1.911228in}}%
\pgfpathlineto{\pgfqpoint{15.169779in}{1.937952in}}%
\pgfpathlineto{\pgfqpoint{15.179713in}{1.989152in}}%
\pgfpathlineto{\pgfqpoint{15.202893in}{2.124312in}}%
\pgfpathlineto{\pgfqpoint{15.209516in}{2.145353in}}%
\pgfpathlineto{\pgfqpoint{15.212827in}{2.148645in}}%
\pgfpathlineto{\pgfqpoint{15.216139in}{2.146623in}}%
\pgfpathlineto{\pgfqpoint{15.219450in}{2.139561in}}%
\pgfpathlineto{\pgfqpoint{15.226073in}{2.113890in}}%
\pgfpathlineto{\pgfqpoint{15.242630in}{2.037132in}}%
\pgfpathlineto{\pgfqpoint{15.252565in}{2.004702in}}%
\pgfpathlineto{\pgfqpoint{15.262499in}{1.980535in}}%
\pgfpathlineto{\pgfqpoint{15.272434in}{1.963327in}}%
\pgfpathlineto{\pgfqpoint{15.279056in}{1.955578in}}%
\pgfpathlineto{\pgfqpoint{15.285679in}{1.950526in}}%
\pgfpathlineto{\pgfqpoint{15.292302in}{1.948378in}}%
\pgfpathlineto{\pgfqpoint{15.298925in}{1.949965in}}%
\pgfpathlineto{\pgfqpoint{15.305548in}{1.956439in}}%
\pgfpathlineto{\pgfqpoint{15.312171in}{1.968729in}}%
\pgfpathlineto{\pgfqpoint{15.318794in}{1.987029in}}%
\pgfpathlineto{\pgfqpoint{15.328728in}{2.023999in}}%
\pgfpathlineto{\pgfqpoint{15.345285in}{2.099640in}}%
\pgfpathlineto{\pgfqpoint{15.401580in}{2.367005in}}%
\pgfpathlineto{\pgfqpoint{15.411514in}{2.400707in}}%
\pgfpathlineto{\pgfqpoint{15.418137in}{2.416523in}}%
\pgfpathlineto{\pgfqpoint{15.424760in}{2.427010in}}%
\pgfpathlineto{\pgfqpoint{15.431383in}{2.432813in}}%
\pgfpathlineto{\pgfqpoint{15.438006in}{2.434607in}}%
\pgfpathlineto{\pgfqpoint{15.444629in}{2.432789in}}%
\pgfpathlineto{\pgfqpoint{15.451252in}{2.427625in}}%
\pgfpathlineto{\pgfqpoint{15.457875in}{2.419553in}}%
\pgfpathlineto{\pgfqpoint{15.467809in}{2.402797in}}%
\pgfpathlineto{\pgfqpoint{15.477743in}{2.379777in}}%
\pgfpathlineto{\pgfqpoint{15.490989in}{2.339681in}}%
\pgfpathlineto{\pgfqpoint{15.504235in}{2.300849in}}%
\pgfpathlineto{\pgfqpoint{15.510858in}{2.287247in}}%
\pgfpathlineto{\pgfqpoint{15.517481in}{2.279414in}}%
\pgfpathlineto{\pgfqpoint{15.520792in}{2.277875in}}%
\pgfpathlineto{\pgfqpoint{15.524104in}{2.277975in}}%
\pgfpathlineto{\pgfqpoint{15.527415in}{2.279726in}}%
\pgfpathlineto{\pgfqpoint{15.530727in}{2.283130in}}%
\pgfpathlineto{\pgfqpoint{15.537350in}{2.294854in}}%
\pgfpathlineto{\pgfqpoint{15.543972in}{2.312993in}}%
\pgfpathlineto{\pgfqpoint{15.550595in}{2.337267in}}%
\pgfpathlineto{\pgfqpoint{15.560530in}{2.384082in}}%
\pgfpathlineto{\pgfqpoint{15.573775in}{2.461377in}}%
\pgfpathlineto{\pgfqpoint{15.600267in}{2.623806in}}%
\pgfpathlineto{\pgfqpoint{15.610201in}{2.668949in}}%
\pgfpathlineto{\pgfqpoint{15.616824in}{2.690770in}}%
\pgfpathlineto{\pgfqpoint{15.623447in}{2.706600in}}%
\pgfpathlineto{\pgfqpoint{15.646627in}{2.750749in}}%
\pgfpathlineto{\pgfqpoint{15.656562in}{2.778876in}}%
\pgfpathlineto{\pgfqpoint{15.669807in}{2.817550in}}%
\pgfpathlineto{\pgfqpoint{15.676430in}{2.830573in}}%
\pgfpathlineto{\pgfqpoint{15.679742in}{2.834499in}}%
\pgfpathlineto{\pgfqpoint{15.683053in}{2.836585in}}%
\pgfpathlineto{\pgfqpoint{15.686365in}{2.836881in}}%
\pgfpathlineto{\pgfqpoint{15.689676in}{2.835528in}}%
\pgfpathlineto{\pgfqpoint{15.696299in}{2.828776in}}%
\pgfpathlineto{\pgfqpoint{15.706233in}{2.812843in}}%
\pgfpathlineto{\pgfqpoint{15.716168in}{2.796948in}}%
\pgfpathlineto{\pgfqpoint{15.722791in}{2.789495in}}%
\pgfpathlineto{\pgfqpoint{15.729414in}{2.785895in}}%
\pgfpathlineto{\pgfqpoint{15.732725in}{2.785723in}}%
\pgfpathlineto{\pgfqpoint{15.739348in}{2.788796in}}%
\pgfpathlineto{\pgfqpoint{15.745971in}{2.796511in}}%
\pgfpathlineto{\pgfqpoint{15.752594in}{2.808950in}}%
\pgfpathlineto{\pgfqpoint{15.759217in}{2.826182in}}%
\pgfpathlineto{\pgfqpoint{15.769151in}{2.860868in}}%
\pgfpathlineto{\pgfqpoint{15.779085in}{2.904741in}}%
\pgfpathlineto{\pgfqpoint{15.805577in}{3.029817in}}%
\pgfpathlineto{\pgfqpoint{15.815511in}{3.061978in}}%
\pgfpathlineto{\pgfqpoint{15.822134in}{3.077023in}}%
\pgfpathlineto{\pgfqpoint{15.832069in}{3.092910in}}%
\pgfpathlineto{\pgfqpoint{15.842003in}{3.104076in}}%
\pgfpathlineto{\pgfqpoint{15.851937in}{3.111447in}}%
\pgfpathlineto{\pgfqpoint{15.858560in}{3.114165in}}%
\pgfpathlineto{\pgfqpoint{15.868494in}{3.115337in}}%
\pgfpathlineto{\pgfqpoint{15.875117in}{3.114307in}}%
\pgfpathlineto{\pgfqpoint{15.881740in}{3.111382in}}%
\pgfpathlineto{\pgfqpoint{15.888363in}{3.105640in}}%
\pgfpathlineto{\pgfqpoint{15.894986in}{3.095959in}}%
\pgfpathlineto{\pgfqpoint{15.901609in}{3.081430in}}%
\pgfpathlineto{\pgfqpoint{15.908232in}{3.061761in}}%
\pgfpathlineto{\pgfqpoint{15.918166in}{3.024045in}}%
\pgfpathlineto{\pgfqpoint{15.941346in}{2.928293in}}%
\pgfpathlineto{\pgfqpoint{15.947969in}{2.907786in}}%
\pgfpathlineto{\pgfqpoint{15.954592in}{2.892862in}}%
\pgfpathlineto{\pgfqpoint{15.961215in}{2.884019in}}%
\pgfpathlineto{\pgfqpoint{15.964527in}{2.881888in}}%
\pgfpathlineto{\pgfqpoint{15.967838in}{2.881229in}}%
\pgfpathlineto{\pgfqpoint{15.971149in}{2.881970in}}%
\pgfpathlineto{\pgfqpoint{15.977772in}{2.887249in}}%
\pgfpathlineto{\pgfqpoint{15.984395in}{2.896747in}}%
\pgfpathlineto{\pgfqpoint{15.994330in}{2.916298in}}%
\pgfpathlineto{\pgfqpoint{16.024133in}{2.980675in}}%
\pgfpathlineto{\pgfqpoint{16.034067in}{2.997519in}}%
\pgfpathlineto{\pgfqpoint{16.040690in}{3.005962in}}%
\pgfpathlineto{\pgfqpoint{16.047313in}{3.011443in}}%
\pgfpathlineto{\pgfqpoint{16.053936in}{3.013745in}}%
\pgfpathlineto{\pgfqpoint{16.060559in}{3.013294in}}%
\pgfpathlineto{\pgfqpoint{16.073804in}{3.008312in}}%
\pgfpathlineto{\pgfqpoint{16.087050in}{3.004290in}}%
\pgfpathlineto{\pgfqpoint{16.103607in}{3.000682in}}%
\pgfpathlineto{\pgfqpoint{16.110230in}{2.997508in}}%
\pgfpathlineto{\pgfqpoint{16.116853in}{2.992347in}}%
\pgfpathlineto{\pgfqpoint{16.123476in}{2.984765in}}%
\pgfpathlineto{\pgfqpoint{16.130099in}{2.974509in}}%
\pgfpathlineto{\pgfqpoint{16.140033in}{2.954009in}}%
\pgfpathlineto{\pgfqpoint{16.149968in}{2.928006in}}%
\pgfpathlineto{\pgfqpoint{16.169836in}{2.867279in}}%
\pgfpathlineto{\pgfqpoint{16.186394in}{2.819375in}}%
\pgfpathlineto{\pgfqpoint{16.196328in}{2.795723in}}%
\pgfpathlineto{\pgfqpoint{16.206262in}{2.777623in}}%
\pgfpathlineto{\pgfqpoint{16.212885in}{2.769124in}}%
\pgfpathlineto{\pgfqpoint{16.219508in}{2.763458in}}%
\pgfpathlineto{\pgfqpoint{16.226131in}{2.760260in}}%
\pgfpathlineto{\pgfqpoint{16.232754in}{2.758992in}}%
\pgfpathlineto{\pgfqpoint{16.242688in}{2.759637in}}%
\pgfpathlineto{\pgfqpoint{16.252623in}{2.762416in}}%
\pgfpathlineto{\pgfqpoint{16.265868in}{2.768424in}}%
\pgfpathlineto{\pgfqpoint{16.285737in}{2.777897in}}%
\pgfpathlineto{\pgfqpoint{16.292360in}{2.778427in}}%
\pgfpathlineto{\pgfqpoint{16.298983in}{2.775544in}}%
\pgfpathlineto{\pgfqpoint{16.305606in}{2.767942in}}%
\pgfpathlineto{\pgfqpoint{16.312229in}{2.754931in}}%
\pgfpathlineto{\pgfqpoint{16.318852in}{2.736717in}}%
\pgfpathlineto{\pgfqpoint{16.328786in}{2.701906in}}%
\pgfpathlineto{\pgfqpoint{16.368523in}{2.549916in}}%
\pgfpathlineto{\pgfqpoint{16.378458in}{2.522675in}}%
\pgfpathlineto{\pgfqpoint{16.385081in}{2.509665in}}%
\pgfpathlineto{\pgfqpoint{16.391704in}{2.501683in}}%
\pgfpathlineto{\pgfqpoint{16.395015in}{2.499693in}}%
\pgfpathlineto{\pgfqpoint{16.398326in}{2.499030in}}%
\pgfpathlineto{\pgfqpoint{16.401638in}{2.499661in}}%
\pgfpathlineto{\pgfqpoint{16.408261in}{2.504620in}}%
\pgfpathlineto{\pgfqpoint{16.414884in}{2.514266in}}%
\pgfpathlineto{\pgfqpoint{16.421507in}{2.528441in}}%
\pgfpathlineto{\pgfqpoint{16.428130in}{2.547029in}}%
\pgfpathlineto{\pgfqpoint{16.438064in}{2.582583in}}%
\pgfpathlineto{\pgfqpoint{16.451310in}{2.641236in}}%
\pgfpathlineto{\pgfqpoint{16.484424in}{2.797313in}}%
\pgfpathlineto{\pgfqpoint{16.494359in}{2.830677in}}%
\pgfpathlineto{\pgfqpoint{16.500981in}{2.845329in}}%
\pgfpathlineto{\pgfqpoint{16.504293in}{2.849904in}}%
\pgfpathlineto{\pgfqpoint{16.507604in}{2.852477in}}%
\pgfpathlineto{\pgfqpoint{16.510916in}{2.852960in}}%
\pgfpathlineto{\pgfqpoint{16.514227in}{2.851306in}}%
\pgfpathlineto{\pgfqpoint{16.517539in}{2.847508in}}%
\pgfpathlineto{\pgfqpoint{16.524162in}{2.833666in}}%
\pgfpathlineto{\pgfqpoint{16.530784in}{2.812200in}}%
\pgfpathlineto{\pgfqpoint{16.540719in}{2.768501in}}%
\pgfpathlineto{\pgfqpoint{16.553965in}{2.696871in}}%
\pgfpathlineto{\pgfqpoint{16.577145in}{2.568012in}}%
\pgfpathlineto{\pgfqpoint{16.590391in}{2.507618in}}%
\pgfpathlineto{\pgfqpoint{16.600325in}{2.472010in}}%
\pgfpathlineto{\pgfqpoint{16.603636in}{2.462117in}}%
\pgfpathlineto{\pgfqpoint{16.603636in}{2.462117in}}%
\pgfusepath{stroke}%
\end{pgfscope}%
\begin{pgfscope}%
\pgfpathrectangle{\pgfqpoint{2.400000in}{1.081300in}}{\pgfqpoint{14.880000in}{7.569100in}}%
\pgfusepath{clip}%
\pgfsetrectcap%
\pgfsetroundjoin%
\pgfsetlinewidth{1.505625pt}%
\definecolor{currentstroke}{rgb}{0.549020,0.337255,0.294118}%
\pgfsetstrokecolor{currentstroke}%
\pgfsetdash{}{0pt}%
\pgfpathmoveto{\pgfqpoint{3.076364in}{1.425350in}}%
\pgfpathlineto{\pgfqpoint{3.132658in}{1.434243in}}%
\pgfpathlineto{\pgfqpoint{3.159150in}{1.436270in}}%
\pgfpathlineto{\pgfqpoint{3.182330in}{1.435953in}}%
\pgfpathlineto{\pgfqpoint{3.212133in}{1.433049in}}%
\pgfpathlineto{\pgfqpoint{3.241936in}{1.430443in}}%
\pgfpathlineto{\pgfqpoint{3.261805in}{1.431414in}}%
\pgfpathlineto{\pgfqpoint{3.328034in}{1.437336in}}%
\pgfpathlineto{\pgfqpoint{3.351214in}{1.436519in}}%
\pgfpathlineto{\pgfqpoint{3.420754in}{1.431970in}}%
\pgfpathlineto{\pgfqpoint{3.536655in}{1.440496in}}%
\pgfpathlineto{\pgfqpoint{3.566458in}{1.446978in}}%
\pgfpathlineto{\pgfqpoint{3.635999in}{1.463227in}}%
\pgfpathlineto{\pgfqpoint{3.662490in}{1.466793in}}%
\pgfpathlineto{\pgfqpoint{3.688982in}{1.467964in}}%
\pgfpathlineto{\pgfqpoint{3.718785in}{1.466748in}}%
\pgfpathlineto{\pgfqpoint{3.761834in}{1.462335in}}%
\pgfpathlineto{\pgfqpoint{3.831374in}{1.455202in}}%
\pgfpathlineto{\pgfqpoint{3.867800in}{1.453935in}}%
\pgfpathlineto{\pgfqpoint{3.914160in}{1.454931in}}%
\pgfpathlineto{\pgfqpoint{4.003570in}{1.457276in}}%
\pgfpathlineto{\pgfqpoint{4.073110in}{1.456407in}}%
\pgfpathlineto{\pgfqpoint{4.179076in}{1.452962in}}%
\pgfpathlineto{\pgfqpoint{4.222125in}{1.449282in}}%
\pgfpathlineto{\pgfqpoint{4.314846in}{1.440003in}}%
\pgfpathlineto{\pgfqpoint{4.344649in}{1.440571in}}%
\pgfpathlineto{\pgfqpoint{4.427435in}{1.443855in}}%
\pgfpathlineto{\pgfqpoint{4.536713in}{1.444572in}}%
\pgfpathlineto{\pgfqpoint{4.576450in}{1.448896in}}%
\pgfpathlineto{\pgfqpoint{4.649302in}{1.457550in}}%
\pgfpathlineto{\pgfqpoint{4.679105in}{1.458693in}}%
\pgfpathlineto{\pgfqpoint{4.708908in}{1.457655in}}%
\pgfpathlineto{\pgfqpoint{4.751957in}{1.453484in}}%
\pgfpathlineto{\pgfqpoint{4.808252in}{1.448276in}}%
\pgfpathlineto{\pgfqpoint{4.844678in}{1.447315in}}%
\pgfpathlineto{\pgfqpoint{4.887727in}{1.448641in}}%
\pgfpathlineto{\pgfqpoint{4.940710in}{1.452623in}}%
\pgfpathlineto{\pgfqpoint{5.023496in}{1.459133in}}%
\pgfpathlineto{\pgfqpoint{5.059922in}{1.459261in}}%
\pgfpathlineto{\pgfqpoint{5.099660in}{1.456857in}}%
\pgfpathlineto{\pgfqpoint{5.222183in}{1.447364in}}%
\pgfpathlineto{\pgfqpoint{5.271855in}{1.446984in}}%
\pgfpathlineto{\pgfqpoint{5.324838in}{1.448854in}}%
\pgfpathlineto{\pgfqpoint{5.530148in}{1.458475in}}%
\pgfpathlineto{\pgfqpoint{5.569885in}{1.458331in}}%
\pgfpathlineto{\pgfqpoint{5.609623in}{1.455846in}}%
\pgfpathlineto{\pgfqpoint{5.655983in}{1.450492in}}%
\pgfpathlineto{\pgfqpoint{5.748704in}{1.438563in}}%
\pgfpathlineto{\pgfqpoint{5.771884in}{1.438719in}}%
\pgfpathlineto{\pgfqpoint{5.801687in}{1.441544in}}%
\pgfpathlineto{\pgfqpoint{5.867916in}{1.448864in}}%
\pgfpathlineto{\pgfqpoint{5.897719in}{1.449567in}}%
\pgfpathlineto{\pgfqpoint{5.927522in}{1.448099in}}%
\pgfpathlineto{\pgfqpoint{6.023554in}{1.441148in}}%
\pgfpathlineto{\pgfqpoint{6.056669in}{1.442684in}}%
\pgfpathlineto{\pgfqpoint{6.165946in}{1.450039in}}%
\pgfpathlineto{\pgfqpoint{6.295093in}{1.453179in}}%
\pgfpathlineto{\pgfqpoint{6.354699in}{1.455757in}}%
\pgfpathlineto{\pgfqpoint{6.404371in}{1.455371in}}%
\pgfpathlineto{\pgfqpoint{6.543452in}{1.452732in}}%
\pgfpathlineto{\pgfqpoint{6.583189in}{1.456175in}}%
\pgfpathlineto{\pgfqpoint{6.626238in}{1.459512in}}%
\pgfpathlineto{\pgfqpoint{6.656041in}{1.459686in}}%
\pgfpathlineto{\pgfqpoint{6.692467in}{1.457537in}}%
\pgfpathlineto{\pgfqpoint{6.762007in}{1.452762in}}%
\pgfpathlineto{\pgfqpoint{6.795122in}{1.453252in}}%
\pgfpathlineto{\pgfqpoint{6.828236in}{1.456221in}}%
\pgfpathlineto{\pgfqpoint{6.930891in}{1.468021in}}%
\pgfpathlineto{\pgfqpoint{6.957383in}{1.467280in}}%
\pgfpathlineto{\pgfqpoint{6.987186in}{1.464011in}}%
\pgfpathlineto{\pgfqpoint{7.033546in}{1.456109in}}%
\pgfpathlineto{\pgfqpoint{7.073284in}{1.450111in}}%
\pgfpathlineto{\pgfqpoint{7.103087in}{1.447767in}}%
\pgfpathlineto{\pgfqpoint{7.132890in}{1.447739in}}%
\pgfpathlineto{\pgfqpoint{7.172627in}{1.450385in}}%
\pgfpathlineto{\pgfqpoint{7.212365in}{1.452611in}}%
\pgfpathlineto{\pgfqpoint{7.238856in}{1.451790in}}%
\pgfpathlineto{\pgfqpoint{7.265348in}{1.448598in}}%
\pgfpathlineto{\pgfqpoint{7.328265in}{1.439220in}}%
\pgfpathlineto{\pgfqpoint{7.348134in}{1.439724in}}%
\pgfpathlineto{\pgfqpoint{7.411052in}{1.443647in}}%
\pgfpathlineto{\pgfqpoint{7.437543in}{1.441951in}}%
\pgfpathlineto{\pgfqpoint{7.473969in}{1.436883in}}%
\pgfpathlineto{\pgfqpoint{7.507084in}{1.432674in}}%
\pgfpathlineto{\pgfqpoint{7.526952in}{1.432247in}}%
\pgfpathlineto{\pgfqpoint{7.556755in}{1.434326in}}%
\pgfpathlineto{\pgfqpoint{7.593181in}{1.438952in}}%
\pgfpathlineto{\pgfqpoint{7.629607in}{1.445867in}}%
\pgfpathlineto{\pgfqpoint{7.675968in}{1.454575in}}%
\pgfpathlineto{\pgfqpoint{7.699148in}{1.456593in}}%
\pgfpathlineto{\pgfqpoint{7.719016in}{1.456303in}}%
\pgfpathlineto{\pgfqpoint{7.742197in}{1.453565in}}%
\pgfpathlineto{\pgfqpoint{7.772000in}{1.447310in}}%
\pgfpathlineto{\pgfqpoint{7.811737in}{1.438915in}}%
\pgfpathlineto{\pgfqpoint{7.831606in}{1.436967in}}%
\pgfpathlineto{\pgfqpoint{7.904458in}{1.432642in}}%
\pgfpathlineto{\pgfqpoint{7.930949in}{1.429679in}}%
\pgfpathlineto{\pgfqpoint{7.944195in}{1.431095in}}%
\pgfpathlineto{\pgfqpoint{7.977309in}{1.438680in}}%
\pgfpathlineto{\pgfqpoint{8.003801in}{1.443443in}}%
\pgfpathlineto{\pgfqpoint{8.023670in}{1.444942in}}%
\pgfpathlineto{\pgfqpoint{8.043538in}{1.444346in}}%
\pgfpathlineto{\pgfqpoint{8.066719in}{1.441352in}}%
\pgfpathlineto{\pgfqpoint{8.106456in}{1.435629in}}%
\pgfpathlineto{\pgfqpoint{8.123013in}{1.436050in}}%
\pgfpathlineto{\pgfqpoint{8.156128in}{1.440321in}}%
\pgfpathlineto{\pgfqpoint{8.179308in}{1.442203in}}%
\pgfpathlineto{\pgfqpoint{8.202488in}{1.441772in}}%
\pgfpathlineto{\pgfqpoint{8.232291in}{1.438390in}}%
\pgfpathlineto{\pgfqpoint{8.255471in}{1.436476in}}%
\pgfpathlineto{\pgfqpoint{8.268717in}{1.437393in}}%
\pgfpathlineto{\pgfqpoint{8.285274in}{1.440676in}}%
\pgfpathlineto{\pgfqpoint{8.348192in}{1.455721in}}%
\pgfpathlineto{\pgfqpoint{8.371372in}{1.458049in}}%
\pgfpathlineto{\pgfqpoint{8.394552in}{1.458175in}}%
\pgfpathlineto{\pgfqpoint{8.427667in}{1.455763in}}%
\pgfpathlineto{\pgfqpoint{8.487273in}{1.450966in}}%
\pgfpathlineto{\pgfqpoint{8.513764in}{1.451263in}}%
\pgfpathlineto{\pgfqpoint{8.536944in}{1.453915in}}%
\pgfpathlineto{\pgfqpoint{8.560125in}{1.458916in}}%
\pgfpathlineto{\pgfqpoint{8.603173in}{1.471287in}}%
\pgfpathlineto{\pgfqpoint{8.632977in}{1.478736in}}%
\pgfpathlineto{\pgfqpoint{8.656157in}{1.482435in}}%
\pgfpathlineto{\pgfqpoint{8.679337in}{1.483899in}}%
\pgfpathlineto{\pgfqpoint{8.705828in}{1.483024in}}%
\pgfpathlineto{\pgfqpoint{8.738943in}{1.479338in}}%
\pgfpathlineto{\pgfqpoint{8.801860in}{1.471900in}}%
\pgfpathlineto{\pgfqpoint{8.831664in}{1.470985in}}%
\pgfpathlineto{\pgfqpoint{8.864778in}{1.472417in}}%
\pgfpathlineto{\pgfqpoint{8.921073in}{1.475523in}}%
\pgfpathlineto{\pgfqpoint{8.947564in}{1.474402in}}%
\pgfpathlineto{\pgfqpoint{8.970744in}{1.471213in}}%
\pgfpathlineto{\pgfqpoint{8.993925in}{1.465916in}}%
\pgfpathlineto{\pgfqpoint{9.023728in}{1.456586in}}%
\pgfpathlineto{\pgfqpoint{9.073399in}{1.440179in}}%
\pgfpathlineto{\pgfqpoint{9.089957in}{1.437465in}}%
\pgfpathlineto{\pgfqpoint{9.103202in}{1.437460in}}%
\pgfpathlineto{\pgfqpoint{9.123071in}{1.440050in}}%
\pgfpathlineto{\pgfqpoint{9.172743in}{1.447550in}}%
\pgfpathlineto{\pgfqpoint{9.205857in}{1.449776in}}%
\pgfpathlineto{\pgfqpoint{9.252218in}{1.450216in}}%
\pgfpathlineto{\pgfqpoint{9.291955in}{1.451433in}}%
\pgfpathlineto{\pgfqpoint{9.318447in}{1.454514in}}%
\pgfpathlineto{\pgfqpoint{9.351561in}{1.460915in}}%
\pgfpathlineto{\pgfqpoint{9.407856in}{1.472351in}}%
\pgfpathlineto{\pgfqpoint{9.434347in}{1.475453in}}%
\pgfpathlineto{\pgfqpoint{9.460839in}{1.476370in}}%
\pgfpathlineto{\pgfqpoint{9.490642in}{1.475038in}}%
\pgfpathlineto{\pgfqpoint{9.537002in}{1.470246in}}%
\pgfpathlineto{\pgfqpoint{9.583363in}{1.466126in}}%
\pgfpathlineto{\pgfqpoint{9.616477in}{1.465526in}}%
\pgfpathlineto{\pgfqpoint{9.649592in}{1.467407in}}%
\pgfpathlineto{\pgfqpoint{9.695952in}{1.472836in}}%
\pgfpathlineto{\pgfqpoint{9.735689in}{1.476937in}}%
\pgfpathlineto{\pgfqpoint{9.762181in}{1.477509in}}%
\pgfpathlineto{\pgfqpoint{9.785361in}{1.475803in}}%
\pgfpathlineto{\pgfqpoint{9.808541in}{1.471817in}}%
\pgfpathlineto{\pgfqpoint{9.835033in}{1.464752in}}%
\pgfpathlineto{\pgfqpoint{9.871459in}{1.452315in}}%
\pgfpathlineto{\pgfqpoint{9.921131in}{1.435437in}}%
\pgfpathlineto{\pgfqpoint{9.950934in}{1.427728in}}%
\pgfpathlineto{\pgfqpoint{9.960868in}{1.426532in}}%
\pgfpathlineto{\pgfqpoint{9.977425in}{1.428732in}}%
\pgfpathlineto{\pgfqpoint{10.010540in}{1.432870in}}%
\pgfpathlineto{\pgfqpoint{10.050277in}{1.435373in}}%
\pgfpathlineto{\pgfqpoint{10.126440in}{1.439434in}}%
\pgfpathlineto{\pgfqpoint{10.176112in}{1.444907in}}%
\pgfpathlineto{\pgfqpoint{10.239030in}{1.454238in}}%
\pgfpathlineto{\pgfqpoint{10.295324in}{1.462159in}}%
\pgfpathlineto{\pgfqpoint{10.328439in}{1.464557in}}%
\pgfpathlineto{\pgfqpoint{10.358242in}{1.464391in}}%
\pgfpathlineto{\pgfqpoint{10.388045in}{1.461827in}}%
\pgfpathlineto{\pgfqpoint{10.431094in}{1.455343in}}%
\pgfpathlineto{\pgfqpoint{10.460897in}{1.451724in}}%
\pgfpathlineto{\pgfqpoint{10.484077in}{1.451194in}}%
\pgfpathlineto{\pgfqpoint{10.510569in}{1.453190in}}%
\pgfpathlineto{\pgfqpoint{10.623158in}{1.465330in}}%
\pgfpathlineto{\pgfqpoint{10.682764in}{1.467328in}}%
\pgfpathlineto{\pgfqpoint{10.755616in}{1.467750in}}%
\pgfpathlineto{\pgfqpoint{10.792042in}{1.465911in}}%
\pgfpathlineto{\pgfqpoint{10.825156in}{1.461987in}}%
\pgfpathlineto{\pgfqpoint{10.861582in}{1.455196in}}%
\pgfpathlineto{\pgfqpoint{10.907943in}{1.446544in}}%
\pgfpathlineto{\pgfqpoint{10.927811in}{1.445021in}}%
\pgfpathlineto{\pgfqpoint{10.950991in}{1.445747in}}%
\pgfpathlineto{\pgfqpoint{10.984106in}{1.449584in}}%
\pgfpathlineto{\pgfqpoint{11.037089in}{1.455651in}}%
\pgfpathlineto{\pgfqpoint{11.083449in}{1.458455in}}%
\pgfpathlineto{\pgfqpoint{11.146367in}{1.462156in}}%
\pgfpathlineto{\pgfqpoint{11.176170in}{1.466375in}}%
\pgfpathlineto{\pgfqpoint{11.212596in}{1.474134in}}%
\pgfpathlineto{\pgfqpoint{11.262268in}{1.484745in}}%
\pgfpathlineto{\pgfqpoint{11.288759in}{1.488037in}}%
\pgfpathlineto{\pgfqpoint{11.311939in}{1.488846in}}%
\pgfpathlineto{\pgfqpoint{11.338431in}{1.487359in}}%
\pgfpathlineto{\pgfqpoint{11.371546in}{1.482836in}}%
\pgfpathlineto{\pgfqpoint{11.421217in}{1.475843in}}%
\pgfpathlineto{\pgfqpoint{11.444397in}{1.474750in}}%
\pgfpathlineto{\pgfqpoint{11.467578in}{1.475759in}}%
\pgfpathlineto{\pgfqpoint{11.494069in}{1.479223in}}%
\pgfpathlineto{\pgfqpoint{11.563610in}{1.490014in}}%
\pgfpathlineto{\pgfqpoint{11.583478in}{1.490347in}}%
\pgfpathlineto{\pgfqpoint{11.603347in}{1.488531in}}%
\pgfpathlineto{\pgfqpoint{11.623216in}{1.484505in}}%
\pgfpathlineto{\pgfqpoint{11.646396in}{1.477337in}}%
\pgfpathlineto{\pgfqpoint{11.679510in}{1.464150in}}%
\pgfpathlineto{\pgfqpoint{11.715936in}{1.449997in}}%
\pgfpathlineto{\pgfqpoint{11.735805in}{1.444792in}}%
\pgfpathlineto{\pgfqpoint{11.752362in}{1.442618in}}%
\pgfpathlineto{\pgfqpoint{11.772231in}{1.442225in}}%
\pgfpathlineto{\pgfqpoint{11.828526in}{1.444712in}}%
\pgfpathlineto{\pgfqpoint{11.990787in}{1.451058in}}%
\pgfpathlineto{\pgfqpoint{12.047081in}{1.450015in}}%
\pgfpathlineto{\pgfqpoint{12.080196in}{1.447176in}}%
\pgfpathlineto{\pgfqpoint{12.123245in}{1.440625in}}%
\pgfpathlineto{\pgfqpoint{12.146425in}{1.438099in}}%
\pgfpathlineto{\pgfqpoint{12.166294in}{1.438185in}}%
\pgfpathlineto{\pgfqpoint{12.249080in}{1.442205in}}%
\pgfpathlineto{\pgfqpoint{12.292129in}{1.442149in}}%
\pgfpathlineto{\pgfqpoint{12.318620in}{1.444854in}}%
\pgfpathlineto{\pgfqpoint{12.368292in}{1.450586in}}%
\pgfpathlineto{\pgfqpoint{12.391472in}{1.450671in}}%
\pgfpathlineto{\pgfqpoint{12.414652in}{1.448452in}}%
\pgfpathlineto{\pgfqpoint{12.444455in}{1.442953in}}%
\pgfpathlineto{\pgfqpoint{12.484193in}{1.435347in}}%
\pgfpathlineto{\pgfqpoint{12.504061in}{1.434100in}}%
\pgfpathlineto{\pgfqpoint{12.537176in}{1.435200in}}%
\pgfpathlineto{\pgfqpoint{12.596782in}{1.438233in}}%
\pgfpathlineto{\pgfqpoint{12.619962in}{1.442876in}}%
\pgfpathlineto{\pgfqpoint{12.696126in}{1.460662in}}%
\pgfpathlineto{\pgfqpoint{12.719306in}{1.462554in}}%
\pgfpathlineto{\pgfqpoint{12.742486in}{1.461933in}}%
\pgfpathlineto{\pgfqpoint{12.768977in}{1.458574in}}%
\pgfpathlineto{\pgfqpoint{12.848452in}{1.446144in}}%
\pgfpathlineto{\pgfqpoint{12.871632in}{1.446024in}}%
\pgfpathlineto{\pgfqpoint{12.894813in}{1.448118in}}%
\pgfpathlineto{\pgfqpoint{12.924616in}{1.453172in}}%
\pgfpathlineto{\pgfqpoint{12.987533in}{1.464487in}}%
\pgfpathlineto{\pgfqpoint{13.014025in}{1.466707in}}%
\pgfpathlineto{\pgfqpoint{13.043828in}{1.466784in}}%
\pgfpathlineto{\pgfqpoint{13.080254in}{1.464357in}}%
\pgfpathlineto{\pgfqpoint{13.202777in}{1.454333in}}%
\pgfpathlineto{\pgfqpoint{13.242515in}{1.453596in}}%
\pgfpathlineto{\pgfqpoint{13.295498in}{1.455475in}}%
\pgfpathlineto{\pgfqpoint{13.335235in}{1.455809in}}%
\pgfpathlineto{\pgfqpoint{13.371661in}{1.453823in}}%
\pgfpathlineto{\pgfqpoint{13.494185in}{1.444553in}}%
\pgfpathlineto{\pgfqpoint{13.530611in}{1.445394in}}%
\pgfpathlineto{\pgfqpoint{13.603463in}{1.447978in}}%
\pgfpathlineto{\pgfqpoint{13.633266in}{1.446443in}}%
\pgfpathlineto{\pgfqpoint{13.669692in}{1.442217in}}%
\pgfpathlineto{\pgfqpoint{13.729298in}{1.434814in}}%
\pgfpathlineto{\pgfqpoint{13.759101in}{1.433895in}}%
\pgfpathlineto{\pgfqpoint{13.825330in}{1.435290in}}%
\pgfpathlineto{\pgfqpoint{13.861756in}{1.437498in}}%
\pgfpathlineto{\pgfqpoint{13.894870in}{1.441826in}}%
\pgfpathlineto{\pgfqpoint{14.007460in}{1.459215in}}%
\pgfpathlineto{\pgfqpoint{14.043886in}{1.461135in}}%
\pgfpathlineto{\pgfqpoint{14.116738in}{1.461951in}}%
\pgfpathlineto{\pgfqpoint{14.199524in}{1.464065in}}%
\pgfpathlineto{\pgfqpoint{14.245884in}{1.467384in}}%
\pgfpathlineto{\pgfqpoint{14.318736in}{1.473271in}}%
\pgfpathlineto{\pgfqpoint{14.345228in}{1.472790in}}%
\pgfpathlineto{\pgfqpoint{14.371719in}{1.470001in}}%
\pgfpathlineto{\pgfqpoint{14.401522in}{1.464428in}}%
\pgfpathlineto{\pgfqpoint{14.461128in}{1.452268in}}%
\pgfpathlineto{\pgfqpoint{14.484308in}{1.450310in}}%
\pgfpathlineto{\pgfqpoint{14.507489in}{1.450558in}}%
\pgfpathlineto{\pgfqpoint{14.543915in}{1.453648in}}%
\pgfpathlineto{\pgfqpoint{14.606832in}{1.459195in}}%
\pgfpathlineto{\pgfqpoint{14.639947in}{1.459748in}}%
\pgfpathlineto{\pgfqpoint{14.676373in}{1.457898in}}%
\pgfpathlineto{\pgfqpoint{14.769093in}{1.451486in}}%
\pgfpathlineto{\pgfqpoint{14.828699in}{1.451444in}}%
\pgfpathlineto{\pgfqpoint{14.871748in}{1.453037in}}%
\pgfpathlineto{\pgfqpoint{14.918108in}{1.457348in}}%
\pgfpathlineto{\pgfqpoint{14.957846in}{1.460484in}}%
\pgfpathlineto{\pgfqpoint{14.987649in}{1.460545in}}%
\pgfpathlineto{\pgfqpoint{15.017452in}{1.458340in}}%
\pgfpathlineto{\pgfqpoint{15.067124in}{1.451893in}}%
\pgfpathlineto{\pgfqpoint{15.110172in}{1.447097in}}%
\pgfpathlineto{\pgfqpoint{15.146598in}{1.445224in}}%
\pgfpathlineto{\pgfqpoint{15.282368in}{1.441108in}}%
\pgfpathlineto{\pgfqpoint{15.345285in}{1.436939in}}%
\pgfpathlineto{\pgfqpoint{15.388334in}{1.436860in}}%
\pgfpathlineto{\pgfqpoint{15.527415in}{1.438246in}}%
\pgfpathlineto{\pgfqpoint{15.563841in}{1.435308in}}%
\pgfpathlineto{\pgfqpoint{15.593644in}{1.433359in}}%
\pgfpathlineto{\pgfqpoint{15.610201in}{1.434544in}}%
\pgfpathlineto{\pgfqpoint{15.633382in}{1.438908in}}%
\pgfpathlineto{\pgfqpoint{15.706233in}{1.454625in}}%
\pgfpathlineto{\pgfqpoint{15.739348in}{1.458684in}}%
\pgfpathlineto{\pgfqpoint{15.779085in}{1.460989in}}%
\pgfpathlineto{\pgfqpoint{15.868494in}{1.465169in}}%
\pgfpathlineto{\pgfqpoint{15.908232in}{1.470005in}}%
\pgfpathlineto{\pgfqpoint{15.981084in}{1.479815in}}%
\pgfpathlineto{\pgfqpoint{16.004264in}{1.480403in}}%
\pgfpathlineto{\pgfqpoint{16.027444in}{1.478852in}}%
\pgfpathlineto{\pgfqpoint{16.050624in}{1.475175in}}%
\pgfpathlineto{\pgfqpoint{16.080427in}{1.468052in}}%
\pgfpathlineto{\pgfqpoint{16.140033in}{1.453126in}}%
\pgfpathlineto{\pgfqpoint{16.166525in}{1.449422in}}%
\pgfpathlineto{\pgfqpoint{16.193017in}{1.448098in}}%
\pgfpathlineto{\pgfqpoint{16.222820in}{1.449014in}}%
\pgfpathlineto{\pgfqpoint{16.262557in}{1.452808in}}%
\pgfpathlineto{\pgfqpoint{16.302294in}{1.456124in}}%
\pgfpathlineto{\pgfqpoint{16.325475in}{1.456029in}}%
\pgfpathlineto{\pgfqpoint{16.348655in}{1.453673in}}%
\pgfpathlineto{\pgfqpoint{16.371835in}{1.448957in}}%
\pgfpathlineto{\pgfqpoint{16.398326in}{1.441213in}}%
\pgfpathlineto{\pgfqpoint{16.431441in}{1.431239in}}%
\pgfpathlineto{\pgfqpoint{16.441375in}{1.430628in}}%
\pgfpathlineto{\pgfqpoint{16.454621in}{1.433052in}}%
\pgfpathlineto{\pgfqpoint{16.500981in}{1.444122in}}%
\pgfpathlineto{\pgfqpoint{16.524162in}{1.446976in}}%
\pgfpathlineto{\pgfqpoint{16.524162in}{1.446976in}}%
\pgfusepath{stroke}%
\end{pgfscope}%
\begin{pgfscope}%
\pgfpathrectangle{\pgfqpoint{2.400000in}{1.081300in}}{\pgfqpoint{14.880000in}{7.569100in}}%
\pgfusepath{clip}%
\pgfsetrectcap%
\pgfsetroundjoin%
\pgfsetlinewidth{1.505625pt}%
\definecolor{currentstroke}{rgb}{0.890196,0.466667,0.760784}%
\pgfsetstrokecolor{currentstroke}%
\pgfsetdash{}{0pt}%
\pgfpathmoveto{\pgfqpoint{3.076364in}{1.425350in}}%
\pgfpathlineto{\pgfqpoint{3.126035in}{1.764713in}}%
\pgfpathlineto{\pgfqpoint{3.142593in}{1.865290in}}%
\pgfpathlineto{\pgfqpoint{3.155838in}{1.934238in}}%
\pgfpathlineto{\pgfqpoint{3.165773in}{1.976091in}}%
\pgfpathlineto{\pgfqpoint{3.175707in}{2.007190in}}%
\pgfpathlineto{\pgfqpoint{3.182330in}{2.021114in}}%
\pgfpathlineto{\pgfqpoint{3.188953in}{2.029232in}}%
\pgfpathlineto{\pgfqpoint{3.192264in}{2.031077in}}%
\pgfpathlineto{\pgfqpoint{3.195576in}{2.031455in}}%
\pgfpathlineto{\pgfqpoint{3.198887in}{2.030384in}}%
\pgfpathlineto{\pgfqpoint{3.205510in}{2.024037in}}%
\pgfpathlineto{\pgfqpoint{3.212133in}{2.012406in}}%
\pgfpathlineto{\pgfqpoint{3.218756in}{1.995973in}}%
\pgfpathlineto{\pgfqpoint{3.228690in}{1.963534in}}%
\pgfpathlineto{\pgfqpoint{3.241936in}{1.908782in}}%
\pgfpathlineto{\pgfqpoint{3.258493in}{1.828307in}}%
\pgfpathlineto{\pgfqpoint{3.288296in}{1.680630in}}%
\pgfpathlineto{\pgfqpoint{3.298231in}{1.643921in}}%
\pgfpathlineto{\pgfqpoint{3.304854in}{1.628017in}}%
\pgfpathlineto{\pgfqpoint{3.308165in}{1.623373in}}%
\pgfpathlineto{\pgfqpoint{3.311477in}{1.621145in}}%
\pgfpathlineto{\pgfqpoint{3.314788in}{1.621369in}}%
\pgfpathlineto{\pgfqpoint{3.318099in}{1.623968in}}%
\pgfpathlineto{\pgfqpoint{3.324722in}{1.635475in}}%
\pgfpathlineto{\pgfqpoint{3.331345in}{1.653414in}}%
\pgfpathlineto{\pgfqpoint{3.344591in}{1.698412in}}%
\pgfpathlineto{\pgfqpoint{3.357837in}{1.741856in}}%
\pgfpathlineto{\pgfqpoint{3.367771in}{1.766421in}}%
\pgfpathlineto{\pgfqpoint{3.374394in}{1.777175in}}%
\pgfpathlineto{\pgfqpoint{3.381017in}{1.782741in}}%
\pgfpathlineto{\pgfqpoint{3.384328in}{1.783477in}}%
\pgfpathlineto{\pgfqpoint{3.387640in}{1.782826in}}%
\pgfpathlineto{\pgfqpoint{3.390951in}{1.780795in}}%
\pgfpathlineto{\pgfqpoint{3.397574in}{1.772693in}}%
\pgfpathlineto{\pgfqpoint{3.404197in}{1.759548in}}%
\pgfpathlineto{\pgfqpoint{3.414132in}{1.731916in}}%
\pgfpathlineto{\pgfqpoint{3.440623in}{1.648409in}}%
\pgfpathlineto{\pgfqpoint{3.443935in}{1.643143in}}%
\pgfpathlineto{\pgfqpoint{3.447246in}{1.640633in}}%
\pgfpathlineto{\pgfqpoint{3.450557in}{1.641300in}}%
\pgfpathlineto{\pgfqpoint{3.453869in}{1.645415in}}%
\pgfpathlineto{\pgfqpoint{3.457180in}{1.653056in}}%
\pgfpathlineto{\pgfqpoint{3.463803in}{1.678330in}}%
\pgfpathlineto{\pgfqpoint{3.470426in}{1.714812in}}%
\pgfpathlineto{\pgfqpoint{3.480361in}{1.784088in}}%
\pgfpathlineto{\pgfqpoint{3.500229in}{1.946605in}}%
\pgfpathlineto{\pgfqpoint{3.520098in}{2.107000in}}%
\pgfpathlineto{\pgfqpoint{3.533344in}{2.198586in}}%
\pgfpathlineto{\pgfqpoint{3.543278in}{2.254914in}}%
\pgfpathlineto{\pgfqpoint{3.553212in}{2.298820in}}%
\pgfpathlineto{\pgfqpoint{3.559835in}{2.320730in}}%
\pgfpathlineto{\pgfqpoint{3.566458in}{2.336625in}}%
\pgfpathlineto{\pgfqpoint{3.573081in}{2.346523in}}%
\pgfpathlineto{\pgfqpoint{3.576393in}{2.349256in}}%
\pgfpathlineto{\pgfqpoint{3.579704in}{2.350539in}}%
\pgfpathlineto{\pgfqpoint{3.583015in}{2.350398in}}%
\pgfpathlineto{\pgfqpoint{3.586327in}{2.348862in}}%
\pgfpathlineto{\pgfqpoint{3.592950in}{2.341739in}}%
\pgfpathlineto{\pgfqpoint{3.599573in}{2.329459in}}%
\pgfpathlineto{\pgfqpoint{3.606196in}{2.312339in}}%
\pgfpathlineto{\pgfqpoint{3.616130in}{2.278324in}}%
\pgfpathlineto{\pgfqpoint{3.626064in}{2.235380in}}%
\pgfpathlineto{\pgfqpoint{3.639310in}{2.166428in}}%
\pgfpathlineto{\pgfqpoint{3.655867in}{2.066093in}}%
\pgfpathlineto{\pgfqpoint{3.705539in}{1.752223in}}%
\pgfpathlineto{\pgfqpoint{3.715473in}{1.705665in}}%
\pgfpathlineto{\pgfqpoint{3.725408in}{1.670647in}}%
\pgfpathlineto{\pgfqpoint{3.732031in}{1.654340in}}%
\pgfpathlineto{\pgfqpoint{3.738654in}{1.643456in}}%
\pgfpathlineto{\pgfqpoint{3.745277in}{1.637251in}}%
\pgfpathlineto{\pgfqpoint{3.751899in}{1.634609in}}%
\pgfpathlineto{\pgfqpoint{3.758522in}{1.634279in}}%
\pgfpathlineto{\pgfqpoint{3.785014in}{1.636379in}}%
\pgfpathlineto{\pgfqpoint{3.794948in}{1.634126in}}%
\pgfpathlineto{\pgfqpoint{3.821440in}{1.625838in}}%
\pgfpathlineto{\pgfqpoint{3.828063in}{1.627326in}}%
\pgfpathlineto{\pgfqpoint{3.834686in}{1.632193in}}%
\pgfpathlineto{\pgfqpoint{3.841309in}{1.641227in}}%
\pgfpathlineto{\pgfqpoint{3.847931in}{1.654829in}}%
\pgfpathlineto{\pgfqpoint{3.854554in}{1.672950in}}%
\pgfpathlineto{\pgfqpoint{3.864489in}{1.707550in}}%
\pgfpathlineto{\pgfqpoint{3.877735in}{1.763379in}}%
\pgfpathlineto{\pgfqpoint{3.910849in}{1.907872in}}%
\pgfpathlineto{\pgfqpoint{3.920783in}{1.942325in}}%
\pgfpathlineto{\pgfqpoint{3.930718in}{1.969632in}}%
\pgfpathlineto{\pgfqpoint{3.937341in}{1.983422in}}%
\pgfpathlineto{\pgfqpoint{3.943964in}{1.993511in}}%
\pgfpathlineto{\pgfqpoint{3.950586in}{1.999858in}}%
\pgfpathlineto{\pgfqpoint{3.957209in}{2.002494in}}%
\pgfpathlineto{\pgfqpoint{3.963832in}{2.001520in}}%
\pgfpathlineto{\pgfqpoint{3.970455in}{1.997095in}}%
\pgfpathlineto{\pgfqpoint{3.977078in}{1.989428in}}%
\pgfpathlineto{\pgfqpoint{3.983701in}{1.978773in}}%
\pgfpathlineto{\pgfqpoint{3.993635in}{1.957846in}}%
\pgfpathlineto{\pgfqpoint{4.006881in}{1.922564in}}%
\pgfpathlineto{\pgfqpoint{4.026750in}{1.860548in}}%
\pgfpathlineto{\pgfqpoint{4.049930in}{1.789378in}}%
\pgfpathlineto{\pgfqpoint{4.063176in}{1.756079in}}%
\pgfpathlineto{\pgfqpoint{4.073110in}{1.736372in}}%
\pgfpathlineto{\pgfqpoint{4.083044in}{1.721410in}}%
\pgfpathlineto{\pgfqpoint{4.092979in}{1.710807in}}%
\pgfpathlineto{\pgfqpoint{4.102913in}{1.703813in}}%
\pgfpathlineto{\pgfqpoint{4.112847in}{1.699579in}}%
\pgfpathlineto{\pgfqpoint{4.122782in}{1.697383in}}%
\pgfpathlineto{\pgfqpoint{4.136028in}{1.696908in}}%
\pgfpathlineto{\pgfqpoint{4.149273in}{1.699146in}}%
\pgfpathlineto{\pgfqpoint{4.159208in}{1.702994in}}%
\pgfpathlineto{\pgfqpoint{4.169142in}{1.709213in}}%
\pgfpathlineto{\pgfqpoint{4.179076in}{1.718271in}}%
\pgfpathlineto{\pgfqpoint{4.189011in}{1.730495in}}%
\pgfpathlineto{\pgfqpoint{4.198945in}{1.745955in}}%
\pgfpathlineto{\pgfqpoint{4.212191in}{1.771120in}}%
\pgfpathlineto{\pgfqpoint{4.232060in}{1.815324in}}%
\pgfpathlineto{\pgfqpoint{4.255240in}{1.867055in}}%
\pgfpathlineto{\pgfqpoint{4.268486in}{1.891638in}}%
\pgfpathlineto{\pgfqpoint{4.278420in}{1.905964in}}%
\pgfpathlineto{\pgfqpoint{4.288354in}{1.915981in}}%
\pgfpathlineto{\pgfqpoint{4.294977in}{1.920024in}}%
\pgfpathlineto{\pgfqpoint{4.301600in}{1.921856in}}%
\pgfpathlineto{\pgfqpoint{4.308223in}{1.921445in}}%
\pgfpathlineto{\pgfqpoint{4.314846in}{1.918803in}}%
\pgfpathlineto{\pgfqpoint{4.321469in}{1.913991in}}%
\pgfpathlineto{\pgfqpoint{4.331403in}{1.902950in}}%
\pgfpathlineto{\pgfqpoint{4.341337in}{1.887826in}}%
\pgfpathlineto{\pgfqpoint{4.354583in}{1.862670in}}%
\pgfpathlineto{\pgfqpoint{4.381075in}{1.804570in}}%
\pgfpathlineto{\pgfqpoint{4.397632in}{1.771053in}}%
\pgfpathlineto{\pgfqpoint{4.410878in}{1.749681in}}%
\pgfpathlineto{\pgfqpoint{4.420812in}{1.737718in}}%
\pgfpathlineto{\pgfqpoint{4.430747in}{1.729381in}}%
\pgfpathlineto{\pgfqpoint{4.440681in}{1.724509in}}%
\pgfpathlineto{\pgfqpoint{4.450615in}{1.722854in}}%
\pgfpathlineto{\pgfqpoint{4.460550in}{1.724263in}}%
\pgfpathlineto{\pgfqpoint{4.470484in}{1.728786in}}%
\pgfpathlineto{\pgfqpoint{4.480418in}{1.736690in}}%
\pgfpathlineto{\pgfqpoint{4.490353in}{1.748367in}}%
\pgfpathlineto{\pgfqpoint{4.500287in}{1.764175in}}%
\pgfpathlineto{\pgfqpoint{4.510221in}{1.784273in}}%
\pgfpathlineto{\pgfqpoint{4.523467in}{1.817400in}}%
\pgfpathlineto{\pgfqpoint{4.540024in}{1.866830in}}%
\pgfpathlineto{\pgfqpoint{4.583073in}{2.002006in}}%
\pgfpathlineto{\pgfqpoint{4.596319in}{2.034951in}}%
\pgfpathlineto{\pgfqpoint{4.606253in}{2.054802in}}%
\pgfpathlineto{\pgfqpoint{4.616188in}{2.070035in}}%
\pgfpathlineto{\pgfqpoint{4.626122in}{2.080487in}}%
\pgfpathlineto{\pgfqpoint{4.632745in}{2.084808in}}%
\pgfpathlineto{\pgfqpoint{4.639368in}{2.087065in}}%
\pgfpathlineto{\pgfqpoint{4.645991in}{2.087335in}}%
\pgfpathlineto{\pgfqpoint{4.652614in}{2.085717in}}%
\pgfpathlineto{\pgfqpoint{4.659237in}{2.082331in}}%
\pgfpathlineto{\pgfqpoint{4.669171in}{2.074232in}}%
\pgfpathlineto{\pgfqpoint{4.679105in}{2.062973in}}%
\pgfpathlineto{\pgfqpoint{4.692351in}{2.044062in}}%
\pgfpathlineto{\pgfqpoint{4.712220in}{2.010577in}}%
\pgfpathlineto{\pgfqpoint{4.738711in}{1.965852in}}%
\pgfpathlineto{\pgfqpoint{4.751957in}{1.947598in}}%
\pgfpathlineto{\pgfqpoint{4.761892in}{1.936966in}}%
\pgfpathlineto{\pgfqpoint{4.771826in}{1.929386in}}%
\pgfpathlineto{\pgfqpoint{4.781760in}{1.924985in}}%
\pgfpathlineto{\pgfqpoint{4.791695in}{1.923640in}}%
\pgfpathlineto{\pgfqpoint{4.801629in}{1.925019in}}%
\pgfpathlineto{\pgfqpoint{4.811563in}{1.928650in}}%
\pgfpathlineto{\pgfqpoint{4.824809in}{1.936095in}}%
\pgfpathlineto{\pgfqpoint{4.841366in}{1.948057in}}%
\pgfpathlineto{\pgfqpoint{4.864547in}{1.967580in}}%
\pgfpathlineto{\pgfqpoint{4.884415in}{1.986770in}}%
\pgfpathlineto{\pgfqpoint{4.904284in}{2.009171in}}%
\pgfpathlineto{\pgfqpoint{4.934087in}{2.047459in}}%
\pgfpathlineto{\pgfqpoint{4.953956in}{2.071043in}}%
\pgfpathlineto{\pgfqpoint{4.967202in}{2.082947in}}%
\pgfpathlineto{\pgfqpoint{4.977136in}{2.088906in}}%
\pgfpathlineto{\pgfqpoint{4.987070in}{2.091869in}}%
\pgfpathlineto{\pgfqpoint{4.997005in}{2.091607in}}%
\pgfpathlineto{\pgfqpoint{5.006939in}{2.088057in}}%
\pgfpathlineto{\pgfqpoint{5.016873in}{2.081322in}}%
\pgfpathlineto{\pgfqpoint{5.026808in}{2.071662in}}%
\pgfpathlineto{\pgfqpoint{5.040053in}{2.054964in}}%
\pgfpathlineto{\pgfqpoint{5.056611in}{2.029934in}}%
\pgfpathlineto{\pgfqpoint{5.086414in}{1.984111in}}%
\pgfpathlineto{\pgfqpoint{5.099660in}{1.968238in}}%
\pgfpathlineto{\pgfqpoint{5.109594in}{1.959550in}}%
\pgfpathlineto{\pgfqpoint{5.119528in}{1.953989in}}%
\pgfpathlineto{\pgfqpoint{5.129463in}{1.951575in}}%
\pgfpathlineto{\pgfqpoint{5.139397in}{1.952060in}}%
\pgfpathlineto{\pgfqpoint{5.149331in}{1.955005in}}%
\pgfpathlineto{\pgfqpoint{5.162577in}{1.961817in}}%
\pgfpathlineto{\pgfqpoint{5.179134in}{1.973217in}}%
\pgfpathlineto{\pgfqpoint{5.215560in}{2.002084in}}%
\pgfpathlineto{\pgfqpoint{5.271855in}{2.046378in}}%
\pgfpathlineto{\pgfqpoint{5.291724in}{2.059242in}}%
\pgfpathlineto{\pgfqpoint{5.308281in}{2.067123in}}%
\pgfpathlineto{\pgfqpoint{5.321527in}{2.071047in}}%
\pgfpathlineto{\pgfqpoint{5.334772in}{2.072690in}}%
\pgfpathlineto{\pgfqpoint{5.348018in}{2.072121in}}%
\pgfpathlineto{\pgfqpoint{5.361264in}{2.069578in}}%
\pgfpathlineto{\pgfqpoint{5.377821in}{2.064136in}}%
\pgfpathlineto{\pgfqpoint{5.397690in}{2.055192in}}%
\pgfpathlineto{\pgfqpoint{5.424182in}{2.040608in}}%
\pgfpathlineto{\pgfqpoint{5.453985in}{2.021480in}}%
\pgfpathlineto{\pgfqpoint{5.480476in}{2.001943in}}%
\pgfpathlineto{\pgfqpoint{5.506968in}{1.979814in}}%
\pgfpathlineto{\pgfqpoint{5.536771in}{1.951764in}}%
\pgfpathlineto{\pgfqpoint{5.569885in}{1.917217in}}%
\pgfpathlineto{\pgfqpoint{5.603000in}{1.883195in}}%
\pgfpathlineto{\pgfqpoint{5.616246in}{1.872136in}}%
\pgfpathlineto{\pgfqpoint{5.626180in}{1.865856in}}%
\pgfpathlineto{\pgfqpoint{5.636114in}{1.861978in}}%
\pgfpathlineto{\pgfqpoint{5.646049in}{1.861155in}}%
\pgfpathlineto{\pgfqpoint{5.652672in}{1.862619in}}%
\pgfpathlineto{\pgfqpoint{5.659295in}{1.865892in}}%
\pgfpathlineto{\pgfqpoint{5.665917in}{1.871113in}}%
\pgfpathlineto{\pgfqpoint{5.675852in}{1.882806in}}%
\pgfpathlineto{\pgfqpoint{5.685786in}{1.899221in}}%
\pgfpathlineto{\pgfqpoint{5.695720in}{1.920194in}}%
\pgfpathlineto{\pgfqpoint{5.708966in}{1.954409in}}%
\pgfpathlineto{\pgfqpoint{5.725524in}{2.004522in}}%
\pgfpathlineto{\pgfqpoint{5.761949in}{2.119018in}}%
\pgfpathlineto{\pgfqpoint{5.775195in}{2.152871in}}%
\pgfpathlineto{\pgfqpoint{5.785130in}{2.172834in}}%
\pgfpathlineto{\pgfqpoint{5.795064in}{2.187147in}}%
\pgfpathlineto{\pgfqpoint{5.801687in}{2.193224in}}%
\pgfpathlineto{\pgfqpoint{5.808310in}{2.196384in}}%
\pgfpathlineto{\pgfqpoint{5.814933in}{2.196572in}}%
\pgfpathlineto{\pgfqpoint{5.821556in}{2.193798in}}%
\pgfpathlineto{\pgfqpoint{5.828178in}{2.188138in}}%
\pgfpathlineto{\pgfqpoint{5.834801in}{2.179736in}}%
\pgfpathlineto{\pgfqpoint{5.844736in}{2.162451in}}%
\pgfpathlineto{\pgfqpoint{5.854670in}{2.140398in}}%
\pgfpathlineto{\pgfqpoint{5.867916in}{2.105644in}}%
\pgfpathlineto{\pgfqpoint{5.904342in}{2.005373in}}%
\pgfpathlineto{\pgfqpoint{5.914276in}{1.983872in}}%
\pgfpathlineto{\pgfqpoint{5.924211in}{1.967217in}}%
\pgfpathlineto{\pgfqpoint{5.930833in}{1.959128in}}%
\pgfpathlineto{\pgfqpoint{5.937456in}{1.953515in}}%
\pgfpathlineto{\pgfqpoint{5.944079in}{1.950301in}}%
\pgfpathlineto{\pgfqpoint{5.950702in}{1.949297in}}%
\pgfpathlineto{\pgfqpoint{5.957325in}{1.950211in}}%
\pgfpathlineto{\pgfqpoint{5.967259in}{1.954356in}}%
\pgfpathlineto{\pgfqpoint{5.983817in}{1.965008in}}%
\pgfpathlineto{\pgfqpoint{5.997062in}{1.972853in}}%
\pgfpathlineto{\pgfqpoint{6.006997in}{1.976212in}}%
\pgfpathlineto{\pgfqpoint{6.013620in}{1.976636in}}%
\pgfpathlineto{\pgfqpoint{6.020243in}{1.975320in}}%
\pgfpathlineto{\pgfqpoint{6.026865in}{1.972095in}}%
\pgfpathlineto{\pgfqpoint{6.033488in}{1.966874in}}%
\pgfpathlineto{\pgfqpoint{6.043423in}{1.955333in}}%
\pgfpathlineto{\pgfqpoint{6.053357in}{1.939790in}}%
\pgfpathlineto{\pgfqpoint{6.069914in}{1.908283in}}%
\pgfpathlineto{\pgfqpoint{6.083160in}{1.883812in}}%
\pgfpathlineto{\pgfqpoint{6.089783in}{1.874305in}}%
\pgfpathlineto{\pgfqpoint{6.096406in}{1.868032in}}%
\pgfpathlineto{\pgfqpoint{6.103029in}{1.866013in}}%
\pgfpathlineto{\pgfqpoint{6.106340in}{1.866870in}}%
\pgfpathlineto{\pgfqpoint{6.109652in}{1.869080in}}%
\pgfpathlineto{\pgfqpoint{6.116275in}{1.877745in}}%
\pgfpathlineto{\pgfqpoint{6.122898in}{1.892113in}}%
\pgfpathlineto{\pgfqpoint{6.129520in}{1.911884in}}%
\pgfpathlineto{\pgfqpoint{6.139455in}{1.950222in}}%
\pgfpathlineto{\pgfqpoint{6.152701in}{2.012821in}}%
\pgfpathlineto{\pgfqpoint{6.195749in}{2.228056in}}%
\pgfpathlineto{\pgfqpoint{6.208995in}{2.280359in}}%
\pgfpathlineto{\pgfqpoint{6.218930in}{2.312557in}}%
\pgfpathlineto{\pgfqpoint{6.228864in}{2.338351in}}%
\pgfpathlineto{\pgfqpoint{6.238798in}{2.357692in}}%
\pgfpathlineto{\pgfqpoint{6.245421in}{2.367074in}}%
\pgfpathlineto{\pgfqpoint{6.252044in}{2.373745in}}%
\pgfpathlineto{\pgfqpoint{6.258667in}{2.377826in}}%
\pgfpathlineto{\pgfqpoint{6.265290in}{2.379455in}}%
\pgfpathlineto{\pgfqpoint{6.271913in}{2.378793in}}%
\pgfpathlineto{\pgfqpoint{6.278536in}{2.376017in}}%
\pgfpathlineto{\pgfqpoint{6.285159in}{2.371317in}}%
\pgfpathlineto{\pgfqpoint{6.295093in}{2.361103in}}%
\pgfpathlineto{\pgfqpoint{6.305027in}{2.347741in}}%
\pgfpathlineto{\pgfqpoint{6.318273in}{2.326360in}}%
\pgfpathlineto{\pgfqpoint{6.364633in}{2.246501in}}%
\pgfpathlineto{\pgfqpoint{6.374568in}{2.233980in}}%
\pgfpathlineto{\pgfqpoint{6.384502in}{2.224900in}}%
\pgfpathlineto{\pgfqpoint{6.391125in}{2.221128in}}%
\pgfpathlineto{\pgfqpoint{6.397748in}{2.219422in}}%
\pgfpathlineto{\pgfqpoint{6.404371in}{2.219977in}}%
\pgfpathlineto{\pgfqpoint{6.410994in}{2.222970in}}%
\pgfpathlineto{\pgfqpoint{6.417617in}{2.228564in}}%
\pgfpathlineto{\pgfqpoint{6.424239in}{2.236902in}}%
\pgfpathlineto{\pgfqpoint{6.430862in}{2.248107in}}%
\pgfpathlineto{\pgfqpoint{6.440797in}{2.270498in}}%
\pgfpathlineto{\pgfqpoint{6.450731in}{2.299776in}}%
\pgfpathlineto{\pgfqpoint{6.460665in}{2.336016in}}%
\pgfpathlineto{\pgfqpoint{6.473911in}{2.395033in}}%
\pgfpathlineto{\pgfqpoint{6.487157in}{2.465783in}}%
\pgfpathlineto{\pgfqpoint{6.503714in}{2.569339in}}%
\pgfpathlineto{\pgfqpoint{6.523583in}{2.711613in}}%
\pgfpathlineto{\pgfqpoint{6.560009in}{2.979790in}}%
\pgfpathlineto{\pgfqpoint{6.583189in}{3.126564in}}%
\pgfpathlineto{\pgfqpoint{6.599746in}{3.219195in}}%
\pgfpathlineto{\pgfqpoint{6.612992in}{3.279872in}}%
\pgfpathlineto{\pgfqpoint{6.622926in}{3.315661in}}%
\pgfpathlineto{\pgfqpoint{6.632861in}{3.342687in}}%
\pgfpathlineto{\pgfqpoint{6.639484in}{3.355856in}}%
\pgfpathlineto{\pgfqpoint{6.646107in}{3.365294in}}%
\pgfpathlineto{\pgfqpoint{6.652729in}{3.371260in}}%
\pgfpathlineto{\pgfqpoint{6.659352in}{3.374135in}}%
\pgfpathlineto{\pgfqpoint{6.665975in}{3.374407in}}%
\pgfpathlineto{\pgfqpoint{6.672598in}{3.372639in}}%
\pgfpathlineto{\pgfqpoint{6.682533in}{3.367398in}}%
\pgfpathlineto{\pgfqpoint{6.695778in}{3.357642in}}%
\pgfpathlineto{\pgfqpoint{6.705713in}{3.348074in}}%
\pgfpathlineto{\pgfqpoint{6.712336in}{3.339565in}}%
\pgfpathlineto{\pgfqpoint{6.718958in}{3.328340in}}%
\pgfpathlineto{\pgfqpoint{6.725581in}{3.313532in}}%
\pgfpathlineto{\pgfqpoint{6.732204in}{3.294428in}}%
\pgfpathlineto{\pgfqpoint{6.742139in}{3.256696in}}%
\pgfpathlineto{\pgfqpoint{6.752073in}{3.207517in}}%
\pgfpathlineto{\pgfqpoint{6.762007in}{3.147131in}}%
\pgfpathlineto{\pgfqpoint{6.775253in}{3.050880in}}%
\pgfpathlineto{\pgfqpoint{6.791810in}{2.910397in}}%
\pgfpathlineto{\pgfqpoint{6.848105in}{2.406852in}}%
\pgfpathlineto{\pgfqpoint{6.858039in}{2.343952in}}%
\pgfpathlineto{\pgfqpoint{6.867974in}{2.297328in}}%
\pgfpathlineto{\pgfqpoint{6.874597in}{2.275708in}}%
\pgfpathlineto{\pgfqpoint{6.881220in}{2.261318in}}%
\pgfpathlineto{\pgfqpoint{6.887842in}{2.253493in}}%
\pgfpathlineto{\pgfqpoint{6.891154in}{2.251783in}}%
\pgfpathlineto{\pgfqpoint{6.894465in}{2.251390in}}%
\pgfpathlineto{\pgfqpoint{6.901088in}{2.254100in}}%
\pgfpathlineto{\pgfqpoint{6.907711in}{2.260726in}}%
\pgfpathlineto{\pgfqpoint{6.914334in}{2.270419in}}%
\pgfpathlineto{\pgfqpoint{6.924268in}{2.289004in}}%
\pgfpathlineto{\pgfqpoint{6.964006in}{2.370107in}}%
\pgfpathlineto{\pgfqpoint{6.970629in}{2.378885in}}%
\pgfpathlineto{\pgfqpoint{6.977252in}{2.384827in}}%
\pgfpathlineto{\pgfqpoint{6.983874in}{2.387431in}}%
\pgfpathlineto{\pgfqpoint{6.990497in}{2.386275in}}%
\pgfpathlineto{\pgfqpoint{6.997120in}{2.381065in}}%
\pgfpathlineto{\pgfqpoint{7.003743in}{2.371691in}}%
\pgfpathlineto{\pgfqpoint{7.010366in}{2.358261in}}%
\pgfpathlineto{\pgfqpoint{7.020300in}{2.331310in}}%
\pgfpathlineto{\pgfqpoint{7.033546in}{2.285971in}}%
\pgfpathlineto{\pgfqpoint{7.063349in}{2.177123in}}%
\pgfpathlineto{\pgfqpoint{7.073284in}{2.149124in}}%
\pgfpathlineto{\pgfqpoint{7.079907in}{2.134829in}}%
\pgfpathlineto{\pgfqpoint{7.086529in}{2.124423in}}%
\pgfpathlineto{\pgfqpoint{7.093152in}{2.117937in}}%
\pgfpathlineto{\pgfqpoint{7.099775in}{2.115136in}}%
\pgfpathlineto{\pgfqpoint{7.106398in}{2.115570in}}%
\pgfpathlineto{\pgfqpoint{7.113021in}{2.118645in}}%
\pgfpathlineto{\pgfqpoint{7.122955in}{2.126764in}}%
\pgfpathlineto{\pgfqpoint{7.159381in}{2.162548in}}%
\pgfpathlineto{\pgfqpoint{7.169316in}{2.167966in}}%
\pgfpathlineto{\pgfqpoint{7.179250in}{2.170421in}}%
\pgfpathlineto{\pgfqpoint{7.189184in}{2.170144in}}%
\pgfpathlineto{\pgfqpoint{7.202430in}{2.166725in}}%
\pgfpathlineto{\pgfqpoint{7.228922in}{2.158168in}}%
\pgfpathlineto{\pgfqpoint{7.242168in}{2.156728in}}%
\pgfpathlineto{\pgfqpoint{7.255413in}{2.157969in}}%
\pgfpathlineto{\pgfqpoint{7.268659in}{2.161586in}}%
\pgfpathlineto{\pgfqpoint{7.281905in}{2.167277in}}%
\pgfpathlineto{\pgfqpoint{7.295151in}{2.175143in}}%
\pgfpathlineto{\pgfqpoint{7.308397in}{2.185823in}}%
\pgfpathlineto{\pgfqpoint{7.318331in}{2.196328in}}%
\pgfpathlineto{\pgfqpoint{7.328265in}{2.209553in}}%
\pgfpathlineto{\pgfqpoint{7.338200in}{2.225923in}}%
\pgfpathlineto{\pgfqpoint{7.351445in}{2.252816in}}%
\pgfpathlineto{\pgfqpoint{7.368003in}{2.292882in}}%
\pgfpathlineto{\pgfqpoint{7.394494in}{2.358200in}}%
\pgfpathlineto{\pgfqpoint{7.407740in}{2.384105in}}%
\pgfpathlineto{\pgfqpoint{7.417674in}{2.398492in}}%
\pgfpathlineto{\pgfqpoint{7.427609in}{2.408023in}}%
\pgfpathlineto{\pgfqpoint{7.434232in}{2.411701in}}%
\pgfpathlineto{\pgfqpoint{7.440855in}{2.413430in}}%
\pgfpathlineto{\pgfqpoint{7.450789in}{2.413046in}}%
\pgfpathlineto{\pgfqpoint{7.464035in}{2.409198in}}%
\pgfpathlineto{\pgfqpoint{7.480592in}{2.404247in}}%
\pgfpathlineto{\pgfqpoint{7.490526in}{2.403577in}}%
\pgfpathlineto{\pgfqpoint{7.500461in}{2.405641in}}%
\pgfpathlineto{\pgfqpoint{7.510395in}{2.410666in}}%
\pgfpathlineto{\pgfqpoint{7.520329in}{2.418342in}}%
\pgfpathlineto{\pgfqpoint{7.540198in}{2.437952in}}%
\pgfpathlineto{\pgfqpoint{7.553444in}{2.449585in}}%
\pgfpathlineto{\pgfqpoint{7.560067in}{2.453463in}}%
\pgfpathlineto{\pgfqpoint{7.566690in}{2.455369in}}%
\pgfpathlineto{\pgfqpoint{7.573313in}{2.454826in}}%
\pgfpathlineto{\pgfqpoint{7.579935in}{2.451405in}}%
\pgfpathlineto{\pgfqpoint{7.586558in}{2.444750in}}%
\pgfpathlineto{\pgfqpoint{7.593181in}{2.434591in}}%
\pgfpathlineto{\pgfqpoint{7.599804in}{2.420769in}}%
\pgfpathlineto{\pgfqpoint{7.609739in}{2.393169in}}%
\pgfpathlineto{\pgfqpoint{7.619673in}{2.357974in}}%
\pgfpathlineto{\pgfqpoint{7.636230in}{2.287639in}}%
\pgfpathlineto{\pgfqpoint{7.652787in}{2.217275in}}%
\pgfpathlineto{\pgfqpoint{7.662722in}{2.185159in}}%
\pgfpathlineto{\pgfqpoint{7.669345in}{2.171656in}}%
\pgfpathlineto{\pgfqpoint{7.672656in}{2.167878in}}%
\pgfpathlineto{\pgfqpoint{7.675968in}{2.166293in}}%
\pgfpathlineto{\pgfqpoint{7.679279in}{2.167017in}}%
\pgfpathlineto{\pgfqpoint{7.682590in}{2.170124in}}%
\pgfpathlineto{\pgfqpoint{7.685902in}{2.175646in}}%
\pgfpathlineto{\pgfqpoint{7.692525in}{2.193846in}}%
\pgfpathlineto{\pgfqpoint{7.699148in}{2.221042in}}%
\pgfpathlineto{\pgfqpoint{7.709082in}{2.276214in}}%
\pgfpathlineto{\pgfqpoint{7.722328in}{2.368786in}}%
\pgfpathlineto{\pgfqpoint{7.762065in}{2.666784in}}%
\pgfpathlineto{\pgfqpoint{7.772000in}{2.725658in}}%
\pgfpathlineto{\pgfqpoint{7.781934in}{2.772473in}}%
\pgfpathlineto{\pgfqpoint{7.788557in}{2.795821in}}%
\pgfpathlineto{\pgfqpoint{7.795180in}{2.812130in}}%
\pgfpathlineto{\pgfqpoint{7.801803in}{2.820823in}}%
\pgfpathlineto{\pgfqpoint{7.805114in}{2.822156in}}%
\pgfpathlineto{\pgfqpoint{7.808426in}{2.821413in}}%
\pgfpathlineto{\pgfqpoint{7.811737in}{2.818555in}}%
\pgfpathlineto{\pgfqpoint{7.815048in}{2.813557in}}%
\pgfpathlineto{\pgfqpoint{7.821671in}{2.797111in}}%
\pgfpathlineto{\pgfqpoint{7.828294in}{2.772206in}}%
\pgfpathlineto{\pgfqpoint{7.834917in}{2.739313in}}%
\pgfpathlineto{\pgfqpoint{7.844851in}{2.677020in}}%
\pgfpathlineto{\pgfqpoint{7.861409in}{2.551779in}}%
\pgfpathlineto{\pgfqpoint{7.877966in}{2.430675in}}%
\pgfpathlineto{\pgfqpoint{7.887900in}{2.376865in}}%
\pgfpathlineto{\pgfqpoint{7.894523in}{2.353123in}}%
\pgfpathlineto{\pgfqpoint{7.901146in}{2.339902in}}%
\pgfpathlineto{\pgfqpoint{7.904458in}{2.337037in}}%
\pgfpathlineto{\pgfqpoint{7.907769in}{2.336377in}}%
\pgfpathlineto{\pgfqpoint{7.911080in}{2.337593in}}%
\pgfpathlineto{\pgfqpoint{7.917703in}{2.343956in}}%
\pgfpathlineto{\pgfqpoint{7.927638in}{2.355700in}}%
\pgfpathlineto{\pgfqpoint{7.930949in}{2.358118in}}%
\pgfpathlineto{\pgfqpoint{7.934261in}{2.359066in}}%
\pgfpathlineto{\pgfqpoint{7.937572in}{2.358289in}}%
\pgfpathlineto{\pgfqpoint{7.940883in}{2.355721in}}%
\pgfpathlineto{\pgfqpoint{7.947506in}{2.346051in}}%
\pgfpathlineto{\pgfqpoint{7.957441in}{2.328922in}}%
\pgfpathlineto{\pgfqpoint{7.960752in}{2.325731in}}%
\pgfpathlineto{\pgfqpoint{7.964064in}{2.325113in}}%
\pgfpathlineto{\pgfqpoint{7.967375in}{2.327608in}}%
\pgfpathlineto{\pgfqpoint{7.970687in}{2.333489in}}%
\pgfpathlineto{\pgfqpoint{7.977309in}{2.354986in}}%
\pgfpathlineto{\pgfqpoint{7.987244in}{2.403597in}}%
\pgfpathlineto{\pgfqpoint{8.000490in}{2.468083in}}%
\pgfpathlineto{\pgfqpoint{8.007112in}{2.489711in}}%
\pgfpathlineto{\pgfqpoint{8.010424in}{2.496750in}}%
\pgfpathlineto{\pgfqpoint{8.013735in}{2.501082in}}%
\pgfpathlineto{\pgfqpoint{8.017047in}{2.502643in}}%
\pgfpathlineto{\pgfqpoint{8.020358in}{2.501408in}}%
\pgfpathlineto{\pgfqpoint{8.023670in}{2.497386in}}%
\pgfpathlineto{\pgfqpoint{8.026981in}{2.490605in}}%
\pgfpathlineto{\pgfqpoint{8.033604in}{2.468977in}}%
\pgfpathlineto{\pgfqpoint{8.040227in}{2.437219in}}%
\pgfpathlineto{\pgfqpoint{8.050161in}{2.374649in}}%
\pgfpathlineto{\pgfqpoint{8.060096in}{2.310409in}}%
\pgfpathlineto{\pgfqpoint{8.063407in}{2.295055in}}%
\pgfpathlineto{\pgfqpoint{8.066719in}{2.286038in}}%
\pgfpathlineto{\pgfqpoint{8.070030in}{2.285251in}}%
\pgfpathlineto{\pgfqpoint{8.073341in}{2.293912in}}%
\pgfpathlineto{\pgfqpoint{8.076653in}{2.312121in}}%
\pgfpathlineto{\pgfqpoint{8.083276in}{2.371332in}}%
\pgfpathlineto{\pgfqpoint{8.096522in}{2.508509in}}%
\pgfpathlineto{\pgfqpoint{8.103145in}{2.557060in}}%
\pgfpathlineto{\pgfqpoint{8.109767in}{2.590694in}}%
\pgfpathlineto{\pgfqpoint{8.129636in}{2.674677in}}%
\pgfpathlineto{\pgfqpoint{8.136259in}{2.715833in}}%
\pgfpathlineto{\pgfqpoint{8.146193in}{2.796317in}}%
\pgfpathlineto{\pgfqpoint{8.156128in}{2.896272in}}%
\pgfpathlineto{\pgfqpoint{8.172685in}{3.089831in}}%
\pgfpathlineto{\pgfqpoint{8.215734in}{3.610049in}}%
\pgfpathlineto{\pgfqpoint{8.232291in}{3.777406in}}%
\pgfpathlineto{\pgfqpoint{8.245537in}{3.886427in}}%
\pgfpathlineto{\pgfqpoint{8.255471in}{3.952435in}}%
\pgfpathlineto{\pgfqpoint{8.265406in}{4.005502in}}%
\pgfpathlineto{\pgfqpoint{8.275340in}{4.046254in}}%
\pgfpathlineto{\pgfqpoint{8.281963in}{4.066734in}}%
\pgfpathlineto{\pgfqpoint{8.288586in}{4.081941in}}%
\pgfpathlineto{\pgfqpoint{8.295209in}{4.092028in}}%
\pgfpathlineto{\pgfqpoint{8.301832in}{4.097266in}}%
\pgfpathlineto{\pgfqpoint{8.308454in}{4.098042in}}%
\pgfpathlineto{\pgfqpoint{8.315077in}{4.094827in}}%
\pgfpathlineto{\pgfqpoint{8.321700in}{4.088129in}}%
\pgfpathlineto{\pgfqpoint{8.328323in}{4.078462in}}%
\pgfpathlineto{\pgfqpoint{8.338257in}{4.059468in}}%
\pgfpathlineto{\pgfqpoint{8.351503in}{4.028004in}}%
\pgfpathlineto{\pgfqpoint{8.368061in}{3.982228in}}%
\pgfpathlineto{\pgfqpoint{8.384618in}{3.930827in}}%
\pgfpathlineto{\pgfqpoint{8.401175in}{3.871961in}}%
\pgfpathlineto{\pgfqpoint{8.417732in}{3.803391in}}%
\pgfpathlineto{\pgfqpoint{8.467404in}{3.586183in}}%
\pgfpathlineto{\pgfqpoint{8.480650in}{3.541108in}}%
\pgfpathlineto{\pgfqpoint{8.490584in}{3.513851in}}%
\pgfpathlineto{\pgfqpoint{8.500519in}{3.492896in}}%
\pgfpathlineto{\pgfqpoint{8.507141in}{3.482594in}}%
\pgfpathlineto{\pgfqpoint{8.513764in}{3.475264in}}%
\pgfpathlineto{\pgfqpoint{8.520387in}{3.470885in}}%
\pgfpathlineto{\pgfqpoint{8.527010in}{3.469388in}}%
\pgfpathlineto{\pgfqpoint{8.533633in}{3.470665in}}%
\pgfpathlineto{\pgfqpoint{8.540256in}{3.474558in}}%
\pgfpathlineto{\pgfqpoint{8.546879in}{3.480864in}}%
\pgfpathlineto{\pgfqpoint{8.556813in}{3.494277in}}%
\pgfpathlineto{\pgfqpoint{8.570059in}{3.517788in}}%
\pgfpathlineto{\pgfqpoint{8.606485in}{3.588011in}}%
\pgfpathlineto{\pgfqpoint{8.616419in}{3.600574in}}%
\pgfpathlineto{\pgfqpoint{8.623042in}{3.606001in}}%
\pgfpathlineto{\pgfqpoint{8.629665in}{3.608954in}}%
\pgfpathlineto{\pgfqpoint{8.636288in}{3.609538in}}%
\pgfpathlineto{\pgfqpoint{8.642911in}{3.607978in}}%
\pgfpathlineto{\pgfqpoint{8.649534in}{3.604558in}}%
\pgfpathlineto{\pgfqpoint{8.659468in}{3.596584in}}%
\pgfpathlineto{\pgfqpoint{8.672714in}{3.582041in}}%
\pgfpathlineto{\pgfqpoint{8.692583in}{3.555661in}}%
\pgfpathlineto{\pgfqpoint{8.715763in}{3.525521in}}%
\pgfpathlineto{\pgfqpoint{8.729009in}{3.512327in}}%
\pgfpathlineto{\pgfqpoint{8.738943in}{3.505351in}}%
\pgfpathlineto{\pgfqpoint{8.748877in}{3.500950in}}%
\pgfpathlineto{\pgfqpoint{8.758812in}{3.498859in}}%
\pgfpathlineto{\pgfqpoint{8.768746in}{3.498676in}}%
\pgfpathlineto{\pgfqpoint{8.781992in}{3.500683in}}%
\pgfpathlineto{\pgfqpoint{8.801860in}{3.506552in}}%
\pgfpathlineto{\pgfqpoint{8.821729in}{3.512374in}}%
\pgfpathlineto{\pgfqpoint{8.834975in}{3.514066in}}%
\pgfpathlineto{\pgfqpoint{8.848221in}{3.513273in}}%
\pgfpathlineto{\pgfqpoint{8.864778in}{3.509619in}}%
\pgfpathlineto{\pgfqpoint{8.894581in}{3.500005in}}%
\pgfpathlineto{\pgfqpoint{8.914450in}{3.491877in}}%
\pgfpathlineto{\pgfqpoint{8.927696in}{3.484639in}}%
\pgfpathlineto{\pgfqpoint{8.940941in}{3.474871in}}%
\pgfpathlineto{\pgfqpoint{8.950876in}{3.465241in}}%
\pgfpathlineto{\pgfqpoint{8.960810in}{3.453145in}}%
\pgfpathlineto{\pgfqpoint{8.970744in}{3.438185in}}%
\pgfpathlineto{\pgfqpoint{8.980679in}{3.419982in}}%
\pgfpathlineto{\pgfqpoint{8.990613in}{3.398089in}}%
\pgfpathlineto{\pgfqpoint{9.000547in}{3.371912in}}%
\pgfpathlineto{\pgfqpoint{9.010482in}{3.340777in}}%
\pgfpathlineto{\pgfqpoint{9.023728in}{3.290567in}}%
\pgfpathlineto{\pgfqpoint{9.036973in}{3.229793in}}%
\pgfpathlineto{\pgfqpoint{9.050219in}{3.158545in}}%
\pgfpathlineto{\pgfqpoint{9.066776in}{3.056388in}}%
\pgfpathlineto{\pgfqpoint{9.096580in}{2.852758in}}%
\pgfpathlineto{\pgfqpoint{9.116448in}{2.725015in}}%
\pgfpathlineto{\pgfqpoint{9.129694in}{2.653834in}}%
\pgfpathlineto{\pgfqpoint{9.139628in}{2.610588in}}%
\pgfpathlineto{\pgfqpoint{9.149563in}{2.576753in}}%
\pgfpathlineto{\pgfqpoint{9.159497in}{2.551754in}}%
\pgfpathlineto{\pgfqpoint{9.169431in}{2.533726in}}%
\pgfpathlineto{\pgfqpoint{9.182677in}{2.515640in}}%
\pgfpathlineto{\pgfqpoint{9.235660in}{2.451090in}}%
\pgfpathlineto{\pgfqpoint{9.242283in}{2.446935in}}%
\pgfpathlineto{\pgfqpoint{9.248906in}{2.445178in}}%
\pgfpathlineto{\pgfqpoint{9.255529in}{2.446068in}}%
\pgfpathlineto{\pgfqpoint{9.262152in}{2.449669in}}%
\pgfpathlineto{\pgfqpoint{9.268775in}{2.455884in}}%
\pgfpathlineto{\pgfqpoint{9.278709in}{2.469603in}}%
\pgfpathlineto{\pgfqpoint{9.288644in}{2.487643in}}%
\pgfpathlineto{\pgfqpoint{9.301889in}{2.516330in}}%
\pgfpathlineto{\pgfqpoint{9.364807in}{2.660271in}}%
\pgfpathlineto{\pgfqpoint{9.378053in}{2.682764in}}%
\pgfpathlineto{\pgfqpoint{9.387987in}{2.695051in}}%
\pgfpathlineto{\pgfqpoint{9.394610in}{2.700243in}}%
\pgfpathlineto{\pgfqpoint{9.401233in}{2.702550in}}%
\pgfpathlineto{\pgfqpoint{9.407856in}{2.701656in}}%
\pgfpathlineto{\pgfqpoint{9.414479in}{2.697343in}}%
\pgfpathlineto{\pgfqpoint{9.421102in}{2.689476in}}%
\pgfpathlineto{\pgfqpoint{9.427724in}{2.677993in}}%
\pgfpathlineto{\pgfqpoint{9.437659in}{2.654145in}}%
\pgfpathlineto{\pgfqpoint{9.447593in}{2.623264in}}%
\pgfpathlineto{\pgfqpoint{9.460839in}{2.574414in}}%
\pgfpathlineto{\pgfqpoint{9.507199in}{2.395441in}}%
\pgfpathlineto{\pgfqpoint{9.523757in}{2.342107in}}%
\pgfpathlineto{\pgfqpoint{9.540314in}{2.296892in}}%
\pgfpathlineto{\pgfqpoint{9.553560in}{2.266672in}}%
\pgfpathlineto{\pgfqpoint{9.566805in}{2.241705in}}%
\pgfpathlineto{\pgfqpoint{9.580051in}{2.222016in}}%
\pgfpathlineto{\pgfqpoint{9.589986in}{2.210750in}}%
\pgfpathlineto{\pgfqpoint{9.599920in}{2.202500in}}%
\pgfpathlineto{\pgfqpoint{9.609854in}{2.197294in}}%
\pgfpathlineto{\pgfqpoint{9.619789in}{2.195189in}}%
\pgfpathlineto{\pgfqpoint{9.629723in}{2.196262in}}%
\pgfpathlineto{\pgfqpoint{9.639657in}{2.200566in}}%
\pgfpathlineto{\pgfqpoint{9.649592in}{2.208110in}}%
\pgfpathlineto{\pgfqpoint{9.659526in}{2.218847in}}%
\pgfpathlineto{\pgfqpoint{9.669460in}{2.232643in}}%
\pgfpathlineto{\pgfqpoint{9.682706in}{2.255323in}}%
\pgfpathlineto{\pgfqpoint{9.699263in}{2.289042in}}%
\pgfpathlineto{\pgfqpoint{9.765492in}{2.431291in}}%
\pgfpathlineto{\pgfqpoint{9.778738in}{2.452207in}}%
\pgfpathlineto{\pgfqpoint{9.788673in}{2.464654in}}%
\pgfpathlineto{\pgfqpoint{9.798607in}{2.474076in}}%
\pgfpathlineto{\pgfqpoint{9.808541in}{2.480599in}}%
\pgfpathlineto{\pgfqpoint{9.821787in}{2.485839in}}%
\pgfpathlineto{\pgfqpoint{9.841656in}{2.492680in}}%
\pgfpathlineto{\pgfqpoint{9.848279in}{2.497265in}}%
\pgfpathlineto{\pgfqpoint{9.854902in}{2.504572in}}%
\pgfpathlineto{\pgfqpoint{9.861524in}{2.515831in}}%
\pgfpathlineto{\pgfqpoint{9.868147in}{2.532440in}}%
\pgfpathlineto{\pgfqpoint{9.874770in}{2.555878in}}%
\pgfpathlineto{\pgfqpoint{9.881393in}{2.587590in}}%
\pgfpathlineto{\pgfqpoint{9.888016in}{2.628938in}}%
\pgfpathlineto{\pgfqpoint{9.894639in}{2.681506in}}%
\pgfpathlineto{\pgfqpoint{9.901262in}{2.748107in}}%
\pgfpathlineto{\pgfqpoint{9.907885in}{2.834289in}}%
\pgfpathlineto{\pgfqpoint{9.914508in}{2.948008in}}%
\pgfpathlineto{\pgfqpoint{9.924442in}{3.175504in}}%
\pgfpathlineto{\pgfqpoint{9.937688in}{3.491885in}}%
\pgfpathlineto{\pgfqpoint{9.944311in}{3.594993in}}%
\pgfpathlineto{\pgfqpoint{9.947622in}{3.626439in}}%
\pgfpathlineto{\pgfqpoint{9.950934in}{3.644804in}}%
\pgfpathlineto{\pgfqpoint{9.954245in}{3.651250in}}%
\pgfpathlineto{\pgfqpoint{9.957556in}{3.647352in}}%
\pgfpathlineto{\pgfqpoint{9.960868in}{3.634866in}}%
\pgfpathlineto{\pgfqpoint{9.967491in}{3.591092in}}%
\pgfpathlineto{\pgfqpoint{9.977425in}{3.500738in}}%
\pgfpathlineto{\pgfqpoint{9.993982in}{3.349579in}}%
\pgfpathlineto{\pgfqpoint{10.003917in}{3.279952in}}%
\pgfpathlineto{\pgfqpoint{10.010540in}{3.245021in}}%
\pgfpathlineto{\pgfqpoint{10.017163in}{3.219206in}}%
\pgfpathlineto{\pgfqpoint{10.023785in}{3.201939in}}%
\pgfpathlineto{\pgfqpoint{10.030408in}{3.192449in}}%
\pgfpathlineto{\pgfqpoint{10.033720in}{3.190374in}}%
\pgfpathlineto{\pgfqpoint{10.037031in}{3.189959in}}%
\pgfpathlineto{\pgfqpoint{10.040343in}{3.191132in}}%
\pgfpathlineto{\pgfqpoint{10.043654in}{3.193839in}}%
\pgfpathlineto{\pgfqpoint{10.050277in}{3.203761in}}%
\pgfpathlineto{\pgfqpoint{10.056900in}{3.219798in}}%
\pgfpathlineto{\pgfqpoint{10.063523in}{3.242409in}}%
\pgfpathlineto{\pgfqpoint{10.070146in}{3.272175in}}%
\pgfpathlineto{\pgfqpoint{10.080080in}{3.330329in}}%
\pgfpathlineto{\pgfqpoint{10.093326in}{3.425973in}}%
\pgfpathlineto{\pgfqpoint{10.126440in}{3.670086in}}%
\pgfpathlineto{\pgfqpoint{10.159555in}{3.897277in}}%
\pgfpathlineto{\pgfqpoint{10.176112in}{3.997678in}}%
\pgfpathlineto{\pgfqpoint{10.189358in}{4.065274in}}%
\pgfpathlineto{\pgfqpoint{10.205915in}{4.135338in}}%
\pgfpathlineto{\pgfqpoint{10.225784in}{4.217235in}}%
\pgfpathlineto{\pgfqpoint{10.252276in}{4.332702in}}%
\pgfpathlineto{\pgfqpoint{10.262210in}{4.365889in}}%
\pgfpathlineto{\pgfqpoint{10.272144in}{4.390903in}}%
\pgfpathlineto{\pgfqpoint{10.282079in}{4.409969in}}%
\pgfpathlineto{\pgfqpoint{10.308570in}{4.456308in}}%
\pgfpathlineto{\pgfqpoint{10.318505in}{4.479167in}}%
\pgfpathlineto{\pgfqpoint{10.328439in}{4.507000in}}%
\pgfpathlineto{\pgfqpoint{10.341685in}{4.551366in}}%
\pgfpathlineto{\pgfqpoint{10.381422in}{4.693465in}}%
\pgfpathlineto{\pgfqpoint{10.391356in}{4.718083in}}%
\pgfpathlineto{\pgfqpoint{10.397979in}{4.729726in}}%
\pgfpathlineto{\pgfqpoint{10.404602in}{4.737356in}}%
\pgfpathlineto{\pgfqpoint{10.411225in}{4.741175in}}%
\pgfpathlineto{\pgfqpoint{10.417848in}{4.741613in}}%
\pgfpathlineto{\pgfqpoint{10.424471in}{4.739147in}}%
\pgfpathlineto{\pgfqpoint{10.431094in}{4.734131in}}%
\pgfpathlineto{\pgfqpoint{10.437717in}{4.726740in}}%
\pgfpathlineto{\pgfqpoint{10.447651in}{4.711362in}}%
\pgfpathlineto{\pgfqpoint{10.457585in}{4.691376in}}%
\pgfpathlineto{\pgfqpoint{10.490700in}{4.617789in}}%
\pgfpathlineto{\pgfqpoint{10.497323in}{4.607855in}}%
\pgfpathlineto{\pgfqpoint{10.503946in}{4.601084in}}%
\pgfpathlineto{\pgfqpoint{10.510569in}{4.597805in}}%
\pgfpathlineto{\pgfqpoint{10.517191in}{4.598129in}}%
\pgfpathlineto{\pgfqpoint{10.523814in}{4.601944in}}%
\pgfpathlineto{\pgfqpoint{10.530437in}{4.608927in}}%
\pgfpathlineto{\pgfqpoint{10.540372in}{4.624186in}}%
\pgfpathlineto{\pgfqpoint{10.556929in}{4.656442in}}%
\pgfpathlineto{\pgfqpoint{10.576798in}{4.693902in}}%
\pgfpathlineto{\pgfqpoint{10.606601in}{4.743720in}}%
\pgfpathlineto{\pgfqpoint{10.616535in}{4.756084in}}%
\pgfpathlineto{\pgfqpoint{10.623158in}{4.760986in}}%
\pgfpathlineto{\pgfqpoint{10.629781in}{4.763312in}}%
\pgfpathlineto{\pgfqpoint{10.652961in}{4.767426in}}%
\pgfpathlineto{\pgfqpoint{10.659584in}{4.772658in}}%
\pgfpathlineto{\pgfqpoint{10.666207in}{4.781098in}}%
\pgfpathlineto{\pgfqpoint{10.672830in}{4.792725in}}%
\pgfpathlineto{\pgfqpoint{10.682764in}{4.815005in}}%
\pgfpathlineto{\pgfqpoint{10.715878in}{4.895327in}}%
\pgfpathlineto{\pgfqpoint{10.725813in}{4.910583in}}%
\pgfpathlineto{\pgfqpoint{10.732436in}{4.917001in}}%
\pgfpathlineto{\pgfqpoint{10.739059in}{4.920478in}}%
\pgfpathlineto{\pgfqpoint{10.745682in}{4.921306in}}%
\pgfpathlineto{\pgfqpoint{10.752304in}{4.919845in}}%
\pgfpathlineto{\pgfqpoint{10.758927in}{4.916391in}}%
\pgfpathlineto{\pgfqpoint{10.768862in}{4.907813in}}%
\pgfpathlineto{\pgfqpoint{10.778796in}{4.895345in}}%
\pgfpathlineto{\pgfqpoint{10.788730in}{4.879341in}}%
\pgfpathlineto{\pgfqpoint{10.805288in}{4.847339in}}%
\pgfpathlineto{\pgfqpoint{10.828468in}{4.802480in}}%
\pgfpathlineto{\pgfqpoint{10.838402in}{4.787215in}}%
\pgfpathlineto{\pgfqpoint{10.848336in}{4.775825in}}%
\pgfpathlineto{\pgfqpoint{10.854959in}{4.770647in}}%
\pgfpathlineto{\pgfqpoint{10.861582in}{4.767467in}}%
\pgfpathlineto{\pgfqpoint{10.868205in}{4.766262in}}%
\pgfpathlineto{\pgfqpoint{10.874828in}{4.766954in}}%
\pgfpathlineto{\pgfqpoint{10.881451in}{4.769433in}}%
\pgfpathlineto{\pgfqpoint{10.891385in}{4.776237in}}%
\pgfpathlineto{\pgfqpoint{10.901320in}{4.786421in}}%
\pgfpathlineto{\pgfqpoint{10.911254in}{4.799682in}}%
\pgfpathlineto{\pgfqpoint{10.924500in}{4.821320in}}%
\pgfpathlineto{\pgfqpoint{10.960926in}{4.884234in}}%
\pgfpathlineto{\pgfqpoint{10.984106in}{4.921908in}}%
\pgfpathlineto{\pgfqpoint{10.997352in}{4.948745in}}%
\pgfpathlineto{\pgfqpoint{11.010598in}{4.981152in}}%
\pgfpathlineto{\pgfqpoint{11.043712in}{5.067910in}}%
\pgfpathlineto{\pgfqpoint{11.053646in}{5.086591in}}%
\pgfpathlineto{\pgfqpoint{11.060269in}{5.095474in}}%
\pgfpathlineto{\pgfqpoint{11.066892in}{5.101523in}}%
\pgfpathlineto{\pgfqpoint{11.073515in}{5.105137in}}%
\pgfpathlineto{\pgfqpoint{11.080138in}{5.106822in}}%
\pgfpathlineto{\pgfqpoint{11.086761in}{5.106872in}}%
\pgfpathlineto{\pgfqpoint{11.093384in}{5.105152in}}%
\pgfpathlineto{\pgfqpoint{11.100007in}{5.101167in}}%
\pgfpathlineto{\pgfqpoint{11.106630in}{5.094367in}}%
\pgfpathlineto{\pgfqpoint{11.113252in}{5.084457in}}%
\pgfpathlineto{\pgfqpoint{11.123187in}{5.064104in}}%
\pgfpathlineto{\pgfqpoint{11.136433in}{5.029704in}}%
\pgfpathlineto{\pgfqpoint{11.162924in}{4.957680in}}%
\pgfpathlineto{\pgfqpoint{11.172859in}{4.935894in}}%
\pgfpathlineto{\pgfqpoint{11.182793in}{4.919046in}}%
\pgfpathlineto{\pgfqpoint{11.192727in}{4.907528in}}%
\pgfpathlineto{\pgfqpoint{11.199350in}{4.902624in}}%
\pgfpathlineto{\pgfqpoint{11.209285in}{4.898573in}}%
\pgfpathlineto{\pgfqpoint{11.222530in}{4.896600in}}%
\pgfpathlineto{\pgfqpoint{11.235776in}{4.894586in}}%
\pgfpathlineto{\pgfqpoint{11.245710in}{4.890530in}}%
\pgfpathlineto{\pgfqpoint{11.252333in}{4.885093in}}%
\pgfpathlineto{\pgfqpoint{11.258956in}{4.875847in}}%
\pgfpathlineto{\pgfqpoint{11.265579in}{4.860932in}}%
\pgfpathlineto{\pgfqpoint{11.272202in}{4.838308in}}%
\pgfpathlineto{\pgfqpoint{11.278825in}{4.806085in}}%
\pgfpathlineto{\pgfqpoint{11.285448in}{4.762929in}}%
\pgfpathlineto{\pgfqpoint{11.295382in}{4.677678in}}%
\pgfpathlineto{\pgfqpoint{11.321874in}{4.420399in}}%
\pgfpathlineto{\pgfqpoint{11.328497in}{4.382195in}}%
\pgfpathlineto{\pgfqpoint{11.335120in}{4.360905in}}%
\pgfpathlineto{\pgfqpoint{11.338431in}{4.355881in}}%
\pgfpathlineto{\pgfqpoint{11.341743in}{4.353902in}}%
\pgfpathlineto{\pgfqpoint{11.345054in}{4.354341in}}%
\pgfpathlineto{\pgfqpoint{11.351677in}{4.359908in}}%
\pgfpathlineto{\pgfqpoint{11.364923in}{4.374094in}}%
\pgfpathlineto{\pgfqpoint{11.368234in}{4.375715in}}%
\pgfpathlineto{\pgfqpoint{11.371546in}{4.375930in}}%
\pgfpathlineto{\pgfqpoint{11.374857in}{4.374508in}}%
\pgfpathlineto{\pgfqpoint{11.378168in}{4.371261in}}%
\pgfpathlineto{\pgfqpoint{11.381480in}{4.366032in}}%
\pgfpathlineto{\pgfqpoint{11.388103in}{4.349184in}}%
\pgfpathlineto{\pgfqpoint{11.394726in}{4.323420in}}%
\pgfpathlineto{\pgfqpoint{11.401349in}{4.288881in}}%
\pgfpathlineto{\pgfqpoint{11.411283in}{4.223160in}}%
\pgfpathlineto{\pgfqpoint{11.437775in}{4.030467in}}%
\pgfpathlineto{\pgfqpoint{11.447709in}{3.976139in}}%
\pgfpathlineto{\pgfqpoint{11.457643in}{3.933753in}}%
\pgfpathlineto{\pgfqpoint{11.470889in}{3.887774in}}%
\pgfpathlineto{\pgfqpoint{11.500692in}{3.787992in}}%
\pgfpathlineto{\pgfqpoint{11.520561in}{3.712063in}}%
\pgfpathlineto{\pgfqpoint{11.540430in}{3.626933in}}%
\pgfpathlineto{\pgfqpoint{11.563610in}{3.516882in}}%
\pgfpathlineto{\pgfqpoint{11.583478in}{3.424932in}}%
\pgfpathlineto{\pgfqpoint{11.590101in}{3.401494in}}%
\pgfpathlineto{\pgfqpoint{11.596724in}{3.386145in}}%
\pgfpathlineto{\pgfqpoint{11.600036in}{3.382407in}}%
\pgfpathlineto{\pgfqpoint{11.603347in}{3.381614in}}%
\pgfpathlineto{\pgfqpoint{11.606659in}{3.383812in}}%
\pgfpathlineto{\pgfqpoint{11.609970in}{3.388859in}}%
\pgfpathlineto{\pgfqpoint{11.616593in}{3.405976in}}%
\pgfpathlineto{\pgfqpoint{11.639773in}{3.480754in}}%
\pgfpathlineto{\pgfqpoint{11.646396in}{3.493185in}}%
\pgfpathlineto{\pgfqpoint{11.653019in}{3.499934in}}%
\pgfpathlineto{\pgfqpoint{11.656330in}{3.501289in}}%
\pgfpathlineto{\pgfqpoint{11.659642in}{3.501385in}}%
\pgfpathlineto{\pgfqpoint{11.666265in}{3.498067in}}%
\pgfpathlineto{\pgfqpoint{11.672888in}{3.490465in}}%
\pgfpathlineto{\pgfqpoint{11.679510in}{3.478961in}}%
\pgfpathlineto{\pgfqpoint{11.689445in}{3.455015in}}%
\pgfpathlineto{\pgfqpoint{11.699379in}{3.423738in}}%
\pgfpathlineto{\pgfqpoint{11.712625in}{3.371980in}}%
\pgfpathlineto{\pgfqpoint{11.725871in}{3.310689in}}%
\pgfpathlineto{\pgfqpoint{11.745739in}{3.206643in}}%
\pgfpathlineto{\pgfqpoint{11.772231in}{3.066067in}}%
\pgfpathlineto{\pgfqpoint{11.782165in}{3.023255in}}%
\pgfpathlineto{\pgfqpoint{11.788788in}{3.001608in}}%
\pgfpathlineto{\pgfqpoint{11.795411in}{2.987221in}}%
\pgfpathlineto{\pgfqpoint{11.798723in}{2.983067in}}%
\pgfpathlineto{\pgfqpoint{11.802034in}{2.981004in}}%
\pgfpathlineto{\pgfqpoint{11.805345in}{2.981006in}}%
\pgfpathlineto{\pgfqpoint{11.808657in}{2.982986in}}%
\pgfpathlineto{\pgfqpoint{11.815280in}{2.992276in}}%
\pgfpathlineto{\pgfqpoint{11.821903in}{3.007357in}}%
\pgfpathlineto{\pgfqpoint{11.831837in}{3.037232in}}%
\pgfpathlineto{\pgfqpoint{11.848394in}{3.096127in}}%
\pgfpathlineto{\pgfqpoint{11.878197in}{3.203087in}}%
\pgfpathlineto{\pgfqpoint{11.891443in}{3.243879in}}%
\pgfpathlineto{\pgfqpoint{11.904689in}{3.277493in}}%
\pgfpathlineto{\pgfqpoint{11.914623in}{3.297246in}}%
\pgfpathlineto{\pgfqpoint{11.924558in}{3.312129in}}%
\pgfpathlineto{\pgfqpoint{11.934492in}{3.322368in}}%
\pgfpathlineto{\pgfqpoint{11.944426in}{3.328743in}}%
\pgfpathlineto{\pgfqpoint{11.974229in}{3.343576in}}%
\pgfpathlineto{\pgfqpoint{11.984164in}{3.353465in}}%
\pgfpathlineto{\pgfqpoint{11.997410in}{3.370519in}}%
\pgfpathlineto{\pgfqpoint{12.030524in}{3.414928in}}%
\pgfpathlineto{\pgfqpoint{12.050393in}{3.437705in}}%
\pgfpathlineto{\pgfqpoint{12.066950in}{3.453120in}}%
\pgfpathlineto{\pgfqpoint{12.080196in}{3.462062in}}%
\pgfpathlineto{\pgfqpoint{12.090130in}{3.466345in}}%
\pgfpathlineto{\pgfqpoint{12.100065in}{3.468366in}}%
\pgfpathlineto{\pgfqpoint{12.109999in}{3.468075in}}%
\pgfpathlineto{\pgfqpoint{12.119933in}{3.465533in}}%
\pgfpathlineto{\pgfqpoint{12.133179in}{3.459085in}}%
\pgfpathlineto{\pgfqpoint{12.166294in}{3.439856in}}%
\pgfpathlineto{\pgfqpoint{12.176228in}{3.437875in}}%
\pgfpathlineto{\pgfqpoint{12.186162in}{3.438410in}}%
\pgfpathlineto{\pgfqpoint{12.199408in}{3.441782in}}%
\pgfpathlineto{\pgfqpoint{12.232523in}{3.451606in}}%
\pgfpathlineto{\pgfqpoint{12.242457in}{3.452387in}}%
\pgfpathlineto{\pgfqpoint{12.252391in}{3.451165in}}%
\pgfpathlineto{\pgfqpoint{12.262326in}{3.447452in}}%
\pgfpathlineto{\pgfqpoint{12.272260in}{3.440934in}}%
\pgfpathlineto{\pgfqpoint{12.282194in}{3.431540in}}%
\pgfpathlineto{\pgfqpoint{12.295440in}{3.414897in}}%
\pgfpathlineto{\pgfqpoint{12.311997in}{3.389024in}}%
\pgfpathlineto{\pgfqpoint{12.351735in}{3.323147in}}%
\pgfpathlineto{\pgfqpoint{12.361669in}{3.311623in}}%
\pgfpathlineto{\pgfqpoint{12.368292in}{3.306511in}}%
\pgfpathlineto{\pgfqpoint{12.374915in}{3.303930in}}%
\pgfpathlineto{\pgfqpoint{12.381538in}{3.304172in}}%
\pgfpathlineto{\pgfqpoint{12.388161in}{3.307375in}}%
\pgfpathlineto{\pgfqpoint{12.394784in}{3.313514in}}%
\pgfpathlineto{\pgfqpoint{12.401406in}{3.322426in}}%
\pgfpathlineto{\pgfqpoint{12.411341in}{3.340394in}}%
\pgfpathlineto{\pgfqpoint{12.424587in}{3.371107in}}%
\pgfpathlineto{\pgfqpoint{12.444455in}{3.425200in}}%
\pgfpathlineto{\pgfqpoint{12.464324in}{3.478719in}}%
\pgfpathlineto{\pgfqpoint{12.477570in}{3.508518in}}%
\pgfpathlineto{\pgfqpoint{12.487504in}{3.525726in}}%
\pgfpathlineto{\pgfqpoint{12.494127in}{3.534226in}}%
\pgfpathlineto{\pgfqpoint{12.500750in}{3.540117in}}%
\pgfpathlineto{\pgfqpoint{12.507373in}{3.543264in}}%
\pgfpathlineto{\pgfqpoint{12.513996in}{3.543597in}}%
\pgfpathlineto{\pgfqpoint{12.520619in}{3.541102in}}%
\pgfpathlineto{\pgfqpoint{12.527242in}{3.535817in}}%
\pgfpathlineto{\pgfqpoint{12.533864in}{3.527830in}}%
\pgfpathlineto{\pgfqpoint{12.543799in}{3.511081in}}%
\pgfpathlineto{\pgfqpoint{12.553733in}{3.489180in}}%
\pgfpathlineto{\pgfqpoint{12.566979in}{3.453400in}}%
\pgfpathlineto{\pgfqpoint{12.583536in}{3.401441in}}%
\pgfpathlineto{\pgfqpoint{12.639831in}{3.218125in}}%
\pgfpathlineto{\pgfqpoint{12.653077in}{3.184632in}}%
\pgfpathlineto{\pgfqpoint{12.663011in}{3.164877in}}%
\pgfpathlineto{\pgfqpoint{12.672945in}{3.150601in}}%
\pgfpathlineto{\pgfqpoint{12.679568in}{3.144381in}}%
\pgfpathlineto{\pgfqpoint{12.686191in}{3.140872in}}%
\pgfpathlineto{\pgfqpoint{12.692814in}{3.140044in}}%
\pgfpathlineto{\pgfqpoint{12.699437in}{3.141788in}}%
\pgfpathlineto{\pgfqpoint{12.706060in}{3.145921in}}%
\pgfpathlineto{\pgfqpoint{12.712683in}{3.152206in}}%
\pgfpathlineto{\pgfqpoint{12.722617in}{3.165077in}}%
\pgfpathlineto{\pgfqpoint{12.735863in}{3.187241in}}%
\pgfpathlineto{\pgfqpoint{12.752420in}{3.220311in}}%
\pgfpathlineto{\pgfqpoint{12.798780in}{3.316964in}}%
\pgfpathlineto{\pgfqpoint{12.808715in}{3.331204in}}%
\pgfpathlineto{\pgfqpoint{12.815338in}{3.338035in}}%
\pgfpathlineto{\pgfqpoint{12.821961in}{3.342496in}}%
\pgfpathlineto{\pgfqpoint{12.828584in}{3.344493in}}%
\pgfpathlineto{\pgfqpoint{12.835206in}{3.344015in}}%
\pgfpathlineto{\pgfqpoint{12.841829in}{3.341117in}}%
\pgfpathlineto{\pgfqpoint{12.848452in}{3.335902in}}%
\pgfpathlineto{\pgfqpoint{12.858387in}{3.324020in}}%
\pgfpathlineto{\pgfqpoint{12.868321in}{3.307699in}}%
\pgfpathlineto{\pgfqpoint{12.878255in}{3.287438in}}%
\pgfpathlineto{\pgfqpoint{12.891501in}{3.255108in}}%
\pgfpathlineto{\pgfqpoint{12.908058in}{3.207562in}}%
\pgfpathlineto{\pgfqpoint{12.927927in}{3.142700in}}%
\pgfpathlineto{\pgfqpoint{12.967664in}{3.009500in}}%
\pgfpathlineto{\pgfqpoint{12.980910in}{2.973946in}}%
\pgfpathlineto{\pgfqpoint{12.990845in}{2.953565in}}%
\pgfpathlineto{\pgfqpoint{12.997467in}{2.943563in}}%
\pgfpathlineto{\pgfqpoint{13.004090in}{2.936615in}}%
\pgfpathlineto{\pgfqpoint{13.010713in}{2.932732in}}%
\pgfpathlineto{\pgfqpoint{13.017336in}{2.931810in}}%
\pgfpathlineto{\pgfqpoint{13.023959in}{2.933673in}}%
\pgfpathlineto{\pgfqpoint{13.030582in}{2.938107in}}%
\pgfpathlineto{\pgfqpoint{13.037205in}{2.944901in}}%
\pgfpathlineto{\pgfqpoint{13.047139in}{2.959104in}}%
\pgfpathlineto{\pgfqpoint{13.057074in}{2.977625in}}%
\pgfpathlineto{\pgfqpoint{13.070319in}{3.008071in}}%
\pgfpathlineto{\pgfqpoint{13.086877in}{3.052921in}}%
\pgfpathlineto{\pgfqpoint{13.119991in}{3.145254in}}%
\pgfpathlineto{\pgfqpoint{13.133237in}{3.174055in}}%
\pgfpathlineto{\pgfqpoint{13.143171in}{3.189735in}}%
\pgfpathlineto{\pgfqpoint{13.149794in}{3.196828in}}%
\pgfpathlineto{\pgfqpoint{13.156417in}{3.200992in}}%
\pgfpathlineto{\pgfqpoint{13.163040in}{3.202098in}}%
\pgfpathlineto{\pgfqpoint{13.169663in}{3.200090in}}%
\pgfpathlineto{\pgfqpoint{13.176286in}{3.194987in}}%
\pgfpathlineto{\pgfqpoint{13.182909in}{3.186895in}}%
\pgfpathlineto{\pgfqpoint{13.192843in}{3.169596in}}%
\pgfpathlineto{\pgfqpoint{13.202777in}{3.147048in}}%
\pgfpathlineto{\pgfqpoint{13.219335in}{3.101888in}}%
\pgfpathlineto{\pgfqpoint{13.245826in}{3.028414in}}%
\pgfpathlineto{\pgfqpoint{13.259072in}{2.999448in}}%
\pgfpathlineto{\pgfqpoint{13.269006in}{2.983574in}}%
\pgfpathlineto{\pgfqpoint{13.275629in}{2.976277in}}%
\pgfpathlineto{\pgfqpoint{13.282252in}{2.971836in}}%
\pgfpathlineto{\pgfqpoint{13.288875in}{2.970384in}}%
\pgfpathlineto{\pgfqpoint{13.295498in}{2.971993in}}%
\pgfpathlineto{\pgfqpoint{13.302121in}{2.976673in}}%
\pgfpathlineto{\pgfqpoint{13.308744in}{2.984371in}}%
\pgfpathlineto{\pgfqpoint{13.315367in}{2.994971in}}%
\pgfpathlineto{\pgfqpoint{13.325301in}{3.015918in}}%
\pgfpathlineto{\pgfqpoint{13.335235in}{3.042218in}}%
\pgfpathlineto{\pgfqpoint{13.348481in}{3.084035in}}%
\pgfpathlineto{\pgfqpoint{13.365038in}{3.144194in}}%
\pgfpathlineto{\pgfqpoint{13.388219in}{3.237475in}}%
\pgfpathlineto{\pgfqpoint{13.461070in}{3.538635in}}%
\pgfpathlineto{\pgfqpoint{13.477628in}{3.594874in}}%
\pgfpathlineto{\pgfqpoint{13.494185in}{3.643215in}}%
\pgfpathlineto{\pgfqpoint{13.507431in}{3.676090in}}%
\pgfpathlineto{\pgfqpoint{13.520677in}{3.703775in}}%
\pgfpathlineto{\pgfqpoint{13.533922in}{3.726102in}}%
\pgfpathlineto{\pgfqpoint{13.543857in}{3.739154in}}%
\pgfpathlineto{\pgfqpoint{13.553791in}{3.748884in}}%
\pgfpathlineto{\pgfqpoint{13.563725in}{3.755165in}}%
\pgfpathlineto{\pgfqpoint{13.573660in}{3.757904in}}%
\pgfpathlineto{\pgfqpoint{13.580283in}{3.757730in}}%
\pgfpathlineto{\pgfqpoint{13.586906in}{3.755944in}}%
\pgfpathlineto{\pgfqpoint{13.596840in}{3.750236in}}%
\pgfpathlineto{\pgfqpoint{13.606774in}{3.740927in}}%
\pgfpathlineto{\pgfqpoint{13.616709in}{3.728134in}}%
\pgfpathlineto{\pgfqpoint{13.629954in}{3.706132in}}%
\pgfpathlineto{\pgfqpoint{13.646512in}{3.672416in}}%
\pgfpathlineto{\pgfqpoint{13.706118in}{3.543320in}}%
\pgfpathlineto{\pgfqpoint{13.719364in}{3.521994in}}%
\pgfpathlineto{\pgfqpoint{13.729298in}{3.509511in}}%
\pgfpathlineto{\pgfqpoint{13.739232in}{3.500395in}}%
\pgfpathlineto{\pgfqpoint{13.749167in}{3.494720in}}%
\pgfpathlineto{\pgfqpoint{13.759101in}{3.492312in}}%
\pgfpathlineto{\pgfqpoint{13.769035in}{3.492783in}}%
\pgfpathlineto{\pgfqpoint{13.778970in}{3.495588in}}%
\pgfpathlineto{\pgfqpoint{13.792215in}{3.501842in}}%
\pgfpathlineto{\pgfqpoint{13.835264in}{3.524974in}}%
\pgfpathlineto{\pgfqpoint{13.855133in}{3.531653in}}%
\pgfpathlineto{\pgfqpoint{13.875002in}{3.538670in}}%
\pgfpathlineto{\pgfqpoint{13.891559in}{3.547274in}}%
\pgfpathlineto{\pgfqpoint{13.911428in}{3.557622in}}%
\pgfpathlineto{\pgfqpoint{13.921362in}{3.560220in}}%
\pgfpathlineto{\pgfqpoint{13.927985in}{3.560277in}}%
\pgfpathlineto{\pgfqpoint{13.934608in}{3.558719in}}%
\pgfpathlineto{\pgfqpoint{13.941231in}{3.555365in}}%
\pgfpathlineto{\pgfqpoint{13.947854in}{3.550083in}}%
\pgfpathlineto{\pgfqpoint{13.957788in}{3.538349in}}%
\pgfpathlineto{\pgfqpoint{13.967722in}{3.521913in}}%
\pgfpathlineto{\pgfqpoint{13.977657in}{3.500798in}}%
\pgfpathlineto{\pgfqpoint{13.990902in}{3.465697in}}%
\pgfpathlineto{\pgfqpoint{14.004148in}{3.423387in}}%
\pgfpathlineto{\pgfqpoint{14.020705in}{3.361984in}}%
\pgfpathlineto{\pgfqpoint{14.043886in}{3.264899in}}%
\pgfpathlineto{\pgfqpoint{14.083623in}{3.095210in}}%
\pgfpathlineto{\pgfqpoint{14.100180in}{3.035196in}}%
\pgfpathlineto{\pgfqpoint{14.116738in}{2.984925in}}%
\pgfpathlineto{\pgfqpoint{14.139918in}{2.923841in}}%
\pgfpathlineto{\pgfqpoint{14.146541in}{2.910295in}}%
\pgfpathlineto{\pgfqpoint{14.149852in}{2.906037in}}%
\pgfpathlineto{\pgfqpoint{14.153163in}{2.904528in}}%
\pgfpathlineto{\pgfqpoint{14.156475in}{2.906680in}}%
\pgfpathlineto{\pgfqpoint{14.159786in}{2.913332in}}%
\pgfpathlineto{\pgfqpoint{14.163098in}{2.925077in}}%
\pgfpathlineto{\pgfqpoint{14.169721in}{2.964134in}}%
\pgfpathlineto{\pgfqpoint{14.179655in}{3.051414in}}%
\pgfpathlineto{\pgfqpoint{14.196212in}{3.204287in}}%
\pgfpathlineto{\pgfqpoint{14.206147in}{3.273694in}}%
\pgfpathlineto{\pgfqpoint{14.212770in}{3.308320in}}%
\pgfpathlineto{\pgfqpoint{14.219392in}{3.333866in}}%
\pgfpathlineto{\pgfqpoint{14.226015in}{3.350671in}}%
\pgfpathlineto{\pgfqpoint{14.232638in}{3.359065in}}%
\pgfpathlineto{\pgfqpoint{14.235950in}{3.360207in}}%
\pgfpathlineto{\pgfqpoint{14.239261in}{3.359366in}}%
\pgfpathlineto{\pgfqpoint{14.242573in}{3.356585in}}%
\pgfpathlineto{\pgfqpoint{14.249196in}{3.345378in}}%
\pgfpathlineto{\pgfqpoint{14.255818in}{3.326973in}}%
\pgfpathlineto{\pgfqpoint{14.262441in}{3.301803in}}%
\pgfpathlineto{\pgfqpoint{14.272376in}{3.252448in}}%
\pgfpathlineto{\pgfqpoint{14.282310in}{3.190841in}}%
\pgfpathlineto{\pgfqpoint{14.295556in}{3.093190in}}%
\pgfpathlineto{\pgfqpoint{14.312113in}{2.953548in}}%
\pgfpathlineto{\pgfqpoint{14.341916in}{2.695718in}}%
\pgfpathlineto{\pgfqpoint{14.351850in}{2.628585in}}%
\pgfpathlineto{\pgfqpoint{14.358473in}{2.595586in}}%
\pgfpathlineto{\pgfqpoint{14.365096in}{2.573764in}}%
\pgfpathlineto{\pgfqpoint{14.368408in}{2.567143in}}%
\pgfpathlineto{\pgfqpoint{14.371719in}{2.563266in}}%
\pgfpathlineto{\pgfqpoint{14.375031in}{2.561952in}}%
\pgfpathlineto{\pgfqpoint{14.378342in}{2.562969in}}%
\pgfpathlineto{\pgfqpoint{14.381653in}{2.566062in}}%
\pgfpathlineto{\pgfqpoint{14.388276in}{2.577452in}}%
\pgfpathlineto{\pgfqpoint{14.394899in}{2.594321in}}%
\pgfpathlineto{\pgfqpoint{14.404834in}{2.627221in}}%
\pgfpathlineto{\pgfqpoint{14.418079in}{2.681827in}}%
\pgfpathlineto{\pgfqpoint{14.434637in}{2.762614in}}%
\pgfpathlineto{\pgfqpoint{14.471063in}{2.946004in}}%
\pgfpathlineto{\pgfqpoint{14.484308in}{2.999937in}}%
\pgfpathlineto{\pgfqpoint{14.494243in}{3.033084in}}%
\pgfpathlineto{\pgfqpoint{14.504177in}{3.059224in}}%
\pgfpathlineto{\pgfqpoint{14.514111in}{3.078235in}}%
\pgfpathlineto{\pgfqpoint{14.524046in}{3.090979in}}%
\pgfpathlineto{\pgfqpoint{14.533980in}{3.099426in}}%
\pgfpathlineto{\pgfqpoint{14.557160in}{3.114133in}}%
\pgfpathlineto{\pgfqpoint{14.570406in}{3.120509in}}%
\pgfpathlineto{\pgfqpoint{14.577029in}{3.121724in}}%
\pgfpathlineto{\pgfqpoint{14.583652in}{3.120895in}}%
\pgfpathlineto{\pgfqpoint{14.590275in}{3.117635in}}%
\pgfpathlineto{\pgfqpoint{14.596898in}{3.111728in}}%
\pgfpathlineto{\pgfqpoint{14.603521in}{3.103113in}}%
\pgfpathlineto{\pgfqpoint{14.613455in}{3.085285in}}%
\pgfpathlineto{\pgfqpoint{14.623389in}{3.062153in}}%
\pgfpathlineto{\pgfqpoint{14.636635in}{3.024793in}}%
\pgfpathlineto{\pgfqpoint{14.679684in}{2.894346in}}%
\pgfpathlineto{\pgfqpoint{14.689618in}{2.873930in}}%
\pgfpathlineto{\pgfqpoint{14.696241in}{2.864641in}}%
\pgfpathlineto{\pgfqpoint{14.702864in}{2.859420in}}%
\pgfpathlineto{\pgfqpoint{14.706176in}{2.858467in}}%
\pgfpathlineto{\pgfqpoint{14.709487in}{2.858673in}}%
\pgfpathlineto{\pgfqpoint{14.716110in}{2.862663in}}%
\pgfpathlineto{\pgfqpoint{14.722733in}{2.871496in}}%
\pgfpathlineto{\pgfqpoint{14.729356in}{2.885130in}}%
\pgfpathlineto{\pgfqpoint{14.735979in}{2.903412in}}%
\pgfpathlineto{\pgfqpoint{14.745913in}{2.939004in}}%
\pgfpathlineto{\pgfqpoint{14.755847in}{2.983270in}}%
\pgfpathlineto{\pgfqpoint{14.772405in}{3.070154in}}%
\pgfpathlineto{\pgfqpoint{14.795585in}{3.191644in}}%
\pgfpathlineto{\pgfqpoint{14.805519in}{3.234113in}}%
\pgfpathlineto{\pgfqpoint{14.815453in}{3.267662in}}%
\pgfpathlineto{\pgfqpoint{14.822076in}{3.284511in}}%
\pgfpathlineto{\pgfqpoint{14.828699in}{3.296721in}}%
\pgfpathlineto{\pgfqpoint{14.835322in}{3.304195in}}%
\pgfpathlineto{\pgfqpoint{14.841945in}{3.306911in}}%
\pgfpathlineto{\pgfqpoint{14.845256in}{3.306496in}}%
\pgfpathlineto{\pgfqpoint{14.851879in}{3.302184in}}%
\pgfpathlineto{\pgfqpoint{14.858502in}{3.293356in}}%
\pgfpathlineto{\pgfqpoint{14.865125in}{3.280228in}}%
\pgfpathlineto{\pgfqpoint{14.875060in}{3.253096in}}%
\pgfpathlineto{\pgfqpoint{14.884994in}{3.218097in}}%
\pgfpathlineto{\pgfqpoint{14.898240in}{3.161589in}}%
\pgfpathlineto{\pgfqpoint{14.914797in}{3.080249in}}%
\pgfpathlineto{\pgfqpoint{14.947911in}{2.913440in}}%
\pgfpathlineto{\pgfqpoint{14.957846in}{2.872064in}}%
\pgfpathlineto{\pgfqpoint{14.967780in}{2.839533in}}%
\pgfpathlineto{\pgfqpoint{14.974403in}{2.824033in}}%
\pgfpathlineto{\pgfqpoint{14.981026in}{2.814114in}}%
\pgfpathlineto{\pgfqpoint{14.984337in}{2.811335in}}%
\pgfpathlineto{\pgfqpoint{14.987649in}{2.810023in}}%
\pgfpathlineto{\pgfqpoint{14.990960in}{2.810166in}}%
\pgfpathlineto{\pgfqpoint{14.994272in}{2.811740in}}%
\pgfpathlineto{\pgfqpoint{15.000895in}{2.819035in}}%
\pgfpathlineto{\pgfqpoint{15.007518in}{2.831559in}}%
\pgfpathlineto{\pgfqpoint{15.014140in}{2.848923in}}%
\pgfpathlineto{\pgfqpoint{15.024075in}{2.883211in}}%
\pgfpathlineto{\pgfqpoint{15.034009in}{2.926240in}}%
\pgfpathlineto{\pgfqpoint{15.047255in}{2.994572in}}%
\pgfpathlineto{\pgfqpoint{15.070435in}{3.130142in}}%
\pgfpathlineto{\pgfqpoint{15.090304in}{3.241235in}}%
\pgfpathlineto{\pgfqpoint{15.103550in}{3.301889in}}%
\pgfpathlineto{\pgfqpoint{15.113484in}{3.336797in}}%
\pgfpathlineto{\pgfqpoint{15.120107in}{3.354180in}}%
\pgfpathlineto{\pgfqpoint{15.126730in}{3.366487in}}%
\pgfpathlineto{\pgfqpoint{15.133353in}{3.373523in}}%
\pgfpathlineto{\pgfqpoint{15.136664in}{3.375032in}}%
\pgfpathlineto{\pgfqpoint{15.139976in}{3.375199in}}%
\pgfpathlineto{\pgfqpoint{15.143287in}{3.374035in}}%
\pgfpathlineto{\pgfqpoint{15.149910in}{3.367800in}}%
\pgfpathlineto{\pgfqpoint{15.156533in}{3.356623in}}%
\pgfpathlineto{\pgfqpoint{15.163156in}{3.341018in}}%
\pgfpathlineto{\pgfqpoint{15.173090in}{3.310957in}}%
\pgfpathlineto{\pgfqpoint{15.189647in}{3.250621in}}%
\pgfpathlineto{\pgfqpoint{15.219450in}{3.131736in}}%
\pgfpathlineto{\pgfqpoint{15.236008in}{3.056829in}}%
\pgfpathlineto{\pgfqpoint{15.262499in}{2.923077in}}%
\pgfpathlineto{\pgfqpoint{15.279056in}{2.843604in}}%
\pgfpathlineto{\pgfqpoint{15.292302in}{2.790661in}}%
\pgfpathlineto{\pgfqpoint{15.302237in}{2.759257in}}%
\pgfpathlineto{\pgfqpoint{15.312171in}{2.735671in}}%
\pgfpathlineto{\pgfqpoint{15.318794in}{2.724487in}}%
\pgfpathlineto{\pgfqpoint{15.325417in}{2.717073in}}%
\pgfpathlineto{\pgfqpoint{15.332040in}{2.713569in}}%
\pgfpathlineto{\pgfqpoint{15.338663in}{2.714133in}}%
\pgfpathlineto{\pgfqpoint{15.345285in}{2.718933in}}%
\pgfpathlineto{\pgfqpoint{15.351908in}{2.728113in}}%
\pgfpathlineto{\pgfqpoint{15.358531in}{2.741759in}}%
\pgfpathlineto{\pgfqpoint{15.365154in}{2.759843in}}%
\pgfpathlineto{\pgfqpoint{15.375088in}{2.794818in}}%
\pgfpathlineto{\pgfqpoint{15.388334in}{2.853550in}}%
\pgfpathlineto{\pgfqpoint{15.411514in}{2.972294in}}%
\pgfpathlineto{\pgfqpoint{15.428072in}{3.053365in}}%
\pgfpathlineto{\pgfqpoint{15.441317in}{3.108476in}}%
\pgfpathlineto{\pgfqpoint{15.451252in}{3.141930in}}%
\pgfpathlineto{\pgfqpoint{15.461186in}{3.167554in}}%
\pgfpathlineto{\pgfqpoint{15.467809in}{3.179961in}}%
\pgfpathlineto{\pgfqpoint{15.474432in}{3.188478in}}%
\pgfpathlineto{\pgfqpoint{15.481055in}{3.193043in}}%
\pgfpathlineto{\pgfqpoint{15.487678in}{3.193652in}}%
\pgfpathlineto{\pgfqpoint{15.494301in}{3.190357in}}%
\pgfpathlineto{\pgfqpoint{15.500924in}{3.183269in}}%
\pgfpathlineto{\pgfqpoint{15.507546in}{3.172553in}}%
\pgfpathlineto{\pgfqpoint{15.517481in}{3.150169in}}%
\pgfpathlineto{\pgfqpoint{15.527415in}{3.121055in}}%
\pgfpathlineto{\pgfqpoint{15.540661in}{3.073835in}}%
\pgfpathlineto{\pgfqpoint{15.567153in}{2.965865in}}%
\pgfpathlineto{\pgfqpoint{15.580398in}{2.916600in}}%
\pgfpathlineto{\pgfqpoint{15.590333in}{2.886938in}}%
\pgfpathlineto{\pgfqpoint{15.600267in}{2.865338in}}%
\pgfpathlineto{\pgfqpoint{15.606890in}{2.855461in}}%
\pgfpathlineto{\pgfqpoint{15.613513in}{2.848794in}}%
\pgfpathlineto{\pgfqpoint{15.620136in}{2.844765in}}%
\pgfpathlineto{\pgfqpoint{15.626759in}{2.842744in}}%
\pgfpathlineto{\pgfqpoint{15.636693in}{2.842207in}}%
\pgfpathlineto{\pgfqpoint{15.653250in}{2.844566in}}%
\pgfpathlineto{\pgfqpoint{15.679742in}{2.851196in}}%
\pgfpathlineto{\pgfqpoint{15.696299in}{2.857422in}}%
\pgfpathlineto{\pgfqpoint{15.709545in}{2.864733in}}%
\pgfpathlineto{\pgfqpoint{15.722791in}{2.875066in}}%
\pgfpathlineto{\pgfqpoint{15.736036in}{2.889158in}}%
\pgfpathlineto{\pgfqpoint{15.749282in}{2.907563in}}%
\pgfpathlineto{\pgfqpoint{15.762528in}{2.930846in}}%
\pgfpathlineto{\pgfqpoint{15.775774in}{2.960043in}}%
\pgfpathlineto{\pgfqpoint{15.785708in}{2.986939in}}%
\pgfpathlineto{\pgfqpoint{15.795643in}{3.019229in}}%
\pgfpathlineto{\pgfqpoint{15.808888in}{3.071733in}}%
\pgfpathlineto{\pgfqpoint{15.825446in}{3.150142in}}%
\pgfpathlineto{\pgfqpoint{15.851937in}{3.277674in}}%
\pgfpathlineto{\pgfqpoint{15.865183in}{3.328264in}}%
\pgfpathlineto{\pgfqpoint{15.875117in}{3.357018in}}%
\pgfpathlineto{\pgfqpoint{15.881740in}{3.371449in}}%
\pgfpathlineto{\pgfqpoint{15.888363in}{3.382055in}}%
\pgfpathlineto{\pgfqpoint{15.894986in}{3.388901in}}%
\pgfpathlineto{\pgfqpoint{15.901609in}{3.392109in}}%
\pgfpathlineto{\pgfqpoint{15.908232in}{3.391834in}}%
\pgfpathlineto{\pgfqpoint{15.914855in}{3.388253in}}%
\pgfpathlineto{\pgfqpoint{15.921478in}{3.381546in}}%
\pgfpathlineto{\pgfqpoint{15.928101in}{3.371898in}}%
\pgfpathlineto{\pgfqpoint{15.938035in}{3.352314in}}%
\pgfpathlineto{\pgfqpoint{15.947969in}{3.327168in}}%
\pgfpathlineto{\pgfqpoint{15.961215in}{3.286210in}}%
\pgfpathlineto{\pgfqpoint{15.977772in}{3.226093in}}%
\pgfpathlineto{\pgfqpoint{16.010887in}{3.102290in}}%
\pgfpathlineto{\pgfqpoint{16.020821in}{3.072872in}}%
\pgfpathlineto{\pgfqpoint{16.030756in}{3.050079in}}%
\pgfpathlineto{\pgfqpoint{16.037378in}{3.038925in}}%
\pgfpathlineto{\pgfqpoint{16.044001in}{3.031045in}}%
\pgfpathlineto{\pgfqpoint{16.050624in}{3.026367in}}%
\pgfpathlineto{\pgfqpoint{16.057247in}{3.024770in}}%
\pgfpathlineto{\pgfqpoint{16.063870in}{3.026130in}}%
\pgfpathlineto{\pgfqpoint{16.070493in}{3.030338in}}%
\pgfpathlineto{\pgfqpoint{16.077116in}{3.037315in}}%
\pgfpathlineto{\pgfqpoint{16.083739in}{3.046997in}}%
\pgfpathlineto{\pgfqpoint{16.093673in}{3.066457in}}%
\pgfpathlineto{\pgfqpoint{16.103607in}{3.091571in}}%
\pgfpathlineto{\pgfqpoint{16.116853in}{3.132954in}}%
\pgfpathlineto{\pgfqpoint{16.133410in}{3.194631in}}%
\pgfpathlineto{\pgfqpoint{16.163214in}{3.318512in}}%
\pgfpathlineto{\pgfqpoint{16.186394in}{3.410138in}}%
\pgfpathlineto{\pgfqpoint{16.199639in}{3.455009in}}%
\pgfpathlineto{\pgfqpoint{16.212885in}{3.491545in}}%
\pgfpathlineto{\pgfqpoint{16.222820in}{3.511965in}}%
\pgfpathlineto{\pgfqpoint{16.229443in}{3.521582in}}%
\pgfpathlineto{\pgfqpoint{16.236065in}{3.527552in}}%
\pgfpathlineto{\pgfqpoint{16.242688in}{3.529498in}}%
\pgfpathlineto{\pgfqpoint{16.249311in}{3.527080in}}%
\pgfpathlineto{\pgfqpoint{16.255934in}{3.520027in}}%
\pgfpathlineto{\pgfqpoint{16.262557in}{3.508157in}}%
\pgfpathlineto{\pgfqpoint{16.269180in}{3.491386in}}%
\pgfpathlineto{\pgfqpoint{16.275803in}{3.469730in}}%
\pgfpathlineto{\pgfqpoint{16.285737in}{3.428356in}}%
\pgfpathlineto{\pgfqpoint{16.295672in}{3.377005in}}%
\pgfpathlineto{\pgfqpoint{16.308917in}{3.295053in}}%
\pgfpathlineto{\pgfqpoint{16.325475in}{3.176373in}}%
\pgfpathlineto{\pgfqpoint{16.378458in}{2.781998in}}%
\pgfpathlineto{\pgfqpoint{16.391704in}{2.704730in}}%
\pgfpathlineto{\pgfqpoint{16.401638in}{2.658387in}}%
\pgfpathlineto{\pgfqpoint{16.408261in}{2.634283in}}%
\pgfpathlineto{\pgfqpoint{16.414884in}{2.616351in}}%
\pgfpathlineto{\pgfqpoint{16.421507in}{2.605093in}}%
\pgfpathlineto{\pgfqpoint{16.424818in}{2.602071in}}%
\pgfpathlineto{\pgfqpoint{16.428130in}{2.600813in}}%
\pgfpathlineto{\pgfqpoint{16.431441in}{2.601318in}}%
\pgfpathlineto{\pgfqpoint{16.434752in}{2.603573in}}%
\pgfpathlineto{\pgfqpoint{16.441375in}{2.613196in}}%
\pgfpathlineto{\pgfqpoint{16.447998in}{2.629301in}}%
\pgfpathlineto{\pgfqpoint{16.454621in}{2.651367in}}%
\pgfpathlineto{\pgfqpoint{16.464555in}{2.694331in}}%
\pgfpathlineto{\pgfqpoint{16.474490in}{2.747288in}}%
\pgfpathlineto{\pgfqpoint{16.487736in}{2.829982in}}%
\pgfpathlineto{\pgfqpoint{16.507604in}{2.970580in}}%
\pgfpathlineto{\pgfqpoint{16.540719in}{3.208643in}}%
\pgfpathlineto{\pgfqpoint{16.553965in}{3.290090in}}%
\pgfpathlineto{\pgfqpoint{16.563899in}{3.341313in}}%
\pgfpathlineto{\pgfqpoint{16.573833in}{3.382106in}}%
\pgfpathlineto{\pgfqpoint{16.573833in}{3.382106in}}%
\pgfusepath{stroke}%
\end{pgfscope}%
\begin{pgfscope}%
\pgfpathrectangle{\pgfqpoint{2.400000in}{1.081300in}}{\pgfqpoint{14.880000in}{7.569100in}}%
\pgfusepath{clip}%
\pgfsetrectcap%
\pgfsetroundjoin%
\pgfsetlinewidth{1.505625pt}%
\definecolor{currentstroke}{rgb}{0.498039,0.498039,0.498039}%
\pgfsetstrokecolor{currentstroke}%
\pgfsetdash{}{0pt}%
\pgfpathmoveto{\pgfqpoint{3.076364in}{1.425350in}}%
\pgfpathlineto{\pgfqpoint{3.198887in}{1.429589in}}%
\pgfpathlineto{\pgfqpoint{3.291608in}{1.432555in}}%
\pgfpathlineto{\pgfqpoint{3.400886in}{1.436824in}}%
\pgfpathlineto{\pgfqpoint{3.616130in}{1.438877in}}%
\pgfpathlineto{\pgfqpoint{3.675736in}{1.437298in}}%
\pgfpathlineto{\pgfqpoint{3.771768in}{1.431715in}}%
\pgfpathlineto{\pgfqpoint{3.828063in}{1.429503in}}%
\pgfpathlineto{\pgfqpoint{3.874423in}{1.430233in}}%
\pgfpathlineto{\pgfqpoint{3.983701in}{1.432741in}}%
\pgfpathlineto{\pgfqpoint{4.162519in}{1.433090in}}%
\pgfpathlineto{\pgfqpoint{4.357895in}{1.428528in}}%
\pgfpathlineto{\pgfqpoint{4.530090in}{1.431975in}}%
\pgfpathlineto{\pgfqpoint{4.712220in}{1.431392in}}%
\pgfpathlineto{\pgfqpoint{4.808252in}{1.431695in}}%
\pgfpathlineto{\pgfqpoint{4.897661in}{1.431986in}}%
\pgfpathlineto{\pgfqpoint{5.301658in}{1.440866in}}%
\pgfpathlineto{\pgfqpoint{5.381133in}{1.438887in}}%
\pgfpathlineto{\pgfqpoint{5.579820in}{1.433649in}}%
\pgfpathlineto{\pgfqpoint{5.646049in}{1.433663in}}%
\pgfpathlineto{\pgfqpoint{5.781818in}{1.435167in}}%
\pgfpathlineto{\pgfqpoint{5.940768in}{1.433563in}}%
\pgfpathlineto{\pgfqpoint{6.252044in}{1.439964in}}%
\pgfpathlineto{\pgfqpoint{6.301716in}{1.437528in}}%
\pgfpathlineto{\pgfqpoint{6.420928in}{1.430213in}}%
\pgfpathlineto{\pgfqpoint{6.480534in}{1.431073in}}%
\pgfpathlineto{\pgfqpoint{6.543452in}{1.431165in}}%
\pgfpathlineto{\pgfqpoint{6.616304in}{1.428552in}}%
\pgfpathlineto{\pgfqpoint{6.662664in}{1.427917in}}%
\pgfpathlineto{\pgfqpoint{6.775253in}{1.428190in}}%
\pgfpathlineto{\pgfqpoint{6.858039in}{1.427516in}}%
\pgfpathlineto{\pgfqpoint{6.930891in}{1.430647in}}%
\pgfpathlineto{\pgfqpoint{7.020300in}{1.433801in}}%
\pgfpathlineto{\pgfqpoint{7.086529in}{1.433651in}}%
\pgfpathlineto{\pgfqpoint{7.179250in}{1.433401in}}%
\pgfpathlineto{\pgfqpoint{7.295151in}{1.436102in}}%
\pgfpathlineto{\pgfqpoint{7.364691in}{1.437042in}}%
\pgfpathlineto{\pgfqpoint{7.414363in}{1.435213in}}%
\pgfpathlineto{\pgfqpoint{7.500461in}{1.431373in}}%
\pgfpathlineto{\pgfqpoint{7.636230in}{1.429890in}}%
\pgfpathlineto{\pgfqpoint{7.692525in}{1.428958in}}%
\pgfpathlineto{\pgfqpoint{7.785245in}{1.431115in}}%
\pgfpathlineto{\pgfqpoint{7.871343in}{1.432161in}}%
\pgfpathlineto{\pgfqpoint{7.927638in}{1.430243in}}%
\pgfpathlineto{\pgfqpoint{8.023670in}{1.426353in}}%
\pgfpathlineto{\pgfqpoint{8.070030in}{1.428788in}}%
\pgfpathlineto{\pgfqpoint{8.205799in}{1.436870in}}%
\pgfpathlineto{\pgfqpoint{8.272028in}{1.437173in}}%
\pgfpathlineto{\pgfqpoint{8.344880in}{1.438110in}}%
\pgfpathlineto{\pgfqpoint{8.447535in}{1.440306in}}%
\pgfpathlineto{\pgfqpoint{8.493896in}{1.438433in}}%
\pgfpathlineto{\pgfqpoint{8.556813in}{1.433234in}}%
\pgfpathlineto{\pgfqpoint{8.616419in}{1.429015in}}%
\pgfpathlineto{\pgfqpoint{8.656157in}{1.428500in}}%
\pgfpathlineto{\pgfqpoint{8.709140in}{1.430295in}}%
\pgfpathlineto{\pgfqpoint{8.781992in}{1.432816in}}%
\pgfpathlineto{\pgfqpoint{8.821729in}{1.431576in}}%
\pgfpathlineto{\pgfqpoint{8.894581in}{1.427981in}}%
\pgfpathlineto{\pgfqpoint{9.000547in}{1.431734in}}%
\pgfpathlineto{\pgfqpoint{9.099891in}{1.431815in}}%
\pgfpathlineto{\pgfqpoint{9.146251in}{1.434487in}}%
\pgfpathlineto{\pgfqpoint{9.275398in}{1.443751in}}%
\pgfpathlineto{\pgfqpoint{9.315135in}{1.442710in}}%
\pgfpathlineto{\pgfqpoint{9.457528in}{1.436036in}}%
\pgfpathlineto{\pgfqpoint{9.507199in}{1.437660in}}%
\pgfpathlineto{\pgfqpoint{9.609854in}{1.441897in}}%
\pgfpathlineto{\pgfqpoint{9.652903in}{1.440852in}}%
\pgfpathlineto{\pgfqpoint{9.848279in}{1.432531in}}%
\pgfpathlineto{\pgfqpoint{9.917819in}{1.432432in}}%
\pgfpathlineto{\pgfqpoint{9.954245in}{1.434640in}}%
\pgfpathlineto{\pgfqpoint{10.076769in}{1.444352in}}%
\pgfpathlineto{\pgfqpoint{10.119818in}{1.443918in}}%
\pgfpathlineto{\pgfqpoint{10.255587in}{1.440188in}}%
\pgfpathlineto{\pgfqpoint{10.407914in}{1.442081in}}%
\pgfpathlineto{\pgfqpoint{10.474143in}{1.438397in}}%
\pgfpathlineto{\pgfqpoint{10.576798in}{1.432726in}}%
\pgfpathlineto{\pgfqpoint{10.669518in}{1.430354in}}%
\pgfpathlineto{\pgfqpoint{10.719190in}{1.430781in}}%
\pgfpathlineto{\pgfqpoint{10.765550in}{1.433722in}}%
\pgfpathlineto{\pgfqpoint{10.848336in}{1.439371in}}%
\pgfpathlineto{\pgfqpoint{10.894697in}{1.440025in}}%
\pgfpathlineto{\pgfqpoint{10.954303in}{1.438310in}}%
\pgfpathlineto{\pgfqpoint{11.043712in}{1.435843in}}%
\pgfpathlineto{\pgfqpoint{11.229153in}{1.433459in}}%
\pgfpathlineto{\pgfqpoint{11.302005in}{1.431269in}}%
\pgfpathlineto{\pgfqpoint{11.374857in}{1.431982in}}%
\pgfpathlineto{\pgfqpoint{11.487446in}{1.432537in}}%
\pgfpathlineto{\pgfqpoint{11.547052in}{1.434116in}}%
\pgfpathlineto{\pgfqpoint{11.633150in}{1.436939in}}%
\pgfpathlineto{\pgfqpoint{11.679510in}{1.435854in}}%
\pgfpathlineto{\pgfqpoint{11.739117in}{1.431961in}}%
\pgfpathlineto{\pgfqpoint{11.808657in}{1.427828in}}%
\pgfpathlineto{\pgfqpoint{11.881509in}{1.426304in}}%
\pgfpathlineto{\pgfqpoint{11.951049in}{1.427248in}}%
\pgfpathlineto{\pgfqpoint{12.063639in}{1.428006in}}%
\pgfpathlineto{\pgfqpoint{12.146425in}{1.426740in}}%
\pgfpathlineto{\pgfqpoint{12.159671in}{1.426472in}}%
\pgfpathlineto{\pgfqpoint{12.159671in}{1.426472in}}%
\pgfusepath{stroke}%
\end{pgfscope}%
\begin{pgfscope}%
\pgfpathrectangle{\pgfqpoint{2.400000in}{1.081300in}}{\pgfqpoint{14.880000in}{7.569100in}}%
\pgfusepath{clip}%
\pgfsetrectcap%
\pgfsetroundjoin%
\pgfsetlinewidth{1.505625pt}%
\definecolor{currentstroke}{rgb}{0.737255,0.741176,0.133333}%
\pgfsetstrokecolor{currentstroke}%
\pgfsetdash{}{0pt}%
\pgfpathmoveto{\pgfqpoint{3.076364in}{1.425350in}}%
\pgfpathlineto{\pgfqpoint{3.126035in}{1.543844in}}%
\pgfpathlineto{\pgfqpoint{3.142593in}{1.576833in}}%
\pgfpathlineto{\pgfqpoint{3.152527in}{1.592840in}}%
\pgfpathlineto{\pgfqpoint{3.162461in}{1.604950in}}%
\pgfpathlineto{\pgfqpoint{3.169084in}{1.610442in}}%
\pgfpathlineto{\pgfqpoint{3.175707in}{1.613629in}}%
\pgfpathlineto{\pgfqpoint{3.182330in}{1.614375in}}%
\pgfpathlineto{\pgfqpoint{3.188953in}{1.612632in}}%
\pgfpathlineto{\pgfqpoint{3.195576in}{1.608443in}}%
\pgfpathlineto{\pgfqpoint{3.202199in}{1.601939in}}%
\pgfpathlineto{\pgfqpoint{3.212133in}{1.588301in}}%
\pgfpathlineto{\pgfqpoint{3.222067in}{1.570797in}}%
\pgfpathlineto{\pgfqpoint{3.235313in}{1.543130in}}%
\pgfpathlineto{\pgfqpoint{3.268428in}{1.469725in}}%
\pgfpathlineto{\pgfqpoint{3.275051in}{1.460878in}}%
\pgfpathlineto{\pgfqpoint{3.278362in}{1.459071in}}%
\pgfpathlineto{\pgfqpoint{3.281674in}{1.459434in}}%
\pgfpathlineto{\pgfqpoint{3.284985in}{1.461856in}}%
\pgfpathlineto{\pgfqpoint{3.291608in}{1.471095in}}%
\pgfpathlineto{\pgfqpoint{3.304854in}{1.496461in}}%
\pgfpathlineto{\pgfqpoint{3.321411in}{1.527448in}}%
\pgfpathlineto{\pgfqpoint{3.331345in}{1.542324in}}%
\pgfpathlineto{\pgfqpoint{3.341280in}{1.552975in}}%
\pgfpathlineto{\pgfqpoint{3.347903in}{1.557312in}}%
\pgfpathlineto{\pgfqpoint{3.354525in}{1.559309in}}%
\pgfpathlineto{\pgfqpoint{3.361148in}{1.559074in}}%
\pgfpathlineto{\pgfqpoint{3.371083in}{1.555576in}}%
\pgfpathlineto{\pgfqpoint{3.381017in}{1.551547in}}%
\pgfpathlineto{\pgfqpoint{3.387640in}{1.551561in}}%
\pgfpathlineto{\pgfqpoint{3.390951in}{1.553212in}}%
\pgfpathlineto{\pgfqpoint{3.394263in}{1.556303in}}%
\pgfpathlineto{\pgfqpoint{3.400886in}{1.567467in}}%
\pgfpathlineto{\pgfqpoint{3.407509in}{1.585577in}}%
\pgfpathlineto{\pgfqpoint{3.414132in}{1.610035in}}%
\pgfpathlineto{\pgfqpoint{3.424066in}{1.656046in}}%
\pgfpathlineto{\pgfqpoint{3.440623in}{1.747821in}}%
\pgfpathlineto{\pgfqpoint{3.477049in}{1.957374in}}%
\pgfpathlineto{\pgfqpoint{3.490295in}{2.020160in}}%
\pgfpathlineto{\pgfqpoint{3.500229in}{2.058188in}}%
\pgfpathlineto{\pgfqpoint{3.510164in}{2.086659in}}%
\pgfpathlineto{\pgfqpoint{3.516786in}{2.099670in}}%
\pgfpathlineto{\pgfqpoint{3.523409in}{2.107556in}}%
\pgfpathlineto{\pgfqpoint{3.526721in}{2.109511in}}%
\pgfpathlineto{\pgfqpoint{3.530032in}{2.110118in}}%
\pgfpathlineto{\pgfqpoint{3.533344in}{2.109368in}}%
\pgfpathlineto{\pgfqpoint{3.536655in}{2.107258in}}%
\pgfpathlineto{\pgfqpoint{3.543278in}{2.098981in}}%
\pgfpathlineto{\pgfqpoint{3.549901in}{2.085386in}}%
\pgfpathlineto{\pgfqpoint{3.556524in}{2.066659in}}%
\pgfpathlineto{\pgfqpoint{3.566458in}{2.029519in}}%
\pgfpathlineto{\pgfqpoint{3.576393in}{1.982507in}}%
\pgfpathlineto{\pgfqpoint{3.589638in}{1.906668in}}%
\pgfpathlineto{\pgfqpoint{3.606196in}{1.795492in}}%
\pgfpathlineto{\pgfqpoint{3.639310in}{1.562784in}}%
\pgfpathlineto{\pgfqpoint{3.642622in}{1.547634in}}%
\pgfpathlineto{\pgfqpoint{3.645933in}{1.537948in}}%
\pgfpathlineto{\pgfqpoint{3.649244in}{1.535305in}}%
\pgfpathlineto{\pgfqpoint{3.652556in}{1.540261in}}%
\pgfpathlineto{\pgfqpoint{3.655867in}{1.551946in}}%
\pgfpathlineto{\pgfqpoint{3.662490in}{1.588907in}}%
\pgfpathlineto{\pgfqpoint{3.675736in}{1.685121in}}%
\pgfpathlineto{\pgfqpoint{3.698916in}{1.856431in}}%
\pgfpathlineto{\pgfqpoint{3.712162in}{1.939064in}}%
\pgfpathlineto{\pgfqpoint{3.722096in}{1.989129in}}%
\pgfpathlineto{\pgfqpoint{3.732031in}{2.026843in}}%
\pgfpathlineto{\pgfqpoint{3.738654in}{2.044396in}}%
\pgfpathlineto{\pgfqpoint{3.745277in}{2.055519in}}%
\pgfpathlineto{\pgfqpoint{3.748588in}{2.058606in}}%
\pgfpathlineto{\pgfqpoint{3.751899in}{2.060026in}}%
\pgfpathlineto{\pgfqpoint{3.755211in}{2.059771in}}%
\pgfpathlineto{\pgfqpoint{3.758522in}{2.057842in}}%
\pgfpathlineto{\pgfqpoint{3.761834in}{2.054247in}}%
\pgfpathlineto{\pgfqpoint{3.768457in}{2.042124in}}%
\pgfpathlineto{\pgfqpoint{3.775080in}{2.023595in}}%
\pgfpathlineto{\pgfqpoint{3.781702in}{1.998971in}}%
\pgfpathlineto{\pgfqpoint{3.791637in}{1.951602in}}%
\pgfpathlineto{\pgfqpoint{3.804883in}{1.872643in}}%
\pgfpathlineto{\pgfqpoint{3.828063in}{1.723914in}}%
\pgfpathlineto{\pgfqpoint{3.834686in}{1.694305in}}%
\pgfpathlineto{\pgfqpoint{3.837997in}{1.685050in}}%
\pgfpathlineto{\pgfqpoint{3.841309in}{1.680491in}}%
\pgfpathlineto{\pgfqpoint{3.844620in}{1.681116in}}%
\pgfpathlineto{\pgfqpoint{3.847931in}{1.687066in}}%
\pgfpathlineto{\pgfqpoint{3.851243in}{1.698103in}}%
\pgfpathlineto{\pgfqpoint{3.857866in}{1.733120in}}%
\pgfpathlineto{\pgfqpoint{3.867800in}{1.807219in}}%
\pgfpathlineto{\pgfqpoint{3.904226in}{2.105951in}}%
\pgfpathlineto{\pgfqpoint{3.914160in}{2.165696in}}%
\pgfpathlineto{\pgfqpoint{3.924095in}{2.209970in}}%
\pgfpathlineto{\pgfqpoint{3.930718in}{2.230225in}}%
\pgfpathlineto{\pgfqpoint{3.937341in}{2.242861in}}%
\pgfpathlineto{\pgfqpoint{3.940652in}{2.246312in}}%
\pgfpathlineto{\pgfqpoint{3.943964in}{2.247862in}}%
\pgfpathlineto{\pgfqpoint{3.947275in}{2.247527in}}%
\pgfpathlineto{\pgfqpoint{3.950586in}{2.245330in}}%
\pgfpathlineto{\pgfqpoint{3.953898in}{2.241299in}}%
\pgfpathlineto{\pgfqpoint{3.960521in}{2.227870in}}%
\pgfpathlineto{\pgfqpoint{3.967144in}{2.207551in}}%
\pgfpathlineto{\pgfqpoint{3.973767in}{2.180711in}}%
\pgfpathlineto{\pgfqpoint{3.983701in}{2.129145in}}%
\pgfpathlineto{\pgfqpoint{3.993635in}{2.065443in}}%
\pgfpathlineto{\pgfqpoint{4.006881in}{1.964875in}}%
\pgfpathlineto{\pgfqpoint{4.026750in}{1.792783in}}%
\pgfpathlineto{\pgfqpoint{4.043307in}{1.654746in}}%
\pgfpathlineto{\pgfqpoint{4.049930in}{1.614701in}}%
\pgfpathlineto{\pgfqpoint{4.053241in}{1.601914in}}%
\pgfpathlineto{\pgfqpoint{4.056553in}{1.595561in}}%
\pgfpathlineto{\pgfqpoint{4.059864in}{1.596283in}}%
\pgfpathlineto{\pgfqpoint{4.063176in}{1.603846in}}%
\pgfpathlineto{\pgfqpoint{4.069799in}{1.635130in}}%
\pgfpathlineto{\pgfqpoint{4.079733in}{1.703820in}}%
\pgfpathlineto{\pgfqpoint{4.106225in}{1.896569in}}%
\pgfpathlineto{\pgfqpoint{4.116159in}{1.954508in}}%
\pgfpathlineto{\pgfqpoint{4.126093in}{2.000598in}}%
\pgfpathlineto{\pgfqpoint{4.132716in}{2.024123in}}%
\pgfpathlineto{\pgfqpoint{4.139339in}{2.041619in}}%
\pgfpathlineto{\pgfqpoint{4.145962in}{2.052968in}}%
\pgfpathlineto{\pgfqpoint{4.152585in}{2.058143in}}%
\pgfpathlineto{\pgfqpoint{4.155896in}{2.058432in}}%
\pgfpathlineto{\pgfqpoint{4.159208in}{2.057208in}}%
\pgfpathlineto{\pgfqpoint{4.162519in}{2.054495in}}%
\pgfpathlineto{\pgfqpoint{4.169142in}{2.044725in}}%
\pgfpathlineto{\pgfqpoint{4.175765in}{2.029440in}}%
\pgfpathlineto{\pgfqpoint{4.182388in}{2.009084in}}%
\pgfpathlineto{\pgfqpoint{4.192322in}{1.970373in}}%
\pgfpathlineto{\pgfqpoint{4.205568in}{1.908023in}}%
\pgfpathlineto{\pgfqpoint{4.225437in}{1.812816in}}%
\pgfpathlineto{\pgfqpoint{4.232060in}{1.787894in}}%
\pgfpathlineto{\pgfqpoint{4.238683in}{1.769823in}}%
\pgfpathlineto{\pgfqpoint{4.245305in}{1.760208in}}%
\pgfpathlineto{\pgfqpoint{4.248617in}{1.758819in}}%
\pgfpathlineto{\pgfqpoint{4.251928in}{1.759689in}}%
\pgfpathlineto{\pgfqpoint{4.255240in}{1.762706in}}%
\pgfpathlineto{\pgfqpoint{4.261863in}{1.774412in}}%
\pgfpathlineto{\pgfqpoint{4.268486in}{1.792005in}}%
\pgfpathlineto{\pgfqpoint{4.281731in}{1.836302in}}%
\pgfpathlineto{\pgfqpoint{4.298289in}{1.891067in}}%
\pgfpathlineto{\pgfqpoint{4.308223in}{1.916967in}}%
\pgfpathlineto{\pgfqpoint{4.314846in}{1.930069in}}%
\pgfpathlineto{\pgfqpoint{4.321469in}{1.939481in}}%
\pgfpathlineto{\pgfqpoint{4.328092in}{1.945060in}}%
\pgfpathlineto{\pgfqpoint{4.334715in}{1.946773in}}%
\pgfpathlineto{\pgfqpoint{4.341337in}{1.944680in}}%
\pgfpathlineto{\pgfqpoint{4.347960in}{1.938915in}}%
\pgfpathlineto{\pgfqpoint{4.354583in}{1.929675in}}%
\pgfpathlineto{\pgfqpoint{4.361206in}{1.917210in}}%
\pgfpathlineto{\pgfqpoint{4.371141in}{1.893131in}}%
\pgfpathlineto{\pgfqpoint{4.384386in}{1.852850in}}%
\pgfpathlineto{\pgfqpoint{4.400944in}{1.794003in}}%
\pgfpathlineto{\pgfqpoint{4.430747in}{1.686338in}}%
\pgfpathlineto{\pgfqpoint{4.443992in}{1.647011in}}%
\pgfpathlineto{\pgfqpoint{4.453927in}{1.623658in}}%
\pgfpathlineto{\pgfqpoint{4.463861in}{1.606128in}}%
\pgfpathlineto{\pgfqpoint{4.473795in}{1.594008in}}%
\pgfpathlineto{\pgfqpoint{4.483730in}{1.586094in}}%
\pgfpathlineto{\pgfqpoint{4.496976in}{1.579377in}}%
\pgfpathlineto{\pgfqpoint{4.516844in}{1.572241in}}%
\pgfpathlineto{\pgfqpoint{4.526779in}{1.569998in}}%
\pgfpathlineto{\pgfqpoint{4.533402in}{1.569832in}}%
\pgfpathlineto{\pgfqpoint{4.540024in}{1.571453in}}%
\pgfpathlineto{\pgfqpoint{4.546647in}{1.575508in}}%
\pgfpathlineto{\pgfqpoint{4.553270in}{1.582512in}}%
\pgfpathlineto{\pgfqpoint{4.559893in}{1.592722in}}%
\pgfpathlineto{\pgfqpoint{4.569828in}{1.613853in}}%
\pgfpathlineto{\pgfqpoint{4.579762in}{1.640746in}}%
\pgfpathlineto{\pgfqpoint{4.599631in}{1.703537in}}%
\pgfpathlineto{\pgfqpoint{4.619499in}{1.764068in}}%
\pgfpathlineto{\pgfqpoint{4.629434in}{1.788900in}}%
\pgfpathlineto{\pgfqpoint{4.639368in}{1.808262in}}%
\pgfpathlineto{\pgfqpoint{4.645991in}{1.817645in}}%
\pgfpathlineto{\pgfqpoint{4.652614in}{1.823970in}}%
\pgfpathlineto{\pgfqpoint{4.659237in}{1.827114in}}%
\pgfpathlineto{\pgfqpoint{4.665860in}{1.827024in}}%
\pgfpathlineto{\pgfqpoint{4.672482in}{1.823719in}}%
\pgfpathlineto{\pgfqpoint{4.679105in}{1.817284in}}%
\pgfpathlineto{\pgfqpoint{4.685728in}{1.807876in}}%
\pgfpathlineto{\pgfqpoint{4.695663in}{1.788690in}}%
\pgfpathlineto{\pgfqpoint{4.705597in}{1.764346in}}%
\pgfpathlineto{\pgfqpoint{4.722154in}{1.716059in}}%
\pgfpathlineto{\pgfqpoint{4.748646in}{1.637321in}}%
\pgfpathlineto{\pgfqpoint{4.758580in}{1.614040in}}%
\pgfpathlineto{\pgfqpoint{4.768515in}{1.596969in}}%
\pgfpathlineto{\pgfqpoint{4.775137in}{1.589393in}}%
\pgfpathlineto{\pgfqpoint{4.781760in}{1.584693in}}%
\pgfpathlineto{\pgfqpoint{4.788383in}{1.582407in}}%
\pgfpathlineto{\pgfqpoint{4.798318in}{1.582058in}}%
\pgfpathlineto{\pgfqpoint{4.824809in}{1.583748in}}%
\pgfpathlineto{\pgfqpoint{4.834744in}{1.581327in}}%
\pgfpathlineto{\pgfqpoint{4.844678in}{1.576648in}}%
\pgfpathlineto{\pgfqpoint{4.871169in}{1.561160in}}%
\pgfpathlineto{\pgfqpoint{4.877792in}{1.560067in}}%
\pgfpathlineto{\pgfqpoint{4.884415in}{1.561715in}}%
\pgfpathlineto{\pgfqpoint{4.891038in}{1.566707in}}%
\pgfpathlineto{\pgfqpoint{4.897661in}{1.575259in}}%
\pgfpathlineto{\pgfqpoint{4.904284in}{1.587173in}}%
\pgfpathlineto{\pgfqpoint{4.914218in}{1.610153in}}%
\pgfpathlineto{\pgfqpoint{4.930776in}{1.656091in}}%
\pgfpathlineto{\pgfqpoint{4.950644in}{1.710579in}}%
\pgfpathlineto{\pgfqpoint{4.960579in}{1.733091in}}%
\pgfpathlineto{\pgfqpoint{4.970513in}{1.750492in}}%
\pgfpathlineto{\pgfqpoint{4.977136in}{1.758711in}}%
\pgfpathlineto{\pgfqpoint{4.983759in}{1.763939in}}%
\pgfpathlineto{\pgfqpoint{4.990382in}{1.766001in}}%
\pgfpathlineto{\pgfqpoint{4.997005in}{1.764783in}}%
\pgfpathlineto{\pgfqpoint{5.003627in}{1.760233in}}%
\pgfpathlineto{\pgfqpoint{5.010250in}{1.752361in}}%
\pgfpathlineto{\pgfqpoint{5.016873in}{1.741233in}}%
\pgfpathlineto{\pgfqpoint{5.026808in}{1.718728in}}%
\pgfpathlineto{\pgfqpoint{5.036742in}{1.689880in}}%
\pgfpathlineto{\pgfqpoint{5.049988in}{1.643385in}}%
\pgfpathlineto{\pgfqpoint{5.083102in}{1.520795in}}%
\pgfpathlineto{\pgfqpoint{5.089725in}{1.508956in}}%
\pgfpathlineto{\pgfqpoint{5.093037in}{1.507308in}}%
\pgfpathlineto{\pgfqpoint{5.096348in}{1.508741in}}%
\pgfpathlineto{\pgfqpoint{5.099660in}{1.512986in}}%
\pgfpathlineto{\pgfqpoint{5.106282in}{1.527743in}}%
\pgfpathlineto{\pgfqpoint{5.119528in}{1.568059in}}%
\pgfpathlineto{\pgfqpoint{5.132774in}{1.607552in}}%
\pgfpathlineto{\pgfqpoint{5.142708in}{1.631740in}}%
\pgfpathlineto{\pgfqpoint{5.152643in}{1.649620in}}%
\pgfpathlineto{\pgfqpoint{5.159266in}{1.657723in}}%
\pgfpathlineto{\pgfqpoint{5.165889in}{1.662745in}}%
\pgfpathlineto{\pgfqpoint{5.172511in}{1.664803in}}%
\pgfpathlineto{\pgfqpoint{5.179134in}{1.664146in}}%
\pgfpathlineto{\pgfqpoint{5.185757in}{1.661168in}}%
\pgfpathlineto{\pgfqpoint{5.195692in}{1.653609in}}%
\pgfpathlineto{\pgfqpoint{5.212249in}{1.639777in}}%
\pgfpathlineto{\pgfqpoint{5.218872in}{1.636919in}}%
\pgfpathlineto{\pgfqpoint{5.225495in}{1.637241in}}%
\pgfpathlineto{\pgfqpoint{5.232118in}{1.641538in}}%
\pgfpathlineto{\pgfqpoint{5.238740in}{1.650137in}}%
\pgfpathlineto{\pgfqpoint{5.245363in}{1.662832in}}%
\pgfpathlineto{\pgfqpoint{5.255298in}{1.688053in}}%
\pgfpathlineto{\pgfqpoint{5.275166in}{1.748687in}}%
\pgfpathlineto{\pgfqpoint{5.288412in}{1.786442in}}%
\pgfpathlineto{\pgfqpoint{5.298347in}{1.809094in}}%
\pgfpathlineto{\pgfqpoint{5.304969in}{1.820496in}}%
\pgfpathlineto{\pgfqpoint{5.311592in}{1.828497in}}%
\pgfpathlineto{\pgfqpoint{5.318215in}{1.832845in}}%
\pgfpathlineto{\pgfqpoint{5.324838in}{1.833391in}}%
\pgfpathlineto{\pgfqpoint{5.331461in}{1.830096in}}%
\pgfpathlineto{\pgfqpoint{5.338084in}{1.823019in}}%
\pgfpathlineto{\pgfqpoint{5.344707in}{1.812324in}}%
\pgfpathlineto{\pgfqpoint{5.354641in}{1.790109in}}%
\pgfpathlineto{\pgfqpoint{5.364575in}{1.761754in}}%
\pgfpathlineto{\pgfqpoint{5.401001in}{1.649317in}}%
\pgfpathlineto{\pgfqpoint{5.407624in}{1.638973in}}%
\pgfpathlineto{\pgfqpoint{5.410936in}{1.636377in}}%
\pgfpathlineto{\pgfqpoint{5.414247in}{1.635662in}}%
\pgfpathlineto{\pgfqpoint{5.417559in}{1.636865in}}%
\pgfpathlineto{\pgfqpoint{5.420870in}{1.639953in}}%
\pgfpathlineto{\pgfqpoint{5.427493in}{1.651298in}}%
\pgfpathlineto{\pgfqpoint{5.434116in}{1.668282in}}%
\pgfpathlineto{\pgfqpoint{5.444050in}{1.700493in}}%
\pgfpathlineto{\pgfqpoint{5.470542in}{1.791381in}}%
\pgfpathlineto{\pgfqpoint{5.480476in}{1.817870in}}%
\pgfpathlineto{\pgfqpoint{5.487099in}{1.831501in}}%
\pgfpathlineto{\pgfqpoint{5.493722in}{1.841422in}}%
\pgfpathlineto{\pgfqpoint{5.500345in}{1.847320in}}%
\pgfpathlineto{\pgfqpoint{5.506968in}{1.848964in}}%
\pgfpathlineto{\pgfqpoint{5.513591in}{1.846213in}}%
\pgfpathlineto{\pgfqpoint{5.520214in}{1.839015in}}%
\pgfpathlineto{\pgfqpoint{5.526837in}{1.827417in}}%
\pgfpathlineto{\pgfqpoint{5.533459in}{1.811584in}}%
\pgfpathlineto{\pgfqpoint{5.543394in}{1.780613in}}%
\pgfpathlineto{\pgfqpoint{5.556640in}{1.729091in}}%
\pgfpathlineto{\pgfqpoint{5.573197in}{1.662778in}}%
\pgfpathlineto{\pgfqpoint{5.579820in}{1.643542in}}%
\pgfpathlineto{\pgfqpoint{5.583131in}{1.637325in}}%
\pgfpathlineto{\pgfqpoint{5.586443in}{1.634018in}}%
\pgfpathlineto{\pgfqpoint{5.589754in}{1.633982in}}%
\pgfpathlineto{\pgfqpoint{5.593066in}{1.637405in}}%
\pgfpathlineto{\pgfqpoint{5.596377in}{1.644268in}}%
\pgfpathlineto{\pgfqpoint{5.603000in}{1.667342in}}%
\pgfpathlineto{\pgfqpoint{5.609623in}{1.700263in}}%
\pgfpathlineto{\pgfqpoint{5.622869in}{1.782788in}}%
\pgfpathlineto{\pgfqpoint{5.649360in}{1.955510in}}%
\pgfpathlineto{\pgfqpoint{5.662606in}{2.026135in}}%
\pgfpathlineto{\pgfqpoint{5.672540in}{2.067497in}}%
\pgfpathlineto{\pgfqpoint{5.679163in}{2.088727in}}%
\pgfpathlineto{\pgfqpoint{5.685786in}{2.104547in}}%
\pgfpathlineto{\pgfqpoint{5.692409in}{2.114777in}}%
\pgfpathlineto{\pgfqpoint{5.699032in}{2.119330in}}%
\pgfpathlineto{\pgfqpoint{5.702343in}{2.119475in}}%
\pgfpathlineto{\pgfqpoint{5.705655in}{2.118205in}}%
\pgfpathlineto{\pgfqpoint{5.712278in}{2.111477in}}%
\pgfpathlineto{\pgfqpoint{5.718901in}{2.099299in}}%
\pgfpathlineto{\pgfqpoint{5.725524in}{2.081893in}}%
\pgfpathlineto{\pgfqpoint{5.735458in}{2.046639in}}%
\pgfpathlineto{\pgfqpoint{5.745392in}{2.001572in}}%
\pgfpathlineto{\pgfqpoint{5.758638in}{1.929130in}}%
\pgfpathlineto{\pgfqpoint{5.781818in}{1.784329in}}%
\pgfpathlineto{\pgfqpoint{5.795064in}{1.706383in}}%
\pgfpathlineto{\pgfqpoint{5.804998in}{1.661328in}}%
\pgfpathlineto{\pgfqpoint{5.811621in}{1.642456in}}%
\pgfpathlineto{\pgfqpoint{5.814933in}{1.637203in}}%
\pgfpathlineto{\pgfqpoint{5.818244in}{1.634893in}}%
\pgfpathlineto{\pgfqpoint{5.821556in}{1.635459in}}%
\pgfpathlineto{\pgfqpoint{5.824867in}{1.638685in}}%
\pgfpathlineto{\pgfqpoint{5.831490in}{1.651715in}}%
\pgfpathlineto{\pgfqpoint{5.841424in}{1.681528in}}%
\pgfpathlineto{\pgfqpoint{5.861293in}{1.745531in}}%
\pgfpathlineto{\pgfqpoint{5.871227in}{1.769067in}}%
\pgfpathlineto{\pgfqpoint{5.877850in}{1.779693in}}%
\pgfpathlineto{\pgfqpoint{5.884473in}{1.785776in}}%
\pgfpathlineto{\pgfqpoint{5.887785in}{1.787037in}}%
\pgfpathlineto{\pgfqpoint{5.891096in}{1.787091in}}%
\pgfpathlineto{\pgfqpoint{5.894407in}{1.785934in}}%
\pgfpathlineto{\pgfqpoint{5.901030in}{1.780010in}}%
\pgfpathlineto{\pgfqpoint{5.907653in}{1.769404in}}%
\pgfpathlineto{\pgfqpoint{5.914276in}{1.754395in}}%
\pgfpathlineto{\pgfqpoint{5.924211in}{1.724611in}}%
\pgfpathlineto{\pgfqpoint{5.937456in}{1.675084in}}%
\pgfpathlineto{\pgfqpoint{5.954014in}{1.611663in}}%
\pgfpathlineto{\pgfqpoint{5.960636in}{1.592883in}}%
\pgfpathlineto{\pgfqpoint{5.963948in}{1.586476in}}%
\pgfpathlineto{\pgfqpoint{5.967259in}{1.582577in}}%
\pgfpathlineto{\pgfqpoint{5.970571in}{1.581444in}}%
\pgfpathlineto{\pgfqpoint{5.973882in}{1.583162in}}%
\pgfpathlineto{\pgfqpoint{5.977194in}{1.587618in}}%
\pgfpathlineto{\pgfqpoint{5.983817in}{1.603528in}}%
\pgfpathlineto{\pgfqpoint{5.993751in}{1.638824in}}%
\pgfpathlineto{\pgfqpoint{6.020243in}{1.742082in}}%
\pgfpathlineto{\pgfqpoint{6.030177in}{1.771509in}}%
\pgfpathlineto{\pgfqpoint{6.036800in}{1.786499in}}%
\pgfpathlineto{\pgfqpoint{6.043423in}{1.797402in}}%
\pgfpathlineto{\pgfqpoint{6.050046in}{1.804018in}}%
\pgfpathlineto{\pgfqpoint{6.056669in}{1.806249in}}%
\pgfpathlineto{\pgfqpoint{6.063291in}{1.804087in}}%
\pgfpathlineto{\pgfqpoint{6.069914in}{1.797602in}}%
\pgfpathlineto{\pgfqpoint{6.076537in}{1.786940in}}%
\pgfpathlineto{\pgfqpoint{6.083160in}{1.772314in}}%
\pgfpathlineto{\pgfqpoint{6.093094in}{1.743567in}}%
\pgfpathlineto{\pgfqpoint{6.103029in}{1.707742in}}%
\pgfpathlineto{\pgfqpoint{6.116275in}{1.651612in}}%
\pgfpathlineto{\pgfqpoint{6.149389in}{1.503549in}}%
\pgfpathlineto{\pgfqpoint{6.156012in}{1.484306in}}%
\pgfpathlineto{\pgfqpoint{6.159323in}{1.479108in}}%
\pgfpathlineto{\pgfqpoint{6.162635in}{1.477684in}}%
\pgfpathlineto{\pgfqpoint{6.165946in}{1.479879in}}%
\pgfpathlineto{\pgfqpoint{6.172569in}{1.491488in}}%
\pgfpathlineto{\pgfqpoint{6.189127in}{1.527220in}}%
\pgfpathlineto{\pgfqpoint{6.195749in}{1.537029in}}%
\pgfpathlineto{\pgfqpoint{6.202372in}{1.542705in}}%
\pgfpathlineto{\pgfqpoint{6.205684in}{1.543844in}}%
\pgfpathlineto{\pgfqpoint{6.208995in}{1.543801in}}%
\pgfpathlineto{\pgfqpoint{6.212307in}{1.542560in}}%
\pgfpathlineto{\pgfqpoint{6.218930in}{1.536492in}}%
\pgfpathlineto{\pgfqpoint{6.225552in}{1.525887in}}%
\pgfpathlineto{\pgfqpoint{6.235487in}{1.503635in}}%
\pgfpathlineto{\pgfqpoint{6.242110in}{1.488979in}}%
\pgfpathlineto{\pgfqpoint{6.245421in}{1.484454in}}%
\pgfpathlineto{\pgfqpoint{6.248733in}{1.483809in}}%
\pgfpathlineto{\pgfqpoint{6.252044in}{1.488157in}}%
\pgfpathlineto{\pgfqpoint{6.255356in}{1.497467in}}%
\pgfpathlineto{\pgfqpoint{6.261978in}{1.527334in}}%
\pgfpathlineto{\pgfqpoint{6.271913in}{1.588740in}}%
\pgfpathlineto{\pgfqpoint{6.285159in}{1.686684in}}%
\pgfpathlineto{\pgfqpoint{6.334830in}{2.070600in}}%
\pgfpathlineto{\pgfqpoint{6.344765in}{2.129497in}}%
\pgfpathlineto{\pgfqpoint{6.354699in}{2.176892in}}%
\pgfpathlineto{\pgfqpoint{6.361322in}{2.201239in}}%
\pgfpathlineto{\pgfqpoint{6.367945in}{2.219300in}}%
\pgfpathlineto{\pgfqpoint{6.374568in}{2.230751in}}%
\pgfpathlineto{\pgfqpoint{6.377879in}{2.233920in}}%
\pgfpathlineto{\pgfqpoint{6.381191in}{2.235351in}}%
\pgfpathlineto{\pgfqpoint{6.384502in}{2.235028in}}%
\pgfpathlineto{\pgfqpoint{6.387814in}{2.232937in}}%
\pgfpathlineto{\pgfqpoint{6.391125in}{2.229071in}}%
\pgfpathlineto{\pgfqpoint{6.397748in}{2.216001in}}%
\pgfpathlineto{\pgfqpoint{6.404371in}{2.195846in}}%
\pgfpathlineto{\pgfqpoint{6.410994in}{2.168705in}}%
\pgfpathlineto{\pgfqpoint{6.420928in}{2.115294in}}%
\pgfpathlineto{\pgfqpoint{6.430862in}{2.047483in}}%
\pgfpathlineto{\pgfqpoint{6.444108in}{1.937125in}}%
\pgfpathlineto{\pgfqpoint{6.460665in}{1.775256in}}%
\pgfpathlineto{\pgfqpoint{6.477223in}{1.615151in}}%
\pgfpathlineto{\pgfqpoint{6.480534in}{1.592082in}}%
\pgfpathlineto{\pgfqpoint{6.483846in}{1.577024in}}%
\pgfpathlineto{\pgfqpoint{6.487157in}{1.572848in}}%
\pgfpathlineto{\pgfqpoint{6.490468in}{1.580696in}}%
\pgfpathlineto{\pgfqpoint{6.493780in}{1.599068in}}%
\pgfpathlineto{\pgfqpoint{6.500403in}{1.656120in}}%
\pgfpathlineto{\pgfqpoint{6.516960in}{1.838189in}}%
\pgfpathlineto{\pgfqpoint{6.536829in}{2.049085in}}%
\pgfpathlineto{\pgfqpoint{6.550075in}{2.163102in}}%
\pgfpathlineto{\pgfqpoint{6.560009in}{2.228506in}}%
\pgfpathlineto{\pgfqpoint{6.566632in}{2.260975in}}%
\pgfpathlineto{\pgfqpoint{6.573255in}{2.283772in}}%
\pgfpathlineto{\pgfqpoint{6.579878in}{2.296391in}}%
\pgfpathlineto{\pgfqpoint{6.583189in}{2.298765in}}%
\pgfpathlineto{\pgfqpoint{6.586501in}{2.298471in}}%
\pgfpathlineto{\pgfqpoint{6.589812in}{2.295488in}}%
\pgfpathlineto{\pgfqpoint{6.593123in}{2.289804in}}%
\pgfpathlineto{\pgfqpoint{6.599746in}{2.270340in}}%
\pgfpathlineto{\pgfqpoint{6.606369in}{2.240193in}}%
\pgfpathlineto{\pgfqpoint{6.612992in}{2.199647in}}%
\pgfpathlineto{\pgfqpoint{6.622926in}{2.120368in}}%
\pgfpathlineto{\pgfqpoint{6.632861in}{2.021017in}}%
\pgfpathlineto{\pgfqpoint{6.646107in}{1.863068in}}%
\pgfpathlineto{\pgfqpoint{6.669287in}{1.565800in}}%
\pgfpathlineto{\pgfqpoint{6.672598in}{1.540053in}}%
\pgfpathlineto{\pgfqpoint{6.675910in}{1.533741in}}%
\pgfpathlineto{\pgfqpoint{6.679221in}{1.550165in}}%
\pgfpathlineto{\pgfqpoint{6.685844in}{1.622774in}}%
\pgfpathlineto{\pgfqpoint{6.735516in}{2.273426in}}%
\pgfpathlineto{\pgfqpoint{6.748762in}{2.402208in}}%
\pgfpathlineto{\pgfqpoint{6.758696in}{2.480905in}}%
\pgfpathlineto{\pgfqpoint{6.768630in}{2.543361in}}%
\pgfpathlineto{\pgfqpoint{6.778565in}{2.589029in}}%
\pgfpathlineto{\pgfqpoint{6.785187in}{2.609963in}}%
\pgfpathlineto{\pgfqpoint{6.791810in}{2.623207in}}%
\pgfpathlineto{\pgfqpoint{6.795122in}{2.626932in}}%
\pgfpathlineto{\pgfqpoint{6.798433in}{2.628725in}}%
\pgfpathlineto{\pgfqpoint{6.801745in}{2.628584in}}%
\pgfpathlineto{\pgfqpoint{6.805056in}{2.626514in}}%
\pgfpathlineto{\pgfqpoint{6.808368in}{2.622518in}}%
\pgfpathlineto{\pgfqpoint{6.814991in}{2.608786in}}%
\pgfpathlineto{\pgfqpoint{6.821613in}{2.587483in}}%
\pgfpathlineto{\pgfqpoint{6.828236in}{2.558758in}}%
\pgfpathlineto{\pgfqpoint{6.838171in}{2.502222in}}%
\pgfpathlineto{\pgfqpoint{6.848105in}{2.430459in}}%
\pgfpathlineto{\pgfqpoint{6.861351in}{2.313576in}}%
\pgfpathlineto{\pgfqpoint{6.877908in}{2.140169in}}%
\pgfpathlineto{\pgfqpoint{6.920957in}{1.660991in}}%
\pgfpathlineto{\pgfqpoint{6.927580in}{1.614373in}}%
\pgfpathlineto{\pgfqpoint{6.930891in}{1.600410in}}%
\pgfpathlineto{\pgfqpoint{6.934203in}{1.594285in}}%
\pgfpathlineto{\pgfqpoint{6.937514in}{1.596232in}}%
\pgfpathlineto{\pgfqpoint{6.940826in}{1.605342in}}%
\pgfpathlineto{\pgfqpoint{6.947449in}{1.638351in}}%
\pgfpathlineto{\pgfqpoint{6.977252in}{1.820749in}}%
\pgfpathlineto{\pgfqpoint{6.987186in}{1.861433in}}%
\pgfpathlineto{\pgfqpoint{6.993809in}{1.880479in}}%
\pgfpathlineto{\pgfqpoint{7.000432in}{1.892977in}}%
\pgfpathlineto{\pgfqpoint{7.007055in}{1.899075in}}%
\pgfpathlineto{\pgfqpoint{7.010366in}{1.899814in}}%
\pgfpathlineto{\pgfqpoint{7.013678in}{1.899077in}}%
\pgfpathlineto{\pgfqpoint{7.016989in}{1.896924in}}%
\pgfpathlineto{\pgfqpoint{7.023612in}{1.888634in}}%
\pgfpathlineto{\pgfqpoint{7.030235in}{1.875530in}}%
\pgfpathlineto{\pgfqpoint{7.040169in}{1.848304in}}%
\pgfpathlineto{\pgfqpoint{7.053415in}{1.801650in}}%
\pgfpathlineto{\pgfqpoint{7.103087in}{1.612587in}}%
\pgfpathlineto{\pgfqpoint{7.113021in}{1.586905in}}%
\pgfpathlineto{\pgfqpoint{7.119644in}{1.573942in}}%
\pgfpathlineto{\pgfqpoint{7.126267in}{1.564520in}}%
\pgfpathlineto{\pgfqpoint{7.132890in}{1.558563in}}%
\pgfpathlineto{\pgfqpoint{7.139513in}{1.555732in}}%
\pgfpathlineto{\pgfqpoint{7.146136in}{1.555470in}}%
\pgfpathlineto{\pgfqpoint{7.156070in}{1.558447in}}%
\pgfpathlineto{\pgfqpoint{7.172627in}{1.567408in}}%
\pgfpathlineto{\pgfqpoint{7.192496in}{1.577448in}}%
\pgfpathlineto{\pgfqpoint{7.205742in}{1.581779in}}%
\pgfpathlineto{\pgfqpoint{7.222299in}{1.584783in}}%
\pgfpathlineto{\pgfqpoint{7.248790in}{1.589019in}}%
\pgfpathlineto{\pgfqpoint{7.262036in}{1.593903in}}%
\pgfpathlineto{\pgfqpoint{7.271971in}{1.599824in}}%
\pgfpathlineto{\pgfqpoint{7.281905in}{1.608098in}}%
\pgfpathlineto{\pgfqpoint{7.291839in}{1.619009in}}%
\pgfpathlineto{\pgfqpoint{7.301774in}{1.632811in}}%
\pgfpathlineto{\pgfqpoint{7.311708in}{1.649758in}}%
\pgfpathlineto{\pgfqpoint{7.324954in}{1.677628in}}%
\pgfpathlineto{\pgfqpoint{7.338200in}{1.711649in}}%
\pgfpathlineto{\pgfqpoint{7.354757in}{1.761649in}}%
\pgfpathlineto{\pgfqpoint{7.391183in}{1.876856in}}%
\pgfpathlineto{\pgfqpoint{7.401117in}{1.901002in}}%
\pgfpathlineto{\pgfqpoint{7.407740in}{1.913412in}}%
\pgfpathlineto{\pgfqpoint{7.414363in}{1.922222in}}%
\pgfpathlineto{\pgfqpoint{7.420986in}{1.926935in}}%
\pgfpathlineto{\pgfqpoint{7.424297in}{1.927616in}}%
\pgfpathlineto{\pgfqpoint{7.427609in}{1.927117in}}%
\pgfpathlineto{\pgfqpoint{7.430920in}{1.925394in}}%
\pgfpathlineto{\pgfqpoint{7.437543in}{1.918142in}}%
\pgfpathlineto{\pgfqpoint{7.444166in}{1.905640in}}%
\pgfpathlineto{\pgfqpoint{7.450789in}{1.887768in}}%
\pgfpathlineto{\pgfqpoint{7.457412in}{1.864513in}}%
\pgfpathlineto{\pgfqpoint{7.467346in}{1.819792in}}%
\pgfpathlineto{\pgfqpoint{7.477281in}{1.764127in}}%
\pgfpathlineto{\pgfqpoint{7.490526in}{1.676366in}}%
\pgfpathlineto{\pgfqpoint{7.507084in}{1.565038in}}%
\pgfpathlineto{\pgfqpoint{7.510395in}{1.549515in}}%
\pgfpathlineto{\pgfqpoint{7.513706in}{1.540294in}}%
\pgfpathlineto{\pgfqpoint{7.517018in}{1.539473in}}%
\pgfpathlineto{\pgfqpoint{7.520329in}{1.547670in}}%
\pgfpathlineto{\pgfqpoint{7.523641in}{1.563621in}}%
\pgfpathlineto{\pgfqpoint{7.530264in}{1.610718in}}%
\pgfpathlineto{\pgfqpoint{7.543510in}{1.730899in}}%
\pgfpathlineto{\pgfqpoint{7.579935in}{2.075118in}}%
\pgfpathlineto{\pgfqpoint{7.596493in}{2.206358in}}%
\pgfpathlineto{\pgfqpoint{7.609739in}{2.291543in}}%
\pgfpathlineto{\pgfqpoint{7.619673in}{2.341734in}}%
\pgfpathlineto{\pgfqpoint{7.629607in}{2.379011in}}%
\pgfpathlineto{\pgfqpoint{7.636230in}{2.396309in}}%
\pgfpathlineto{\pgfqpoint{7.642853in}{2.407389in}}%
\pgfpathlineto{\pgfqpoint{7.646164in}{2.410570in}}%
\pgfpathlineto{\pgfqpoint{7.649476in}{2.412173in}}%
\pgfpathlineto{\pgfqpoint{7.652787in}{2.412198in}}%
\pgfpathlineto{\pgfqpoint{7.656099in}{2.410651in}}%
\pgfpathlineto{\pgfqpoint{7.659410in}{2.407542in}}%
\pgfpathlineto{\pgfqpoint{7.666033in}{2.396699in}}%
\pgfpathlineto{\pgfqpoint{7.672656in}{2.379838in}}%
\pgfpathlineto{\pgfqpoint{7.679279in}{2.357204in}}%
\pgfpathlineto{\pgfqpoint{7.689213in}{2.313140in}}%
\pgfpathlineto{\pgfqpoint{7.699148in}{2.258199in}}%
\pgfpathlineto{\pgfqpoint{7.712393in}{2.171121in}}%
\pgfpathlineto{\pgfqpoint{7.732262in}{2.021982in}}%
\pgfpathlineto{\pgfqpoint{7.755442in}{1.848687in}}%
\pgfpathlineto{\pgfqpoint{7.765377in}{1.786113in}}%
\pgfpathlineto{\pgfqpoint{7.775311in}{1.737198in}}%
\pgfpathlineto{\pgfqpoint{7.781934in}{1.714189in}}%
\pgfpathlineto{\pgfqpoint{7.788557in}{1.699506in}}%
\pgfpathlineto{\pgfqpoint{7.791868in}{1.695186in}}%
\pgfpathlineto{\pgfqpoint{7.795180in}{1.692701in}}%
\pgfpathlineto{\pgfqpoint{7.798491in}{1.691850in}}%
\pgfpathlineto{\pgfqpoint{7.801803in}{1.692395in}}%
\pgfpathlineto{\pgfqpoint{7.808426in}{1.696628in}}%
\pgfpathlineto{\pgfqpoint{7.821671in}{1.710670in}}%
\pgfpathlineto{\pgfqpoint{7.831606in}{1.719858in}}%
\pgfpathlineto{\pgfqpoint{7.838229in}{1.723405in}}%
\pgfpathlineto{\pgfqpoint{7.844851in}{1.724251in}}%
\pgfpathlineto{\pgfqpoint{7.851474in}{1.722168in}}%
\pgfpathlineto{\pgfqpoint{7.858097in}{1.717170in}}%
\pgfpathlineto{\pgfqpoint{7.864720in}{1.709491in}}%
\pgfpathlineto{\pgfqpoint{7.874655in}{1.694005in}}%
\pgfpathlineto{\pgfqpoint{7.894523in}{1.659720in}}%
\pgfpathlineto{\pgfqpoint{7.901146in}{1.652117in}}%
\pgfpathlineto{\pgfqpoint{7.907769in}{1.649051in}}%
\pgfpathlineto{\pgfqpoint{7.911080in}{1.649649in}}%
\pgfpathlineto{\pgfqpoint{7.914392in}{1.651824in}}%
\pgfpathlineto{\pgfqpoint{7.921015in}{1.661077in}}%
\pgfpathlineto{\pgfqpoint{7.927638in}{1.676629in}}%
\pgfpathlineto{\pgfqpoint{7.934261in}{1.697628in}}%
\pgfpathlineto{\pgfqpoint{7.944195in}{1.736685in}}%
\pgfpathlineto{\pgfqpoint{7.964064in}{1.827245in}}%
\pgfpathlineto{\pgfqpoint{7.983932in}{1.914049in}}%
\pgfpathlineto{\pgfqpoint{7.997178in}{1.961770in}}%
\pgfpathlineto{\pgfqpoint{8.007112in}{1.990104in}}%
\pgfpathlineto{\pgfqpoint{8.017047in}{2.011238in}}%
\pgfpathlineto{\pgfqpoint{8.023670in}{2.021146in}}%
\pgfpathlineto{\pgfqpoint{8.030293in}{2.027681in}}%
\pgfpathlineto{\pgfqpoint{8.036916in}{2.030887in}}%
\pgfpathlineto{\pgfqpoint{8.043538in}{2.030864in}}%
\pgfpathlineto{\pgfqpoint{8.050161in}{2.027765in}}%
\pgfpathlineto{\pgfqpoint{8.056784in}{2.021787in}}%
\pgfpathlineto{\pgfqpoint{8.063407in}{2.013167in}}%
\pgfpathlineto{\pgfqpoint{8.073341in}{1.995877in}}%
\pgfpathlineto{\pgfqpoint{8.086587in}{1.966263in}}%
\pgfpathlineto{\pgfqpoint{8.103145in}{1.922267in}}%
\pgfpathlineto{\pgfqpoint{8.149505in}{1.794735in}}%
\pgfpathlineto{\pgfqpoint{8.162751in}{1.765188in}}%
\pgfpathlineto{\pgfqpoint{8.172685in}{1.746885in}}%
\pgfpathlineto{\pgfqpoint{8.182619in}{1.732674in}}%
\pgfpathlineto{\pgfqpoint{8.189242in}{1.725820in}}%
\pgfpathlineto{\pgfqpoint{8.195865in}{1.721268in}}%
\pgfpathlineto{\pgfqpoint{8.202488in}{1.719130in}}%
\pgfpathlineto{\pgfqpoint{8.209111in}{1.719431in}}%
\pgfpathlineto{\pgfqpoint{8.215734in}{1.722085in}}%
\pgfpathlineto{\pgfqpoint{8.222357in}{1.726886in}}%
\pgfpathlineto{\pgfqpoint{8.232291in}{1.737387in}}%
\pgfpathlineto{\pgfqpoint{8.248848in}{1.759864in}}%
\pgfpathlineto{\pgfqpoint{8.265406in}{1.781345in}}%
\pgfpathlineto{\pgfqpoint{8.275340in}{1.790301in}}%
\pgfpathlineto{\pgfqpoint{8.281963in}{1.793639in}}%
\pgfpathlineto{\pgfqpoint{8.288586in}{1.794399in}}%
\pgfpathlineto{\pgfqpoint{8.295209in}{1.792237in}}%
\pgfpathlineto{\pgfqpoint{8.301832in}{1.786878in}}%
\pgfpathlineto{\pgfqpoint{8.308454in}{1.778116in}}%
\pgfpathlineto{\pgfqpoint{8.315077in}{1.765821in}}%
\pgfpathlineto{\pgfqpoint{8.325012in}{1.740691in}}%
\pgfpathlineto{\pgfqpoint{8.334946in}{1.707827in}}%
\pgfpathlineto{\pgfqpoint{8.348192in}{1.653760in}}%
\pgfpathlineto{\pgfqpoint{8.371372in}{1.553478in}}%
\pgfpathlineto{\pgfqpoint{8.377995in}{1.537763in}}%
\pgfpathlineto{\pgfqpoint{8.381306in}{1.535706in}}%
\pgfpathlineto{\pgfqpoint{8.384618in}{1.538215in}}%
\pgfpathlineto{\pgfqpoint{8.387929in}{1.545145in}}%
\pgfpathlineto{\pgfqpoint{8.394552in}{1.569599in}}%
\pgfpathlineto{\pgfqpoint{8.404486in}{1.621622in}}%
\pgfpathlineto{\pgfqpoint{8.440912in}{1.829709in}}%
\pgfpathlineto{\pgfqpoint{8.454158in}{1.888502in}}%
\pgfpathlineto{\pgfqpoint{8.464093in}{1.922611in}}%
\pgfpathlineto{\pgfqpoint{8.470715in}{1.940031in}}%
\pgfpathlineto{\pgfqpoint{8.477338in}{1.952940in}}%
\pgfpathlineto{\pgfqpoint{8.483961in}{1.961185in}}%
\pgfpathlineto{\pgfqpoint{8.490584in}{1.964665in}}%
\pgfpathlineto{\pgfqpoint{8.493896in}{1.964600in}}%
\pgfpathlineto{\pgfqpoint{8.497207in}{1.963327in}}%
\pgfpathlineto{\pgfqpoint{8.503830in}{1.957169in}}%
\pgfpathlineto{\pgfqpoint{8.510453in}{1.946232in}}%
\pgfpathlineto{\pgfqpoint{8.517076in}{1.930606in}}%
\pgfpathlineto{\pgfqpoint{8.527010in}{1.898675in}}%
\pgfpathlineto{\pgfqpoint{8.536944in}{1.857137in}}%
\pgfpathlineto{\pgfqpoint{8.546879in}{1.806860in}}%
\pgfpathlineto{\pgfqpoint{8.560125in}{1.728197in}}%
\pgfpathlineto{\pgfqpoint{8.579993in}{1.592929in}}%
\pgfpathlineto{\pgfqpoint{8.596551in}{1.482986in}}%
\pgfpathlineto{\pgfqpoint{8.599862in}{1.470382in}}%
\pgfpathlineto{\pgfqpoint{8.603173in}{1.469612in}}%
\pgfpathlineto{\pgfqpoint{8.606485in}{1.480913in}}%
\pgfpathlineto{\pgfqpoint{8.616419in}{1.540000in}}%
\pgfpathlineto{\pgfqpoint{8.636288in}{1.660702in}}%
\pgfpathlineto{\pgfqpoint{8.646222in}{1.710148in}}%
\pgfpathlineto{\pgfqpoint{8.656157in}{1.749395in}}%
\pgfpathlineto{\pgfqpoint{8.662780in}{1.769201in}}%
\pgfpathlineto{\pgfqpoint{8.669402in}{1.783578in}}%
\pgfpathlineto{\pgfqpoint{8.676025in}{1.792341in}}%
\pgfpathlineto{\pgfqpoint{8.679337in}{1.794587in}}%
\pgfpathlineto{\pgfqpoint{8.682648in}{1.795404in}}%
\pgfpathlineto{\pgfqpoint{8.685960in}{1.794796in}}%
\pgfpathlineto{\pgfqpoint{8.689271in}{1.792775in}}%
\pgfpathlineto{\pgfqpoint{8.695894in}{1.784563in}}%
\pgfpathlineto{\pgfqpoint{8.702517in}{1.770967in}}%
\pgfpathlineto{\pgfqpoint{8.709140in}{1.752281in}}%
\pgfpathlineto{\pgfqpoint{8.719074in}{1.715581in}}%
\pgfpathlineto{\pgfqpoint{8.729009in}{1.670048in}}%
\pgfpathlineto{\pgfqpoint{8.745566in}{1.581563in}}%
\pgfpathlineto{\pgfqpoint{8.758812in}{1.514507in}}%
\pgfpathlineto{\pgfqpoint{8.762123in}{1.503203in}}%
\pgfpathlineto{\pgfqpoint{8.765435in}{1.496950in}}%
\pgfpathlineto{\pgfqpoint{8.768746in}{1.497120in}}%
\pgfpathlineto{\pgfqpoint{8.772057in}{1.503611in}}%
\pgfpathlineto{\pgfqpoint{8.778680in}{1.529318in}}%
\pgfpathlineto{\pgfqpoint{8.815106in}{1.703730in}}%
\pgfpathlineto{\pgfqpoint{8.825041in}{1.735543in}}%
\pgfpathlineto{\pgfqpoint{8.831664in}{1.750977in}}%
\pgfpathlineto{\pgfqpoint{8.838286in}{1.761494in}}%
\pgfpathlineto{\pgfqpoint{8.844909in}{1.766956in}}%
\pgfpathlineto{\pgfqpoint{8.848221in}{1.767773in}}%
\pgfpathlineto{\pgfqpoint{8.851532in}{1.767318in}}%
\pgfpathlineto{\pgfqpoint{8.858155in}{1.762628in}}%
\pgfpathlineto{\pgfqpoint{8.864778in}{1.753020in}}%
\pgfpathlineto{\pgfqpoint{8.871401in}{1.738708in}}%
\pgfpathlineto{\pgfqpoint{8.881335in}{1.709071in}}%
\pgfpathlineto{\pgfqpoint{8.891270in}{1.670786in}}%
\pgfpathlineto{\pgfqpoint{8.904515in}{1.609175in}}%
\pgfpathlineto{\pgfqpoint{8.931007in}{1.480768in}}%
\pgfpathlineto{\pgfqpoint{8.934318in}{1.474307in}}%
\pgfpathlineto{\pgfqpoint{8.937630in}{1.475330in}}%
\pgfpathlineto{\pgfqpoint{8.940941in}{1.483368in}}%
\pgfpathlineto{\pgfqpoint{8.947564in}{1.510873in}}%
\pgfpathlineto{\pgfqpoint{8.977367in}{1.650941in}}%
\pgfpathlineto{\pgfqpoint{8.987302in}{1.686095in}}%
\pgfpathlineto{\pgfqpoint{8.997236in}{1.711875in}}%
\pgfpathlineto{\pgfqpoint{9.003859in}{1.723143in}}%
\pgfpathlineto{\pgfqpoint{9.010482in}{1.729274in}}%
\pgfpathlineto{\pgfqpoint{9.013793in}{1.730326in}}%
\pgfpathlineto{\pgfqpoint{9.017105in}{1.729999in}}%
\pgfpathlineto{\pgfqpoint{9.020416in}{1.728271in}}%
\pgfpathlineto{\pgfqpoint{9.027039in}{1.720555in}}%
\pgfpathlineto{\pgfqpoint{9.033662in}{1.707129in}}%
\pgfpathlineto{\pgfqpoint{9.040285in}{1.688067in}}%
\pgfpathlineto{\pgfqpoint{9.050219in}{1.649509in}}%
\pgfpathlineto{\pgfqpoint{9.063465in}{1.583560in}}%
\pgfpathlineto{\pgfqpoint{9.073399in}{1.534367in}}%
\pgfpathlineto{\pgfqpoint{9.076711in}{1.523204in}}%
\pgfpathlineto{\pgfqpoint{9.080022in}{1.518106in}}%
\pgfpathlineto{\pgfqpoint{9.083334in}{1.520959in}}%
\pgfpathlineto{\pgfqpoint{9.086645in}{1.531883in}}%
\pgfpathlineto{\pgfqpoint{9.093268in}{1.571486in}}%
\pgfpathlineto{\pgfqpoint{9.103202in}{1.653481in}}%
\pgfpathlineto{\pgfqpoint{9.126383in}{1.874650in}}%
\pgfpathlineto{\pgfqpoint{9.146251in}{2.056831in}}%
\pgfpathlineto{\pgfqpoint{9.159497in}{2.160304in}}%
\pgfpathlineto{\pgfqpoint{9.169431in}{2.223912in}}%
\pgfpathlineto{\pgfqpoint{9.179366in}{2.273108in}}%
\pgfpathlineto{\pgfqpoint{9.185989in}{2.297049in}}%
\pgfpathlineto{\pgfqpoint{9.192612in}{2.313448in}}%
\pgfpathlineto{\pgfqpoint{9.199234in}{2.322026in}}%
\pgfpathlineto{\pgfqpoint{9.202546in}{2.323322in}}%
\pgfpathlineto{\pgfqpoint{9.205857in}{2.322603in}}%
\pgfpathlineto{\pgfqpoint{9.209169in}{2.319861in}}%
\pgfpathlineto{\pgfqpoint{9.212480in}{2.315097in}}%
\pgfpathlineto{\pgfqpoint{9.219103in}{2.299528in}}%
\pgfpathlineto{\pgfqpoint{9.225726in}{2.276009in}}%
\pgfpathlineto{\pgfqpoint{9.232349in}{2.244747in}}%
\pgfpathlineto{\pgfqpoint{9.242283in}{2.184017in}}%
\pgfpathlineto{\pgfqpoint{9.252218in}{2.107929in}}%
\pgfpathlineto{\pgfqpoint{9.265463in}{1.985719in}}%
\pgfpathlineto{\pgfqpoint{9.282021in}{1.807167in}}%
\pgfpathlineto{\pgfqpoint{9.311824in}{1.466277in}}%
\pgfpathlineto{\pgfqpoint{9.315135in}{1.463059in}}%
\pgfpathlineto{\pgfqpoint{9.321758in}{1.525869in}}%
\pgfpathlineto{\pgfqpoint{9.341627in}{1.736401in}}%
\pgfpathlineto{\pgfqpoint{9.354873in}{1.854071in}}%
\pgfpathlineto{\pgfqpoint{9.364807in}{1.925113in}}%
\pgfpathlineto{\pgfqpoint{9.374741in}{1.979026in}}%
\pgfpathlineto{\pgfqpoint{9.381364in}{2.004662in}}%
\pgfpathlineto{\pgfqpoint{9.387987in}{2.021665in}}%
\pgfpathlineto{\pgfqpoint{9.391299in}{2.026862in}}%
\pgfpathlineto{\pgfqpoint{9.394610in}{2.029835in}}%
\pgfpathlineto{\pgfqpoint{9.397921in}{2.030577in}}%
\pgfpathlineto{\pgfqpoint{9.401233in}{2.029090in}}%
\pgfpathlineto{\pgfqpoint{9.404544in}{2.025383in}}%
\pgfpathlineto{\pgfqpoint{9.411167in}{2.011385in}}%
\pgfpathlineto{\pgfqpoint{9.417790in}{1.988810in}}%
\pgfpathlineto{\pgfqpoint{9.424413in}{1.958012in}}%
\pgfpathlineto{\pgfqpoint{9.434347in}{1.897522in}}%
\pgfpathlineto{\pgfqpoint{9.444282in}{1.822109in}}%
\pgfpathlineto{\pgfqpoint{9.474085in}{1.579496in}}%
\pgfpathlineto{\pgfqpoint{9.477396in}{1.571847in}}%
\pgfpathlineto{\pgfqpoint{9.480708in}{1.575553in}}%
\pgfpathlineto{\pgfqpoint{9.484019in}{1.590321in}}%
\pgfpathlineto{\pgfqpoint{9.490642in}{1.643602in}}%
\pgfpathlineto{\pgfqpoint{9.500576in}{1.752007in}}%
\pgfpathlineto{\pgfqpoint{9.540314in}{2.211624in}}%
\pgfpathlineto{\pgfqpoint{9.553560in}{2.334114in}}%
\pgfpathlineto{\pgfqpoint{9.563494in}{2.408786in}}%
\pgfpathlineto{\pgfqpoint{9.573428in}{2.466694in}}%
\pgfpathlineto{\pgfqpoint{9.580051in}{2.495355in}}%
\pgfpathlineto{\pgfqpoint{9.586674in}{2.515786in}}%
\pgfpathlineto{\pgfqpoint{9.593297in}{2.527882in}}%
\pgfpathlineto{\pgfqpoint{9.596608in}{2.530803in}}%
\pgfpathlineto{\pgfqpoint{9.599920in}{2.531654in}}%
\pgfpathlineto{\pgfqpoint{9.603231in}{2.530454in}}%
\pgfpathlineto{\pgfqpoint{9.606543in}{2.527230in}}%
\pgfpathlineto{\pgfqpoint{9.613166in}{2.514850in}}%
\pgfpathlineto{\pgfqpoint{9.619789in}{2.494858in}}%
\pgfpathlineto{\pgfqpoint{9.626411in}{2.467697in}}%
\pgfpathlineto{\pgfqpoint{9.636346in}{2.414689in}}%
\pgfpathlineto{\pgfqpoint{9.646280in}{2.348845in}}%
\pgfpathlineto{\pgfqpoint{9.659526in}{2.245210in}}%
\pgfpathlineto{\pgfqpoint{9.679395in}{2.068220in}}%
\pgfpathlineto{\pgfqpoint{9.715821in}{1.738483in}}%
\pgfpathlineto{\pgfqpoint{9.732378in}{1.611185in}}%
\pgfpathlineto{\pgfqpoint{9.742312in}{1.547603in}}%
\pgfpathlineto{\pgfqpoint{9.752247in}{1.498167in}}%
\pgfpathlineto{\pgfqpoint{9.758869in}{1.478086in}}%
\pgfpathlineto{\pgfqpoint{9.762181in}{1.473853in}}%
\pgfpathlineto{\pgfqpoint{9.765492in}{1.473736in}}%
\pgfpathlineto{\pgfqpoint{9.768804in}{1.476855in}}%
\pgfpathlineto{\pgfqpoint{9.778738in}{1.493958in}}%
\pgfpathlineto{\pgfqpoint{9.788673in}{1.509582in}}%
\pgfpathlineto{\pgfqpoint{9.795295in}{1.516400in}}%
\pgfpathlineto{\pgfqpoint{9.801918in}{1.520028in}}%
\pgfpathlineto{\pgfqpoint{9.808541in}{1.520646in}}%
\pgfpathlineto{\pgfqpoint{9.815164in}{1.518684in}}%
\pgfpathlineto{\pgfqpoint{9.825098in}{1.512417in}}%
\pgfpathlineto{\pgfqpoint{9.838344in}{1.503593in}}%
\pgfpathlineto{\pgfqpoint{9.844967in}{1.502303in}}%
\pgfpathlineto{\pgfqpoint{9.851590in}{1.504939in}}%
\pgfpathlineto{\pgfqpoint{9.858213in}{1.511926in}}%
\pgfpathlineto{\pgfqpoint{9.864836in}{1.522694in}}%
\pgfpathlineto{\pgfqpoint{9.874770in}{1.543525in}}%
\pgfpathlineto{\pgfqpoint{9.904573in}{1.609838in}}%
\pgfpathlineto{\pgfqpoint{9.914508in}{1.626283in}}%
\pgfpathlineto{\pgfqpoint{9.924442in}{1.637865in}}%
\pgfpathlineto{\pgfqpoint{9.931065in}{1.642579in}}%
\pgfpathlineto{\pgfqpoint{9.937688in}{1.644778in}}%
\pgfpathlineto{\pgfqpoint{9.944311in}{1.644451in}}%
\pgfpathlineto{\pgfqpoint{9.950934in}{1.641656in}}%
\pgfpathlineto{\pgfqpoint{9.957556in}{1.636522in}}%
\pgfpathlineto{\pgfqpoint{9.967491in}{1.624894in}}%
\pgfpathlineto{\pgfqpoint{9.977425in}{1.609458in}}%
\pgfpathlineto{\pgfqpoint{10.013851in}{1.546824in}}%
\pgfpathlineto{\pgfqpoint{10.020474in}{1.539963in}}%
\pgfpathlineto{\pgfqpoint{10.027097in}{1.535933in}}%
\pgfpathlineto{\pgfqpoint{10.033720in}{1.534796in}}%
\pgfpathlineto{\pgfqpoint{10.040343in}{1.536213in}}%
\pgfpathlineto{\pgfqpoint{10.050277in}{1.541740in}}%
\pgfpathlineto{\pgfqpoint{10.070146in}{1.557290in}}%
\pgfpathlineto{\pgfqpoint{10.093326in}{1.576475in}}%
\pgfpathlineto{\pgfqpoint{10.103260in}{1.587871in}}%
\pgfpathlineto{\pgfqpoint{10.109883in}{1.598062in}}%
\pgfpathlineto{\pgfqpoint{10.116506in}{1.611046in}}%
\pgfpathlineto{\pgfqpoint{10.126440in}{1.636662in}}%
\pgfpathlineto{\pgfqpoint{10.136375in}{1.670071in}}%
\pgfpathlineto{\pgfqpoint{10.146309in}{1.710813in}}%
\pgfpathlineto{\pgfqpoint{10.159555in}{1.774450in}}%
\pgfpathlineto{\pgfqpoint{10.182735in}{1.900196in}}%
\pgfpathlineto{\pgfqpoint{10.205915in}{2.023270in}}%
\pgfpathlineto{\pgfqpoint{10.219161in}{2.082636in}}%
\pgfpathlineto{\pgfqpoint{10.229095in}{2.118439in}}%
\pgfpathlineto{\pgfqpoint{10.239030in}{2.144840in}}%
\pgfpathlineto{\pgfqpoint{10.245653in}{2.156459in}}%
\pgfpathlineto{\pgfqpoint{10.252276in}{2.162836in}}%
\pgfpathlineto{\pgfqpoint{10.255587in}{2.163955in}}%
\pgfpathlineto{\pgfqpoint{10.258898in}{2.163646in}}%
\pgfpathlineto{\pgfqpoint{10.262210in}{2.161878in}}%
\pgfpathlineto{\pgfqpoint{10.268833in}{2.153866in}}%
\pgfpathlineto{\pgfqpoint{10.275456in}{2.139763in}}%
\pgfpathlineto{\pgfqpoint{10.282079in}{2.119490in}}%
\pgfpathlineto{\pgfqpoint{10.288701in}{2.093054in}}%
\pgfpathlineto{\pgfqpoint{10.298636in}{2.042087in}}%
\pgfpathlineto{\pgfqpoint{10.308570in}{1.978287in}}%
\pgfpathlineto{\pgfqpoint{10.321816in}{1.875969in}}%
\pgfpathlineto{\pgfqpoint{10.351619in}{1.629159in}}%
\pgfpathlineto{\pgfqpoint{10.354930in}{1.613067in}}%
\pgfpathlineto{\pgfqpoint{10.358242in}{1.604110in}}%
\pgfpathlineto{\pgfqpoint{10.361553in}{1.603696in}}%
\pgfpathlineto{\pgfqpoint{10.364865in}{1.612119in}}%
\pgfpathlineto{\pgfqpoint{10.368176in}{1.628426in}}%
\pgfpathlineto{\pgfqpoint{10.374799in}{1.678083in}}%
\pgfpathlineto{\pgfqpoint{10.388045in}{1.809466in}}%
\pgfpathlineto{\pgfqpoint{10.414537in}{2.083128in}}%
\pgfpathlineto{\pgfqpoint{10.427782in}{2.199736in}}%
\pgfpathlineto{\pgfqpoint{10.437717in}{2.272733in}}%
\pgfpathlineto{\pgfqpoint{10.447651in}{2.331152in}}%
\pgfpathlineto{\pgfqpoint{10.454274in}{2.361242in}}%
\pgfpathlineto{\pgfqpoint{10.460897in}{2.383835in}}%
\pgfpathlineto{\pgfqpoint{10.467520in}{2.398669in}}%
\pgfpathlineto{\pgfqpoint{10.470831in}{2.403116in}}%
\pgfpathlineto{\pgfqpoint{10.474143in}{2.405563in}}%
\pgfpathlineto{\pgfqpoint{10.477454in}{2.405998in}}%
\pgfpathlineto{\pgfqpoint{10.480766in}{2.404419in}}%
\pgfpathlineto{\pgfqpoint{10.484077in}{2.400825in}}%
\pgfpathlineto{\pgfqpoint{10.490700in}{2.387630in}}%
\pgfpathlineto{\pgfqpoint{10.497323in}{2.366537in}}%
\pgfpathlineto{\pgfqpoint{10.503946in}{2.337766in}}%
\pgfpathlineto{\pgfqpoint{10.513880in}{2.280944in}}%
\pgfpathlineto{\pgfqpoint{10.523814in}{2.209137in}}%
\pgfpathlineto{\pgfqpoint{10.537060in}{2.093935in}}%
\pgfpathlineto{\pgfqpoint{10.560240in}{1.862493in}}%
\pgfpathlineto{\pgfqpoint{10.573486in}{1.740608in}}%
\pgfpathlineto{\pgfqpoint{10.580109in}{1.696351in}}%
\pgfpathlineto{\pgfqpoint{10.583420in}{1.681693in}}%
\pgfpathlineto{\pgfqpoint{10.586732in}{1.673309in}}%
\pgfpathlineto{\pgfqpoint{10.590043in}{1.671757in}}%
\pgfpathlineto{\pgfqpoint{10.593355in}{1.677019in}}%
\pgfpathlineto{\pgfqpoint{10.596666in}{1.688481in}}%
\pgfpathlineto{\pgfqpoint{10.603289in}{1.725850in}}%
\pgfpathlineto{\pgfqpoint{10.613224in}{1.802388in}}%
\pgfpathlineto{\pgfqpoint{10.639715in}{2.016931in}}%
\pgfpathlineto{\pgfqpoint{10.649649in}{2.082035in}}%
\pgfpathlineto{\pgfqpoint{10.659584in}{2.134055in}}%
\pgfpathlineto{\pgfqpoint{10.666207in}{2.160690in}}%
\pgfpathlineto{\pgfqpoint{10.672830in}{2.180530in}}%
\pgfpathlineto{\pgfqpoint{10.679453in}{2.193381in}}%
\pgfpathlineto{\pgfqpoint{10.682764in}{2.197153in}}%
\pgfpathlineto{\pgfqpoint{10.686075in}{2.199151in}}%
\pgfpathlineto{\pgfqpoint{10.689387in}{2.199377in}}%
\pgfpathlineto{\pgfqpoint{10.692698in}{2.197840in}}%
\pgfpathlineto{\pgfqpoint{10.696010in}{2.194555in}}%
\pgfpathlineto{\pgfqpoint{10.702633in}{2.182836in}}%
\pgfpathlineto{\pgfqpoint{10.709256in}{2.164469in}}%
\pgfpathlineto{\pgfqpoint{10.715878in}{2.139816in}}%
\pgfpathlineto{\pgfqpoint{10.725813in}{2.092145in}}%
\pgfpathlineto{\pgfqpoint{10.739059in}{2.012325in}}%
\pgfpathlineto{\pgfqpoint{10.772173in}{1.795208in}}%
\pgfpathlineto{\pgfqpoint{10.778796in}{1.767040in}}%
\pgfpathlineto{\pgfqpoint{10.782107in}{1.757571in}}%
\pgfpathlineto{\pgfqpoint{10.785419in}{1.751697in}}%
\pgfpathlineto{\pgfqpoint{10.788730in}{1.749661in}}%
\pgfpathlineto{\pgfqpoint{10.792042in}{1.751545in}}%
\pgfpathlineto{\pgfqpoint{10.795353in}{1.757253in}}%
\pgfpathlineto{\pgfqpoint{10.801976in}{1.779003in}}%
\pgfpathlineto{\pgfqpoint{10.808599in}{1.811730in}}%
\pgfpathlineto{\pgfqpoint{10.818533in}{1.873570in}}%
\pgfpathlineto{\pgfqpoint{10.851648in}{2.092910in}}%
\pgfpathlineto{\pgfqpoint{10.864894in}{2.160953in}}%
\pgfpathlineto{\pgfqpoint{10.874828in}{2.199844in}}%
\pgfpathlineto{\pgfqpoint{10.881451in}{2.219412in}}%
\pgfpathlineto{\pgfqpoint{10.888074in}{2.233686in}}%
\pgfpathlineto{\pgfqpoint{10.894697in}{2.242587in}}%
\pgfpathlineto{\pgfqpoint{10.901320in}{2.246115in}}%
\pgfpathlineto{\pgfqpoint{10.904631in}{2.245886in}}%
\pgfpathlineto{\pgfqpoint{10.907943in}{2.244350in}}%
\pgfpathlineto{\pgfqpoint{10.914565in}{2.237448in}}%
\pgfpathlineto{\pgfqpoint{10.921188in}{2.225639in}}%
\pgfpathlineto{\pgfqpoint{10.927811in}{2.209224in}}%
\pgfpathlineto{\pgfqpoint{10.937746in}{2.176779in}}%
\pgfpathlineto{\pgfqpoint{10.947680in}{2.136281in}}%
\pgfpathlineto{\pgfqpoint{10.960926in}{2.072763in}}%
\pgfpathlineto{\pgfqpoint{10.990729in}{1.912972in}}%
\pgfpathlineto{\pgfqpoint{11.007286in}{1.830366in}}%
\pgfpathlineto{\pgfqpoint{11.020532in}{1.775442in}}%
\pgfpathlineto{\pgfqpoint{11.030466in}{1.743152in}}%
\pgfpathlineto{\pgfqpoint{11.040401in}{1.719289in}}%
\pgfpathlineto{\pgfqpoint{11.047023in}{1.707935in}}%
\pgfpathlineto{\pgfqpoint{11.053646in}{1.699831in}}%
\pgfpathlineto{\pgfqpoint{11.060269in}{1.694450in}}%
\pgfpathlineto{\pgfqpoint{11.070204in}{1.690114in}}%
\pgfpathlineto{\pgfqpoint{11.083449in}{1.688048in}}%
\pgfpathlineto{\pgfqpoint{11.103318in}{1.685316in}}%
\pgfpathlineto{\pgfqpoint{11.116564in}{1.680968in}}%
\pgfpathlineto{\pgfqpoint{11.133121in}{1.672509in}}%
\pgfpathlineto{\pgfqpoint{11.159613in}{1.657862in}}%
\pgfpathlineto{\pgfqpoint{11.169547in}{1.654424in}}%
\pgfpathlineto{\pgfqpoint{11.179481in}{1.653117in}}%
\pgfpathlineto{\pgfqpoint{11.189416in}{1.654275in}}%
\pgfpathlineto{\pgfqpoint{11.199350in}{1.657924in}}%
\pgfpathlineto{\pgfqpoint{11.209285in}{1.663793in}}%
\pgfpathlineto{\pgfqpoint{11.222530in}{1.674207in}}%
\pgfpathlineto{\pgfqpoint{11.275514in}{1.719762in}}%
\pgfpathlineto{\pgfqpoint{11.292071in}{1.729600in}}%
\pgfpathlineto{\pgfqpoint{11.311939in}{1.738451in}}%
\pgfpathlineto{\pgfqpoint{11.358300in}{1.757181in}}%
\pgfpathlineto{\pgfqpoint{11.374857in}{1.766880in}}%
\pgfpathlineto{\pgfqpoint{11.391414in}{1.779620in}}%
\pgfpathlineto{\pgfqpoint{11.404660in}{1.792403in}}%
\pgfpathlineto{\pgfqpoint{11.421217in}{1.811788in}}%
\pgfpathlineto{\pgfqpoint{11.437775in}{1.834599in}}%
\pgfpathlineto{\pgfqpoint{11.470889in}{1.885397in}}%
\pgfpathlineto{\pgfqpoint{11.490758in}{1.913584in}}%
\pgfpathlineto{\pgfqpoint{11.504004in}{1.928742in}}%
\pgfpathlineto{\pgfqpoint{11.513938in}{1.937267in}}%
\pgfpathlineto{\pgfqpoint{11.523872in}{1.942768in}}%
\pgfpathlineto{\pgfqpoint{11.530495in}{1.944510in}}%
\pgfpathlineto{\pgfqpoint{11.537118in}{1.944556in}}%
\pgfpathlineto{\pgfqpoint{11.543741in}{1.942781in}}%
\pgfpathlineto{\pgfqpoint{11.550364in}{1.939076in}}%
\pgfpathlineto{\pgfqpoint{11.556987in}{1.933347in}}%
\pgfpathlineto{\pgfqpoint{11.566921in}{1.920792in}}%
\pgfpathlineto{\pgfqpoint{11.576855in}{1.903344in}}%
\pgfpathlineto{\pgfqpoint{11.586790in}{1.880975in}}%
\pgfpathlineto{\pgfqpoint{11.600036in}{1.843793in}}%
\pgfpathlineto{\pgfqpoint{11.613281in}{1.799210in}}%
\pgfpathlineto{\pgfqpoint{11.636462in}{1.710839in}}%
\pgfpathlineto{\pgfqpoint{11.649707in}{1.662842in}}%
\pgfpathlineto{\pgfqpoint{11.659642in}{1.634580in}}%
\pgfpathlineto{\pgfqpoint{11.666265in}{1.622194in}}%
\pgfpathlineto{\pgfqpoint{11.669576in}{1.618456in}}%
\pgfpathlineto{\pgfqpoint{11.672888in}{1.616506in}}%
\pgfpathlineto{\pgfqpoint{11.676199in}{1.616382in}}%
\pgfpathlineto{\pgfqpoint{11.679510in}{1.618051in}}%
\pgfpathlineto{\pgfqpoint{11.686133in}{1.626334in}}%
\pgfpathlineto{\pgfqpoint{11.692756in}{1.640037in}}%
\pgfpathlineto{\pgfqpoint{11.702691in}{1.667156in}}%
\pgfpathlineto{\pgfqpoint{11.735805in}{1.764236in}}%
\pgfpathlineto{\pgfqpoint{11.745739in}{1.786132in}}%
\pgfpathlineto{\pgfqpoint{11.755674in}{1.802653in}}%
\pgfpathlineto{\pgfqpoint{11.762297in}{1.810471in}}%
\pgfpathlineto{\pgfqpoint{11.768920in}{1.815676in}}%
\pgfpathlineto{\pgfqpoint{11.775542in}{1.818278in}}%
\pgfpathlineto{\pgfqpoint{11.782165in}{1.818330in}}%
\pgfpathlineto{\pgfqpoint{11.788788in}{1.815924in}}%
\pgfpathlineto{\pgfqpoint{11.795411in}{1.811184in}}%
\pgfpathlineto{\pgfqpoint{11.802034in}{1.804264in}}%
\pgfpathlineto{\pgfqpoint{11.811968in}{1.790191in}}%
\pgfpathlineto{\pgfqpoint{11.821903in}{1.772314in}}%
\pgfpathlineto{\pgfqpoint{11.835149in}{1.743946in}}%
\pgfpathlineto{\pgfqpoint{11.884820in}{1.629942in}}%
\pgfpathlineto{\pgfqpoint{11.894755in}{1.613984in}}%
\pgfpathlineto{\pgfqpoint{11.904689in}{1.602534in}}%
\pgfpathlineto{\pgfqpoint{11.911312in}{1.597462in}}%
\pgfpathlineto{\pgfqpoint{11.921246in}{1.593229in}}%
\pgfpathlineto{\pgfqpoint{11.931181in}{1.591905in}}%
\pgfpathlineto{\pgfqpoint{11.954361in}{1.590807in}}%
\pgfpathlineto{\pgfqpoint{11.964295in}{1.587533in}}%
\pgfpathlineto{\pgfqpoint{11.974229in}{1.581087in}}%
\pgfpathlineto{\pgfqpoint{11.984164in}{1.571122in}}%
\pgfpathlineto{\pgfqpoint{11.994098in}{1.557974in}}%
\pgfpathlineto{\pgfqpoint{12.017278in}{1.525029in}}%
\pgfpathlineto{\pgfqpoint{12.023901in}{1.520198in}}%
\pgfpathlineto{\pgfqpoint{12.027213in}{1.519536in}}%
\pgfpathlineto{\pgfqpoint{12.030524in}{1.520244in}}%
\pgfpathlineto{\pgfqpoint{12.033836in}{1.522382in}}%
\pgfpathlineto{\pgfqpoint{12.040458in}{1.530783in}}%
\pgfpathlineto{\pgfqpoint{12.047081in}{1.543797in}}%
\pgfpathlineto{\pgfqpoint{12.057016in}{1.568911in}}%
\pgfpathlineto{\pgfqpoint{12.093442in}{1.667168in}}%
\pgfpathlineto{\pgfqpoint{12.103376in}{1.687606in}}%
\pgfpathlineto{\pgfqpoint{12.113310in}{1.703843in}}%
\pgfpathlineto{\pgfqpoint{12.123245in}{1.715795in}}%
\pgfpathlineto{\pgfqpoint{12.133179in}{1.723672in}}%
\pgfpathlineto{\pgfqpoint{12.143113in}{1.727916in}}%
\pgfpathlineto{\pgfqpoint{12.153048in}{1.729132in}}%
\pgfpathlineto{\pgfqpoint{12.162982in}{1.728034in}}%
\pgfpathlineto{\pgfqpoint{12.176228in}{1.724279in}}%
\pgfpathlineto{\pgfqpoint{12.209342in}{1.713747in}}%
\pgfpathlineto{\pgfqpoint{12.222588in}{1.712326in}}%
\pgfpathlineto{\pgfqpoint{12.235834in}{1.713430in}}%
\pgfpathlineto{\pgfqpoint{12.249080in}{1.717118in}}%
\pgfpathlineto{\pgfqpoint{12.262326in}{1.723268in}}%
\pgfpathlineto{\pgfqpoint{12.275571in}{1.731839in}}%
\pgfpathlineto{\pgfqpoint{12.288817in}{1.743092in}}%
\pgfpathlineto{\pgfqpoint{12.302063in}{1.757670in}}%
\pgfpathlineto{\pgfqpoint{12.315309in}{1.776492in}}%
\pgfpathlineto{\pgfqpoint{12.328555in}{1.800433in}}%
\pgfpathlineto{\pgfqpoint{12.341800in}{1.829885in}}%
\pgfpathlineto{\pgfqpoint{12.358358in}{1.873682in}}%
\pgfpathlineto{\pgfqpoint{12.411341in}{2.022046in}}%
\pgfpathlineto{\pgfqpoint{12.421275in}{2.041582in}}%
\pgfpathlineto{\pgfqpoint{12.431210in}{2.055854in}}%
\pgfpathlineto{\pgfqpoint{12.437832in}{2.062050in}}%
\pgfpathlineto{\pgfqpoint{12.444455in}{2.065376in}}%
\pgfpathlineto{\pgfqpoint{12.451078in}{2.065696in}}%
\pgfpathlineto{\pgfqpoint{12.457701in}{2.062921in}}%
\pgfpathlineto{\pgfqpoint{12.464324in}{2.057006in}}%
\pgfpathlineto{\pgfqpoint{12.470947in}{2.047952in}}%
\pgfpathlineto{\pgfqpoint{12.477570in}{2.035804in}}%
\pgfpathlineto{\pgfqpoint{12.487504in}{2.011982in}}%
\pgfpathlineto{\pgfqpoint{12.497439in}{1.981863in}}%
\pgfpathlineto{\pgfqpoint{12.510684in}{1.933053in}}%
\pgfpathlineto{\pgfqpoint{12.527242in}{1.861058in}}%
\pgfpathlineto{\pgfqpoint{12.580225in}{1.618554in}}%
\pgfpathlineto{\pgfqpoint{12.590159in}{1.589040in}}%
\pgfpathlineto{\pgfqpoint{12.596782in}{1.576028in}}%
\pgfpathlineto{\pgfqpoint{12.603405in}{1.568882in}}%
\pgfpathlineto{\pgfqpoint{12.606716in}{1.567410in}}%
\pgfpathlineto{\pgfqpoint{12.613339in}{1.568010in}}%
\pgfpathlineto{\pgfqpoint{12.619962in}{1.572059in}}%
\pgfpathlineto{\pgfqpoint{12.646454in}{1.594010in}}%
\pgfpathlineto{\pgfqpoint{12.653077in}{1.596333in}}%
\pgfpathlineto{\pgfqpoint{12.659700in}{1.596580in}}%
\pgfpathlineto{\pgfqpoint{12.666322in}{1.594758in}}%
\pgfpathlineto{\pgfqpoint{12.676257in}{1.588662in}}%
\pgfpathlineto{\pgfqpoint{12.699437in}{1.569962in}}%
\pgfpathlineto{\pgfqpoint{12.706060in}{1.568251in}}%
\pgfpathlineto{\pgfqpoint{12.712683in}{1.570461in}}%
\pgfpathlineto{\pgfqpoint{12.719306in}{1.577406in}}%
\pgfpathlineto{\pgfqpoint{12.725929in}{1.589231in}}%
\pgfpathlineto{\pgfqpoint{12.732551in}{1.605464in}}%
\pgfpathlineto{\pgfqpoint{12.742486in}{1.636222in}}%
\pgfpathlineto{\pgfqpoint{12.759043in}{1.697064in}}%
\pgfpathlineto{\pgfqpoint{12.785535in}{1.795391in}}%
\pgfpathlineto{\pgfqpoint{12.798780in}{1.837084in}}%
\pgfpathlineto{\pgfqpoint{12.808715in}{1.863125in}}%
\pgfpathlineto{\pgfqpoint{12.818649in}{1.884012in}}%
\pgfpathlineto{\pgfqpoint{12.828584in}{1.899399in}}%
\pgfpathlineto{\pgfqpoint{12.835206in}{1.906529in}}%
\pgfpathlineto{\pgfqpoint{12.841829in}{1.911163in}}%
\pgfpathlineto{\pgfqpoint{12.848452in}{1.913347in}}%
\pgfpathlineto{\pgfqpoint{12.855075in}{1.913160in}}%
\pgfpathlineto{\pgfqpoint{12.861698in}{1.910710in}}%
\pgfpathlineto{\pgfqpoint{12.868321in}{1.906133in}}%
\pgfpathlineto{\pgfqpoint{12.878255in}{1.895628in}}%
\pgfpathlineto{\pgfqpoint{12.888190in}{1.881319in}}%
\pgfpathlineto{\pgfqpoint{12.901435in}{1.857563in}}%
\pgfpathlineto{\pgfqpoint{12.921304in}{1.815794in}}%
\pgfpathlineto{\pgfqpoint{12.951107in}{1.752942in}}%
\pgfpathlineto{\pgfqpoint{12.964353in}{1.729886in}}%
\pgfpathlineto{\pgfqpoint{12.974287in}{1.715933in}}%
\pgfpathlineto{\pgfqpoint{12.984222in}{1.705366in}}%
\pgfpathlineto{\pgfqpoint{12.994156in}{1.698544in}}%
\pgfpathlineto{\pgfqpoint{13.000779in}{1.696222in}}%
\pgfpathlineto{\pgfqpoint{13.007402in}{1.695775in}}%
\pgfpathlineto{\pgfqpoint{13.014025in}{1.697281in}}%
\pgfpathlineto{\pgfqpoint{13.020648in}{1.700816in}}%
\pgfpathlineto{\pgfqpoint{13.027271in}{1.706452in}}%
\pgfpathlineto{\pgfqpoint{13.037205in}{1.718955in}}%
\pgfpathlineto{\pgfqpoint{13.047139in}{1.736367in}}%
\pgfpathlineto{\pgfqpoint{13.057074in}{1.758548in}}%
\pgfpathlineto{\pgfqpoint{13.070319in}{1.794833in}}%
\pgfpathlineto{\pgfqpoint{13.086877in}{1.848539in}}%
\pgfpathlineto{\pgfqpoint{13.129925in}{1.995232in}}%
\pgfpathlineto{\pgfqpoint{13.139860in}{2.021841in}}%
\pgfpathlineto{\pgfqpoint{13.149794in}{2.042649in}}%
\pgfpathlineto{\pgfqpoint{13.156417in}{2.052638in}}%
\pgfpathlineto{\pgfqpoint{13.163040in}{2.059110in}}%
\pgfpathlineto{\pgfqpoint{13.169663in}{2.061769in}}%
\pgfpathlineto{\pgfqpoint{13.176286in}{2.060380in}}%
\pgfpathlineto{\pgfqpoint{13.182909in}{2.054780in}}%
\pgfpathlineto{\pgfqpoint{13.189532in}{2.044886in}}%
\pgfpathlineto{\pgfqpoint{13.196154in}{2.030704in}}%
\pgfpathlineto{\pgfqpoint{13.206089in}{2.001640in}}%
\pgfpathlineto{\pgfqpoint{13.216023in}{1.963940in}}%
\pgfpathlineto{\pgfqpoint{13.229269in}{1.902514in}}%
\pgfpathlineto{\pgfqpoint{13.249138in}{1.796182in}}%
\pgfpathlineto{\pgfqpoint{13.265695in}{1.709402in}}%
\pgfpathlineto{\pgfqpoint{13.275629in}{1.666546in}}%
\pgfpathlineto{\pgfqpoint{13.282252in}{1.644911in}}%
\pgfpathlineto{\pgfqpoint{13.288875in}{1.630366in}}%
\pgfpathlineto{\pgfqpoint{13.292186in}{1.625971in}}%
\pgfpathlineto{\pgfqpoint{13.295498in}{1.623476in}}%
\pgfpathlineto{\pgfqpoint{13.298809in}{1.622775in}}%
\pgfpathlineto{\pgfqpoint{13.302121in}{1.623697in}}%
\pgfpathlineto{\pgfqpoint{13.308744in}{1.629479in}}%
\pgfpathlineto{\pgfqpoint{13.318678in}{1.644143in}}%
\pgfpathlineto{\pgfqpoint{13.335235in}{1.670139in}}%
\pgfpathlineto{\pgfqpoint{13.341858in}{1.677751in}}%
\pgfpathlineto{\pgfqpoint{13.348481in}{1.682862in}}%
\pgfpathlineto{\pgfqpoint{13.355104in}{1.685255in}}%
\pgfpathlineto{\pgfqpoint{13.361727in}{1.684961in}}%
\pgfpathlineto{\pgfqpoint{13.368350in}{1.682250in}}%
\pgfpathlineto{\pgfqpoint{13.378284in}{1.674894in}}%
\pgfpathlineto{\pgfqpoint{13.391530in}{1.664095in}}%
\pgfpathlineto{\pgfqpoint{13.398153in}{1.661441in}}%
\pgfpathlineto{\pgfqpoint{13.401464in}{1.661538in}}%
\pgfpathlineto{\pgfqpoint{13.404776in}{1.662865in}}%
\pgfpathlineto{\pgfqpoint{13.408087in}{1.665607in}}%
\pgfpathlineto{\pgfqpoint{13.414710in}{1.675897in}}%
\pgfpathlineto{\pgfqpoint{13.421333in}{1.693066in}}%
\pgfpathlineto{\pgfqpoint{13.427956in}{1.717021in}}%
\pgfpathlineto{\pgfqpoint{13.437890in}{1.764084in}}%
\pgfpathlineto{\pgfqpoint{13.451136in}{1.842126in}}%
\pgfpathlineto{\pgfqpoint{13.474316in}{1.997770in}}%
\pgfpathlineto{\pgfqpoint{13.494185in}{2.126890in}}%
\pgfpathlineto{\pgfqpoint{13.507431in}{2.200460in}}%
\pgfpathlineto{\pgfqpoint{13.517365in}{2.245377in}}%
\pgfpathlineto{\pgfqpoint{13.527299in}{2.279295in}}%
\pgfpathlineto{\pgfqpoint{13.533922in}{2.294918in}}%
\pgfpathlineto{\pgfqpoint{13.540545in}{2.304409in}}%
\pgfpathlineto{\pgfqpoint{13.543857in}{2.306730in}}%
\pgfpathlineto{\pgfqpoint{13.547168in}{2.307376in}}%
\pgfpathlineto{\pgfqpoint{13.550480in}{2.306310in}}%
\pgfpathlineto{\pgfqpoint{13.553791in}{2.303500in}}%
\pgfpathlineto{\pgfqpoint{13.560414in}{2.292549in}}%
\pgfpathlineto{\pgfqpoint{13.567037in}{2.274391in}}%
\pgfpathlineto{\pgfqpoint{13.573660in}{2.248999in}}%
\pgfpathlineto{\pgfqpoint{13.583594in}{2.197600in}}%
\pgfpathlineto{\pgfqpoint{13.593528in}{2.131087in}}%
\pgfpathlineto{\pgfqpoint{13.606774in}{2.021995in}}%
\pgfpathlineto{\pgfqpoint{13.626643in}{1.830998in}}%
\pgfpathlineto{\pgfqpoint{13.639889in}{1.710438in}}%
\pgfpathlineto{\pgfqpoint{13.646512in}{1.667240in}}%
\pgfpathlineto{\pgfqpoint{13.649823in}{1.654015in}}%
\pgfpathlineto{\pgfqpoint{13.653135in}{1.647995in}}%
\pgfpathlineto{\pgfqpoint{13.656446in}{1.649768in}}%
\pgfpathlineto{\pgfqpoint{13.659757in}{1.659080in}}%
\pgfpathlineto{\pgfqpoint{13.666380in}{1.696021in}}%
\pgfpathlineto{\pgfqpoint{13.676315in}{1.778023in}}%
\pgfpathlineto{\pgfqpoint{13.706118in}{2.046767in}}%
\pgfpathlineto{\pgfqpoint{13.719364in}{2.142840in}}%
\pgfpathlineto{\pgfqpoint{13.729298in}{2.200153in}}%
\pgfpathlineto{\pgfqpoint{13.739232in}{2.243714in}}%
\pgfpathlineto{\pgfqpoint{13.745855in}{2.264892in}}%
\pgfpathlineto{\pgfqpoint{13.752478in}{2.279757in}}%
\pgfpathlineto{\pgfqpoint{13.759101in}{2.288378in}}%
\pgfpathlineto{\pgfqpoint{13.762412in}{2.290388in}}%
\pgfpathlineto{\pgfqpoint{13.765724in}{2.290896in}}%
\pgfpathlineto{\pgfqpoint{13.769035in}{2.289928in}}%
\pgfpathlineto{\pgfqpoint{13.772347in}{2.287516in}}%
\pgfpathlineto{\pgfqpoint{13.778970in}{2.278502in}}%
\pgfpathlineto{\pgfqpoint{13.785593in}{2.264174in}}%
\pgfpathlineto{\pgfqpoint{13.792215in}{2.244894in}}%
\pgfpathlineto{\pgfqpoint{13.802150in}{2.207587in}}%
\pgfpathlineto{\pgfqpoint{13.812084in}{2.161563in}}%
\pgfpathlineto{\pgfqpoint{13.825330in}{2.089467in}}%
\pgfpathlineto{\pgfqpoint{13.845199in}{1.966682in}}%
\pgfpathlineto{\pgfqpoint{13.878313in}{1.758765in}}%
\pgfpathlineto{\pgfqpoint{13.891559in}{1.687826in}}%
\pgfpathlineto{\pgfqpoint{13.901493in}{1.644011in}}%
\pgfpathlineto{\pgfqpoint{13.911428in}{1.610717in}}%
\pgfpathlineto{\pgfqpoint{13.918051in}{1.595084in}}%
\pgfpathlineto{\pgfqpoint{13.924673in}{1.584621in}}%
\pgfpathlineto{\pgfqpoint{13.931296in}{1.578636in}}%
\pgfpathlineto{\pgfqpoint{13.937919in}{1.575967in}}%
\pgfpathlineto{\pgfqpoint{13.947854in}{1.575229in}}%
\pgfpathlineto{\pgfqpoint{13.961099in}{1.574494in}}%
\pgfpathlineto{\pgfqpoint{13.971034in}{1.571546in}}%
\pgfpathlineto{\pgfqpoint{13.990902in}{1.563580in}}%
\pgfpathlineto{\pgfqpoint{13.997525in}{1.564753in}}%
\pgfpathlineto{\pgfqpoint{14.000837in}{1.567067in}}%
\pgfpathlineto{\pgfqpoint{14.007460in}{1.576155in}}%
\pgfpathlineto{\pgfqpoint{14.014083in}{1.591910in}}%
\pgfpathlineto{\pgfqpoint{14.020705in}{1.614328in}}%
\pgfpathlineto{\pgfqpoint{14.030640in}{1.658748in}}%
\pgfpathlineto{\pgfqpoint{14.043886in}{1.732507in}}%
\pgfpathlineto{\pgfqpoint{14.067066in}{1.879695in}}%
\pgfpathlineto{\pgfqpoint{14.090246in}{2.021957in}}%
\pgfpathlineto{\pgfqpoint{14.103492in}{2.091537in}}%
\pgfpathlineto{\pgfqpoint{14.116738in}{2.148633in}}%
\pgfpathlineto{\pgfqpoint{14.126672in}{2.182098in}}%
\pgfpathlineto{\pgfqpoint{14.136606in}{2.207094in}}%
\pgfpathlineto{\pgfqpoint{14.143229in}{2.219024in}}%
\pgfpathlineto{\pgfqpoint{14.149852in}{2.227242in}}%
\pgfpathlineto{\pgfqpoint{14.156475in}{2.231879in}}%
\pgfpathlineto{\pgfqpoint{14.163098in}{2.233118in}}%
\pgfpathlineto{\pgfqpoint{14.169721in}{2.231194in}}%
\pgfpathlineto{\pgfqpoint{14.176344in}{2.226379in}}%
\pgfpathlineto{\pgfqpoint{14.182967in}{2.218979in}}%
\pgfpathlineto{\pgfqpoint{14.192901in}{2.203756in}}%
\pgfpathlineto{\pgfqpoint{14.206147in}{2.177584in}}%
\pgfpathlineto{\pgfqpoint{14.226015in}{2.131263in}}%
\pgfpathlineto{\pgfqpoint{14.255818in}{2.060997in}}%
\pgfpathlineto{\pgfqpoint{14.272376in}{2.026905in}}%
\pgfpathlineto{\pgfqpoint{14.288933in}{1.997793in}}%
\pgfpathlineto{\pgfqpoint{14.305490in}{1.973544in}}%
\pgfpathlineto{\pgfqpoint{14.322047in}{1.953524in}}%
\pgfpathlineto{\pgfqpoint{14.338605in}{1.936989in}}%
\pgfpathlineto{\pgfqpoint{14.355162in}{1.923267in}}%
\pgfpathlineto{\pgfqpoint{14.375031in}{1.909580in}}%
\pgfpathlineto{\pgfqpoint{14.418079in}{1.881542in}}%
\pgfpathlineto{\pgfqpoint{14.431325in}{1.869494in}}%
\pgfpathlineto{\pgfqpoint{14.444571in}{1.853669in}}%
\pgfpathlineto{\pgfqpoint{14.454505in}{1.838613in}}%
\pgfpathlineto{\pgfqpoint{14.467751in}{1.813622in}}%
\pgfpathlineto{\pgfqpoint{14.480997in}{1.782822in}}%
\pgfpathlineto{\pgfqpoint{14.497554in}{1.737328in}}%
\pgfpathlineto{\pgfqpoint{14.530669in}{1.643146in}}%
\pgfpathlineto{\pgfqpoint{14.540603in}{1.622661in}}%
\pgfpathlineto{\pgfqpoint{14.547226in}{1.613084in}}%
\pgfpathlineto{\pgfqpoint{14.553849in}{1.607175in}}%
\pgfpathlineto{\pgfqpoint{14.560472in}{1.604854in}}%
\pgfpathlineto{\pgfqpoint{14.567095in}{1.605633in}}%
\pgfpathlineto{\pgfqpoint{14.573718in}{1.608728in}}%
\pgfpathlineto{\pgfqpoint{14.590275in}{1.620679in}}%
\pgfpathlineto{\pgfqpoint{14.600209in}{1.626840in}}%
\pgfpathlineto{\pgfqpoint{14.610144in}{1.630028in}}%
\pgfpathlineto{\pgfqpoint{14.616766in}{1.630012in}}%
\pgfpathlineto{\pgfqpoint{14.623389in}{1.628125in}}%
\pgfpathlineto{\pgfqpoint{14.630012in}{1.624354in}}%
\pgfpathlineto{\pgfqpoint{14.639947in}{1.615374in}}%
\pgfpathlineto{\pgfqpoint{14.649881in}{1.603044in}}%
\pgfpathlineto{\pgfqpoint{14.682995in}{1.557473in}}%
\pgfpathlineto{\pgfqpoint{14.689618in}{1.553062in}}%
\pgfpathlineto{\pgfqpoint{14.696241in}{1.552217in}}%
\pgfpathlineto{\pgfqpoint{14.702864in}{1.555365in}}%
\pgfpathlineto{\pgfqpoint{14.709487in}{1.562420in}}%
\pgfpathlineto{\pgfqpoint{14.716110in}{1.572848in}}%
\pgfpathlineto{\pgfqpoint{14.726044in}{1.593118in}}%
\pgfpathlineto{\pgfqpoint{14.745913in}{1.641052in}}%
\pgfpathlineto{\pgfqpoint{14.762470in}{1.679481in}}%
\pgfpathlineto{\pgfqpoint{14.775716in}{1.704776in}}%
\pgfpathlineto{\pgfqpoint{14.785650in}{1.719027in}}%
\pgfpathlineto{\pgfqpoint{14.792273in}{1.725825in}}%
\pgfpathlineto{\pgfqpoint{14.798896in}{1.730218in}}%
\pgfpathlineto{\pgfqpoint{14.805519in}{1.732022in}}%
\pgfpathlineto{\pgfqpoint{14.812142in}{1.731082in}}%
\pgfpathlineto{\pgfqpoint{14.818765in}{1.727269in}}%
\pgfpathlineto{\pgfqpoint{14.825388in}{1.720483in}}%
\pgfpathlineto{\pgfqpoint{14.832011in}{1.710663in}}%
\pgfpathlineto{\pgfqpoint{14.841945in}{1.690200in}}%
\pgfpathlineto{\pgfqpoint{14.851879in}{1.662975in}}%
\pgfpathlineto{\pgfqpoint{14.861814in}{1.629360in}}%
\pgfpathlineto{\pgfqpoint{14.875060in}{1.575855in}}%
\pgfpathlineto{\pgfqpoint{14.901551in}{1.460549in}}%
\pgfpathlineto{\pgfqpoint{14.904863in}{1.456714in}}%
\pgfpathlineto{\pgfqpoint{14.908174in}{1.462068in}}%
\pgfpathlineto{\pgfqpoint{14.914797in}{1.488141in}}%
\pgfpathlineto{\pgfqpoint{14.951223in}{1.656796in}}%
\pgfpathlineto{\pgfqpoint{14.964469in}{1.703806in}}%
\pgfpathlineto{\pgfqpoint{14.974403in}{1.731352in}}%
\pgfpathlineto{\pgfqpoint{14.984337in}{1.751768in}}%
\pgfpathlineto{\pgfqpoint{14.990960in}{1.761326in}}%
\pgfpathlineto{\pgfqpoint{14.997583in}{1.767647in}}%
\pgfpathlineto{\pgfqpoint{15.004206in}{1.770780in}}%
\pgfpathlineto{\pgfqpoint{15.010829in}{1.770805in}}%
\pgfpathlineto{\pgfqpoint{15.017452in}{1.767831in}}%
\pgfpathlineto{\pgfqpoint{15.024075in}{1.761986in}}%
\pgfpathlineto{\pgfqpoint{15.030698in}{1.753413in}}%
\pgfpathlineto{\pgfqpoint{15.040632in}{1.735786in}}%
\pgfpathlineto{\pgfqpoint{15.050566in}{1.712946in}}%
\pgfpathlineto{\pgfqpoint{15.063812in}{1.675464in}}%
\pgfpathlineto{\pgfqpoint{15.080369in}{1.619661in}}%
\pgfpathlineto{\pgfqpoint{15.120107in}{1.477549in}}%
\pgfpathlineto{\pgfqpoint{15.123418in}{1.471208in}}%
\pgfpathlineto{\pgfqpoint{15.126730in}{1.468400in}}%
\pgfpathlineto{\pgfqpoint{15.130041in}{1.469700in}}%
\pgfpathlineto{\pgfqpoint{15.133353in}{1.474660in}}%
\pgfpathlineto{\pgfqpoint{15.139976in}{1.491294in}}%
\pgfpathlineto{\pgfqpoint{15.176401in}{1.598536in}}%
\pgfpathlineto{\pgfqpoint{15.186336in}{1.620215in}}%
\pgfpathlineto{\pgfqpoint{15.196270in}{1.637101in}}%
\pgfpathlineto{\pgfqpoint{15.206205in}{1.648988in}}%
\pgfpathlineto{\pgfqpoint{15.212827in}{1.654146in}}%
\pgfpathlineto{\pgfqpoint{15.219450in}{1.657175in}}%
\pgfpathlineto{\pgfqpoint{15.226073in}{1.658201in}}%
\pgfpathlineto{\pgfqpoint{15.232696in}{1.657401in}}%
\pgfpathlineto{\pgfqpoint{15.242630in}{1.653265in}}%
\pgfpathlineto{\pgfqpoint{15.252565in}{1.646407in}}%
\pgfpathlineto{\pgfqpoint{15.269122in}{1.631594in}}%
\pgfpathlineto{\pgfqpoint{15.295614in}{1.607677in}}%
\pgfpathlineto{\pgfqpoint{15.315482in}{1.593305in}}%
\pgfpathlineto{\pgfqpoint{15.375088in}{1.556047in}}%
\pgfpathlineto{\pgfqpoint{15.398269in}{1.543012in}}%
\pgfpathlineto{\pgfqpoint{15.414826in}{1.536332in}}%
\pgfpathlineto{\pgfqpoint{15.428072in}{1.533194in}}%
\pgfpathlineto{\pgfqpoint{15.441317in}{1.532172in}}%
\pgfpathlineto{\pgfqpoint{15.441317in}{1.532172in}}%
\pgfusepath{stroke}%
\end{pgfscope}%
\begin{pgfscope}%
\pgfpathrectangle{\pgfqpoint{2.400000in}{1.081300in}}{\pgfqpoint{14.880000in}{7.569100in}}%
\pgfusepath{clip}%
\pgfsetrectcap%
\pgfsetroundjoin%
\pgfsetlinewidth{1.505625pt}%
\definecolor{currentstroke}{rgb}{0.090196,0.745098,0.811765}%
\pgfsetstrokecolor{currentstroke}%
\pgfsetdash{}{0pt}%
\pgfpathmoveto{\pgfqpoint{3.076364in}{1.425350in}}%
\pgfpathlineto{\pgfqpoint{3.271739in}{1.430187in}}%
\pgfpathlineto{\pgfqpoint{3.500229in}{1.432953in}}%
\pgfpathlineto{\pgfqpoint{3.702228in}{1.432628in}}%
\pgfpathlineto{\pgfqpoint{3.785014in}{1.433257in}}%
\pgfpathlineto{\pgfqpoint{3.854554in}{1.431226in}}%
\pgfpathlineto{\pgfqpoint{3.930718in}{1.429491in}}%
\pgfpathlineto{\pgfqpoint{4.013504in}{1.430232in}}%
\pgfpathlineto{\pgfqpoint{4.139339in}{1.433919in}}%
\pgfpathlineto{\pgfqpoint{4.285043in}{1.439310in}}%
\pgfpathlineto{\pgfqpoint{4.357895in}{1.438631in}}%
\pgfpathlineto{\pgfqpoint{4.612876in}{1.432289in}}%
\pgfpathlineto{\pgfqpoint{4.682417in}{1.430290in}}%
\pgfpathlineto{\pgfqpoint{4.732089in}{1.431353in}}%
\pgfpathlineto{\pgfqpoint{4.811563in}{1.435644in}}%
\pgfpathlineto{\pgfqpoint{4.934087in}{1.442469in}}%
\pgfpathlineto{\pgfqpoint{4.987070in}{1.442625in}}%
\pgfpathlineto{\pgfqpoint{5.036742in}{1.440509in}}%
\pgfpathlineto{\pgfqpoint{5.106282in}{1.434931in}}%
\pgfpathlineto{\pgfqpoint{5.185757in}{1.429066in}}%
\pgfpathlineto{\pgfqpoint{5.238740in}{1.427511in}}%
\pgfpathlineto{\pgfqpoint{5.328150in}{1.427423in}}%
\pgfpathlineto{\pgfqpoint{5.487099in}{1.429517in}}%
\pgfpathlineto{\pgfqpoint{5.563262in}{1.430729in}}%
\pgfpathlineto{\pgfqpoint{5.695720in}{1.433097in}}%
\pgfpathlineto{\pgfqpoint{5.752015in}{1.432002in}}%
\pgfpathlineto{\pgfqpoint{5.828178in}{1.429618in}}%
\pgfpathlineto{\pgfqpoint{5.867916in}{1.432363in}}%
\pgfpathlineto{\pgfqpoint{5.910965in}{1.434694in}}%
\pgfpathlineto{\pgfqpoint{5.950702in}{1.434456in}}%
\pgfpathlineto{\pgfqpoint{6.046734in}{1.432408in}}%
\pgfpathlineto{\pgfqpoint{6.129520in}{1.433650in}}%
\pgfpathlineto{\pgfqpoint{6.301716in}{1.432077in}}%
\pgfpathlineto{\pgfqpoint{6.401059in}{1.432228in}}%
\pgfpathlineto{\pgfqpoint{6.586501in}{1.430511in}}%
\pgfpathlineto{\pgfqpoint{6.728893in}{1.432215in}}%
\pgfpathlineto{\pgfqpoint{6.821613in}{1.430547in}}%
\pgfpathlineto{\pgfqpoint{7.003743in}{1.430996in}}%
\pgfpathlineto{\pgfqpoint{7.066661in}{1.430396in}}%
\pgfpathlineto{\pgfqpoint{7.175939in}{1.431923in}}%
\pgfpathlineto{\pgfqpoint{7.344823in}{1.426817in}}%
\pgfpathlineto{\pgfqpoint{7.394494in}{1.429093in}}%
\pgfpathlineto{\pgfqpoint{7.460723in}{1.431823in}}%
\pgfpathlineto{\pgfqpoint{7.517018in}{1.431586in}}%
\pgfpathlineto{\pgfqpoint{7.583247in}{1.431390in}}%
\pgfpathlineto{\pgfqpoint{7.675968in}{1.434533in}}%
\pgfpathlineto{\pgfqpoint{7.735574in}{1.435515in}}%
\pgfpathlineto{\pgfqpoint{7.785245in}{1.433990in}}%
\pgfpathlineto{\pgfqpoint{7.861409in}{1.431243in}}%
\pgfpathlineto{\pgfqpoint{7.997178in}{1.430903in}}%
\pgfpathlineto{\pgfqpoint{8.109767in}{1.429169in}}%
\pgfpathlineto{\pgfqpoint{8.156128in}{1.430233in}}%
\pgfpathlineto{\pgfqpoint{8.232291in}{1.432322in}}%
\pgfpathlineto{\pgfqpoint{8.285274in}{1.431061in}}%
\pgfpathlineto{\pgfqpoint{8.354815in}{1.429528in}}%
\pgfpathlineto{\pgfqpoint{8.523699in}{1.430145in}}%
\pgfpathlineto{\pgfqpoint{8.616419in}{1.434197in}}%
\pgfpathlineto{\pgfqpoint{8.685960in}{1.434732in}}%
\pgfpathlineto{\pgfqpoint{8.834975in}{1.434832in}}%
\pgfpathlineto{\pgfqpoint{8.917761in}{1.434457in}}%
\pgfpathlineto{\pgfqpoint{9.010482in}{1.434258in}}%
\pgfpathlineto{\pgfqpoint{9.176054in}{1.435585in}}%
\pgfpathlineto{\pgfqpoint{9.265463in}{1.431233in}}%
\pgfpathlineto{\pgfqpoint{9.325070in}{1.429771in}}%
\pgfpathlineto{\pgfqpoint{9.556871in}{1.430084in}}%
\pgfpathlineto{\pgfqpoint{9.646280in}{1.432431in}}%
\pgfpathlineto{\pgfqpoint{9.722444in}{1.431320in}}%
\pgfpathlineto{\pgfqpoint{9.818476in}{1.430484in}}%
\pgfpathlineto{\pgfqpoint{9.960868in}{1.429770in}}%
\pgfpathlineto{\pgfqpoint{10.133063in}{1.427469in}}%
\pgfpathlineto{\pgfqpoint{10.192669in}{1.431043in}}%
\pgfpathlineto{\pgfqpoint{10.278767in}{1.435849in}}%
\pgfpathlineto{\pgfqpoint{10.338373in}{1.436829in}}%
\pgfpathlineto{\pgfqpoint{10.414537in}{1.435511in}}%
\pgfpathlineto{\pgfqpoint{10.573486in}{1.432655in}}%
\pgfpathlineto{\pgfqpoint{10.742370in}{1.432058in}}%
\pgfpathlineto{\pgfqpoint{10.904631in}{1.433327in}}%
\pgfpathlineto{\pgfqpoint{10.964237in}{1.430047in}}%
\pgfpathlineto{\pgfqpoint{11.010598in}{1.427850in}}%
\pgfpathlineto{\pgfqpoint{11.043712in}{1.428889in}}%
\pgfpathlineto{\pgfqpoint{11.113252in}{1.431300in}}%
\pgfpathlineto{\pgfqpoint{11.182793in}{1.430934in}}%
\pgfpathlineto{\pgfqpoint{11.255645in}{1.431074in}}%
\pgfpathlineto{\pgfqpoint{11.315251in}{1.433837in}}%
\pgfpathlineto{\pgfqpoint{11.401349in}{1.437833in}}%
\pgfpathlineto{\pgfqpoint{11.464266in}{1.438210in}}%
\pgfpathlineto{\pgfqpoint{11.629839in}{1.437868in}}%
\pgfpathlineto{\pgfqpoint{11.788788in}{1.440519in}}%
\pgfpathlineto{\pgfqpoint{11.848394in}{1.437507in}}%
\pgfpathlineto{\pgfqpoint{11.987475in}{1.429487in}}%
\pgfpathlineto{\pgfqpoint{12.033836in}{1.429849in}}%
\pgfpathlineto{\pgfqpoint{12.162982in}{1.434401in}}%
\pgfpathlineto{\pgfqpoint{12.262326in}{1.437009in}}%
\pgfpathlineto{\pgfqpoint{12.345112in}{1.439548in}}%
\pgfpathlineto{\pgfqpoint{12.480881in}{1.445405in}}%
\pgfpathlineto{\pgfqpoint{12.530553in}{1.444194in}}%
\pgfpathlineto{\pgfqpoint{12.742486in}{1.435945in}}%
\pgfpathlineto{\pgfqpoint{12.805403in}{1.433482in}}%
\pgfpathlineto{\pgfqpoint{12.858387in}{1.428937in}}%
\pgfpathlineto{\pgfqpoint{12.891501in}{1.426609in}}%
\pgfpathlineto{\pgfqpoint{12.917993in}{1.428662in}}%
\pgfpathlineto{\pgfqpoint{12.970976in}{1.432341in}}%
\pgfpathlineto{\pgfqpoint{13.020648in}{1.433346in}}%
\pgfpathlineto{\pgfqpoint{13.153106in}{1.434295in}}%
\pgfpathlineto{\pgfqpoint{13.312055in}{1.439627in}}%
\pgfpathlineto{\pgfqpoint{13.418022in}{1.438766in}}%
\pgfpathlineto{\pgfqpoint{13.500808in}{1.436700in}}%
\pgfpathlineto{\pgfqpoint{13.639889in}{1.432060in}}%
\pgfpathlineto{\pgfqpoint{13.875002in}{1.429628in}}%
\pgfpathlineto{\pgfqpoint{13.964411in}{1.433658in}}%
\pgfpathlineto{\pgfqpoint{14.040574in}{1.434025in}}%
\pgfpathlineto{\pgfqpoint{14.133295in}{1.435145in}}%
\pgfpathlineto{\pgfqpoint{14.288933in}{1.437816in}}%
\pgfpathlineto{\pgfqpoint{14.341916in}{1.435584in}}%
\pgfpathlineto{\pgfqpoint{14.441260in}{1.430560in}}%
\pgfpathlineto{\pgfqpoint{14.514111in}{1.430979in}}%
\pgfpathlineto{\pgfqpoint{14.669750in}{1.433285in}}%
\pgfpathlineto{\pgfqpoint{14.828699in}{1.437722in}}%
\pgfpathlineto{\pgfqpoint{15.010829in}{1.437757in}}%
\pgfpathlineto{\pgfqpoint{15.093615in}{1.437932in}}%
\pgfpathlineto{\pgfqpoint{15.169779in}{1.435647in}}%
\pgfpathlineto{\pgfqpoint{15.305548in}{1.431285in}}%
\pgfpathlineto{\pgfqpoint{15.378400in}{1.431532in}}%
\pgfpathlineto{\pgfqpoint{15.438006in}{1.430961in}}%
\pgfpathlineto{\pgfqpoint{15.500924in}{1.427694in}}%
\pgfpathlineto{\pgfqpoint{15.537350in}{1.426629in}}%
\pgfpathlineto{\pgfqpoint{15.818823in}{1.433770in}}%
\pgfpathlineto{\pgfqpoint{15.924789in}{1.436183in}}%
\pgfpathlineto{\pgfqpoint{16.106919in}{1.437436in}}%
\pgfpathlineto{\pgfqpoint{16.166525in}{1.437362in}}%
\pgfpathlineto{\pgfqpoint{16.252623in}{1.434382in}}%
\pgfpathlineto{\pgfqpoint{16.332097in}{1.432620in}}%
\pgfpathlineto{\pgfqpoint{16.398326in}{1.433530in}}%
\pgfpathlineto{\pgfqpoint{16.471178in}{1.434173in}}%
\pgfpathlineto{\pgfqpoint{16.510916in}{1.432969in}}%
\pgfpathlineto{\pgfqpoint{16.510916in}{1.432969in}}%
\pgfusepath{stroke}%
\end{pgfscope}%
\begin{pgfscope}%
\pgfpathrectangle{\pgfqpoint{2.400000in}{1.081300in}}{\pgfqpoint{14.880000in}{7.569100in}}%
\pgfusepath{clip}%
\pgfsetrectcap%
\pgfsetroundjoin%
\pgfsetlinewidth{1.505625pt}%
\definecolor{currentstroke}{rgb}{0.121569,0.466667,0.705882}%
\pgfsetstrokecolor{currentstroke}%
\pgfsetdash{}{0pt}%
\pgfpathmoveto{\pgfqpoint{3.076364in}{1.425350in}}%
\pgfpathlineto{\pgfqpoint{3.112790in}{1.542003in}}%
\pgfpathlineto{\pgfqpoint{3.129347in}{1.589136in}}%
\pgfpathlineto{\pgfqpoint{3.142593in}{1.621233in}}%
\pgfpathlineto{\pgfqpoint{3.152527in}{1.640618in}}%
\pgfpathlineto{\pgfqpoint{3.162461in}{1.654875in}}%
\pgfpathlineto{\pgfqpoint{3.169084in}{1.661077in}}%
\pgfpathlineto{\pgfqpoint{3.175707in}{1.664389in}}%
\pgfpathlineto{\pgfqpoint{3.182330in}{1.664696in}}%
\pgfpathlineto{\pgfqpoint{3.188953in}{1.661982in}}%
\pgfpathlineto{\pgfqpoint{3.195576in}{1.656342in}}%
\pgfpathlineto{\pgfqpoint{3.202199in}{1.647974in}}%
\pgfpathlineto{\pgfqpoint{3.212133in}{1.630921in}}%
\pgfpathlineto{\pgfqpoint{3.225379in}{1.601552in}}%
\pgfpathlineto{\pgfqpoint{3.241936in}{1.558034in}}%
\pgfpathlineto{\pgfqpoint{3.278362in}{1.458650in}}%
\pgfpathlineto{\pgfqpoint{3.284985in}{1.446522in}}%
\pgfpathlineto{\pgfqpoint{3.288296in}{1.444354in}}%
\pgfpathlineto{\pgfqpoint{3.291608in}{1.445934in}}%
\pgfpathlineto{\pgfqpoint{3.298231in}{1.456219in}}%
\pgfpathlineto{\pgfqpoint{3.321411in}{1.499999in}}%
\pgfpathlineto{\pgfqpoint{3.331345in}{1.514163in}}%
\pgfpathlineto{\pgfqpoint{3.341280in}{1.524056in}}%
\pgfpathlineto{\pgfqpoint{3.347903in}{1.527980in}}%
\pgfpathlineto{\pgfqpoint{3.354525in}{1.529623in}}%
\pgfpathlineto{\pgfqpoint{3.361148in}{1.528926in}}%
\pgfpathlineto{\pgfqpoint{3.367771in}{1.525893in}}%
\pgfpathlineto{\pgfqpoint{3.374394in}{1.520604in}}%
\pgfpathlineto{\pgfqpoint{3.384328in}{1.508800in}}%
\pgfpathlineto{\pgfqpoint{3.394263in}{1.493161in}}%
\pgfpathlineto{\pgfqpoint{3.417443in}{1.453641in}}%
\pgfpathlineto{\pgfqpoint{3.420754in}{1.451161in}}%
\pgfpathlineto{\pgfqpoint{3.424066in}{1.451066in}}%
\pgfpathlineto{\pgfqpoint{3.427377in}{1.453449in}}%
\pgfpathlineto{\pgfqpoint{3.434000in}{1.463416in}}%
\pgfpathlineto{\pgfqpoint{3.447246in}{1.490807in}}%
\pgfpathlineto{\pgfqpoint{3.467115in}{1.530964in}}%
\pgfpathlineto{\pgfqpoint{3.477049in}{1.547216in}}%
\pgfpathlineto{\pgfqpoint{3.486983in}{1.559662in}}%
\pgfpathlineto{\pgfqpoint{3.493606in}{1.565489in}}%
\pgfpathlineto{\pgfqpoint{3.500229in}{1.569144in}}%
\pgfpathlineto{\pgfqpoint{3.506852in}{1.570517in}}%
\pgfpathlineto{\pgfqpoint{3.513475in}{1.569552in}}%
\pgfpathlineto{\pgfqpoint{3.520098in}{1.566261in}}%
\pgfpathlineto{\pgfqpoint{3.526721in}{1.560730in}}%
\pgfpathlineto{\pgfqpoint{3.536655in}{1.548621in}}%
\pgfpathlineto{\pgfqpoint{3.549901in}{1.526988in}}%
\pgfpathlineto{\pgfqpoint{3.569770in}{1.492736in}}%
\pgfpathlineto{\pgfqpoint{3.576393in}{1.485685in}}%
\pgfpathlineto{\pgfqpoint{3.579704in}{1.484025in}}%
\pgfpathlineto{\pgfqpoint{3.583015in}{1.483859in}}%
\pgfpathlineto{\pgfqpoint{3.586327in}{1.485230in}}%
\pgfpathlineto{\pgfqpoint{3.592950in}{1.492124in}}%
\pgfpathlineto{\pgfqpoint{3.599573in}{1.503084in}}%
\pgfpathlineto{\pgfqpoint{3.612819in}{1.530609in}}%
\pgfpathlineto{\pgfqpoint{3.632687in}{1.572126in}}%
\pgfpathlineto{\pgfqpoint{3.645933in}{1.594533in}}%
\pgfpathlineto{\pgfqpoint{3.655867in}{1.607013in}}%
\pgfpathlineto{\pgfqpoint{3.662490in}{1.612905in}}%
\pgfpathlineto{\pgfqpoint{3.669113in}{1.616676in}}%
\pgfpathlineto{\pgfqpoint{3.675736in}{1.618215in}}%
\pgfpathlineto{\pgfqpoint{3.682359in}{1.617447in}}%
\pgfpathlineto{\pgfqpoint{3.688982in}{1.614351in}}%
\pgfpathlineto{\pgfqpoint{3.695605in}{1.608958in}}%
\pgfpathlineto{\pgfqpoint{3.705539in}{1.596794in}}%
\pgfpathlineto{\pgfqpoint{3.715473in}{1.580355in}}%
\pgfpathlineto{\pgfqpoint{3.732031in}{1.546892in}}%
\pgfpathlineto{\pgfqpoint{3.745277in}{1.521068in}}%
\pgfpathlineto{\pgfqpoint{3.751899in}{1.511946in}}%
\pgfpathlineto{\pgfqpoint{3.758522in}{1.507716in}}%
\pgfpathlineto{\pgfqpoint{3.761834in}{1.507875in}}%
\pgfpathlineto{\pgfqpoint{3.765145in}{1.509619in}}%
\pgfpathlineto{\pgfqpoint{3.771768in}{1.517463in}}%
\pgfpathlineto{\pgfqpoint{3.778391in}{1.529846in}}%
\pgfpathlineto{\pgfqpoint{3.791637in}{1.561880in}}%
\pgfpathlineto{\pgfqpoint{3.814817in}{1.620232in}}%
\pgfpathlineto{\pgfqpoint{3.828063in}{1.646911in}}%
\pgfpathlineto{\pgfqpoint{3.837997in}{1.661306in}}%
\pgfpathlineto{\pgfqpoint{3.844620in}{1.667608in}}%
\pgfpathlineto{\pgfqpoint{3.851243in}{1.670932in}}%
\pgfpathlineto{\pgfqpoint{3.857866in}{1.671027in}}%
\pgfpathlineto{\pgfqpoint{3.864489in}{1.667687in}}%
\pgfpathlineto{\pgfqpoint{3.871112in}{1.660767in}}%
\pgfpathlineto{\pgfqpoint{3.877735in}{1.650200in}}%
\pgfpathlineto{\pgfqpoint{3.884357in}{1.636001in}}%
\pgfpathlineto{\pgfqpoint{3.894292in}{1.608192in}}%
\pgfpathlineto{\pgfqpoint{3.904226in}{1.573466in}}%
\pgfpathlineto{\pgfqpoint{3.927406in}{1.485123in}}%
\pgfpathlineto{\pgfqpoint{3.930718in}{1.479096in}}%
\pgfpathlineto{\pgfqpoint{3.934029in}{1.478768in}}%
\pgfpathlineto{\pgfqpoint{3.937341in}{1.484574in}}%
\pgfpathlineto{\pgfqpoint{3.943964in}{1.509075in}}%
\pgfpathlineto{\pgfqpoint{3.953898in}{1.559649in}}%
\pgfpathlineto{\pgfqpoint{4.000258in}{1.811384in}}%
\pgfpathlineto{\pgfqpoint{4.013504in}{1.869661in}}%
\pgfpathlineto{\pgfqpoint{4.023438in}{1.905897in}}%
\pgfpathlineto{\pgfqpoint{4.033373in}{1.934436in}}%
\pgfpathlineto{\pgfqpoint{4.039996in}{1.948654in}}%
\pgfpathlineto{\pgfqpoint{4.046618in}{1.958702in}}%
\pgfpathlineto{\pgfqpoint{4.053241in}{1.964356in}}%
\pgfpathlineto{\pgfqpoint{4.059864in}{1.965455in}}%
\pgfpathlineto{\pgfqpoint{4.066487in}{1.961911in}}%
\pgfpathlineto{\pgfqpoint{4.073110in}{1.953723in}}%
\pgfpathlineto{\pgfqpoint{4.079733in}{1.940973in}}%
\pgfpathlineto{\pgfqpoint{4.086356in}{1.923835in}}%
\pgfpathlineto{\pgfqpoint{4.096290in}{1.890478in}}%
\pgfpathlineto{\pgfqpoint{4.106225in}{1.849015in}}%
\pgfpathlineto{\pgfqpoint{4.119470in}{1.783811in}}%
\pgfpathlineto{\pgfqpoint{4.155896in}{1.592967in}}%
\pgfpathlineto{\pgfqpoint{4.162519in}{1.571020in}}%
\pgfpathlineto{\pgfqpoint{4.165831in}{1.564297in}}%
\pgfpathlineto{\pgfqpoint{4.169142in}{1.561001in}}%
\pgfpathlineto{\pgfqpoint{4.172454in}{1.561321in}}%
\pgfpathlineto{\pgfqpoint{4.175765in}{1.565136in}}%
\pgfpathlineto{\pgfqpoint{4.182388in}{1.581515in}}%
\pgfpathlineto{\pgfqpoint{4.192322in}{1.619576in}}%
\pgfpathlineto{\pgfqpoint{4.222125in}{1.744438in}}%
\pgfpathlineto{\pgfqpoint{4.232060in}{1.775967in}}%
\pgfpathlineto{\pgfqpoint{4.241994in}{1.799562in}}%
\pgfpathlineto{\pgfqpoint{4.248617in}{1.810563in}}%
\pgfpathlineto{\pgfqpoint{4.255240in}{1.817710in}}%
\pgfpathlineto{\pgfqpoint{4.261863in}{1.821045in}}%
\pgfpathlineto{\pgfqpoint{4.268486in}{1.820687in}}%
\pgfpathlineto{\pgfqpoint{4.275108in}{1.816825in}}%
\pgfpathlineto{\pgfqpoint{4.281731in}{1.809694in}}%
\pgfpathlineto{\pgfqpoint{4.288354in}{1.799573in}}%
\pgfpathlineto{\pgfqpoint{4.298289in}{1.779449in}}%
\pgfpathlineto{\pgfqpoint{4.308223in}{1.754359in}}%
\pgfpathlineto{\pgfqpoint{4.321469in}{1.715102in}}%
\pgfpathlineto{\pgfqpoint{4.344649in}{1.637944in}}%
\pgfpathlineto{\pgfqpoint{4.364518in}{1.573807in}}%
\pgfpathlineto{\pgfqpoint{4.374452in}{1.547411in}}%
\pgfpathlineto{\pgfqpoint{4.381075in}{1.533945in}}%
\pgfpathlineto{\pgfqpoint{4.387698in}{1.525050in}}%
\pgfpathlineto{\pgfqpoint{4.394321in}{1.521480in}}%
\pgfpathlineto{\pgfqpoint{4.397632in}{1.521708in}}%
\pgfpathlineto{\pgfqpoint{4.404255in}{1.525681in}}%
\pgfpathlineto{\pgfqpoint{4.410878in}{1.533211in}}%
\pgfpathlineto{\pgfqpoint{4.424124in}{1.553434in}}%
\pgfpathlineto{\pgfqpoint{4.440681in}{1.578767in}}%
\pgfpathlineto{\pgfqpoint{4.450615in}{1.590992in}}%
\pgfpathlineto{\pgfqpoint{4.460550in}{1.600075in}}%
\pgfpathlineto{\pgfqpoint{4.470484in}{1.605767in}}%
\pgfpathlineto{\pgfqpoint{4.480418in}{1.608096in}}%
\pgfpathlineto{\pgfqpoint{4.490353in}{1.607299in}}%
\pgfpathlineto{\pgfqpoint{4.500287in}{1.603774in}}%
\pgfpathlineto{\pgfqpoint{4.513533in}{1.595755in}}%
\pgfpathlineto{\pgfqpoint{4.530090in}{1.582506in}}%
\pgfpathlineto{\pgfqpoint{4.559893in}{1.558447in}}%
\pgfpathlineto{\pgfqpoint{4.576450in}{1.548294in}}%
\pgfpathlineto{\pgfqpoint{4.596319in}{1.539062in}}%
\pgfpathlineto{\pgfqpoint{4.649302in}{1.518143in}}%
\pgfpathlineto{\pgfqpoint{4.662548in}{1.514575in}}%
\pgfpathlineto{\pgfqpoint{4.672482in}{1.513564in}}%
\pgfpathlineto{\pgfqpoint{4.682417in}{1.514719in}}%
\pgfpathlineto{\pgfqpoint{4.692351in}{1.518674in}}%
\pgfpathlineto{\pgfqpoint{4.702286in}{1.525859in}}%
\pgfpathlineto{\pgfqpoint{4.712220in}{1.536401in}}%
\pgfpathlineto{\pgfqpoint{4.722154in}{1.550143in}}%
\pgfpathlineto{\pgfqpoint{4.735400in}{1.572834in}}%
\pgfpathlineto{\pgfqpoint{4.751957in}{1.606805in}}%
\pgfpathlineto{\pgfqpoint{4.771826in}{1.653398in}}%
\pgfpathlineto{\pgfqpoint{4.801629in}{1.730280in}}%
\pgfpathlineto{\pgfqpoint{4.844678in}{1.841878in}}%
\pgfpathlineto{\pgfqpoint{4.857924in}{1.870881in}}%
\pgfpathlineto{\pgfqpoint{4.867858in}{1.888850in}}%
\pgfpathlineto{\pgfqpoint{4.877792in}{1.902404in}}%
\pgfpathlineto{\pgfqpoint{4.884415in}{1.908445in}}%
\pgfpathlineto{\pgfqpoint{4.891038in}{1.911714in}}%
\pgfpathlineto{\pgfqpoint{4.897661in}{1.911909in}}%
\pgfpathlineto{\pgfqpoint{4.904284in}{1.908762in}}%
\pgfpathlineto{\pgfqpoint{4.910907in}{1.902048in}}%
\pgfpathlineto{\pgfqpoint{4.917530in}{1.891603in}}%
\pgfpathlineto{\pgfqpoint{4.924153in}{1.877335in}}%
\pgfpathlineto{\pgfqpoint{4.934087in}{1.848779in}}%
\pgfpathlineto{\pgfqpoint{4.944021in}{1.812034in}}%
\pgfpathlineto{\pgfqpoint{4.957267in}{1.752113in}}%
\pgfpathlineto{\pgfqpoint{4.990382in}{1.588783in}}%
\pgfpathlineto{\pgfqpoint{4.997005in}{1.570092in}}%
\pgfpathlineto{\pgfqpoint{5.000316in}{1.565694in}}%
\pgfpathlineto{\pgfqpoint{5.003627in}{1.565166in}}%
\pgfpathlineto{\pgfqpoint{5.006939in}{1.568560in}}%
\pgfpathlineto{\pgfqpoint{5.010250in}{1.575585in}}%
\pgfpathlineto{\pgfqpoint{5.016873in}{1.598311in}}%
\pgfpathlineto{\pgfqpoint{5.026808in}{1.645476in}}%
\pgfpathlineto{\pgfqpoint{5.066545in}{1.849908in}}%
\pgfpathlineto{\pgfqpoint{5.079791in}{1.902937in}}%
\pgfpathlineto{\pgfqpoint{5.089725in}{1.934828in}}%
\pgfpathlineto{\pgfqpoint{5.099660in}{1.959192in}}%
\pgfpathlineto{\pgfqpoint{5.106282in}{1.970971in}}%
\pgfpathlineto{\pgfqpoint{5.112905in}{1.979013in}}%
\pgfpathlineto{\pgfqpoint{5.119528in}{1.983199in}}%
\pgfpathlineto{\pgfqpoint{5.126151in}{1.983434in}}%
\pgfpathlineto{\pgfqpoint{5.132774in}{1.979652in}}%
\pgfpathlineto{\pgfqpoint{5.139397in}{1.971814in}}%
\pgfpathlineto{\pgfqpoint{5.146020in}{1.959919in}}%
\pgfpathlineto{\pgfqpoint{5.152643in}{1.944000in}}%
\pgfpathlineto{\pgfqpoint{5.162577in}{1.912759in}}%
\pgfpathlineto{\pgfqpoint{5.172511in}{1.873103in}}%
\pgfpathlineto{\pgfqpoint{5.185757in}{1.808429in}}%
\pgfpathlineto{\pgfqpoint{5.202314in}{1.712518in}}%
\pgfpathlineto{\pgfqpoint{5.228806in}{1.551787in}}%
\pgfpathlineto{\pgfqpoint{5.235429in}{1.524974in}}%
\pgfpathlineto{\pgfqpoint{5.238740in}{1.518382in}}%
\pgfpathlineto{\pgfqpoint{5.242052in}{1.517842in}}%
\pgfpathlineto{\pgfqpoint{5.245363in}{1.523362in}}%
\pgfpathlineto{\pgfqpoint{5.251986in}{1.547957in}}%
\pgfpathlineto{\pgfqpoint{5.261921in}{1.600539in}}%
\pgfpathlineto{\pgfqpoint{5.285101in}{1.726759in}}%
\pgfpathlineto{\pgfqpoint{5.298347in}{1.785128in}}%
\pgfpathlineto{\pgfqpoint{5.308281in}{1.819277in}}%
\pgfpathlineto{\pgfqpoint{5.318215in}{1.844278in}}%
\pgfpathlineto{\pgfqpoint{5.324838in}{1.855658in}}%
\pgfpathlineto{\pgfqpoint{5.331461in}{1.862763in}}%
\pgfpathlineto{\pgfqpoint{5.338084in}{1.865622in}}%
\pgfpathlineto{\pgfqpoint{5.344707in}{1.864317in}}%
\pgfpathlineto{\pgfqpoint{5.351330in}{1.858980in}}%
\pgfpathlineto{\pgfqpoint{5.357953in}{1.849784in}}%
\pgfpathlineto{\pgfqpoint{5.364575in}{1.836941in}}%
\pgfpathlineto{\pgfqpoint{5.374510in}{1.811379in}}%
\pgfpathlineto{\pgfqpoint{5.384444in}{1.779105in}}%
\pgfpathlineto{\pgfqpoint{5.397690in}{1.727514in}}%
\pgfpathlineto{\pgfqpoint{5.417559in}{1.638310in}}%
\pgfpathlineto{\pgfqpoint{5.434116in}{1.564454in}}%
\pgfpathlineto{\pgfqpoint{5.440739in}{1.540555in}}%
\pgfpathlineto{\pgfqpoint{5.447362in}{1.524884in}}%
\pgfpathlineto{\pgfqpoint{5.450673in}{1.521565in}}%
\pgfpathlineto{\pgfqpoint{5.453985in}{1.521750in}}%
\pgfpathlineto{\pgfqpoint{5.457296in}{1.525372in}}%
\pgfpathlineto{\pgfqpoint{5.463919in}{1.541064in}}%
\pgfpathlineto{\pgfqpoint{5.473853in}{1.576958in}}%
\pgfpathlineto{\pgfqpoint{5.506968in}{1.708077in}}%
\pgfpathlineto{\pgfqpoint{5.520214in}{1.748514in}}%
\pgfpathlineto{\pgfqpoint{5.530148in}{1.771652in}}%
\pgfpathlineto{\pgfqpoint{5.540082in}{1.787962in}}%
\pgfpathlineto{\pgfqpoint{5.546705in}{1.794868in}}%
\pgfpathlineto{\pgfqpoint{5.553328in}{1.798545in}}%
\pgfpathlineto{\pgfqpoint{5.559951in}{1.798999in}}%
\pgfpathlineto{\pgfqpoint{5.566574in}{1.796276in}}%
\pgfpathlineto{\pgfqpoint{5.573197in}{1.790466in}}%
\pgfpathlineto{\pgfqpoint{5.579820in}{1.781696in}}%
\pgfpathlineto{\pgfqpoint{5.589754in}{1.763356in}}%
\pgfpathlineto{\pgfqpoint{5.599688in}{1.739407in}}%
\pgfpathlineto{\pgfqpoint{5.612934in}{1.700148in}}%
\pgfpathlineto{\pgfqpoint{5.629491in}{1.642391in}}%
\pgfpathlineto{\pgfqpoint{5.665917in}{1.509776in}}%
\pgfpathlineto{\pgfqpoint{5.672540in}{1.493994in}}%
\pgfpathlineto{\pgfqpoint{5.675852in}{1.489410in}}%
\pgfpathlineto{\pgfqpoint{5.679163in}{1.487712in}}%
\pgfpathlineto{\pgfqpoint{5.682475in}{1.489076in}}%
\pgfpathlineto{\pgfqpoint{5.685786in}{1.493245in}}%
\pgfpathlineto{\pgfqpoint{5.692409in}{1.507750in}}%
\pgfpathlineto{\pgfqpoint{5.702343in}{1.537303in}}%
\pgfpathlineto{\pgfqpoint{5.732146in}{1.630772in}}%
\pgfpathlineto{\pgfqpoint{5.745392in}{1.665606in}}%
\pgfpathlineto{\pgfqpoint{5.758638in}{1.693886in}}%
\pgfpathlineto{\pgfqpoint{5.768572in}{1.710231in}}%
\pgfpathlineto{\pgfqpoint{5.778507in}{1.722146in}}%
\pgfpathlineto{\pgfqpoint{5.785130in}{1.727585in}}%
\pgfpathlineto{\pgfqpoint{5.791753in}{1.731040in}}%
\pgfpathlineto{\pgfqpoint{5.798375in}{1.732564in}}%
\pgfpathlineto{\pgfqpoint{5.804998in}{1.732246in}}%
\pgfpathlineto{\pgfqpoint{5.811621in}{1.730207in}}%
\pgfpathlineto{\pgfqpoint{5.821556in}{1.724270in}}%
\pgfpathlineto{\pgfqpoint{5.831490in}{1.715453in}}%
\pgfpathlineto{\pgfqpoint{5.844736in}{1.700513in}}%
\pgfpathlineto{\pgfqpoint{5.891096in}{1.644490in}}%
\pgfpathlineto{\pgfqpoint{5.904342in}{1.632771in}}%
\pgfpathlineto{\pgfqpoint{5.917588in}{1.623595in}}%
\pgfpathlineto{\pgfqpoint{5.934145in}{1.614747in}}%
\pgfpathlineto{\pgfqpoint{5.977194in}{1.593335in}}%
\pgfpathlineto{\pgfqpoint{5.993751in}{1.582044in}}%
\pgfpathlineto{\pgfqpoint{6.013620in}{1.565418in}}%
\pgfpathlineto{\pgfqpoint{6.036800in}{1.542845in}}%
\pgfpathlineto{\pgfqpoint{6.083160in}{1.496460in}}%
\pgfpathlineto{\pgfqpoint{6.096406in}{1.486525in}}%
\pgfpathlineto{\pgfqpoint{6.106340in}{1.482180in}}%
\pgfpathlineto{\pgfqpoint{6.112963in}{1.481471in}}%
\pgfpathlineto{\pgfqpoint{6.119586in}{1.482836in}}%
\pgfpathlineto{\pgfqpoint{6.126209in}{1.486320in}}%
\pgfpathlineto{\pgfqpoint{6.136143in}{1.495102in}}%
\pgfpathlineto{\pgfqpoint{6.146078in}{1.507184in}}%
\pgfpathlineto{\pgfqpoint{6.162635in}{1.531675in}}%
\pgfpathlineto{\pgfqpoint{6.199061in}{1.587242in}}%
\pgfpathlineto{\pgfqpoint{6.212307in}{1.602988in}}%
\pgfpathlineto{\pgfqpoint{6.222241in}{1.611977in}}%
\pgfpathlineto{\pgfqpoint{6.232175in}{1.618142in}}%
\pgfpathlineto{\pgfqpoint{6.242110in}{1.621257in}}%
\pgfpathlineto{\pgfqpoint{6.252044in}{1.621245in}}%
\pgfpathlineto{\pgfqpoint{6.261978in}{1.618175in}}%
\pgfpathlineto{\pgfqpoint{6.271913in}{1.612258in}}%
\pgfpathlineto{\pgfqpoint{6.281847in}{1.603830in}}%
\pgfpathlineto{\pgfqpoint{6.295093in}{1.589443in}}%
\pgfpathlineto{\pgfqpoint{6.314962in}{1.563596in}}%
\pgfpathlineto{\pgfqpoint{6.348076in}{1.519602in}}%
\pgfpathlineto{\pgfqpoint{6.364633in}{1.501825in}}%
\pgfpathlineto{\pgfqpoint{6.377879in}{1.491238in}}%
\pgfpathlineto{\pgfqpoint{6.387814in}{1.485733in}}%
\pgfpathlineto{\pgfqpoint{6.397748in}{1.482276in}}%
\pgfpathlineto{\pgfqpoint{6.410994in}{1.480341in}}%
\pgfpathlineto{\pgfqpoint{6.427551in}{1.480725in}}%
\pgfpathlineto{\pgfqpoint{6.450731in}{1.483656in}}%
\pgfpathlineto{\pgfqpoint{6.467288in}{1.487750in}}%
\pgfpathlineto{\pgfqpoint{6.477223in}{1.492262in}}%
\pgfpathlineto{\pgfqpoint{6.487157in}{1.499455in}}%
\pgfpathlineto{\pgfqpoint{6.497091in}{1.510180in}}%
\pgfpathlineto{\pgfqpoint{6.507026in}{1.524898in}}%
\pgfpathlineto{\pgfqpoint{6.516960in}{1.543591in}}%
\pgfpathlineto{\pgfqpoint{6.530206in}{1.573955in}}%
\pgfpathlineto{\pgfqpoint{6.546763in}{1.618243in}}%
\pgfpathlineto{\pgfqpoint{6.589812in}{1.737154in}}%
\pgfpathlineto{\pgfqpoint{6.603058in}{1.766493in}}%
\pgfpathlineto{\pgfqpoint{6.612992in}{1.784426in}}%
\pgfpathlineto{\pgfqpoint{6.622926in}{1.798435in}}%
\pgfpathlineto{\pgfqpoint{6.632861in}{1.808351in}}%
\pgfpathlineto{\pgfqpoint{6.642795in}{1.814204in}}%
\pgfpathlineto{\pgfqpoint{6.649418in}{1.815944in}}%
\pgfpathlineto{\pgfqpoint{6.656041in}{1.816065in}}%
\pgfpathlineto{\pgfqpoint{6.665975in}{1.813467in}}%
\pgfpathlineto{\pgfqpoint{6.675910in}{1.807930in}}%
\pgfpathlineto{\pgfqpoint{6.685844in}{1.799904in}}%
\pgfpathlineto{\pgfqpoint{6.699090in}{1.786029in}}%
\pgfpathlineto{\pgfqpoint{6.715647in}{1.764612in}}%
\pgfpathlineto{\pgfqpoint{6.732204in}{1.739470in}}%
\pgfpathlineto{\pgfqpoint{6.752073in}{1.704914in}}%
\pgfpathlineto{\pgfqpoint{6.775253in}{1.659279in}}%
\pgfpathlineto{\pgfqpoint{6.828236in}{1.551327in}}%
\pgfpathlineto{\pgfqpoint{6.841482in}{1.530343in}}%
\pgfpathlineto{\pgfqpoint{6.851416in}{1.517956in}}%
\pgfpathlineto{\pgfqpoint{6.861351in}{1.508920in}}%
\pgfpathlineto{\pgfqpoint{6.871285in}{1.503397in}}%
\pgfpathlineto{\pgfqpoint{6.881220in}{1.501230in}}%
\pgfpathlineto{\pgfqpoint{6.891154in}{1.501956in}}%
\pgfpathlineto{\pgfqpoint{6.901088in}{1.504943in}}%
\pgfpathlineto{\pgfqpoint{6.914334in}{1.511264in}}%
\pgfpathlineto{\pgfqpoint{6.954071in}{1.532126in}}%
\pgfpathlineto{\pgfqpoint{6.967317in}{1.535404in}}%
\pgfpathlineto{\pgfqpoint{6.977252in}{1.535654in}}%
\pgfpathlineto{\pgfqpoint{6.987186in}{1.533781in}}%
\pgfpathlineto{\pgfqpoint{6.997120in}{1.529729in}}%
\pgfpathlineto{\pgfqpoint{7.007055in}{1.523593in}}%
\pgfpathlineto{\pgfqpoint{7.020300in}{1.512642in}}%
\pgfpathlineto{\pgfqpoint{7.053415in}{1.481827in}}%
\pgfpathlineto{\pgfqpoint{7.060038in}{1.478494in}}%
\pgfpathlineto{\pgfqpoint{7.066661in}{1.477600in}}%
\pgfpathlineto{\pgfqpoint{7.073284in}{1.479525in}}%
\pgfpathlineto{\pgfqpoint{7.079907in}{1.484155in}}%
\pgfpathlineto{\pgfqpoint{7.089841in}{1.495054in}}%
\pgfpathlineto{\pgfqpoint{7.103087in}{1.513984in}}%
\pgfpathlineto{\pgfqpoint{7.149447in}{1.584412in}}%
\pgfpathlineto{\pgfqpoint{7.162693in}{1.600258in}}%
\pgfpathlineto{\pgfqpoint{7.175939in}{1.612793in}}%
\pgfpathlineto{\pgfqpoint{7.185873in}{1.619781in}}%
\pgfpathlineto{\pgfqpoint{7.195807in}{1.624619in}}%
\pgfpathlineto{\pgfqpoint{7.205742in}{1.627322in}}%
\pgfpathlineto{\pgfqpoint{7.215676in}{1.627987in}}%
\pgfpathlineto{\pgfqpoint{7.225610in}{1.626776in}}%
\pgfpathlineto{\pgfqpoint{7.238856in}{1.622611in}}%
\pgfpathlineto{\pgfqpoint{7.252102in}{1.616043in}}%
\pgfpathlineto{\pgfqpoint{7.268659in}{1.605258in}}%
\pgfpathlineto{\pgfqpoint{7.288528in}{1.589534in}}%
\pgfpathlineto{\pgfqpoint{7.311708in}{1.568101in}}%
\pgfpathlineto{\pgfqpoint{7.334888in}{1.543545in}}%
\pgfpathlineto{\pgfqpoint{7.387871in}{1.484882in}}%
\pgfpathlineto{\pgfqpoint{7.397806in}{1.477930in}}%
\pgfpathlineto{\pgfqpoint{7.404429in}{1.475518in}}%
\pgfpathlineto{\pgfqpoint{7.411052in}{1.475464in}}%
\pgfpathlineto{\pgfqpoint{7.417674in}{1.478062in}}%
\pgfpathlineto{\pgfqpoint{7.424297in}{1.483292in}}%
\pgfpathlineto{\pgfqpoint{7.434232in}{1.495485in}}%
\pgfpathlineto{\pgfqpoint{7.444166in}{1.511919in}}%
\pgfpathlineto{\pgfqpoint{7.457412in}{1.538886in}}%
\pgfpathlineto{\pgfqpoint{7.473969in}{1.578390in}}%
\pgfpathlineto{\pgfqpoint{7.520329in}{1.693564in}}%
\pgfpathlineto{\pgfqpoint{7.533575in}{1.717994in}}%
\pgfpathlineto{\pgfqpoint{7.543510in}{1.731307in}}%
\pgfpathlineto{\pgfqpoint{7.550132in}{1.737415in}}%
\pgfpathlineto{\pgfqpoint{7.556755in}{1.741146in}}%
\pgfpathlineto{\pgfqpoint{7.563378in}{1.742415in}}%
\pgfpathlineto{\pgfqpoint{7.570001in}{1.741193in}}%
\pgfpathlineto{\pgfqpoint{7.576624in}{1.737499in}}%
\pgfpathlineto{\pgfqpoint{7.583247in}{1.731401in}}%
\pgfpathlineto{\pgfqpoint{7.593181in}{1.718002in}}%
\pgfpathlineto{\pgfqpoint{7.603116in}{1.699981in}}%
\pgfpathlineto{\pgfqpoint{7.616361in}{1.669945in}}%
\pgfpathlineto{\pgfqpoint{7.632919in}{1.625448in}}%
\pgfpathlineto{\pgfqpoint{7.666033in}{1.532345in}}%
\pgfpathlineto{\pgfqpoint{7.675968in}{1.512045in}}%
\pgfpathlineto{\pgfqpoint{7.682590in}{1.504371in}}%
\pgfpathlineto{\pgfqpoint{7.685902in}{1.502898in}}%
\pgfpathlineto{\pgfqpoint{7.689213in}{1.503136in}}%
\pgfpathlineto{\pgfqpoint{7.692525in}{1.505046in}}%
\pgfpathlineto{\pgfqpoint{7.699148in}{1.513238in}}%
\pgfpathlineto{\pgfqpoint{7.705771in}{1.525745in}}%
\pgfpathlineto{\pgfqpoint{7.719016in}{1.557367in}}%
\pgfpathlineto{\pgfqpoint{7.748819in}{1.631244in}}%
\pgfpathlineto{\pgfqpoint{7.762065in}{1.657903in}}%
\pgfpathlineto{\pgfqpoint{7.772000in}{1.673598in}}%
\pgfpathlineto{\pgfqpoint{7.781934in}{1.684861in}}%
\pgfpathlineto{\pgfqpoint{7.788557in}{1.689638in}}%
\pgfpathlineto{\pgfqpoint{7.795180in}{1.692083in}}%
\pgfpathlineto{\pgfqpoint{7.801803in}{1.692110in}}%
\pgfpathlineto{\pgfqpoint{7.808426in}{1.689670in}}%
\pgfpathlineto{\pgfqpoint{7.815048in}{1.684758in}}%
\pgfpathlineto{\pgfqpoint{7.821671in}{1.677416in}}%
\pgfpathlineto{\pgfqpoint{7.831606in}{1.662057in}}%
\pgfpathlineto{\pgfqpoint{7.841540in}{1.641985in}}%
\pgfpathlineto{\pgfqpoint{7.854786in}{1.609426in}}%
\pgfpathlineto{\pgfqpoint{7.884589in}{1.531317in}}%
\pgfpathlineto{\pgfqpoint{7.891212in}{1.519222in}}%
\pgfpathlineto{\pgfqpoint{7.897835in}{1.512066in}}%
\pgfpathlineto{\pgfqpoint{7.901146in}{1.510704in}}%
\pgfpathlineto{\pgfqpoint{7.904458in}{1.510863in}}%
\pgfpathlineto{\pgfqpoint{7.907769in}{1.512458in}}%
\pgfpathlineto{\pgfqpoint{7.914392in}{1.519226in}}%
\pgfpathlineto{\pgfqpoint{7.924326in}{1.534982in}}%
\pgfpathlineto{\pgfqpoint{7.947506in}{1.575304in}}%
\pgfpathlineto{\pgfqpoint{7.957441in}{1.588863in}}%
\pgfpathlineto{\pgfqpoint{7.967375in}{1.598949in}}%
\pgfpathlineto{\pgfqpoint{7.977309in}{1.605337in}}%
\pgfpathlineto{\pgfqpoint{7.983932in}{1.607537in}}%
\pgfpathlineto{\pgfqpoint{7.990555in}{1.608133in}}%
\pgfpathlineto{\pgfqpoint{7.997178in}{1.607188in}}%
\pgfpathlineto{\pgfqpoint{8.007112in}{1.603048in}}%
\pgfpathlineto{\pgfqpoint{8.017047in}{1.595916in}}%
\pgfpathlineto{\pgfqpoint{8.026981in}{1.586143in}}%
\pgfpathlineto{\pgfqpoint{8.040227in}{1.569702in}}%
\pgfpathlineto{\pgfqpoint{8.060096in}{1.540253in}}%
\pgfpathlineto{\pgfqpoint{8.079964in}{1.510936in}}%
\pgfpathlineto{\pgfqpoint{8.089899in}{1.499826in}}%
\pgfpathlineto{\pgfqpoint{8.096522in}{1.495073in}}%
\pgfpathlineto{\pgfqpoint{8.103145in}{1.493011in}}%
\pgfpathlineto{\pgfqpoint{8.109767in}{1.493738in}}%
\pgfpathlineto{\pgfqpoint{8.116390in}{1.496902in}}%
\pgfpathlineto{\pgfqpoint{8.126325in}{1.504749in}}%
\pgfpathlineto{\pgfqpoint{8.156128in}{1.531331in}}%
\pgfpathlineto{\pgfqpoint{8.166062in}{1.537060in}}%
\pgfpathlineto{\pgfqpoint{8.175996in}{1.540308in}}%
\pgfpathlineto{\pgfqpoint{8.185931in}{1.540996in}}%
\pgfpathlineto{\pgfqpoint{8.195865in}{1.539295in}}%
\pgfpathlineto{\pgfqpoint{8.209111in}{1.534068in}}%
\pgfpathlineto{\pgfqpoint{8.238914in}{1.519607in}}%
\pgfpathlineto{\pgfqpoint{8.248848in}{1.517927in}}%
\pgfpathlineto{\pgfqpoint{8.258783in}{1.519357in}}%
\pgfpathlineto{\pgfqpoint{8.268717in}{1.523955in}}%
\pgfpathlineto{\pgfqpoint{8.278651in}{1.531095in}}%
\pgfpathlineto{\pgfqpoint{8.298520in}{1.549041in}}%
\pgfpathlineto{\pgfqpoint{8.318389in}{1.566008in}}%
\pgfpathlineto{\pgfqpoint{8.331635in}{1.574487in}}%
\pgfpathlineto{\pgfqpoint{8.344880in}{1.579884in}}%
\pgfpathlineto{\pgfqpoint{8.354815in}{1.581724in}}%
\pgfpathlineto{\pgfqpoint{8.364749in}{1.581633in}}%
\pgfpathlineto{\pgfqpoint{8.374683in}{1.579642in}}%
\pgfpathlineto{\pgfqpoint{8.387929in}{1.574185in}}%
\pgfpathlineto{\pgfqpoint{8.401175in}{1.565860in}}%
\pgfpathlineto{\pgfqpoint{8.417732in}{1.552308in}}%
\pgfpathlineto{\pgfqpoint{8.457470in}{1.517420in}}%
\pgfpathlineto{\pgfqpoint{8.467404in}{1.511901in}}%
\pgfpathlineto{\pgfqpoint{8.477338in}{1.509141in}}%
\pgfpathlineto{\pgfqpoint{8.487273in}{1.509520in}}%
\pgfpathlineto{\pgfqpoint{8.497207in}{1.512968in}}%
\pgfpathlineto{\pgfqpoint{8.507141in}{1.519046in}}%
\pgfpathlineto{\pgfqpoint{8.520387in}{1.530222in}}%
\pgfpathlineto{\pgfqpoint{8.536944in}{1.547373in}}%
\pgfpathlineto{\pgfqpoint{8.593239in}{1.609263in}}%
\pgfpathlineto{\pgfqpoint{8.606485in}{1.619295in}}%
\pgfpathlineto{\pgfqpoint{8.616419in}{1.624253in}}%
\pgfpathlineto{\pgfqpoint{8.626354in}{1.626527in}}%
\pgfpathlineto{\pgfqpoint{8.636288in}{1.625770in}}%
\pgfpathlineto{\pgfqpoint{8.646222in}{1.621757in}}%
\pgfpathlineto{\pgfqpoint{8.656157in}{1.614417in}}%
\pgfpathlineto{\pgfqpoint{8.666091in}{1.603870in}}%
\pgfpathlineto{\pgfqpoint{8.679337in}{1.585457in}}%
\pgfpathlineto{\pgfqpoint{8.699206in}{1.552286in}}%
\pgfpathlineto{\pgfqpoint{8.712451in}{1.531531in}}%
\pgfpathlineto{\pgfqpoint{8.719074in}{1.523693in}}%
\pgfpathlineto{\pgfqpoint{8.725697in}{1.518769in}}%
\pgfpathlineto{\pgfqpoint{8.732320in}{1.517511in}}%
\pgfpathlineto{\pgfqpoint{8.738943in}{1.520180in}}%
\pgfpathlineto{\pgfqpoint{8.745566in}{1.526419in}}%
\pgfpathlineto{\pgfqpoint{8.755500in}{1.540722in}}%
\pgfpathlineto{\pgfqpoint{8.772057in}{1.570790in}}%
\pgfpathlineto{\pgfqpoint{8.791926in}{1.606167in}}%
\pgfpathlineto{\pgfqpoint{8.805172in}{1.625259in}}%
\pgfpathlineto{\pgfqpoint{8.815106in}{1.636179in}}%
\pgfpathlineto{\pgfqpoint{8.825041in}{1.643786in}}%
\pgfpathlineto{\pgfqpoint{8.834975in}{1.647871in}}%
\pgfpathlineto{\pgfqpoint{8.841598in}{1.648591in}}%
\pgfpathlineto{\pgfqpoint{8.848221in}{1.647706in}}%
\pgfpathlineto{\pgfqpoint{8.854844in}{1.645239in}}%
\pgfpathlineto{\pgfqpoint{8.864778in}{1.638679in}}%
\pgfpathlineto{\pgfqpoint{8.874712in}{1.628909in}}%
\pgfpathlineto{\pgfqpoint{8.884647in}{1.616279in}}%
\pgfpathlineto{\pgfqpoint{8.897893in}{1.595814in}}%
\pgfpathlineto{\pgfqpoint{8.921073in}{1.554375in}}%
\pgfpathlineto{\pgfqpoint{8.937630in}{1.525702in}}%
\pgfpathlineto{\pgfqpoint{8.947564in}{1.511542in}}%
\pgfpathlineto{\pgfqpoint{8.957499in}{1.501373in}}%
\pgfpathlineto{\pgfqpoint{8.964122in}{1.497317in}}%
\pgfpathlineto{\pgfqpoint{8.970744in}{1.495472in}}%
\pgfpathlineto{\pgfqpoint{8.977367in}{1.495530in}}%
\pgfpathlineto{\pgfqpoint{8.987302in}{1.497993in}}%
\pgfpathlineto{\pgfqpoint{9.010482in}{1.504996in}}%
\pgfpathlineto{\pgfqpoint{9.020416in}{1.505390in}}%
\pgfpathlineto{\pgfqpoint{9.030351in}{1.503309in}}%
\pgfpathlineto{\pgfqpoint{9.040285in}{1.498688in}}%
\pgfpathlineto{\pgfqpoint{9.053531in}{1.489284in}}%
\pgfpathlineto{\pgfqpoint{9.070088in}{1.476640in}}%
\pgfpathlineto{\pgfqpoint{9.076711in}{1.473850in}}%
\pgfpathlineto{\pgfqpoint{9.083334in}{1.474019in}}%
\pgfpathlineto{\pgfqpoint{9.089957in}{1.477800in}}%
\pgfpathlineto{\pgfqpoint{9.096580in}{1.484988in}}%
\pgfpathlineto{\pgfqpoint{9.106514in}{1.500455in}}%
\pgfpathlineto{\pgfqpoint{9.123071in}{1.532512in}}%
\pgfpathlineto{\pgfqpoint{9.149563in}{1.584395in}}%
\pgfpathlineto{\pgfqpoint{9.162809in}{1.605907in}}%
\pgfpathlineto{\pgfqpoint{9.172743in}{1.618888in}}%
\pgfpathlineto{\pgfqpoint{9.182677in}{1.628712in}}%
\pgfpathlineto{\pgfqpoint{9.192612in}{1.635125in}}%
\pgfpathlineto{\pgfqpoint{9.202546in}{1.638023in}}%
\pgfpathlineto{\pgfqpoint{9.209169in}{1.638019in}}%
\pgfpathlineto{\pgfqpoint{9.219103in}{1.635236in}}%
\pgfpathlineto{\pgfqpoint{9.229037in}{1.629390in}}%
\pgfpathlineto{\pgfqpoint{9.238972in}{1.620908in}}%
\pgfpathlineto{\pgfqpoint{9.252218in}{1.606452in}}%
\pgfpathlineto{\pgfqpoint{9.301889in}{1.547025in}}%
\pgfpathlineto{\pgfqpoint{9.311824in}{1.539208in}}%
\pgfpathlineto{\pgfqpoint{9.321758in}{1.533973in}}%
\pgfpathlineto{\pgfqpoint{9.331692in}{1.531587in}}%
\pgfpathlineto{\pgfqpoint{9.341627in}{1.532212in}}%
\pgfpathlineto{\pgfqpoint{9.351561in}{1.535951in}}%
\pgfpathlineto{\pgfqpoint{9.361495in}{1.542873in}}%
\pgfpathlineto{\pgfqpoint{9.371430in}{1.552986in}}%
\pgfpathlineto{\pgfqpoint{9.381364in}{1.566185in}}%
\pgfpathlineto{\pgfqpoint{9.394610in}{1.588124in}}%
\pgfpathlineto{\pgfqpoint{9.411167in}{1.620954in}}%
\pgfpathlineto{\pgfqpoint{9.457528in}{1.717940in}}%
\pgfpathlineto{\pgfqpoint{9.470773in}{1.738744in}}%
\pgfpathlineto{\pgfqpoint{9.480708in}{1.750246in}}%
\pgfpathlineto{\pgfqpoint{9.487331in}{1.755609in}}%
\pgfpathlineto{\pgfqpoint{9.493953in}{1.758941in}}%
\pgfpathlineto{\pgfqpoint{9.500576in}{1.760108in}}%
\pgfpathlineto{\pgfqpoint{9.507199in}{1.759002in}}%
\pgfpathlineto{\pgfqpoint{9.513822in}{1.755539in}}%
\pgfpathlineto{\pgfqpoint{9.520445in}{1.749664in}}%
\pgfpathlineto{\pgfqpoint{9.527068in}{1.741350in}}%
\pgfpathlineto{\pgfqpoint{9.537002in}{1.724314in}}%
\pgfpathlineto{\pgfqpoint{9.546937in}{1.701923in}}%
\pgfpathlineto{\pgfqpoint{9.556871in}{1.674447in}}%
\pgfpathlineto{\pgfqpoint{9.570117in}{1.630640in}}%
\pgfpathlineto{\pgfqpoint{9.586674in}{1.566455in}}%
\pgfpathlineto{\pgfqpoint{9.609854in}{1.470844in}}%
\pgfpathlineto{\pgfqpoint{9.613166in}{1.461069in}}%
\pgfpathlineto{\pgfqpoint{9.616477in}{1.456542in}}%
\pgfpathlineto{\pgfqpoint{9.619789in}{1.459544in}}%
\pgfpathlineto{\pgfqpoint{9.626411in}{1.480784in}}%
\pgfpathlineto{\pgfqpoint{9.642969in}{1.553370in}}%
\pgfpathlineto{\pgfqpoint{9.666149in}{1.652775in}}%
\pgfpathlineto{\pgfqpoint{9.682706in}{1.713397in}}%
\pgfpathlineto{\pgfqpoint{9.695952in}{1.752691in}}%
\pgfpathlineto{\pgfqpoint{9.705886in}{1.775695in}}%
\pgfpathlineto{\pgfqpoint{9.715821in}{1.792481in}}%
\pgfpathlineto{\pgfqpoint{9.722444in}{1.799970in}}%
\pgfpathlineto{\pgfqpoint{9.729066in}{1.804362in}}%
\pgfpathlineto{\pgfqpoint{9.735689in}{1.805577in}}%
\pgfpathlineto{\pgfqpoint{9.742312in}{1.803571in}}%
\pgfpathlineto{\pgfqpoint{9.748935in}{1.798343in}}%
\pgfpathlineto{\pgfqpoint{9.755558in}{1.789937in}}%
\pgfpathlineto{\pgfqpoint{9.762181in}{1.778444in}}%
\pgfpathlineto{\pgfqpoint{9.772115in}{1.755754in}}%
\pgfpathlineto{\pgfqpoint{9.782050in}{1.727192in}}%
\pgfpathlineto{\pgfqpoint{9.795295in}{1.681808in}}%
\pgfpathlineto{\pgfqpoint{9.831721in}{1.548319in}}%
\pgfpathlineto{\pgfqpoint{9.838344in}{1.532381in}}%
\pgfpathlineto{\pgfqpoint{9.844967in}{1.523686in}}%
\pgfpathlineto{\pgfqpoint{9.848279in}{1.522548in}}%
\pgfpathlineto{\pgfqpoint{9.851590in}{1.523552in}}%
\pgfpathlineto{\pgfqpoint{9.854902in}{1.526501in}}%
\pgfpathlineto{\pgfqpoint{9.861524in}{1.536971in}}%
\pgfpathlineto{\pgfqpoint{9.871459in}{1.559113in}}%
\pgfpathlineto{\pgfqpoint{9.891327in}{1.605238in}}%
\pgfpathlineto{\pgfqpoint{9.901262in}{1.623576in}}%
\pgfpathlineto{\pgfqpoint{9.911196in}{1.637140in}}%
\pgfpathlineto{\pgfqpoint{9.917819in}{1.643280in}}%
\pgfpathlineto{\pgfqpoint{9.924442in}{1.647019in}}%
\pgfpathlineto{\pgfqpoint{9.931065in}{1.648350in}}%
\pgfpathlineto{\pgfqpoint{9.937688in}{1.647304in}}%
\pgfpathlineto{\pgfqpoint{9.944311in}{1.643948in}}%
\pgfpathlineto{\pgfqpoint{9.950934in}{1.638380in}}%
\pgfpathlineto{\pgfqpoint{9.960868in}{1.626165in}}%
\pgfpathlineto{\pgfqpoint{9.970802in}{1.609796in}}%
\pgfpathlineto{\pgfqpoint{9.984048in}{1.582663in}}%
\pgfpathlineto{\pgfqpoint{10.003917in}{1.534758in}}%
\pgfpathlineto{\pgfqpoint{10.020474in}{1.495269in}}%
\pgfpathlineto{\pgfqpoint{10.027097in}{1.482874in}}%
\pgfpathlineto{\pgfqpoint{10.033720in}{1.475302in}}%
\pgfpathlineto{\pgfqpoint{10.037031in}{1.474073in}}%
\pgfpathlineto{\pgfqpoint{10.040343in}{1.474736in}}%
\pgfpathlineto{\pgfqpoint{10.043654in}{1.477165in}}%
\pgfpathlineto{\pgfqpoint{10.050277in}{1.486107in}}%
\pgfpathlineto{\pgfqpoint{10.060211in}{1.505046in}}%
\pgfpathlineto{\pgfqpoint{10.090014in}{1.565520in}}%
\pgfpathlineto{\pgfqpoint{10.103260in}{1.587813in}}%
\pgfpathlineto{\pgfqpoint{10.116506in}{1.606196in}}%
\pgfpathlineto{\pgfqpoint{10.129752in}{1.620702in}}%
\pgfpathlineto{\pgfqpoint{10.142998in}{1.631654in}}%
\pgfpathlineto{\pgfqpoint{10.156243in}{1.639548in}}%
\pgfpathlineto{\pgfqpoint{10.169489in}{1.644943in}}%
\pgfpathlineto{\pgfqpoint{10.186047in}{1.648961in}}%
\pgfpathlineto{\pgfqpoint{10.202604in}{1.650632in}}%
\pgfpathlineto{\pgfqpoint{10.219161in}{1.650259in}}%
\pgfpathlineto{\pgfqpoint{10.235718in}{1.647694in}}%
\pgfpathlineto{\pgfqpoint{10.248964in}{1.643753in}}%
\pgfpathlineto{\pgfqpoint{10.262210in}{1.637820in}}%
\pgfpathlineto{\pgfqpoint{10.275456in}{1.629660in}}%
\pgfpathlineto{\pgfqpoint{10.288701in}{1.619189in}}%
\pgfpathlineto{\pgfqpoint{10.305259in}{1.603089in}}%
\pgfpathlineto{\pgfqpoint{10.328439in}{1.576720in}}%
\pgfpathlineto{\pgfqpoint{10.354930in}{1.547195in}}%
\pgfpathlineto{\pgfqpoint{10.368176in}{1.535902in}}%
\pgfpathlineto{\pgfqpoint{10.378111in}{1.530100in}}%
\pgfpathlineto{\pgfqpoint{10.388045in}{1.526988in}}%
\pgfpathlineto{\pgfqpoint{10.397979in}{1.526573in}}%
\pgfpathlineto{\pgfqpoint{10.407914in}{1.528506in}}%
\pgfpathlineto{\pgfqpoint{10.421159in}{1.533679in}}%
\pgfpathlineto{\pgfqpoint{10.464208in}{1.553633in}}%
\pgfpathlineto{\pgfqpoint{10.477454in}{1.556586in}}%
\pgfpathlineto{\pgfqpoint{10.487388in}{1.557002in}}%
\pgfpathlineto{\pgfqpoint{10.497323in}{1.555635in}}%
\pgfpathlineto{\pgfqpoint{10.507257in}{1.552307in}}%
\pgfpathlineto{\pgfqpoint{10.517191in}{1.546881in}}%
\pgfpathlineto{\pgfqpoint{10.527126in}{1.539272in}}%
\pgfpathlineto{\pgfqpoint{10.540372in}{1.525726in}}%
\pgfpathlineto{\pgfqpoint{10.553617in}{1.508596in}}%
\pgfpathlineto{\pgfqpoint{10.583420in}{1.466094in}}%
\pgfpathlineto{\pgfqpoint{10.590043in}{1.461574in}}%
\pgfpathlineto{\pgfqpoint{10.593355in}{1.461369in}}%
\pgfpathlineto{\pgfqpoint{10.596666in}{1.462722in}}%
\pgfpathlineto{\pgfqpoint{10.603289in}{1.469576in}}%
\pgfpathlineto{\pgfqpoint{10.613224in}{1.486514in}}%
\pgfpathlineto{\pgfqpoint{10.629781in}{1.521532in}}%
\pgfpathlineto{\pgfqpoint{10.662895in}{1.592071in}}%
\pgfpathlineto{\pgfqpoint{10.679453in}{1.622636in}}%
\pgfpathlineto{\pgfqpoint{10.696010in}{1.648741in}}%
\pgfpathlineto{\pgfqpoint{10.709256in}{1.666167in}}%
\pgfpathlineto{\pgfqpoint{10.722501in}{1.680397in}}%
\pgfpathlineto{\pgfqpoint{10.735747in}{1.691292in}}%
\pgfpathlineto{\pgfqpoint{10.745682in}{1.697155in}}%
\pgfpathlineto{\pgfqpoint{10.755616in}{1.700915in}}%
\pgfpathlineto{\pgfqpoint{10.765550in}{1.702441in}}%
\pgfpathlineto{\pgfqpoint{10.775485in}{1.701596in}}%
\pgfpathlineto{\pgfqpoint{10.785419in}{1.698259in}}%
\pgfpathlineto{\pgfqpoint{10.795353in}{1.692333in}}%
\pgfpathlineto{\pgfqpoint{10.805288in}{1.683778in}}%
\pgfpathlineto{\pgfqpoint{10.815222in}{1.672618in}}%
\pgfpathlineto{\pgfqpoint{10.828468in}{1.653902in}}%
\pgfpathlineto{\pgfqpoint{10.841714in}{1.631334in}}%
\pgfpathlineto{\pgfqpoint{10.861582in}{1.592387in}}%
\pgfpathlineto{\pgfqpoint{10.891385in}{1.532725in}}%
\pgfpathlineto{\pgfqpoint{10.904631in}{1.511447in}}%
\pgfpathlineto{\pgfqpoint{10.914565in}{1.500310in}}%
\pgfpathlineto{\pgfqpoint{10.921188in}{1.495930in}}%
\pgfpathlineto{\pgfqpoint{10.927811in}{1.494203in}}%
\pgfpathlineto{\pgfqpoint{10.934434in}{1.494983in}}%
\pgfpathlineto{\pgfqpoint{10.941057in}{1.497850in}}%
\pgfpathlineto{\pgfqpoint{10.950991in}{1.504875in}}%
\pgfpathlineto{\pgfqpoint{10.970860in}{1.523147in}}%
\pgfpathlineto{\pgfqpoint{10.994040in}{1.544054in}}%
\pgfpathlineto{\pgfqpoint{11.010598in}{1.556191in}}%
\pgfpathlineto{\pgfqpoint{11.023843in}{1.563522in}}%
\pgfpathlineto{\pgfqpoint{11.037089in}{1.568386in}}%
\pgfpathlineto{\pgfqpoint{11.050335in}{1.570553in}}%
\pgfpathlineto{\pgfqpoint{11.063581in}{1.569914in}}%
\pgfpathlineto{\pgfqpoint{11.076827in}{1.566515in}}%
\pgfpathlineto{\pgfqpoint{11.090072in}{1.560596in}}%
\pgfpathlineto{\pgfqpoint{11.106630in}{1.550417in}}%
\pgfpathlineto{\pgfqpoint{11.143056in}{1.526140in}}%
\pgfpathlineto{\pgfqpoint{11.152990in}{1.522141in}}%
\pgfpathlineto{\pgfqpoint{11.162924in}{1.520540in}}%
\pgfpathlineto{\pgfqpoint{11.172859in}{1.521661in}}%
\pgfpathlineto{\pgfqpoint{11.182793in}{1.525379in}}%
\pgfpathlineto{\pgfqpoint{11.196039in}{1.533398in}}%
\pgfpathlineto{\pgfqpoint{11.232465in}{1.558621in}}%
\pgfpathlineto{\pgfqpoint{11.242399in}{1.562871in}}%
\pgfpathlineto{\pgfqpoint{11.252333in}{1.564949in}}%
\pgfpathlineto{\pgfqpoint{11.262268in}{1.564488in}}%
\pgfpathlineto{\pgfqpoint{11.272202in}{1.561223in}}%
\pgfpathlineto{\pgfqpoint{11.282136in}{1.554993in}}%
\pgfpathlineto{\pgfqpoint{11.292071in}{1.545732in}}%
\pgfpathlineto{\pgfqpoint{11.302005in}{1.533466in}}%
\pgfpathlineto{\pgfqpoint{11.315251in}{1.512645in}}%
\pgfpathlineto{\pgfqpoint{11.328497in}{1.487202in}}%
\pgfpathlineto{\pgfqpoint{11.348365in}{1.442685in}}%
\pgfpathlineto{\pgfqpoint{11.351677in}{1.435628in}}%
\pgfpathlineto{\pgfqpoint{11.354988in}{1.432221in}}%
\pgfpathlineto{\pgfqpoint{11.358300in}{1.436856in}}%
\pgfpathlineto{\pgfqpoint{11.371546in}{1.470242in}}%
\pgfpathlineto{\pgfqpoint{11.417906in}{1.591392in}}%
\pgfpathlineto{\pgfqpoint{11.434463in}{1.628607in}}%
\pgfpathlineto{\pgfqpoint{11.451020in}{1.660168in}}%
\pgfpathlineto{\pgfqpoint{11.464266in}{1.680643in}}%
\pgfpathlineto{\pgfqpoint{11.477512in}{1.696421in}}%
\pgfpathlineto{\pgfqpoint{11.487446in}{1.704956in}}%
\pgfpathlineto{\pgfqpoint{11.497381in}{1.710514in}}%
\pgfpathlineto{\pgfqpoint{11.507315in}{1.712966in}}%
\pgfpathlineto{\pgfqpoint{11.517249in}{1.712196in}}%
\pgfpathlineto{\pgfqpoint{11.527184in}{1.708105in}}%
\pgfpathlineto{\pgfqpoint{11.537118in}{1.700623in}}%
\pgfpathlineto{\pgfqpoint{11.547052in}{1.689715in}}%
\pgfpathlineto{\pgfqpoint{11.556987in}{1.675403in}}%
\pgfpathlineto{\pgfqpoint{11.570233in}{1.651189in}}%
\pgfpathlineto{\pgfqpoint{11.583478in}{1.621539in}}%
\pgfpathlineto{\pgfqpoint{11.600036in}{1.578101in}}%
\pgfpathlineto{\pgfqpoint{11.636462in}{1.477487in}}%
\pgfpathlineto{\pgfqpoint{11.643084in}{1.468124in}}%
\pgfpathlineto{\pgfqpoint{11.646396in}{1.467098in}}%
\pgfpathlineto{\pgfqpoint{11.649707in}{1.468812in}}%
\pgfpathlineto{\pgfqpoint{11.656330in}{1.478729in}}%
\pgfpathlineto{\pgfqpoint{11.666265in}{1.501477in}}%
\pgfpathlineto{\pgfqpoint{11.689445in}{1.557353in}}%
\pgfpathlineto{\pgfqpoint{11.702691in}{1.583507in}}%
\pgfpathlineto{\pgfqpoint{11.712625in}{1.598997in}}%
\pgfpathlineto{\pgfqpoint{11.722559in}{1.610528in}}%
\pgfpathlineto{\pgfqpoint{11.732494in}{1.617872in}}%
\pgfpathlineto{\pgfqpoint{11.739117in}{1.620381in}}%
\pgfpathlineto{\pgfqpoint{11.745739in}{1.620969in}}%
\pgfpathlineto{\pgfqpoint{11.752362in}{1.619649in}}%
\pgfpathlineto{\pgfqpoint{11.758985in}{1.616457in}}%
\pgfpathlineto{\pgfqpoint{11.768920in}{1.608301in}}%
\pgfpathlineto{\pgfqpoint{11.778854in}{1.596407in}}%
\pgfpathlineto{\pgfqpoint{11.792100in}{1.575671in}}%
\pgfpathlineto{\pgfqpoint{11.828526in}{1.513034in}}%
\pgfpathlineto{\pgfqpoint{11.835149in}{1.507930in}}%
\pgfpathlineto{\pgfqpoint{11.838460in}{1.507097in}}%
\pgfpathlineto{\pgfqpoint{11.841771in}{1.507516in}}%
\pgfpathlineto{\pgfqpoint{11.848394in}{1.512026in}}%
\pgfpathlineto{\pgfqpoint{11.855017in}{1.520741in}}%
\pgfpathlineto{\pgfqpoint{11.864952in}{1.539089in}}%
\pgfpathlineto{\pgfqpoint{11.891443in}{1.597382in}}%
\pgfpathlineto{\pgfqpoint{11.908000in}{1.630534in}}%
\pgfpathlineto{\pgfqpoint{11.921246in}{1.652535in}}%
\pgfpathlineto{\pgfqpoint{11.934492in}{1.669756in}}%
\pgfpathlineto{\pgfqpoint{11.944426in}{1.679342in}}%
\pgfpathlineto{\pgfqpoint{11.954361in}{1.686021in}}%
\pgfpathlineto{\pgfqpoint{11.964295in}{1.689802in}}%
\pgfpathlineto{\pgfqpoint{11.974229in}{1.690731in}}%
\pgfpathlineto{\pgfqpoint{11.984164in}{1.688878in}}%
\pgfpathlineto{\pgfqpoint{11.994098in}{1.684333in}}%
\pgfpathlineto{\pgfqpoint{12.004032in}{1.677206in}}%
\pgfpathlineto{\pgfqpoint{12.013967in}{1.667620in}}%
\pgfpathlineto{\pgfqpoint{12.027213in}{1.651267in}}%
\pgfpathlineto{\pgfqpoint{12.040458in}{1.631211in}}%
\pgfpathlineto{\pgfqpoint{12.057016in}{1.601706in}}%
\pgfpathlineto{\pgfqpoint{12.076884in}{1.561498in}}%
\pgfpathlineto{\pgfqpoint{12.116622in}{1.478980in}}%
\pgfpathlineto{\pgfqpoint{12.126556in}{1.463369in}}%
\pgfpathlineto{\pgfqpoint{12.133179in}{1.456860in}}%
\pgfpathlineto{\pgfqpoint{12.136490in}{1.455347in}}%
\pgfpathlineto{\pgfqpoint{12.139802in}{1.455121in}}%
\pgfpathlineto{\pgfqpoint{12.146425in}{1.458018in}}%
\pgfpathlineto{\pgfqpoint{12.156359in}{1.466979in}}%
\pgfpathlineto{\pgfqpoint{12.176228in}{1.485783in}}%
\pgfpathlineto{\pgfqpoint{12.186162in}{1.492540in}}%
\pgfpathlineto{\pgfqpoint{12.196097in}{1.497067in}}%
\pgfpathlineto{\pgfqpoint{12.206031in}{1.499541in}}%
\pgfpathlineto{\pgfqpoint{12.219277in}{1.500524in}}%
\pgfpathlineto{\pgfqpoint{12.242457in}{1.501767in}}%
\pgfpathlineto{\pgfqpoint{12.252391in}{1.504850in}}%
\pgfpathlineto{\pgfqpoint{12.262326in}{1.510566in}}%
\pgfpathlineto{\pgfqpoint{12.272260in}{1.518973in}}%
\pgfpathlineto{\pgfqpoint{12.285506in}{1.533575in}}%
\pgfpathlineto{\pgfqpoint{12.308686in}{1.563672in}}%
\pgfpathlineto{\pgfqpoint{12.328555in}{1.588517in}}%
\pgfpathlineto{\pgfqpoint{12.341800in}{1.602353in}}%
\pgfpathlineto{\pgfqpoint{12.355046in}{1.612923in}}%
\pgfpathlineto{\pgfqpoint{12.364981in}{1.618295in}}%
\pgfpathlineto{\pgfqpoint{12.374915in}{1.621247in}}%
\pgfpathlineto{\pgfqpoint{12.384849in}{1.621654in}}%
\pgfpathlineto{\pgfqpoint{12.394784in}{1.619468in}}%
\pgfpathlineto{\pgfqpoint{12.404718in}{1.614729in}}%
\pgfpathlineto{\pgfqpoint{12.414652in}{1.607565in}}%
\pgfpathlineto{\pgfqpoint{12.427898in}{1.594650in}}%
\pgfpathlineto{\pgfqpoint{12.444455in}{1.574320in}}%
\pgfpathlineto{\pgfqpoint{12.484193in}{1.522126in}}%
\pgfpathlineto{\pgfqpoint{12.494127in}{1.512634in}}%
\pgfpathlineto{\pgfqpoint{12.504061in}{1.505990in}}%
\pgfpathlineto{\pgfqpoint{12.513996in}{1.502297in}}%
\pgfpathlineto{\pgfqpoint{12.523930in}{1.501049in}}%
\pgfpathlineto{\pgfqpoint{12.540487in}{1.501912in}}%
\pgfpathlineto{\pgfqpoint{12.590159in}{1.507238in}}%
\pgfpathlineto{\pgfqpoint{12.600093in}{1.511606in}}%
\pgfpathlineto{\pgfqpoint{12.610028in}{1.518744in}}%
\pgfpathlineto{\pgfqpoint{12.619962in}{1.528784in}}%
\pgfpathlineto{\pgfqpoint{12.633208in}{1.546046in}}%
\pgfpathlineto{\pgfqpoint{12.653077in}{1.576983in}}%
\pgfpathlineto{\pgfqpoint{12.682880in}{1.623439in}}%
\pgfpathlineto{\pgfqpoint{12.696126in}{1.640337in}}%
\pgfpathlineto{\pgfqpoint{12.706060in}{1.650433in}}%
\pgfpathlineto{\pgfqpoint{12.715994in}{1.657847in}}%
\pgfpathlineto{\pgfqpoint{12.725929in}{1.662218in}}%
\pgfpathlineto{\pgfqpoint{12.732551in}{1.663292in}}%
\pgfpathlineto{\pgfqpoint{12.739174in}{1.662812in}}%
\pgfpathlineto{\pgfqpoint{12.745797in}{1.660725in}}%
\pgfpathlineto{\pgfqpoint{12.755732in}{1.654517in}}%
\pgfpathlineto{\pgfqpoint{12.765666in}{1.644612in}}%
\pgfpathlineto{\pgfqpoint{12.775600in}{1.631131in}}%
\pgfpathlineto{\pgfqpoint{12.788846in}{1.608150in}}%
\pgfpathlineto{\pgfqpoint{12.805403in}{1.573582in}}%
\pgfpathlineto{\pgfqpoint{12.825272in}{1.532539in}}%
\pgfpathlineto{\pgfqpoint{12.831895in}{1.522788in}}%
\pgfpathlineto{\pgfqpoint{12.838518in}{1.517324in}}%
\pgfpathlineto{\pgfqpoint{12.841829in}{1.516612in}}%
\pgfpathlineto{\pgfqpoint{12.845141in}{1.517360in}}%
\pgfpathlineto{\pgfqpoint{12.848452in}{1.519556in}}%
\pgfpathlineto{\pgfqpoint{12.855075in}{1.527883in}}%
\pgfpathlineto{\pgfqpoint{12.861698in}{1.540362in}}%
\pgfpathlineto{\pgfqpoint{12.874944in}{1.572403in}}%
\pgfpathlineto{\pgfqpoint{12.904747in}{1.648578in}}%
\pgfpathlineto{\pgfqpoint{12.917993in}{1.675919in}}%
\pgfpathlineto{\pgfqpoint{12.927927in}{1.691907in}}%
\pgfpathlineto{\pgfqpoint{12.937861in}{1.703336in}}%
\pgfpathlineto{\pgfqpoint{12.944484in}{1.708184in}}%
\pgfpathlineto{\pgfqpoint{12.951107in}{1.710692in}}%
\pgfpathlineto{\pgfqpoint{12.957730in}{1.710784in}}%
\pgfpathlineto{\pgfqpoint{12.964353in}{1.708422in}}%
\pgfpathlineto{\pgfqpoint{12.970976in}{1.703600in}}%
\pgfpathlineto{\pgfqpoint{12.977599in}{1.696350in}}%
\pgfpathlineto{\pgfqpoint{12.987533in}{1.681086in}}%
\pgfpathlineto{\pgfqpoint{12.997467in}{1.660935in}}%
\pgfpathlineto{\pgfqpoint{13.010713in}{1.627562in}}%
\pgfpathlineto{\pgfqpoint{13.027271in}{1.578422in}}%
\pgfpathlineto{\pgfqpoint{13.053762in}{1.498957in}}%
\pgfpathlineto{\pgfqpoint{13.060385in}{1.484634in}}%
\pgfpathlineto{\pgfqpoint{13.067008in}{1.476647in}}%
\pgfpathlineto{\pgfqpoint{13.070319in}{1.475761in}}%
\pgfpathlineto{\pgfqpoint{13.073631in}{1.476954in}}%
\pgfpathlineto{\pgfqpoint{13.080254in}{1.484175in}}%
\pgfpathlineto{\pgfqpoint{13.093499in}{1.506626in}}%
\pgfpathlineto{\pgfqpoint{13.106745in}{1.527586in}}%
\pgfpathlineto{\pgfqpoint{13.116680in}{1.538800in}}%
\pgfpathlineto{\pgfqpoint{13.123303in}{1.543584in}}%
\pgfpathlineto{\pgfqpoint{13.129925in}{1.546072in}}%
\pgfpathlineto{\pgfqpoint{13.136548in}{1.546238in}}%
\pgfpathlineto{\pgfqpoint{13.143171in}{1.544137in}}%
\pgfpathlineto{\pgfqpoint{13.149794in}{1.539914in}}%
\pgfpathlineto{\pgfqpoint{13.159728in}{1.530188in}}%
\pgfpathlineto{\pgfqpoint{13.182909in}{1.502754in}}%
\pgfpathlineto{\pgfqpoint{13.189532in}{1.499424in}}%
\pgfpathlineto{\pgfqpoint{13.192843in}{1.499649in}}%
\pgfpathlineto{\pgfqpoint{13.196154in}{1.501354in}}%
\pgfpathlineto{\pgfqpoint{13.202777in}{1.509288in}}%
\pgfpathlineto{\pgfqpoint{13.209400in}{1.522565in}}%
\pgfpathlineto{\pgfqpoint{13.219335in}{1.549628in}}%
\pgfpathlineto{\pgfqpoint{13.232580in}{1.593316in}}%
\pgfpathlineto{\pgfqpoint{13.282252in}{1.765126in}}%
\pgfpathlineto{\pgfqpoint{13.295498in}{1.802185in}}%
\pgfpathlineto{\pgfqpoint{13.308744in}{1.832272in}}%
\pgfpathlineto{\pgfqpoint{13.318678in}{1.849484in}}%
\pgfpathlineto{\pgfqpoint{13.328612in}{1.861635in}}%
\pgfpathlineto{\pgfqpoint{13.335235in}{1.866777in}}%
\pgfpathlineto{\pgfqpoint{13.341858in}{1.869493in}}%
\pgfpathlineto{\pgfqpoint{13.348481in}{1.869766in}}%
\pgfpathlineto{\pgfqpoint{13.355104in}{1.867614in}}%
\pgfpathlineto{\pgfqpoint{13.361727in}{1.863089in}}%
\pgfpathlineto{\pgfqpoint{13.368350in}{1.856277in}}%
\pgfpathlineto{\pgfqpoint{13.378284in}{1.842033in}}%
\pgfpathlineto{\pgfqpoint{13.388219in}{1.823428in}}%
\pgfpathlineto{\pgfqpoint{13.401464in}{1.792950in}}%
\pgfpathlineto{\pgfqpoint{13.418022in}{1.748208in}}%
\pgfpathlineto{\pgfqpoint{13.480939in}{1.569587in}}%
\pgfpathlineto{\pgfqpoint{13.497496in}{1.532923in}}%
\pgfpathlineto{\pgfqpoint{13.510742in}{1.508809in}}%
\pgfpathlineto{\pgfqpoint{13.523988in}{1.489500in}}%
\pgfpathlineto{\pgfqpoint{13.537234in}{1.474721in}}%
\pgfpathlineto{\pgfqpoint{13.550480in}{1.463713in}}%
\pgfpathlineto{\pgfqpoint{13.567037in}{1.453534in}}%
\pgfpathlineto{\pgfqpoint{13.583594in}{1.446106in}}%
\pgfpathlineto{\pgfqpoint{13.593528in}{1.443949in}}%
\pgfpathlineto{\pgfqpoint{13.600151in}{1.444330in}}%
\pgfpathlineto{\pgfqpoint{13.606774in}{1.446419in}}%
\pgfpathlineto{\pgfqpoint{13.616709in}{1.452174in}}%
\pgfpathlineto{\pgfqpoint{13.633266in}{1.465302in}}%
\pgfpathlineto{\pgfqpoint{13.653135in}{1.480888in}}%
\pgfpathlineto{\pgfqpoint{13.666380in}{1.488375in}}%
\pgfpathlineto{\pgfqpoint{13.676315in}{1.491552in}}%
\pgfpathlineto{\pgfqpoint{13.686249in}{1.492336in}}%
\pgfpathlineto{\pgfqpoint{13.696183in}{1.490802in}}%
\pgfpathlineto{\pgfqpoint{13.712741in}{1.484918in}}%
\pgfpathlineto{\pgfqpoint{13.722675in}{1.481865in}}%
\pgfpathlineto{\pgfqpoint{13.729298in}{1.481432in}}%
\pgfpathlineto{\pgfqpoint{13.735921in}{1.483021in}}%
\pgfpathlineto{\pgfqpoint{13.742544in}{1.486898in}}%
\pgfpathlineto{\pgfqpoint{13.752478in}{1.496513in}}%
\pgfpathlineto{\pgfqpoint{13.769035in}{1.517984in}}%
\pgfpathlineto{\pgfqpoint{13.785593in}{1.538650in}}%
\pgfpathlineto{\pgfqpoint{13.795527in}{1.547933in}}%
\pgfpathlineto{\pgfqpoint{13.805461in}{1.553752in}}%
\pgfpathlineto{\pgfqpoint{13.812084in}{1.555434in}}%
\pgfpathlineto{\pgfqpoint{13.818707in}{1.555265in}}%
\pgfpathlineto{\pgfqpoint{13.825330in}{1.553239in}}%
\pgfpathlineto{\pgfqpoint{13.831953in}{1.549415in}}%
\pgfpathlineto{\pgfqpoint{13.841887in}{1.540606in}}%
\pgfpathlineto{\pgfqpoint{13.855133in}{1.524366in}}%
\pgfpathlineto{\pgfqpoint{13.878313in}{1.493622in}}%
\pgfpathlineto{\pgfqpoint{13.884936in}{1.488100in}}%
\pgfpathlineto{\pgfqpoint{13.891559in}{1.485960in}}%
\pgfpathlineto{\pgfqpoint{13.898182in}{1.487778in}}%
\pgfpathlineto{\pgfqpoint{13.904805in}{1.493260in}}%
\pgfpathlineto{\pgfqpoint{13.914739in}{1.506327in}}%
\pgfpathlineto{\pgfqpoint{13.931296in}{1.533770in}}%
\pgfpathlineto{\pgfqpoint{13.954476in}{1.572008in}}%
\pgfpathlineto{\pgfqpoint{13.971034in}{1.595127in}}%
\pgfpathlineto{\pgfqpoint{13.984280in}{1.609960in}}%
\pgfpathlineto{\pgfqpoint{13.994214in}{1.618487in}}%
\pgfpathlineto{\pgfqpoint{14.004148in}{1.624395in}}%
\pgfpathlineto{\pgfqpoint{14.014083in}{1.627273in}}%
\pgfpathlineto{\pgfqpoint{14.020705in}{1.627298in}}%
\pgfpathlineto{\pgfqpoint{14.027328in}{1.625667in}}%
\pgfpathlineto{\pgfqpoint{14.033951in}{1.622282in}}%
\pgfpathlineto{\pgfqpoint{14.040574in}{1.617082in}}%
\pgfpathlineto{\pgfqpoint{14.050509in}{1.605897in}}%
\pgfpathlineto{\pgfqpoint{14.060443in}{1.591064in}}%
\pgfpathlineto{\pgfqpoint{14.083623in}{1.552516in}}%
\pgfpathlineto{\pgfqpoint{14.090246in}{1.546162in}}%
\pgfpathlineto{\pgfqpoint{14.093557in}{1.544996in}}%
\pgfpathlineto{\pgfqpoint{14.096869in}{1.545494in}}%
\pgfpathlineto{\pgfqpoint{14.100180in}{1.547816in}}%
\pgfpathlineto{\pgfqpoint{14.103492in}{1.552022in}}%
\pgfpathlineto{\pgfqpoint{14.110115in}{1.565846in}}%
\pgfpathlineto{\pgfqpoint{14.116738in}{1.585819in}}%
\pgfpathlineto{\pgfqpoint{14.126672in}{1.623806in}}%
\pgfpathlineto{\pgfqpoint{14.146541in}{1.712375in}}%
\pgfpathlineto{\pgfqpoint{14.166409in}{1.798118in}}%
\pgfpathlineto{\pgfqpoint{14.179655in}{1.846062in}}%
\pgfpathlineto{\pgfqpoint{14.189589in}{1.874992in}}%
\pgfpathlineto{\pgfqpoint{14.199524in}{1.896919in}}%
\pgfpathlineto{\pgfqpoint{14.206147in}{1.907349in}}%
\pgfpathlineto{\pgfqpoint{14.212770in}{1.914292in}}%
\pgfpathlineto{\pgfqpoint{14.219392in}{1.917686in}}%
\pgfpathlineto{\pgfqpoint{14.226015in}{1.917506in}}%
\pgfpathlineto{\pgfqpoint{14.232638in}{1.913771in}}%
\pgfpathlineto{\pgfqpoint{14.239261in}{1.906535in}}%
\pgfpathlineto{\pgfqpoint{14.245884in}{1.895890in}}%
\pgfpathlineto{\pgfqpoint{14.255818in}{1.873807in}}%
\pgfpathlineto{\pgfqpoint{14.265753in}{1.844894in}}%
\pgfpathlineto{\pgfqpoint{14.278999in}{1.796981in}}%
\pgfpathlineto{\pgfqpoint{14.295556in}{1.725182in}}%
\pgfpathlineto{\pgfqpoint{14.338605in}{1.527915in}}%
\pgfpathlineto{\pgfqpoint{14.345228in}{1.511391in}}%
\pgfpathlineto{\pgfqpoint{14.348539in}{1.507619in}}%
\pgfpathlineto{\pgfqpoint{14.351850in}{1.507314in}}%
\pgfpathlineto{\pgfqpoint{14.355162in}{1.510342in}}%
\pgfpathlineto{\pgfqpoint{14.361785in}{1.524155in}}%
\pgfpathlineto{\pgfqpoint{14.371719in}{1.554853in}}%
\pgfpathlineto{\pgfqpoint{14.391588in}{1.619147in}}%
\pgfpathlineto{\pgfqpoint{14.404834in}{1.653602in}}%
\pgfpathlineto{\pgfqpoint{14.414768in}{1.672986in}}%
\pgfpathlineto{\pgfqpoint{14.421391in}{1.682552in}}%
\pgfpathlineto{\pgfqpoint{14.428014in}{1.689352in}}%
\pgfpathlineto{\pgfqpoint{14.434637in}{1.693361in}}%
\pgfpathlineto{\pgfqpoint{14.441260in}{1.694595in}}%
\pgfpathlineto{\pgfqpoint{14.447882in}{1.693098in}}%
\pgfpathlineto{\pgfqpoint{14.454505in}{1.688947in}}%
\pgfpathlineto{\pgfqpoint{14.461128in}{1.682246in}}%
\pgfpathlineto{\pgfqpoint{14.471063in}{1.667711in}}%
\pgfpathlineto{\pgfqpoint{14.480997in}{1.648322in}}%
\pgfpathlineto{\pgfqpoint{14.494243in}{1.616286in}}%
\pgfpathlineto{\pgfqpoint{14.530669in}{1.521348in}}%
\pgfpathlineto{\pgfqpoint{14.537292in}{1.513090in}}%
\pgfpathlineto{\pgfqpoint{14.540603in}{1.511659in}}%
\pgfpathlineto{\pgfqpoint{14.543915in}{1.512215in}}%
\pgfpathlineto{\pgfqpoint{14.547226in}{1.514724in}}%
\pgfpathlineto{\pgfqpoint{14.553849in}{1.524814in}}%
\pgfpathlineto{\pgfqpoint{14.563783in}{1.548468in}}%
\pgfpathlineto{\pgfqpoint{14.606832in}{1.665776in}}%
\pgfpathlineto{\pgfqpoint{14.616766in}{1.685858in}}%
\pgfpathlineto{\pgfqpoint{14.626701in}{1.701793in}}%
\pgfpathlineto{\pgfqpoint{14.636635in}{1.713379in}}%
\pgfpathlineto{\pgfqpoint{14.646569in}{1.720611in}}%
\pgfpathlineto{\pgfqpoint{14.653192in}{1.723096in}}%
\pgfpathlineto{\pgfqpoint{14.659815in}{1.723820in}}%
\pgfpathlineto{\pgfqpoint{14.666438in}{1.722907in}}%
\pgfpathlineto{\pgfqpoint{14.676373in}{1.718816in}}%
\pgfpathlineto{\pgfqpoint{14.686307in}{1.712010in}}%
\pgfpathlineto{\pgfqpoint{14.699553in}{1.699917in}}%
\pgfpathlineto{\pgfqpoint{14.752536in}{1.647199in}}%
\pgfpathlineto{\pgfqpoint{14.765782in}{1.638113in}}%
\pgfpathlineto{\pgfqpoint{14.782339in}{1.629751in}}%
\pgfpathlineto{\pgfqpoint{14.798896in}{1.624037in}}%
\pgfpathlineto{\pgfqpoint{14.822076in}{1.618748in}}%
\pgfpathlineto{\pgfqpoint{14.871748in}{1.608548in}}%
\pgfpathlineto{\pgfqpoint{14.891617in}{1.601712in}}%
\pgfpathlineto{\pgfqpoint{14.908174in}{1.593564in}}%
\pgfpathlineto{\pgfqpoint{14.921420in}{1.584943in}}%
\pgfpathlineto{\pgfqpoint{14.934666in}{1.574090in}}%
\pgfpathlineto{\pgfqpoint{14.947911in}{1.560752in}}%
\pgfpathlineto{\pgfqpoint{14.964469in}{1.540470in}}%
\pgfpathlineto{\pgfqpoint{14.984337in}{1.511574in}}%
\pgfpathlineto{\pgfqpoint{15.010829in}{1.471881in}}%
\pgfpathlineto{\pgfqpoint{15.017452in}{1.464400in}}%
\pgfpathlineto{\pgfqpoint{15.024075in}{1.459644in}}%
\pgfpathlineto{\pgfqpoint{15.030698in}{1.458549in}}%
\pgfpathlineto{\pgfqpoint{15.037321in}{1.461003in}}%
\pgfpathlineto{\pgfqpoint{15.047255in}{1.468674in}}%
\pgfpathlineto{\pgfqpoint{15.067124in}{1.485625in}}%
\pgfpathlineto{\pgfqpoint{15.077058in}{1.491747in}}%
\pgfpathlineto{\pgfqpoint{15.086992in}{1.495798in}}%
\pgfpathlineto{\pgfqpoint{15.100238in}{1.498748in}}%
\pgfpathlineto{\pgfqpoint{15.116795in}{1.502309in}}%
\pgfpathlineto{\pgfqpoint{15.126730in}{1.507221in}}%
\pgfpathlineto{\pgfqpoint{15.136664in}{1.515861in}}%
\pgfpathlineto{\pgfqpoint{15.146598in}{1.528596in}}%
\pgfpathlineto{\pgfqpoint{15.156533in}{1.544929in}}%
\pgfpathlineto{\pgfqpoint{15.173090in}{1.577532in}}%
\pgfpathlineto{\pgfqpoint{15.219450in}{1.673309in}}%
\pgfpathlineto{\pgfqpoint{15.236008in}{1.701432in}}%
\pgfpathlineto{\pgfqpoint{15.249253in}{1.720134in}}%
\pgfpathlineto{\pgfqpoint{15.262499in}{1.734950in}}%
\pgfpathlineto{\pgfqpoint{15.272434in}{1.743181in}}%
\pgfpathlineto{\pgfqpoint{15.282368in}{1.748620in}}%
\pgfpathlineto{\pgfqpoint{15.292302in}{1.750913in}}%
\pgfpathlineto{\pgfqpoint{15.298925in}{1.750495in}}%
\pgfpathlineto{\pgfqpoint{15.305548in}{1.748373in}}%
\pgfpathlineto{\pgfqpoint{15.312171in}{1.744418in}}%
\pgfpathlineto{\pgfqpoint{15.318794in}{1.738509in}}%
\pgfpathlineto{\pgfqpoint{15.328728in}{1.725748in}}%
\pgfpathlineto{\pgfqpoint{15.338663in}{1.708073in}}%
\pgfpathlineto{\pgfqpoint{15.348597in}{1.685360in}}%
\pgfpathlineto{\pgfqpoint{15.358531in}{1.657697in}}%
\pgfpathlineto{\pgfqpoint{15.371777in}{1.613728in}}%
\pgfpathlineto{\pgfqpoint{15.388334in}{1.549942in}}%
\pgfpathlineto{\pgfqpoint{15.404892in}{1.484585in}}%
\pgfpathlineto{\pgfqpoint{15.411514in}{1.466335in}}%
\pgfpathlineto{\pgfqpoint{15.414826in}{1.463489in}}%
\pgfpathlineto{\pgfqpoint{15.418137in}{1.466475in}}%
\pgfpathlineto{\pgfqpoint{15.421449in}{1.474186in}}%
\pgfpathlineto{\pgfqpoint{15.428072in}{1.496761in}}%
\pgfpathlineto{\pgfqpoint{15.461186in}{1.621732in}}%
\pgfpathlineto{\pgfqpoint{15.474432in}{1.661484in}}%
\pgfpathlineto{\pgfqpoint{15.484366in}{1.685590in}}%
\pgfpathlineto{\pgfqpoint{15.494301in}{1.704500in}}%
\pgfpathlineto{\pgfqpoint{15.504235in}{1.718185in}}%
\pgfpathlineto{\pgfqpoint{15.510858in}{1.724483in}}%
\pgfpathlineto{\pgfqpoint{15.517481in}{1.728627in}}%
\pgfpathlineto{\pgfqpoint{15.524104in}{1.730745in}}%
\pgfpathlineto{\pgfqpoint{15.530727in}{1.730991in}}%
\pgfpathlineto{\pgfqpoint{15.537350in}{1.729541in}}%
\pgfpathlineto{\pgfqpoint{15.547284in}{1.724618in}}%
\pgfpathlineto{\pgfqpoint{15.557218in}{1.717028in}}%
\pgfpathlineto{\pgfqpoint{15.570464in}{1.704032in}}%
\pgfpathlineto{\pgfqpoint{15.623447in}{1.648249in}}%
\pgfpathlineto{\pgfqpoint{15.640004in}{1.636097in}}%
\pgfpathlineto{\pgfqpoint{15.656562in}{1.627044in}}%
\pgfpathlineto{\pgfqpoint{15.679742in}{1.617603in}}%
\pgfpathlineto{\pgfqpoint{15.709545in}{1.605572in}}%
\pgfpathlineto{\pgfqpoint{15.726102in}{1.596637in}}%
\pgfpathlineto{\pgfqpoint{15.742659in}{1.585153in}}%
\pgfpathlineto{\pgfqpoint{15.759217in}{1.570875in}}%
\pgfpathlineto{\pgfqpoint{15.779085in}{1.550314in}}%
\pgfpathlineto{\pgfqpoint{15.805577in}{1.518849in}}%
\pgfpathlineto{\pgfqpoint{15.838691in}{1.479393in}}%
\pgfpathlineto{\pgfqpoint{15.851937in}{1.466501in}}%
\pgfpathlineto{\pgfqpoint{15.861872in}{1.459298in}}%
\pgfpathlineto{\pgfqpoint{15.871806in}{1.455005in}}%
\pgfpathlineto{\pgfqpoint{15.881740in}{1.453711in}}%
\pgfpathlineto{\pgfqpoint{15.894986in}{1.454904in}}%
\pgfpathlineto{\pgfqpoint{15.911543in}{1.456472in}}%
\pgfpathlineto{\pgfqpoint{15.921478in}{1.455707in}}%
\pgfpathlineto{\pgfqpoint{15.931412in}{1.452993in}}%
\pgfpathlineto{\pgfqpoint{15.941346in}{1.448068in}}%
\pgfpathlineto{\pgfqpoint{15.954592in}{1.438265in}}%
\pgfpathlineto{\pgfqpoint{15.961215in}{1.433288in}}%
\pgfpathlineto{\pgfqpoint{15.964527in}{1.432337in}}%
\pgfpathlineto{\pgfqpoint{15.967838in}{1.433630in}}%
\pgfpathlineto{\pgfqpoint{15.974461in}{1.440779in}}%
\pgfpathlineto{\pgfqpoint{15.984395in}{1.455804in}}%
\pgfpathlineto{\pgfqpoint{15.997641in}{1.479758in}}%
\pgfpathlineto{\pgfqpoint{16.017510in}{1.520802in}}%
\pgfpathlineto{\pgfqpoint{16.053936in}{1.597483in}}%
\pgfpathlineto{\pgfqpoint{16.067181in}{1.620685in}}%
\pgfpathlineto{\pgfqpoint{16.080427in}{1.639023in}}%
\pgfpathlineto{\pgfqpoint{16.090362in}{1.648884in}}%
\pgfpathlineto{\pgfqpoint{16.100296in}{1.655028in}}%
\pgfpathlineto{\pgfqpoint{16.106919in}{1.656960in}}%
\pgfpathlineto{\pgfqpoint{16.113542in}{1.657137in}}%
\pgfpathlineto{\pgfqpoint{16.120165in}{1.655570in}}%
\pgfpathlineto{\pgfqpoint{16.126788in}{1.652300in}}%
\pgfpathlineto{\pgfqpoint{16.136722in}{1.644360in}}%
\pgfpathlineto{\pgfqpoint{16.146656in}{1.633082in}}%
\pgfpathlineto{\pgfqpoint{16.159902in}{1.613636in}}%
\pgfpathlineto{\pgfqpoint{16.176459in}{1.584177in}}%
\pgfpathlineto{\pgfqpoint{16.212885in}{1.516982in}}%
\pgfpathlineto{\pgfqpoint{16.222820in}{1.504144in}}%
\pgfpathlineto{\pgfqpoint{16.229443in}{1.498567in}}%
\pgfpathlineto{\pgfqpoint{16.236065in}{1.495826in}}%
\pgfpathlineto{\pgfqpoint{16.242688in}{1.495922in}}%
\pgfpathlineto{\pgfqpoint{16.249311in}{1.498450in}}%
\pgfpathlineto{\pgfqpoint{16.259246in}{1.505338in}}%
\pgfpathlineto{\pgfqpoint{16.292360in}{1.532075in}}%
\pgfpathlineto{\pgfqpoint{16.302294in}{1.536983in}}%
\pgfpathlineto{\pgfqpoint{16.312229in}{1.539501in}}%
\pgfpathlineto{\pgfqpoint{16.322163in}{1.539382in}}%
\pgfpathlineto{\pgfqpoint{16.332097in}{1.536489in}}%
\pgfpathlineto{\pgfqpoint{16.342032in}{1.530784in}}%
\pgfpathlineto{\pgfqpoint{16.351966in}{1.522338in}}%
\pgfpathlineto{\pgfqpoint{16.365212in}{1.507227in}}%
\pgfpathlineto{\pgfqpoint{16.398326in}{1.464551in}}%
\pgfpathlineto{\pgfqpoint{16.401638in}{1.462919in}}%
\pgfpathlineto{\pgfqpoint{16.404949in}{1.462617in}}%
\pgfpathlineto{\pgfqpoint{16.408261in}{1.463743in}}%
\pgfpathlineto{\pgfqpoint{16.414884in}{1.469848in}}%
\pgfpathlineto{\pgfqpoint{16.424818in}{1.485255in}}%
\pgfpathlineto{\pgfqpoint{16.467867in}{1.561679in}}%
\pgfpathlineto{\pgfqpoint{16.481113in}{1.578152in}}%
\pgfpathlineto{\pgfqpoint{16.491047in}{1.587210in}}%
\pgfpathlineto{\pgfqpoint{16.500981in}{1.593430in}}%
\pgfpathlineto{\pgfqpoint{16.510916in}{1.596984in}}%
\pgfpathlineto{\pgfqpoint{16.520850in}{1.598181in}}%
\pgfpathlineto{\pgfqpoint{16.520850in}{1.598181in}}%
\pgfusepath{stroke}%
\end{pgfscope}%
\begin{pgfscope}%
\pgfpathrectangle{\pgfqpoint{2.400000in}{1.081300in}}{\pgfqpoint{14.880000in}{7.569100in}}%
\pgfusepath{clip}%
\pgfsetrectcap%
\pgfsetroundjoin%
\pgfsetlinewidth{1.505625pt}%
\definecolor{currentstroke}{rgb}{1.000000,0.498039,0.054902}%
\pgfsetstrokecolor{currentstroke}%
\pgfsetdash{}{0pt}%
\pgfpathmoveto{\pgfqpoint{3.076364in}{1.425350in}}%
\pgfpathlineto{\pgfqpoint{3.208822in}{1.428228in}}%
\pgfpathlineto{\pgfqpoint{3.324722in}{1.430632in}}%
\pgfpathlineto{\pgfqpoint{3.675736in}{1.431956in}}%
\pgfpathlineto{\pgfqpoint{3.960521in}{1.427438in}}%
\pgfpathlineto{\pgfqpoint{4.016815in}{1.426185in}}%
\pgfpathlineto{\pgfqpoint{4.159208in}{1.431055in}}%
\pgfpathlineto{\pgfqpoint{4.391009in}{1.434039in}}%
\pgfpathlineto{\pgfqpoint{4.457238in}{1.434369in}}%
\pgfpathlineto{\pgfqpoint{4.523467in}{1.432282in}}%
\pgfpathlineto{\pgfqpoint{4.616188in}{1.429339in}}%
\pgfpathlineto{\pgfqpoint{4.685728in}{1.429791in}}%
\pgfpathlineto{\pgfqpoint{4.838055in}{1.431568in}}%
\pgfpathlineto{\pgfqpoint{5.043365in}{1.430349in}}%
\pgfpathlineto{\pgfqpoint{5.169200in}{1.428651in}}%
\pgfpathlineto{\pgfqpoint{5.420870in}{1.431011in}}%
\pgfpathlineto{\pgfqpoint{5.702343in}{1.430505in}}%
\pgfpathlineto{\pgfqpoint{5.795064in}{1.428199in}}%
\pgfpathlineto{\pgfqpoint{5.864604in}{1.427352in}}%
\pgfpathlineto{\pgfqpoint{6.066603in}{1.427503in}}%
\pgfpathlineto{\pgfqpoint{6.116275in}{1.426513in}}%
\pgfpathlineto{\pgfqpoint{6.258667in}{1.429658in}}%
\pgfpathlineto{\pgfqpoint{6.367945in}{1.430796in}}%
\pgfpathlineto{\pgfqpoint{6.467288in}{1.435239in}}%
\pgfpathlineto{\pgfqpoint{6.523583in}{1.436330in}}%
\pgfpathlineto{\pgfqpoint{6.583189in}{1.435063in}}%
\pgfpathlineto{\pgfqpoint{6.685844in}{1.432738in}}%
\pgfpathlineto{\pgfqpoint{6.844794in}{1.431135in}}%
\pgfpathlineto{\pgfqpoint{6.944137in}{1.428884in}}%
\pgfpathlineto{\pgfqpoint{7.245479in}{1.430429in}}%
\pgfpathlineto{\pgfqpoint{7.334888in}{1.430725in}}%
\pgfpathlineto{\pgfqpoint{7.467346in}{1.430543in}}%
\pgfpathlineto{\pgfqpoint{7.599804in}{1.431243in}}%
\pgfpathlineto{\pgfqpoint{7.669345in}{1.429074in}}%
\pgfpathlineto{\pgfqpoint{7.768688in}{1.425894in}}%
\pgfpathlineto{\pgfqpoint{7.831606in}{1.427267in}}%
\pgfpathlineto{\pgfqpoint{7.937572in}{1.429593in}}%
\pgfpathlineto{\pgfqpoint{8.149505in}{1.428023in}}%
\pgfpathlineto{\pgfqpoint{8.205799in}{1.427519in}}%
\pgfpathlineto{\pgfqpoint{8.285274in}{1.426753in}}%
\pgfpathlineto{\pgfqpoint{8.321700in}{1.427923in}}%
\pgfpathlineto{\pgfqpoint{8.417732in}{1.432262in}}%
\pgfpathlineto{\pgfqpoint{8.467404in}{1.431596in}}%
\pgfpathlineto{\pgfqpoint{8.566748in}{1.429586in}}%
\pgfpathlineto{\pgfqpoint{8.629665in}{1.431358in}}%
\pgfpathlineto{\pgfqpoint{8.715763in}{1.433838in}}%
\pgfpathlineto{\pgfqpoint{8.772057in}{1.432889in}}%
\pgfpathlineto{\pgfqpoint{8.897893in}{1.429514in}}%
\pgfpathlineto{\pgfqpoint{9.119760in}{1.430321in}}%
\pgfpathlineto{\pgfqpoint{9.202546in}{1.433619in}}%
\pgfpathlineto{\pgfqpoint{9.258841in}{1.433298in}}%
\pgfpathlineto{\pgfqpoint{9.368118in}{1.431963in}}%
\pgfpathlineto{\pgfqpoint{9.543625in}{1.434088in}}%
\pgfpathlineto{\pgfqpoint{9.695952in}{1.431495in}}%
\pgfpathlineto{\pgfqpoint{9.798607in}{1.431349in}}%
\pgfpathlineto{\pgfqpoint{9.874770in}{1.428283in}}%
\pgfpathlineto{\pgfqpoint{9.924442in}{1.427080in}}%
\pgfpathlineto{\pgfqpoint{10.156243in}{1.429107in}}%
\pgfpathlineto{\pgfqpoint{10.285390in}{1.434020in}}%
\pgfpathlineto{\pgfqpoint{10.364865in}{1.433360in}}%
\pgfpathlineto{\pgfqpoint{10.454274in}{1.433243in}}%
\pgfpathlineto{\pgfqpoint{10.636404in}{1.434944in}}%
\pgfpathlineto{\pgfqpoint{10.878140in}{1.429753in}}%
\pgfpathlineto{\pgfqpoint{11.265579in}{1.430545in}}%
\pgfpathlineto{\pgfqpoint{11.454332in}{1.433074in}}%
\pgfpathlineto{\pgfqpoint{11.679510in}{1.432324in}}%
\pgfpathlineto{\pgfqpoint{11.795411in}{1.431169in}}%
\pgfpathlineto{\pgfqpoint{12.004032in}{1.432486in}}%
\pgfpathlineto{\pgfqpoint{12.109999in}{1.431598in}}%
\pgfpathlineto{\pgfqpoint{12.239145in}{1.431260in}}%
\pgfpathlineto{\pgfqpoint{12.239145in}{1.431260in}}%
\pgfusepath{stroke}%
\end{pgfscope}%
\begin{pgfscope}%
\pgfsetrectcap%
\pgfsetmiterjoin%
\pgfsetlinewidth{0.803000pt}%
\definecolor{currentstroke}{rgb}{0.000000,0.000000,0.000000}%
\pgfsetstrokecolor{currentstroke}%
\pgfsetdash{}{0pt}%
\pgfpathmoveto{\pgfqpoint{2.400000in}{1.081300in}}%
\pgfpathlineto{\pgfqpoint{2.400000in}{8.650400in}}%
\pgfusepath{stroke}%
\end{pgfscope}%
\begin{pgfscope}%
\pgfsetrectcap%
\pgfsetmiterjoin%
\pgfsetlinewidth{0.803000pt}%
\definecolor{currentstroke}{rgb}{0.000000,0.000000,0.000000}%
\pgfsetstrokecolor{currentstroke}%
\pgfsetdash{}{0pt}%
\pgfpathmoveto{\pgfqpoint{17.280000in}{1.081300in}}%
\pgfpathlineto{\pgfqpoint{17.280000in}{8.650400in}}%
\pgfusepath{stroke}%
\end{pgfscope}%
\begin{pgfscope}%
\pgfsetrectcap%
\pgfsetmiterjoin%
\pgfsetlinewidth{0.803000pt}%
\definecolor{currentstroke}{rgb}{0.000000,0.000000,0.000000}%
\pgfsetstrokecolor{currentstroke}%
\pgfsetdash{}{0pt}%
\pgfpathmoveto{\pgfqpoint{2.400000in}{1.081300in}}%
\pgfpathlineto{\pgfqpoint{17.280000in}{1.081300in}}%
\pgfusepath{stroke}%
\end{pgfscope}%
\begin{pgfscope}%
\pgfsetrectcap%
\pgfsetmiterjoin%
\pgfsetlinewidth{0.803000pt}%
\definecolor{currentstroke}{rgb}{0.000000,0.000000,0.000000}%
\pgfsetstrokecolor{currentstroke}%
\pgfsetdash{}{0pt}%
\pgfpathmoveto{\pgfqpoint{2.400000in}{8.650400in}}%
\pgfpathlineto{\pgfqpoint{17.280000in}{8.650400in}}%
\pgfusepath{stroke}%
\end{pgfscope}%
\end{pgfpicture}%
\makeatother%
\endgroup%
}
\end{figure}

aaa \begin{tikzpicture}
	\draw[->] (0, 0) -- (1, 1);
\end{tikzpicture} aaa

\begin{figure}[H]
\centering
\begin{minipage}[t]{.25\linewidth}\centering
\qrh{https://jq.qq.com/?_wv=1027&k=57DLax6}
\end{minipage}
\begin{minipage}[t]{.25\linewidth}\centering
\qrh{http://weixin.qq.com/r/Zjn-54fE5tqZrcOE92x0}
\end{minipage}
\caption*{\heiti\zihao{3}欢迎扫码加群及关注公众号}
\end{figure}

\begin{lstlisting}[language = C]
  #include <stdio.h>

  int main()
  {
      printf("Hello world!");
  }
\end{lstlisting}

``''

\verb”%”

\begin{algorithm}[H]
\caption{粒子群算法}\label{alg:PSO}
\zihao{-5}
\begin{algorithmic}[1]
	\Require 直杆影子端点坐标 $\(x_i^*, y_i^*\), i = 1, 2, \cdots, n$.
	\Ensure 纬度 $\phi$, 经度 $\psi$.
	\State 在纬度 $\phi \in \l[-\dfrac{\pi}{2}, \dfrac{\pi}{2}\r]$, 地方时 $t \in [720, 1380]$ 的范围内随机创建粒子.
\end{algorithmic}
\end{algorithm}

\begin{thebibliography}{1}
%\setlength{\parskip}{0pt}
%\setlength{\baselineskip}{16pt}
%\setlength{\leftskip}{-19pt}
%\setlength{\itemsep}{0em}
%\setlength{\itemindent}{2em}
%\setlength{\labelwidth}{0pt}
%\setlength{\labelsep}{0pt}

\addcontentsline{toc}{section}{参考文献}	% 将 "参考文献" 加入目录中

\bibitem{RongYuan} 袁荣. 常微分方程[M]. 北京: 高等教育出版社, 2012:59,62.
\end{thebibliography}
%\end{multicols}

\begin{center}
{\zihao{-3} \bf Matrix decomposition}

{\zihao{-4} QUE~Jia-hao}

{\zihao{-5} (School of Mathematical Sciences, Beijing Normal University, Beijing 100875, China)}
\end{center}

{\zihao{5} \bf Abstract}~.

{\zihao{5} \bf Key words}~Matrix;~LU decomposition;

\begin{appendices}

\section{\bfseries Some Appendix2}\label{aaa}
The contents...

\end{appendices}

\end{document}
