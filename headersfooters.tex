% !Mode:: "TeX:UTF-8"
\newcommand{\makefirstpageheadrule}{	% 定义首页页眉线绘制命令, 这里为等宽双线
\makebox[0pt][l]{\rule[0.55\baselineskip]{\headwidth}{0.4pt}}%
\rule[0.7\baselineskip]{\headwidth}{0.4pt}}
\newcommand{\makeheadrule}{	% 定义正文页页眉线绘制命令, 单线
\rule[0.7\baselineskip]{\headwidth}{0.4pt}}

\newboolean{first}	% 定义一个布尔变量用于判断是否为首页
\setboolean{first}{true}	% 设定 first 变量初值为 true, 根据布尔变量 first 为 true 或 false 分别执行不同的页眉线绘制命令
\renewcommand{\headrule}{\ifthenelse{\boolean{first}}{\makeheadrule}{\makefirstpageheadrule}}

% \ctexset {today = old}
\newdateformat{monthyeardate}{\monthname[\THEMONTH], \THEYEAR}
\renewcommand{\dateseparator}{\shortdate}
\fancypagestyle{plain}{\setboolean{first}{false}	% 在 plain 样式的定义中将 first 重置为 false
\lhead{\songti\zihao{-5} \number\year ~年 \number\month ~月} \chead{\zihao{-5} 实\quad 变\quad 函\quad 数\quad 笔\quad 记} \rhead{\zihao{-5}\currenttime, \today}%\monthyeardate\today}
\lfoot{} \cfoot{} \rfoot{}}

\pagestyle{fancy}
\fancyhf{}
\lhead{} \chead{\zihao{-5} 阙嘉豪: 矩阵分解} \rhead{\zihao{-5}\thepage}
\lfoot{} \cfoot{} \rfoot{}

% 奇偶页页眉
%\pagestyle{fancy}
%\fancyhf{}
%\fancyhead[LE]{\kaishu\zihao{-5}\thepage\quad \leftmark}
%\fancyhead[RO]{\kaishu\zihao{-5}\rightmark\quad \thepage}
%\renewcommand\sectionmark[1]{%
%\markright{\CTEXifname{\CTEXthesection\quad}{}#1}}
